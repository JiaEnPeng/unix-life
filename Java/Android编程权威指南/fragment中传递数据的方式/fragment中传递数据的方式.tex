% !TeX spellcheck = en_US
%% 字体:方正静蕾简体
%%		 方正粗宋
\documentclass[a4paper,left=2.5cm,right=2.5cm,11pt]{article}

\usepackage[utf8]{inputenc}
\usepackage{fontspec}
\usepackage{cite}
\usepackage{xeCJK}
\usepackage{indentfirst}
\usepackage{titlesec}
\usepackage{longtable}
\usepackage{graphicx}
\usepackage{float}
\usepackage{rotating}
\usepackage{subfigure}
\usepackage{tabu}
\usepackage{amsmath}
\usepackage{setspace}
\usepackage{amsfonts}
\usepackage{appendix}
\usepackage{listings}
\usepackage{xcolor}
\usepackage{geometry}
\setcounter{secnumdepth}{4}
\usepackage{mhchem}
\usepackage{multirow}
\usepackage{extarrows}
\usepackage{hyperref}
\titleformat*{\section}{\LARGE}
\renewcommand\refname{参考文献}
\renewcommand{\abstractname}{\sihao \cjkfzcs 摘{  }要}
%\titleformat{\chapter}{\centering\bfseries\huge\wryh}{}{0.7em}{}{}
%\titleformat{\section}{\LARGE\bf}{\thesection}{1em}{}{}
\titleformat{\subsection}{\Large\bfseries}{\thesubsection}{1em}{}{}
\titleformat{\subsubsection}{\large\bfseries}{\thesubsubsection}{1em}{}{}
\renewcommand{\contentsname}{{\cjkfzcs \centerline{目{  } 录}}}
\setCJKfamilyfont{cjkhwxk}{STXingkai}
\setCJKfamilyfont{cjkfzcs}{STSongti-SC-Regular}
% \setCJKfamilyfont{cjkhwxk}{华文行楷}
% \setCJKfamilyfont{cjkfzcs}{方正粗宋简体}
\newcommand*{\cjkfzcs}{\CJKfamily{cjkfzcs}}
\newcommand*{\cjkhwxk}{\CJKfamily{cjkhwxk}}
\newfontfamily\wryh{Microsoft YaHei}
\newfontfamily\hwzs{STZhongsong}
\newfontfamily\hwst{STSong}
\newfontfamily\hwfs{STFangsong}
\newfontfamily\jljt{MicrosoftYaHei}
\newfontfamily\hwxk{STXingkai}
% \newfontfamily\hwzs{华文中宋}
% \newfontfamily\hwst{华文宋体}
% \newfontfamily\hwfs{华文仿宋}
% \newfontfamily\jljt{方正静蕾简体}
% \newfontfamily\hwxk{华文行楷}
\newcommand{\verylarge}{\fontsize{60pt}{\baselineskip}\selectfont}  
\newcommand{\chuhao}{\fontsize{44.9pt}{\baselineskip}\selectfont}  
\newcommand{\xiaochu}{\fontsize{38.5pt}{\baselineskip}\selectfont}  
\newcommand{\yihao}{\fontsize{27.8pt}{\baselineskip}\selectfont}  
\newcommand{\xiaoyi}{\fontsize{25.7pt}{\baselineskip}\selectfont}  
\newcommand{\erhao}{\fontsize{23.5pt}{\baselineskip}\selectfont}  
\newcommand{\xiaoerhao}{\fontsize{19.3pt}{\baselineskip}\selectfont} 
\newcommand{\sihao}{\fontsize{14pt}{\baselineskip}\selectfont}      % 字号设置  
\newcommand{\xiaosihao}{\fontsize{12pt}{\baselineskip}\selectfont}  % 字号设置  
\newcommand{\wuhao}{\fontsize{10.5pt}{\baselineskip}\selectfont}    % 字号设置  
\newcommand{\xiaowuhao}{\fontsize{9pt}{\baselineskip}\selectfont}   % 字号设置  
\newcommand{\liuhao}{\fontsize{7.875pt}{\baselineskip}\selectfont}  % 字号设置  
\newcommand{\qihao}{\fontsize{5.25pt}{\baselineskip}\selectfont}    % 字号设置 

\usepackage{diagbox}
\usepackage{multirow}
\boldmath
\XeTeXlinebreaklocale "zh"
\XeTeXlinebreakskip = 0pt plus 1pt minus 0.1pt
\definecolor{cred}{rgb}{0.8,0.8,0.8}
\definecolor{cgreen}{rgb}{0,0.3,0}
\definecolor{cpurple}{rgb}{0.5,0,0.35}
\definecolor{cdocblue}{rgb}{0,0,0.3}
\definecolor{cdark}{rgb}{0.95,1.0,1.0}
\lstset{
	language=java,
	numbers=left,
	numberstyle=\tiny\color{black},
	showspaces=false,
	showstringspaces=false,
	basicstyle=\scriptsize,
	keywordstyle=\color{purple},
	commentstyle=\itshape\color{cgreen},
	stringstyle=\color{blue},
	frame=lines,
	% escapeinside=``,
	extendedchars=true, 
	xleftmargin=1em,
	xrightmargin=1em, 
	backgroundcolor=\color{cred},
	aboveskip=1em,
	breaklines=true,
	tabsize=4
} 

\newfontfamily{\consolas}{Consolas}
\newfontfamily{\monaco}{Monaco}
\setmonofont[Mapping={}]{Consolas}	%英文引号之类的正常显示,相当于设置英文字体
\setsansfont{Consolas} %设置英文字体 Monaco, Consolas,  Fantasque Sans Mono
\setmainfont{Times New Roman}

\setCJKmainfont{华文中宋}


\newcommand{\fic}[1]{\begin{figure}[H]
		\center
		\includegraphics[width=0.8\textwidth]{#1}
	\end{figure}}
	
\newcommand{\sizedfic}[2]{\begin{figure}[H]
		\center
		\includegraphics[width=#1\textwidth]{#2}
	\end{figure}}

\newcommand{\codefile}[1]{\lstinputlisting{#1}}

\newcommand{\interval}{\vspace{0.5em}}

% 改变段间隔
\setlength{\parskip}{0.2em}
\linespread{1.1}

\usepackage{lastpage}
\usepackage{fancyhdr}
\pagestyle{fancy}
\lhead{\space \qquad \space}
\chead{fragment中传递数据的方式 \qquad}
\rhead{\qquad\thepage/\pageref{LastPage}}
\begin{document}

% \tableofcontents

% \clearpage

\section{fragment中传递数据的方式}
\subsection{从父fragment启动子fragment}
	现在父fragment为CrimeListFragment,它启动CrimeActivity,随后CrimeActivity启动CrimeFragment。\par

	步骤如下:
	\begin{itemize}
		\item[1.] 使用Intent对象和startActivity()从CrimeListFragment向CrimeActivity传递数据。
		\item[2.] 在CrimeActivity中使用getIntent()获得Intent对象,从而获得其中的数据。
		          随后使用Bundle对象附带数据,然后再将Bundle对象附加给fragment对象。
		\item[3.] 在CrimeFragment中使用getArguments()方法获得Bundle对象,从而获得其中的数据。
	\end{itemize}

	首先从CrimeListFragment向CrimeActivity传递数据:
	\begin{lstlisting}
	// CrimeActivity.java
	public class CrimeActivity extends SingleFragmentActivity
	{
		public static final String EXTRA_CRIME_ID = "com.bignerdranch.android.criminalintent.crime_id";

		public static Intent newIntent(Context packageContext, UUID crimeId)
		{
			Intent intent = new Intent(packageContext, CrimeActivity.class);
			intent.putExtra(EXTRA_CRIME_ID, crimeId);
			return intent;
		}
	}

	// CrimeListFragment.java
	private class CrimeHolder extends RecyclerView.ViewHolder implements View.OnClickListener
	{
		...
		@Override
		public void onClick(View v)
		{
			Intent intent = CrimeActivity.newIntent(getActivity(), mCrime.getId());
			startActivity(intent);
		}
	}
	\end{lstlisting}

	在CrimeActivity中获得传递的数据,然后传递给CrimeFragment:
	\begin{lstlisting}
	// CrimeFragment.java
	public class CrimeFragment extends Fragment
	{
		private static final String ARG_CRIME_ID = "crime_id";

		public static CrimeFragment newInstance(UUID crimeId)
		{
			Bundle args = new Bundle();
			args.putSerializable(ARG_CRIME_ID, crimeId);

			CrimeFragment fragment = new CrimeFragment();
			fragment.setArguments(args);
			return fragment;
		}
	}

	// CrimeActivity.java
	public class CrimeActivity extends SingleFragmentActivity
	{
		private static final String EXTRA_CRIME_ID = "com.bignerdranch.android.criminalintent.crime_id";
		...
		@Override
		protected Fragment createFragment()
		{
			UUID crimeId = (UUID)getIntent().getSerializableExtra(EXTRA_CRIME_ID);
			return CrimeFragment.newInstance(crimeId);
		}
	}
	\end{lstlisting}

	在CrimeFragment中获取传递的数据:
	\begin{lstlisting}
	// CrimeFragment.java
	@Override
	public void onCreate(Bundle savedInstanceState)
	{
		super.onCreate(savedInstanceState);

		UUID crimeId = (UUID)getArguments().getSerializable(ARG_CRIME_ID);
		...
	}
	\end{lstlisting}

\subsection{从子fragment返回到父fragment}
	这里返回数据的方法和activity中返回数据的方法类似,步骤如下:
	\begin{itemize}
		\item[1.] 父fragment使用startActivity()启动子fragment。
		\item[2.] 子fragment使用Activity.setResult()来设置返回的数据。
		\item[3.] 从子fragment返回父fragment,父fragment的onActivityResult()方法将被自动调用。
	\end{itemize}

	需要注意的是,Fragment类只有startActivityForResult()和onActivityResult(),Fragment只能调用Activity.setResult()来设置返回的数据。\par

	父fragment使用startActivity()启动子fragment:
	\begin{lstlisting}
	public class CrimeHolder extends RecyclerView.ViewHolder implements View.OnClickListener
	{
		private static final int REQUEST_CRIME = 1;
		...
		@Override
		public void onClick(View v)
		{
			Intent intent = CrimeActivity.newIntent(getActivity(), mCrime.getId());
			startActivityForResult(intent, REQUEST_CRIME);
		}
	}
	\end{lstlisting}

	子fragment使用Activity.setResult()来设置返回的数据:
	\begin{lstlisting}
	public class CrimeFragment extends Fragment
	{
		private static final String ARG_CRIME_ID = "crime_id";
		public static final String TITLE = "crime_title";
		public static final String CHECKBOX = "crime_checkbox";
		private Intent data;

		@Override
		public void onCreate(Bundle savedInstancesState)
		{
			super.onCreate(savedInstancesState);
			...
			data = new Intent();
		}

		@Override
		public View onCreateView(LayoutInflater inflater, ViewGroup container, Bundle savedInstancesState)
		{
			View v = inflater.inflate(R.layout.fragment_crime, container, false);
			...
			mTitleField.addTextChangedListener(new TextWatcher() {
				@Override
				public void beforeTextChanged(CharSequence s, int start, int count, int after) {

				}

				@Override
				public void onTextChanged(CharSequence s, int start, int before, int count) {
					mCrime.setTitle(s.toString());
					data.putExtra(TITLE, s.toString());
					getActivity().setResult(ACTIVITY.RESULT_OK, data);
				}

				@Override
				public void afterTextChanged(Editable s) {

				}
			});
			...
			mSolvedCheckBox.setOnCheckedChangeListener(new CompoundButton.OnCheckedChangeListener() {
				@Override
				public void onCheckedChanged(CompoundButton buttonView, boolean isChecked)
				{
					mCrime.setSolved(isChecked);
					data.putExtra(TITLE, isChecked);
					getActivity().setResult(ACTIVITY.RESULT_OK, data);
				}
			});

			return v;
		}
		...
	}
	\end{lstlisting}

	从子fragment返回父fragment,父fragment的onActivityResult()方法将被自动调用。
	我们覆盖onActivityResult()方法,来处理Intent对象:
	\begin{lstlisting}
	public class CrimeListFragment extends Fragment
	{
		...
		@Override
		public void onActivityResult(int requestCode, int resultCode, Intent data)
		{
			if(resultCode != ACTIVITY.RESULT_OK)
				return;

			if(requestCode == REQUEST_CRIME)
			{
				String title = data.getStringExtra(CrimeFragment.TITLE);
				boolean isChecked = data.getBooleanExtra(CrimeFragment.CHECKBOX, false);
				...
			}
		}
	}
	\end{lstlisting}

\end{document}
