% !TeX spellcheck = en_US
%% 字体:方正静蕾简体
%%		 方正粗宋
\documentclass[a4paper,left=2.5cm,right=2.5cm,11pt]{article}

\usepackage[utf8]{inputenc}
\usepackage{fontspec}
\usepackage{cite}
\usepackage{xeCJK}
\usepackage{indentfirst}
\usepackage{titlesec}
\usepackage{longtable}
\usepackage{graphicx}
\usepackage{float}
\usepackage{rotating}
\usepackage{subfigure}
\usepackage{tabu}
\usepackage{amsmath}
\usepackage{setspace}
\usepackage{amsfonts}
\usepackage{appendix}
\usepackage{listings}
\usepackage{xcolor}
\usepackage{geometry}
\setcounter{secnumdepth}{4}
\usepackage{mhchem}
\usepackage{multirow}
\usepackage{extarrows}
\usepackage{hyperref}
\titleformat*{\section}{\LARGE}
\renewcommand\refname{参考文献}
\renewcommand{\abstractname}{\sihao \cjkfzcs 摘{  }要}
%\titleformat{\chapter}{\centering\bfseries\huge\wryh}{}{0.7em}{}{}
%\titleformat{\section}{\LARGE\bf}{\thesection}{1em}{}{}
\titleformat{\subsection}{\Large\bfseries}{\thesubsection}{1em}{}{}
\titleformat{\subsubsection}{\large\bfseries}{\thesubsubsection}{1em}{}{}
\renewcommand{\contentsname}{{\cjkfzcs \centerline{目{  } 录}}}
\setCJKfamilyfont{cjkhwxk}{STXingkai}
\setCJKfamilyfont{cjkfzcs}{STSongti-SC-Regular}
% \setCJKfamilyfont{cjkhwxk}{华文行楷}
% \setCJKfamilyfont{cjkfzcs}{方正粗宋简体}
\newcommand*{\cjkfzcs}{\CJKfamily{cjkfzcs}}
\newcommand*{\cjkhwxk}{\CJKfamily{cjkhwxk}}
\newfontfamily\wryh{Microsoft YaHei}
\newfontfamily\hwzs{STZhongsong}
\newfontfamily\hwst{STSong}
\newfontfamily\hwfs{STFangsong}
\newfontfamily\jljt{MicrosoftYaHei}
\newfontfamily\hwxk{STXingkai}
% \newfontfamily\hwzs{华文中宋}
% \newfontfamily\hwst{华文宋体}
% \newfontfamily\hwfs{华文仿宋}
% \newfontfamily\jljt{方正静蕾简体}
% \newfontfamily\hwxk{华文行楷}
\newcommand{\verylarge}{\fontsize{60pt}{\baselineskip}\selectfont}  
\newcommand{\chuhao}{\fontsize{44.9pt}{\baselineskip}\selectfont}  
\newcommand{\xiaochu}{\fontsize{38.5pt}{\baselineskip}\selectfont}  
\newcommand{\yihao}{\fontsize{27.8pt}{\baselineskip}\selectfont}  
\newcommand{\xiaoyi}{\fontsize{25.7pt}{\baselineskip}\selectfont}  
\newcommand{\erhao}{\fontsize{23.5pt}{\baselineskip}\selectfont}  
\newcommand{\xiaoerhao}{\fontsize{19.3pt}{\baselineskip}\selectfont} 
\newcommand{\sihao}{\fontsize{14pt}{\baselineskip}\selectfont}      % 字号设置  
\newcommand{\xiaosihao}{\fontsize{12pt}{\baselineskip}\selectfont}  % 字号设置  
\newcommand{\wuhao}{\fontsize{10.5pt}{\baselineskip}\selectfont}    % 字号设置  
\newcommand{\xiaowuhao}{\fontsize{9pt}{\baselineskip}\selectfont}   % 字号设置  
\newcommand{\liuhao}{\fontsize{7.875pt}{\baselineskip}\selectfont}  % 字号设置  
\newcommand{\qihao}{\fontsize{5.25pt}{\baselineskip}\selectfont}    % 字号设置 

\usepackage{diagbox}
\usepackage{multirow}
\boldmath
\XeTeXlinebreaklocale "zh"
\XeTeXlinebreakskip = 0pt plus 1pt minus 0.1pt
\definecolor{cred}{rgb}{0.8,0.8,0.8}
\definecolor{cgreen}{rgb}{0,0.3,0}
\definecolor{cpurple}{rgb}{0.5,0,0.35}
\definecolor{cdocblue}{rgb}{0,0,0.3}
\definecolor{cdark}{rgb}{0.95,1.0,1.0}
\lstset{
	language=java,
	numbers=left,
	numberstyle=\tiny\color{black},
	showspaces=false,
	showstringspaces=false,
	basicstyle=\scriptsize,
	keywordstyle=\color{purple},
	commentstyle=\itshape\color{cgreen},
	stringstyle=\color{blue},
	frame=lines,
	% escapeinside=``,
	extendedchars=true, 
	xleftmargin=1em,
	xrightmargin=1em, 
	backgroundcolor=\color{cred},
	aboveskip=1em,
	breaklines=true,
	tabsize=4
} 

\newfontfamily{\consolas}{Consolas}
\newfontfamily{\monaco}{Monaco}
\setmonofont[Mapping={}]{Consolas}	%英文引号之类的正常显示,相当于设置英文字体
\setsansfont{Consolas} %设置英文字体 Monaco, Consolas,  Fantasque Sans Mono
\setmainfont{Times New Roman}

\setCJKmainfont{华文中宋}


\newcommand{\fic}[1]{\begin{figure}[H]
		\center
		\includegraphics[width=0.8\textwidth]{#1}
	\end{figure}}
	
\newcommand{\sizedfic}[2]{\begin{figure}[H]
		\center
		\includegraphics[width=#1\textwidth]{#2}
	\end{figure}}

\newcommand{\codefile}[1]{\lstinputlisting{#1}}

\newcommand{\interval}{\vspace{0.5em}}

% 改变段间隔
\setlength{\parskip}{0.2em}
\linespread{1.1}

\usepackage{lastpage}
\usepackage{fancyhdr}
\pagestyle{fancy}
\lhead{\space \qquad \space}
\chead{使用RecyclerView \qquad}
\rhead{\qquad\thepage/\pageref{LastPage}}
\begin{document}

% \tableofcontents

% \clearpage

\section{使用RecyclerView}
	RecyclerView对象用于回收和定位屏幕上的View对象。
	其中定位View对象的任务交给了LayoutManager,LayoutManager可以在屏幕上定位列表项,还能负责定义屏幕滚动行为。
	而RecyclerView对象借助Adapter子类和ViewHolder子类来显示View对象。\par

	使用RecyclerView的步骤如下:
	\begin{itemize}
		\item 指定LayoutManger。
		\item 实现ViewHolder子类。
		\item 实现Adapter子类,需要覆盖其中的onCreateViewHolder()、onBindViewHolder()和getItemCount()三个方法。
		\item 将RecyclerView和Adapter子类相关联。
	\end{itemize}

\subsection{指定LayoutManger}
	LayoutManger有很多版本,比如LinearLayoutManager、GridLayoutManager等。使用例子如下:
	\begin{lstlisting}
	public class CrimeListFragment extends Fragment
	{
		private RecyclerView mCrimeRecycleView;

		@Override
		public View onCreateView(LayoutInflater inflater, ViewGroup container, BUndle savedInstanceState)
		{
			View view = inflater.inflate(R.layout.fragment_crime_list, container, false);
			mCrimeRecycleView = (RecyclerView)view.findViewById(R.id.crime_recycler_view);
			mCrimeRecycleView.setLayoutManager(new LinearLayoutManager(getActivity()));

			return view;
		}
	}
	\end{lstlisting}

\subsection{ViewHolder子类}
	ViewHolder通过itemView域容纳View视图,使用例子如下:
	\begin{lstlisting}
	public class CrimeHolder extends RecyclerView.ViewHolder
	{
		public TextView mTitleTextView;

		public CrimeHolder(View itemView)
		{
			super(itemView);

			mTitleTextView = (TextView)itemView;
		}
	}
	\end{lstlisting}

\subsection{Adapter子类}
	RecyclerView自己不会创建ViewHolder,这个任务交由Adapter子类来完成。\par

	Adapter子类是一个控制器对象,它创建必要的ViewHolder,然后从模型层获取数据,绑定ViewHolder至模型层数据,
	也就是使用模型数据填充视图,然后提供给RecyclerView显示。\par

	当RecyclerView需要新的View视图来显示列表项时,Adapter子类会调用onCreateViewHolder()方法,
	从而创建View视图,然后封装到ViewHolder中。随后Adapter子类再调用onBindViewHolder()方法使用模型数据来填充View视图。\par

	使用例子如下:
	\begin{lstlisting}
	private class CrimeAdapter extends RecyclerView.Adapter<CrimeHolder>
	{
		private List<Crime> mCrimes;

		public CrimeAdapter(List<Crime> crimes)
		{
			mCrimes = crimes;
		}

		@Override
		public CrimeHolder onCreateViewHolder(ViewGroup parent, int viewType)
		{
			LayoutInflater layoutInflater = LayoutInflater.from(getActivity());
			View view = layoutInflater.inflate(android.R.layout.simple_list_item_1, parent, false);
			return new CrimeHolder(view);
		}

		@Override
		public void onBindViewHolder(CrimeHolder holder, int position)
		{
			Crime crime = mCrimes.get(position);
			holder.mTitleTextView.setText(crime.getTitle());
		}

		@Override
		public int getItemCount()
		{
			return mCrimes.size();
		}
	}
	\end{lstlisting}

\subsection{关联Adapter和RecyclerView}
	关联的代码如下:
	\begin{lstlisting}
	public class CrimeListFragment extends Fragment
	{
		private RecyclerView mCrimeRecycleView;
		private CrimeAdapter mAdapter;

		@Override
		public View onCreateView(LayoutInflater inflater, ViewGroup container, Bundle savedInstanceState)
		{
			View view = inflater.inflate(R.layout.fragment_crime_list, container, false);
			mCrimeRecycleView = (RecyclerView)view.findViewById(R.id.crime_recycler_view);
			mCrimeRecycleView.setLayoutManager(new LinearLayoutManager(getActivity()));

			CrimeLab crimeLab = CrimeLab.get(getActivity());
			List<Crime> crimes = crimeLab.getCrimes();
			mAdapter = new CrimeAdapter(crimes);
			mCrimeRecycleView.setAdapter(mAdapter);
		}
	}
	\end{lstlisting}

\subsection{刷新ViewHolder中的数据}
	如果从CrimeListFragment启动CrimeActivity并修改了模型层的数据,返回CrimeListFragment时我们需要实现视图对象数据的更新。\par

	CrimeListFragment使用的是RecyclerView,
	它可以借助Adapter.notifyDataSetChanged()方法来更新ViewHolder中的数据,使用例子如下:
	\begin{lstlisting}
	public class CrimeListFragment extends Fragment
	{
		...
		@Override
		public void onResume()
		{
			super.onResume();
			updateUI();
		}

		private void updateUI()
		{
			CrimeLab crimeLab = CrimeLab.get(getActivity());
			List<Crime> crimes = crimeLab.getCrimes();

			if (mAdapter == null)
			{
				mAdapter = new CrimeAdapter(crimes);
				mCrimeRecyclerView.setAdapter(mAdapter);
			}
			else
				mAdapter.notifyDataSetChanged();
		}
	}
	\end{lstlisting}

\end{document}
