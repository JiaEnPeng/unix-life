% !TeX spellcheck = en_US
%% 字体:方正静蕾简体
%%		 方正粗宋
\documentclass[a4paper,left=2.5cm,right=2.5cm,11pt]{article}

\usepackage[utf8]{inputenc}
\usepackage{fontspec}
\usepackage{cite}
\usepackage{xeCJK}
\usepackage{indentfirst}
\usepackage{titlesec}
\usepackage{longtable}
\usepackage{graphicx}
\usepackage{float}
\usepackage{rotating}
\usepackage{subfigure}
\usepackage{tabu}
\usepackage{amsmath}
\usepackage{setspace}
\usepackage{amsfonts}
\usepackage{appendix}
\usepackage{listings}
\usepackage{xcolor}
\usepackage{geometry}
\setcounter{secnumdepth}{4}
\usepackage{mhchem}
\usepackage{multirow}
\usepackage{extarrows}
\usepackage{hyperref}
\titleformat*{\section}{\LARGE}
\renewcommand\refname{参考文献}
\renewcommand{\abstractname}{\sihao \cjkfzcs 摘{  }要}
%\titleformat{\chapter}{\centering\bfseries\huge\wryh}{}{0.7em}{}{}
%\titleformat{\section}{\LARGE\bf}{\thesection}{1em}{}{}
\titleformat{\subsection}{\Large\bfseries}{\thesubsection}{1em}{}{}
\titleformat{\subsubsection}{\large\bfseries}{\thesubsubsection}{1em}{}{}
\renewcommand{\contentsname}{{\cjkfzcs \centerline{目{  } 录}}}
\setCJKfamilyfont{cjkhwxk}{STXingkai}
\setCJKfamilyfont{cjkfzcs}{STSongti-SC-Regular}
% \setCJKfamilyfont{cjkhwxk}{华文行楷}
% \setCJKfamilyfont{cjkfzcs}{方正粗宋简体}
\newcommand*{\cjkfzcs}{\CJKfamily{cjkfzcs}}
\newcommand*{\cjkhwxk}{\CJKfamily{cjkhwxk}}
\newfontfamily\wryh{Microsoft YaHei}
\newfontfamily\hwzs{STZhongsong}
\newfontfamily\hwst{STSong}
\newfontfamily\hwfs{STFangsong}
\newfontfamily\jljt{MicrosoftYaHei}
\newfontfamily\hwxk{STXingkai}
% \newfontfamily\hwzs{华文中宋}
% \newfontfamily\hwst{华文宋体}
% \newfontfamily\hwfs{华文仿宋}
% \newfontfamily\jljt{方正静蕾简体}
% \newfontfamily\hwxk{华文行楷}
\newcommand{\verylarge}{\fontsize{60pt}{\baselineskip}\selectfont}  
\newcommand{\chuhao}{\fontsize{44.9pt}{\baselineskip}\selectfont}  
\newcommand{\xiaochu}{\fontsize{38.5pt}{\baselineskip}\selectfont}  
\newcommand{\yihao}{\fontsize{27.8pt}{\baselineskip}\selectfont}  
\newcommand{\xiaoyi}{\fontsize{25.7pt}{\baselineskip}\selectfont}  
\newcommand{\erhao}{\fontsize{23.5pt}{\baselineskip}\selectfont}  
\newcommand{\xiaoerhao}{\fontsize{19.3pt}{\baselineskip}\selectfont} 
\newcommand{\sihao}{\fontsize{14pt}{\baselineskip}\selectfont}      % 字号设置  
\newcommand{\xiaosihao}{\fontsize{12pt}{\baselineskip}\selectfont}  % 字号设置  
\newcommand{\wuhao}{\fontsize{10.5pt}{\baselineskip}\selectfont}    % 字号设置  
\newcommand{\xiaowuhao}{\fontsize{9pt}{\baselineskip}\selectfont}   % 字号设置  
\newcommand{\liuhao}{\fontsize{7.875pt}{\baselineskip}\selectfont}  % 字号设置  
\newcommand{\qihao}{\fontsize{5.25pt}{\baselineskip}\selectfont}    % 字号设置 

\usepackage{diagbox}
\usepackage{multirow}
\boldmath
\XeTeXlinebreaklocale "zh"
\XeTeXlinebreakskip = 0pt plus 1pt minus 0.1pt
\definecolor{cred}{rgb}{0.8,0.8,0.8}
\definecolor{cgreen}{rgb}{0,0.3,0}
\definecolor{cpurple}{rgb}{0.5,0,0.35}
\definecolor{cdocblue}{rgb}{0,0,0.3}
\definecolor{cdark}{rgb}{0.95,1.0,1.0}
\lstset{
	language=java,
	numbers=left,
	numberstyle=\tiny\color{black},
	showspaces=false,
	showstringspaces=false,
	basicstyle=\scriptsize,
	keywordstyle=\color{purple},
	commentstyle=\itshape\color{cgreen},
	stringstyle=\color{blue},
	frame=lines,
	% escapeinside=``,
	extendedchars=true, 
	xleftmargin=1em,
	xrightmargin=1em, 
	backgroundcolor=\color{cred},
	aboveskip=1em,
	breaklines=true,
	tabsize=4
} 

\newfontfamily{\consolas}{Consolas}
\newfontfamily{\monaco}{Monaco}
\setmonofont[Mapping={}]{Consolas}	%英文引号之类的正常显示,相当于设置英文字体
\setsansfont{Consolas} %设置英文字体 Monaco, Consolas,  Fantasque Sans Mono
\setmainfont{Times New Roman}

\setCJKmainfont{华文中宋}


\newcommand{\fic}[1]{\begin{figure}[H]
		\center
		\includegraphics[width=0.8\textwidth]{#1}
	\end{figure}}
	
\newcommand{\sizedfic}[2]{\begin{figure}[H]
		\center
		\includegraphics[width=#1\textwidth]{#2}
	\end{figure}}

\newcommand{\codefile}[1]{\lstinputlisting{#1}}

\newcommand{\interval}{\vspace{0.5em}}

% 改变段间隔
\setlength{\parskip}{0.2em}
\linespread{1.1}

\usepackage{lastpage}
\usepackage{fancyhdr}
\pagestyle{fancy}
\lhead{\space \qquad \space}
\chead{在activity销毁与新建之间传递数据 \qquad}
\rhead{\qquad\thepage/\pageref{LastPage}}
\begin{document}

% \tableofcontents

% \clearpage

\section{在activity销毁与新建之间传递数据}
	利用onSaveInstanceState()方法可以传递数据,然后在onCreate()方法中恢复数据:
	\begin{lstlisting}
	import android.support.v7.app.AppCompatActivity;
	import android.os.Bundle;
	import android.view.View;
	import android.widget.Button;
	import android.widget.TextView;
	import android.widget.Toast;

	public class QuizActivity extends AppCompatActivity {
		private Button mTrueButton;
		private Button mFalseButton;
		private Button mNextButton;
		private TextView mQuestionTextView;

		// 键值字符串
		private static final String KEY_INDEX = "index";

		private Question[] mQuestionBank = new Question[]{
				new Question(R.string.question_oceans, true),
				new Question(R.string.question_mideast, false),
				new Question(R.string.question_africa, true),
				new Question(R.string.question_americas, true),
				new Question(R.string.question_asia, true)
		};

		private int mCurrentIndex = 0;

		private void updateQuestion()
		{
			int question_text = mQuestionBank[mCurrentIndex].getTextResId();
			mQuestionTextView.setText(question_text);
		}

		private void checkAnswer(boolean answer)
		{
			boolean rightAnswer = mQuestionBank[mCurrentIndex].isAnswerTrue();

			if(rightAnswer == answer)
				Toast.makeText(QuizActivity.this, R.string.correct_button, Toast.LENGTH_SHORT).show();
			else
				Toast.makeText(QuizActivity.this, R.string.incorrect_button, Toast.LENGTH_SHORT).show();
		}

		@Override
		protected void onCreate(Bundle savedInstanceState) {
			super.onCreate(savedInstanceState);
			setContentView(R.layout.activity_quiz);

			mTrueButton = (Button)findViewById(R.id.true_button);
			mFalseButton = (Button)findViewById(R.id.false_button);
			mNextButton = (Button)findViewById(R.id.next_button);

			mQuestionTextView = (TextView)findViewById(R.id.question_text_view);

			// 恢复数据
			if (savedInstanceState != null)
				mCurrentIndex = savedInstanceState.getInt(KEY_INDEX, 0);

			updateQuestion();

			mTrueButton.setOnClickListener(new View.OnClickListener(){
				@Override
				public void onClick(View v)
				{
					checkAnswer(true);
				}
			});

			mFalseButton.setOnClickListener(new View.OnClickListener(){
				@Override
				public void onClick(View v)
				{
					checkAnswer(false);
				}
			});

			mNextButton.setOnClickListener(new View.OnClickListener(){
				@Override
				public void onClick(View v)
				{
					mCurrentIndex = (mCurrentIndex + 1) % mQuestionBank.length;
					updateQuestion();
				}
			});
		}

		// onSaveInstanceState函数
		@Override
		public void onSaveInstanceState(Bundle outState)
		{
			super.onSaveInstanceState(outState);
			outState.putInt(KEY_INDEX, mCurrentIndex);
		}
	}		
	\end{lstlisting}

\end{document}
