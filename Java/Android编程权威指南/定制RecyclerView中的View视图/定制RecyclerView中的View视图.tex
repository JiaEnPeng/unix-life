% !TeX spellcheck = en_US
%% 字体:方正静蕾简体
%%		 方正粗宋
\documentclass[a4paper,left=2.5cm,right=2.5cm,11pt]{article}

\usepackage[utf8]{inputenc}
\usepackage{fontspec}
\usepackage{cite}
\usepackage{xeCJK}
\usepackage{indentfirst}
\usepackage{titlesec}
\usepackage{longtable}
\usepackage{graphicx}
\usepackage{float}
\usepackage{rotating}
\usepackage{subfigure}
\usepackage{tabu}
\usepackage{amsmath}
\usepackage{setspace}
\usepackage{amsfonts}
\usepackage{appendix}
\usepackage{listings}
\usepackage{xcolor}
\usepackage{geometry}
\setcounter{secnumdepth}{4}
\usepackage{mhchem}
\usepackage{multirow}
\usepackage{extarrows}
\usepackage{hyperref}
\titleformat*{\section}{\LARGE}
\renewcommand\refname{参考文献}
\renewcommand{\abstractname}{\sihao \cjkfzcs 摘{  }要}
%\titleformat{\chapter}{\centering\bfseries\huge\wryh}{}{0.7em}{}{}
%\titleformat{\section}{\LARGE\bf}{\thesection}{1em}{}{}
\titleformat{\subsection}{\Large\bfseries}{\thesubsection}{1em}{}{}
\titleformat{\subsubsection}{\large\bfseries}{\thesubsubsection}{1em}{}{}
\renewcommand{\contentsname}{{\cjkfzcs \centerline{目{  } 录}}}
\setCJKfamilyfont{cjkhwxk}{STXingkai}
\setCJKfamilyfont{cjkfzcs}{STSongti-SC-Regular}
% \setCJKfamilyfont{cjkhwxk}{华文行楷}
% \setCJKfamilyfont{cjkfzcs}{方正粗宋简体}
\newcommand*{\cjkfzcs}{\CJKfamily{cjkfzcs}}
\newcommand*{\cjkhwxk}{\CJKfamily{cjkhwxk}}
\newfontfamily\wryh{Microsoft YaHei}
\newfontfamily\hwzs{STZhongsong}
\newfontfamily\hwst{STSong}
\newfontfamily\hwfs{STFangsong}
\newfontfamily\jljt{MicrosoftYaHei}
\newfontfamily\hwxk{STXingkai}
% \newfontfamily\hwzs{华文中宋}
% \newfontfamily\hwst{华文宋体}
% \newfontfamily\hwfs{华文仿宋}
% \newfontfamily\jljt{方正静蕾简体}
% \newfontfamily\hwxk{华文行楷}
\newcommand{\verylarge}{\fontsize{60pt}{\baselineskip}\selectfont}  
\newcommand{\chuhao}{\fontsize{44.9pt}{\baselineskip}\selectfont}  
\newcommand{\xiaochu}{\fontsize{38.5pt}{\baselineskip}\selectfont}  
\newcommand{\yihao}{\fontsize{27.8pt}{\baselineskip}\selectfont}  
\newcommand{\xiaoyi}{\fontsize{25.7pt}{\baselineskip}\selectfont}  
\newcommand{\erhao}{\fontsize{23.5pt}{\baselineskip}\selectfont}  
\newcommand{\xiaoerhao}{\fontsize{19.3pt}{\baselineskip}\selectfont} 
\newcommand{\sihao}{\fontsize{14pt}{\baselineskip}\selectfont}      % 字号设置  
\newcommand{\xiaosihao}{\fontsize{12pt}{\baselineskip}\selectfont}  % 字号设置  
\newcommand{\wuhao}{\fontsize{10.5pt}{\baselineskip}\selectfont}    % 字号设置  
\newcommand{\xiaowuhao}{\fontsize{9pt}{\baselineskip}\selectfont}   % 字号设置  
\newcommand{\liuhao}{\fontsize{7.875pt}{\baselineskip}\selectfont}  % 字号设置  
\newcommand{\qihao}{\fontsize{5.25pt}{\baselineskip}\selectfont}    % 字号设置 

\usepackage{diagbox}
\usepackage{multirow}
\boldmath
\XeTeXlinebreaklocale "zh"
\XeTeXlinebreakskip = 0pt plus 1pt minus 0.1pt
\definecolor{cred}{rgb}{0.8,0.8,0.8}
\definecolor{cgreen}{rgb}{0,0.3,0}
\definecolor{cpurple}{rgb}{0.5,0,0.35}
\definecolor{cdocblue}{rgb}{0,0,0.3}
\definecolor{cdark}{rgb}{0.95,1.0,1.0}
\lstset{
	language=java,
	numbers=left,
	numberstyle=\tiny\color{black},
	showspaces=false,
	showstringspaces=false,
	basicstyle=\scriptsize,
	keywordstyle=\color{purple},
	commentstyle=\itshape\color{cgreen},
	stringstyle=\color{blue},
	frame=lines,
	% escapeinside=``,
	extendedchars=true, 
	xleftmargin=1em,
	xrightmargin=1em, 
	backgroundcolor=\color{cred},
	aboveskip=1em,
	breaklines=true,
	tabsize=4
} 

\newfontfamily{\consolas}{Consolas}
\newfontfamily{\monaco}{Monaco}
\setmonofont[Mapping={}]{Consolas}	%英文引号之类的正常显示,相当于设置英文字体
\setsansfont{Consolas} %设置英文字体 Monaco, Consolas,  Fantasque Sans Mono
\setmainfont{Times New Roman}

\setCJKmainfont{华文中宋}


\newcommand{\fic}[1]{\begin{figure}[H]
		\center
		\includegraphics[width=0.8\textwidth]{#1}
	\end{figure}}
	
\newcommand{\sizedfic}[2]{\begin{figure}[H]
		\center
		\includegraphics[width=#1\textwidth]{#2}
	\end{figure}}

\newcommand{\codefile}[1]{\lstinputlisting{#1}}

\newcommand{\interval}{\vspace{0.5em}}

% 改变段间隔
\setlength{\parskip}{0.2em}
\linespread{1.1}

\usepackage{lastpage}
\usepackage{fancyhdr}
\pagestyle{fancy}
\lhead{\space \qquad \space}
\chead{定制RecyclerView中的View视图 \qquad}
\rhead{\qquad\thepage/\pageref{LastPage}}
\begin{document}

% \tableofcontents

% \clearpage

\section{定制RecyclerView中的View视图}
	步骤如下:
	\begin{itemize}
		\item 创建一个View的视图页面。
		\item 创建一个容纳View对象的ViewHolder。
		\item 创建一个负责创建和填充ViewHolder的Adapter对象。
		\item 将RecyclerView对象与Adapter类相关联。
	\end{itemize}

	创建一个View的视图页面:
	\begin{lstlisting}[language = xml]
	<?xml version="1.0" encoding="utf-8"?>
	<RelativeLayout
		xmlns:android="http://schemas.android.com/apk/res/android"
		xmlns:tools="http://schemas.android.com/tools"
		android:layout_width="match_parent"
		android:layout_height="wrap_content"
		>

		<CheckBox
			android:id="@+id/list_item_crime_solved_check_box"
			android:layout_width="wrap_content"
			android:layout_height="wrap_content"
			android:layout_alignParentRight="true"
			android:padding="4dp"
			/>

		<TextView
			android:id="@+id/list_item_crime_title_text_view"
			android:layout_width="match_parent"
			android:layout_height="wrap_content"
			android:layout_toLeftOf="@id/list_item_crime_solved_check_box"
			android:textStyle="bold"
			android:padding="4dp"
			tools:text="Crime Title"
			/>

		<TextView
			android:id="@+id/list_item_crime_date_text_view"
			android:layout_width="match_parent"
			android:layout_height="wrap_content"
			android:layout_toLeftOf="@id/list_item_crime_solved_check_box"
			android:layout_below="@id/list_item_crime_title_text_view"
			android:padding="4dp"
			tools:text="Crime date"
			/>

	</RelativeLayout>
	\end{lstlisting}

	创建一个容纳View对象的ViewHolder:
	\begin{lstlisting}
	private class CrimeHolder extends RecyclerView.ViewHolder
	{
		public TextView mTitleTextView;
		public TextView mDateTextView;
		public CheckBox mSolvedCheckBox;

		public CrimeHolder(View itemView)
		{
			super(itemView);

			mTitleTextView = (TextView)itemView.findViewById(R.id.list_item_crime_title_text_view);
			mDateTextView = (TextView)itemView.findViewById(R.id.list_item_crime_date_text_view);
			mSolvedCheckBox = (CheckBox)itemView.findViewById(R.id.list_item_crime_solved_check_box);
		}
	}
	\end{lstlisting}

	创建一个负责创建和填充ViewHolder的Adapter对象:
	\begin{lstlisting}
	private class CrimeAdapter extends RecyclerView.Adapter<CrimeHolder>
    {
        private List<Crime> mCrimes;

        public CrimeAdapter(List<Crime> crimes)
        {
            mCrimes = crimes;
        }

        @Override
        public CrimeHolder onCreateViewHolder(ViewGroup parent, int viewType)
        {
            LayoutInflater layoutInflater = LayoutInflater.from(getActivity());
            View view = layoutInflater.inflate(R.layout.list_item_crime, parent, false);
            return new CrimeHolder(view);
        }

        @Override
        public void onBindViewHolder(CrimeHolder holder, int position)
        {
            Crime crime = mCrimes.get(position);
            holder.mTitleTextView.setText(crime.getTitle());
            holder.mDateTextView.setText(crime.getDate().toString());
            holder.mSolvedCheckBox.setChecked(crime.isSolved());
        }

        @Override
        public int getItemCount()
        {
            return mCrimes.size();
        }
    }
	\end{lstlisting}

\end{document}
