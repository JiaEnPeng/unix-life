% !TeX spellcheck = en_US
%% 字体:方正静蕾简体
%%		 方正粗宋
\documentclass[a4paper,left=2.5cm,right=2.5cm,11pt]{article}

\usepackage[utf8]{inputenc}
\usepackage{fontspec}
\usepackage{cite}
\usepackage{xeCJK}
\usepackage{indentfirst}
\usepackage{titlesec}
\usepackage{longtable}
\usepackage{graphicx}
\usepackage{float}
\usepackage{rotating}
\usepackage{subfigure}
\usepackage{tabu}
\usepackage{amsmath}
\usepackage{setspace}
\usepackage{amsfonts}
\usepackage{appendix}
\usepackage{listings}
\usepackage{xcolor}
\usepackage{geometry}
\setcounter{secnumdepth}{4}
\usepackage{mhchem}
\usepackage{multirow}
\usepackage{extarrows}
\usepackage{hyperref}
\titleformat*{\section}{\LARGE}
\renewcommand\refname{参考文献}
\renewcommand{\abstractname}{\sihao \cjkfzcs 摘{  }要}
%\titleformat{\chapter}{\centering\bfseries\huge\wryh}{}{0.7em}{}{}
%\titleformat{\section}{\LARGE\bf}{\thesection}{1em}{}{}
\titleformat{\subsection}{\Large\bfseries}{\thesubsection}{1em}{}{}
\titleformat{\subsubsection}{\large\bfseries}{\thesubsubsection}{1em}{}{}
\renewcommand{\contentsname}{{\cjkfzcs \centerline{目{  } 录}}}
\setCJKfamilyfont{cjkhwxk}{STXingkai}
\setCJKfamilyfont{cjkfzcs}{STSongti-SC-Regular}
% \setCJKfamilyfont{cjkhwxk}{华文行楷}
% \setCJKfamilyfont{cjkfzcs}{方正粗宋简体}
\newcommand*{\cjkfzcs}{\CJKfamily{cjkfzcs}}
\newcommand*{\cjkhwxk}{\CJKfamily{cjkhwxk}}
\newfontfamily\wryh{Microsoft YaHei}
\newfontfamily\hwzs{STZhongsong}
\newfontfamily\hwst{STSong}
\newfontfamily\hwfs{STFangsong}
\newfontfamily\jljt{MicrosoftYaHei}
\newfontfamily\hwxk{STXingkai}
% \newfontfamily\hwzs{华文中宋}
% \newfontfamily\hwst{华文宋体}
% \newfontfamily\hwfs{华文仿宋}
% \newfontfamily\jljt{方正静蕾简体}
% \newfontfamily\hwxk{华文行楷}
\newcommand{\verylarge}{\fontsize{60pt}{\baselineskip}\selectfont}  
\newcommand{\chuhao}{\fontsize{44.9pt}{\baselineskip}\selectfont}  
\newcommand{\xiaochu}{\fontsize{38.5pt}{\baselineskip}\selectfont}  
\newcommand{\yihao}{\fontsize{27.8pt}{\baselineskip}\selectfont}  
\newcommand{\xiaoyi}{\fontsize{25.7pt}{\baselineskip}\selectfont}  
\newcommand{\erhao}{\fontsize{23.5pt}{\baselineskip}\selectfont}  
\newcommand{\xiaoerhao}{\fontsize{19.3pt}{\baselineskip}\selectfont} 
\newcommand{\sihao}{\fontsize{14pt}{\baselineskip}\selectfont}      % 字号设置  
\newcommand{\xiaosihao}{\fontsize{12pt}{\baselineskip}\selectfont}  % 字号设置  
\newcommand{\wuhao}{\fontsize{10.5pt}{\baselineskip}\selectfont}    % 字号设置  
\newcommand{\xiaowuhao}{\fontsize{9pt}{\baselineskip}\selectfont}   % 字号设置  
\newcommand{\liuhao}{\fontsize{7.875pt}{\baselineskip}\selectfont}  % 字号设置  
\newcommand{\qihao}{\fontsize{5.25pt}{\baselineskip}\selectfont}    % 字号设置 

\usepackage{diagbox}
\usepackage{multirow}
\boldmath
\XeTeXlinebreaklocale "zh"
\XeTeXlinebreakskip = 0pt plus 1pt minus 0.1pt
\definecolor{cred}{rgb}{0.8,0.8,0.8}
\definecolor{cgreen}{rgb}{0,0.3,0}
\definecolor{cpurple}{rgb}{0.5,0,0.35}
\definecolor{cdocblue}{rgb}{0,0,0.3}
\definecolor{cdark}{rgb}{0.95,1.0,1.0}
\lstset{
	language=python,
	numbers=left,
	numberstyle=\tiny\color{black},
	showspaces=false,
	showstringspaces=false,
	basicstyle=\scriptsize,
	keywordstyle=\color{purple},
	commentstyle=\itshape\color{cgreen},
	stringstyle=\color{blue},
	frame=lines,
	% escapeinside=``,
	extendedchars=true, 
	xleftmargin=1em,
	xrightmargin=1em, 
	backgroundcolor=\color{cred},
	aboveskip=1em,
	breaklines=true,
	tabsize=4
} 

\newfontfamily{\consolas}{Consolas}
\newfontfamily{\monaco}{Monaco}
\setmonofont[Mapping={}]{Consolas}	%英文引号之类的正常显示,相当于设置英文字体
\setsansfont{Consolas} %设置英文字体 Monaco, Consolas,  Fantasque Sans Mono
\setmainfont{Times New Roman}

\setCJKmainfont{华文中宋}


\newcommand{\fic}[1]{\begin{figure}[H]
		\center
		\includegraphics[width=0.8\textwidth]{#1}
	\end{figure}}
	
\newcommand{\sizedfic}[2]{\begin{figure}[H]
		\center
		\includegraphics[width=#1\textwidth]{#2}
	\end{figure}}

\newcommand{\codefile}[1]{\lstinputlisting{#1}}

% 改变段间隔
\setlength{\parskip}{0.2em}
\linespread{1.1}

\usepackage{lastpage}
\usepackage{fancyhdr}
\pagestyle{fancy}
\lhead{\space \qquad \space}
\chead{KVM扩容内存、CPU、磁盘 \qquad}
\rhead{\qquad\thepage/\pageref{LastPage}}

\begin{document}

\tableofcontents

\clearpage

\section{KVM扩容内存、CPU、磁盘}
\subsection{查看虚拟机状态}
	使用“dominfo <domain-name>”可以查看虚拟机的状态,其中就可以看到虚拟机当前的内存,如下图所示:
	\sizedfic{0.5}{1.png}

\subsection{KVM扩容内存}
\subsubsection{设置当前使用内存“currentMemory”}
	KVM只能在虚拟机的运行状态下扩展虚拟机的当前可使用内存,命令如下:
	\begin{lstlisting}
	virsh setmem <domain-name> <count>
	\end{lstlisting}

	使用例子如下图所示:
	\sizedfic{0.6}{2.png}

	如果想永久性地更改虚拟机的“currentMemory”,命令如下:
	\begin{lstlisting}
	virsh setmem <domain-name> <count> --config
	\end{lstlisting}

	需要注意的是,虚拟机还有另一个属性,就是最大可使用的内存“memory”,而当前可使用内存是“currentMemory”。
	这里setmem命令是设置“currentMemory”,而“currentMemory”的值不能大于“memory”,否则会出错,如下图所示:
	\sizedfic{0.5}{3.png}

\subsubsection{设置最大可使用内存“memory”}
	KVM只能在虚拟机的关机状态下调整虚拟机的最大可使用内存,命令如下:
	\begin{lstlisting}
	virsh setmaxmem <domain-name> <count>
	\end{lstlisting}

	使用例子如下图所示:
	\sizedfic{0.6}{4.png}

	需要注意的是,必须在虚拟机的关机状态下调整虚拟机的最大可使用内存,否则会报错,如下图所示:
	\fic{5.png}

	还有一点需要知道,如果调整“memory”小于“currentMemory”,那么“currentMemory”也会随之改变,如下图所示:
	\sizedfic{0.5}{6.png}

\subsection{KVM调整CPU数量}
	KVM可以设置虚拟机最多可以使用的CPU数量,命令如下:
	\begin{lstlisting}
	virsh setvcpus <domain-name> --maximum <count> --config
	\end{lstlisting}

	设置这个值以后,需要重启虚拟机才能生效,命令如下:
	\begin{lstlisting}
	destroy <domain-name>
	start <domain-name>
	\end{lstlisting}

	KVM只能在虚拟机的运行状态下调整虚拟机的CPU数量,命令如下:
	\begin{lstlisting}
	virsh setvcpus <domain-name> <count>
	\end{lstlisting}

	使用例子如下图所示:
	\sizedfic{0.5}{7.png}

	如果想永久性地更改虚拟机的CPU数量,命令如下:
	\begin{lstlisting}
	virsh setvcpus <domain-name> <count> --config
	\end{lstlisting}

	需要注意的是,CPU个数只能调大,不能调小,如下图所示:
	\sizedfic{0.5}{8.png}

	如果想减小CPU的个数,可以使用“setvcpus <domain-name> --maximum <count> --config”命令,然后重启虚拟机,如下图所示:
	\sizedfic{0.6}{9.png}

\clearpage

\subsubsection{调整CPU数量需要知道的事}
	虽然KVM可以动态地增加虚拟机CPU的数量,但是虚拟机增加的CPU其实是处于“offline”状态。\par
	举个例子,虚拟机CPU本身状态如下图所示:
	\fic{10.png}

	随后使用“setvcpus”命令设置为4个cpu,结果虽然是分配了4个cpu,但是虚拟机中增加的3个cpu处于“offline”状态,如下图所示:
	\fic{11.png}

	只有虚拟机重启之后,这三个cpu才会投入使用,如下图所示:
	\sizedfic{0.5}{12.png}

\subsubsection{使用qemu-agent动态调整CPU数量}
	首先在宿主机安装qemu-agent:
	\begin{lstlisting}
	sudo apt install qemu-guest-agent
	\end{lstlisting}

	然后在虚拟机的xml配置文件的<devices>中增加如下内容:
	\begin{lstlisting}
	<channel type='unix'>
		<source mode='bind' path='/var/lib/libvirt/qemu/f16x86_64.agent'/>
   		<target type='virtio' name='org.qemu.guest_agent.0'/>
	</channel>
	\end{lstlisting}

	如下图所示:
	\fic{15.png}

	然后在虚拟机中也安装qemu-agent:
	\begin{lstlisting}
	sudo apt install qemu-guest-agent
	\end{lstlisting}

	这样设置以后,就可以动态地增加和减少CPU数量了。\par

	动态地减少CPU数量的命令如下:
	\begin{lstlisting}
	virsh setvcpus --live --guest <domain-name> <count>
	\end{lstlisting}

	使用例子如下图所示:
	\fic{13.png}

	如果要保存这个设置,可以再输入如下语句:
	\begin{lstlisting}
	virsh setvcpus <domain-name> <count> --config
	# 如,setcpus devstack 2 --config
	\end{lstlisting}

	需要知道的是,这里实现的并不是把CPU数量减少了,而是将虚拟机中2个CPU设置为“offline”。
	如果在virsh中查看虚拟机的CPU数量,会发现和原来一样,如下图所示:
	\sizedfic{0.6}{14.png}

	不过这一点我觉得不必计较,因为将CPU设置为“offline”以后,就相当于没有使用这两个CPU了。\par

	使用qemu-agent动态增加CPU数量的命令如下:
	\begin{lstlisting}
	virsh setvcpu <domain-name> <count>
	virsh setvcpu --live --guest <domain-name> <count>
	\end{lstlisting}

	使用例子如下图所示:
	\fic{16.png}

	这样子来增加CPU数量,相当于先增加CPU数量,然后使用qemu-agent将这些CPU设置为“online”状态。\par

	如果想保存这个设置,可以再输入如下语句:
	\begin{lstlisting}
	virsh setvcpus <domain-name> <count> --config
	\end{lstlisting}

	其实上述语句也就是加一个“--config”参数选项,用于修改下一次的启动配置。

\subsection{KVM扩展磁盘}
	KVM扩展磁盘的方式有两种:
	\begin{itemize}
		\item[1.] 添加qcow2磁盘加入虚拟机。
		\item[2.] 直接扩展qcow2磁盘。
	\end{itemize}

\subsubsection{添加qcow2磁盘加入虚拟机}
	使用这个方法需要注意的是,IDE磁盘不支持热拔插,其他磁盘可以。
	如果想添加IDE磁盘,只能静态添加。\par

	步骤如下:
	\begin{itemize}
		\item[1.] 使用qemu-img工具创建一块qcow2磁盘:
		\begin{lstlisting}
	qemu-img create -f qcow2 file.qcow2 <size>
	# 如,qemu-img create -f qcow2 attach.qcow2 50G
		\end{lstlisting}

		\item[2.] 向虚拟机中动态添加一块qcow2磁盘:
		\begin{lstlisting}
	virsh attach-disk <domain-name> <source> <target> --config
	# 使用--config选项是为了永久添加这块磁盘
	# 如,virsh attach-disk devstack /home/sidapeng/kvm/openstack/attach.qcow2 hdb --subdriver qcow2 --config
		\end{lstlisting}

		\item[3.] 到上个步骤,虚拟机的磁盘其实已经扩展成功了,我们可以使用如下语句查看虚拟机拥有的磁盘:
		\begin{lstlisting}
	virsh domblklist <domain>
	# 如,virsh domblklist devstack
		\end{lstlisting}

		到虚拟机中查看,如下图所示:
		\sizedfic{0.7}{17.png}

		可以看出,虚拟机的磁盘空间确实扩大了,但是还需要进行分区。

		\item[4.] 对虚拟机进行分区扩容:
		\begin{lstlisting}
	sudo fdisk /dev/sdb # 从上图可知,新增的是/dev/sdb,它还没分区
		\end{lstlisting}

		具体操作如下图所示:
		\fic{18.png}

		再使用“fdisk -l”语句,就可以看到多了/dev/sdb1分区,如下图所示:
		\sizedfic{0.7}{19.png}

		然后再对/dev/sdb1进行格式化,当前文件系统为ext4,所以使用mkfs.ext4进行格式化:
		\begin{lstlisting}
	sudo mkfs.ext4 /dev/sdb1
		\end{lstlisting}

		最后将这个分区挂载到/mnt下:
		\begin{lstlisting}
	sudo mount -t ext4 /dev/sdb1 /mnt
		\end{lstlisting}

		到此,虚拟机可以说是正式扩容了,如下图所示:
		\sizedfic{0.5}{20.png}

		 mount挂载分区在系统重启之后需要重新挂载,可以通过修改/etc/fstab文件使得挂载永久生效,如下图所示:
		 \fic{21.png}
	\end{itemize}

\subsubsection{直接扩展qcow2磁盘}
	使用这个方法需要注意的是,如果虚拟机的qcow2磁盘拥有快照,那么是无法使用qemu-img工具扩展这个磁盘的。
	还有就是,这个方法无法动态扩容。\par
	步骤如下:
	\begin{itemize}
		\item[1.] 直接扩展虚拟机的qcow2磁盘:
		\begin{lstlisting}
	qemu-img resize file.qcow2 +<size>
	# 比如,qemu-img resize devstack.qcow2 +50G
		\end{lstlisting}

		\item[2.] 重启虚拟机,使用“sudo fdisk -l”命令,就可以看到原先的磁盘多了50G。

		\item[3.] 对磁盘进行分区,和上一小节的第四步骤一样,在此就不再详述。
	\end{itemize}

\end{document}