% !TeX spellcheck = en_US
%% 字体:方正静蕾简体
%%		 方正粗宋
\documentclass[a4paper,left=2.5cm,right=2.5cm,11pt]{article}

\usepackage[utf8]{inputenc}
\usepackage{fontspec}
\usepackage{cite}
\usepackage{xeCJK}
\usepackage{indentfirst}
\usepackage{titlesec}
\usepackage{longtable}
\usepackage{graphicx}
\usepackage{float}
\usepackage{rotating}
\usepackage{subfigure}
\usepackage{tabu}
\usepackage{amsmath}
\usepackage{setspace}
\usepackage{amsfonts}
\usepackage{appendix}
\usepackage{listings}
\usepackage{xcolor}
\usepackage{geometry}
\setcounter{secnumdepth}{4}
\usepackage{mhchem}
\usepackage{multirow}
\usepackage{extarrows}
\usepackage{hyperref}
\titleformat*{\section}{\LARGE}
\renewcommand\refname{参考文献}
\renewcommand{\abstractname}{\sihao \cjkfzcs 摘{  }要}
%\titleformat{\chapter}{\centering\bfseries\huge\wryh}{}{0.7em}{}{}
%\titleformat{\section}{\LARGE\bf}{\thesection}{1em}{}{}
\titleformat{\subsection}{\Large\bfseries}{\thesubsection}{1em}{}{}
\titleformat{\subsubsection}{\large\bfseries}{\thesubsubsection}{1em}{}{}
\renewcommand{\contentsname}{{\cjkfzcs \centerline{目{  } 录}}}
\setCJKfamilyfont{cjkhwxk}{STXingkai}
\setCJKfamilyfont{cjkfzcs}{STSongti-SC-Regular}
% \setCJKfamilyfont{cjkhwxk}{华文行楷}
% \setCJKfamilyfont{cjkfzcs}{方正粗宋简体}
\newcommand*{\cjkfzcs}{\CJKfamily{cjkfzcs}}
\newcommand*{\cjkhwxk}{\CJKfamily{cjkhwxk}}
\newfontfamily\wryh{Microsoft YaHei}
\newfontfamily\hwzs{STZhongsong}
\newfontfamily\hwst{STSong}
\newfontfamily\hwfs{STFangsong}
\newfontfamily\jljt{MicrosoftYaHei}
\newfontfamily\hwxk{STXingkai}
% \newfontfamily\hwzs{华文中宋}
% \newfontfamily\hwst{华文宋体}
% \newfontfamily\hwfs{华文仿宋}
% \newfontfamily\jljt{方正静蕾简体}
% \newfontfamily\hwxk{华文行楷}
\newcommand{\verylarge}{\fontsize{60pt}{\baselineskip}\selectfont}  
\newcommand{\chuhao}{\fontsize{44.9pt}{\baselineskip}\selectfont}  
\newcommand{\xiaochu}{\fontsize{38.5pt}{\baselineskip}\selectfont}  
\newcommand{\yihao}{\fontsize{27.8pt}{\baselineskip}\selectfont}  
\newcommand{\xiaoyi}{\fontsize{25.7pt}{\baselineskip}\selectfont}  
\newcommand{\erhao}{\fontsize{23.5pt}{\baselineskip}\selectfont}  
\newcommand{\xiaoerhao}{\fontsize{19.3pt}{\baselineskip}\selectfont} 
\newcommand{\sihao}{\fontsize{14pt}{\baselineskip}\selectfont}      % 字号设置  
\newcommand{\xiaosihao}{\fontsize{12pt}{\baselineskip}\selectfont}  % 字号设置  
\newcommand{\wuhao}{\fontsize{10.5pt}{\baselineskip}\selectfont}    % 字号设置  
\newcommand{\xiaowuhao}{\fontsize{9pt}{\baselineskip}\selectfont}   % 字号设置  
\newcommand{\liuhao}{\fontsize{7.875pt}{\baselineskip}\selectfont}  % 字号设置  
\newcommand{\qihao}{\fontsize{5.25pt}{\baselineskip}\selectfont}    % 字号设置 

\usepackage{diagbox}
\usepackage{multirow}
\boldmath
\XeTeXlinebreaklocale "zh"
\XeTeXlinebreakskip = 0pt plus 1pt minus 0.1pt
\definecolor{cred}{rgb}{0.8,0.8,0.8}
\definecolor{cgreen}{rgb}{0,0.3,0}
\definecolor{cpurple}{rgb}{0.5,0,0.35}
\definecolor{cdocblue}{rgb}{0,0,0.3}
\definecolor{cdark}{rgb}{0.95,1.0,1.0}
\lstset{
	language=C,
	numbers=left,
	numberstyle=\tiny\color{white},
	showspaces=false,
	showstringspaces=false,
	basicstyle=\scriptsize,
	keywordstyle=\color{purple},
	commentstyle=\itshape\color{cgreen},
	stringstyle=\color{blue},
	frame=lines,
	% escapeinside=``,
	extendedchars=true, 
	xleftmargin=0em,
	xrightmargin=0em, 
	backgroundcolor=\color{cred},
	aboveskip=1em,
	breaklines=true,
	tabsize=4
} 

\newfontfamily{\consolas}{Consolas}
\newfontfamily{\monaco}{Monaco}
\setmonofont[Mapping={}]{Consolas}	%英文引号之类的正常显示,相当于设置英文字体
\setsansfont{Consolas} %设置英文字体 Monaco, Consolas,  Fantasque Sans Mono
\setmainfont{Times New Roman}

\setCJKmainfont{华文中宋}


\newcommand{\fic}[1]{\begin{figure}[H]
		\center
		\includegraphics[width=0.8\textwidth]{#1}
	\end{figure}}
	
\newcommand{\sizedfic}[2]{\begin{figure}[H]
		\center
		\includegraphics[width=#1\textwidth]{#2}
	\end{figure}}

\newcommand{\codefile}[1]{\lstinputlisting{#1}}

\newcommand{\interval}{\vspace{0.5em}}

\newcommand{\tablestart}{
	\interval
	\begin{longtable}{p{2cm}p{10cm}}
	\hline}
\newcommand{\tableend}{
	\hline
	\end{longtable}
	\interval}

% 改变段间隔
\setlength{\parskip}{0.2em}
\linespread{1.1}

\usepackage{lastpage}
\usepackage{fancyhdr}
\pagestyle{fancy}
\lhead{\space \qquad \space}
\chead{kvm管理工具 \qquad}
\rhead{\qquad\thepage/\pageref{LastPage}}
\begin{document}

\tableofcontents

\clearpage

\section{libvirt}
\subsection{libvirt的介绍}
	libvirt是用于管理平台虚拟化技术的应用程序接口、守护进程和管理工具,它不仅提供了对虚拟化客户机的管理,也提供了对虚拟化网络和存储的管理。\par
	在libvirt中有几个重要的概念,如下所示:
	\begin{itemize}
		\item Node又叫做节点,是一个物理机器,上面可能运行着多个虚拟客户机。Hypervisor和Domain都运行在节点之上。
		\item Hypervisor又叫做虚拟机监控器,比如KVM、Xen、VMware、Hyper-V等,是虚拟化中的一个底层软件层,它可以虚拟化一个节点让其运行多个虚拟客户机。
		\item Domain又叫做域,是在Hypervisor上运行的一个客户机操作系统实例。
	\end{itemize}

	libvirt被用于管理节点上的各个域,其中的管理功能包括以下四个部分:
	\begin{itemize}
		\item[1.] 域的管理。包括对域的生命周期的管理以及管理对多种设备类型的热拔插操作。
		\item[2.] 远程节点的管理。libvirt支持多种网络远程传输类型。只要物理节点上运行了libvirtd这个守护进程,远程的管理程序就可以连接到该节点进行管理操作。
		\item[3.] 存储的管理。任何运行了libvirtd守护进程的主机,都可以通过libvirt来管理不同类型的存储。
		\item[4.] 网络的管理。任何运行了libvirtd守护进程的主机,都可以通过libvirt来管理物理的和逻辑的网络接口。
	\end{itemize}

	libvirt由三部分组成:
	\begin{itemize}
		\item 应用程序编程接口库,为其他虚拟机管理工具提供虚拟机管理的程序库支持。
		\item libvirtd守护进程,负责执行对节点上的域的管理工作。
		\item virsh,是libvirt项目中默认的对虚拟机管理的一个命令行工具。
	\end{itemize}

	\clearpage

\subsection{libvirt的安装}
	安装libvirt的步骤如下:
	\begin{itemize}
		\item[1.] 首先检查是否安装过libvirt,命令如下所示:
		\begin{lstlisting}
	which libvirtd
		\end{lstlisting}

		如果有安装过libvirt,就应该先清除之前装过的libvirt,命令如下所示:
		\begin{lstlisting}
	// 仅在ubuntu16.04下试验过
	sudo apt remove libvirt-bin
		\end{lstlisting}

		\item[2.] 下载libvirt的源代码,命令如下所示:
		\begin{lstlisting}
	// 下载日期为2016.12.25,此时最新版本为2.5.0
	wget http://libvirt.org/sources/libvirt-2.5.0.tar.xz
		\end{lstlisting}

		\item[3.] 配置libvirt前,需要安装一些工具。相关命令如下所示:
		\begin{lstlisting}
	sudo apt-get install libpciaccess-dev
	sudo apt-get install libxml++2.6-2v5
	sudo apt-get install libxml++2.6-dev
	sudo apt-get install libyajl-dev
	sudo apt-get install libdevmapper-dev
	sudo apt-get install libnl-3-dev
	sudo apt-get install libnl-route-3-dev
		\end{lstlisting}

		\item[4.] 配置libvirt,命令如下所示:
		\begin{lstlisting}
	./configure
		\end{lstlisting}

		\item[5.] 编译libvirt,命令如下所示:
		\begin{lstlisting}
	make -j 4
		\end{lstlisting}

		\item[6.] 安装libvirt,命令如下所示:
		\begin{lstlisting}
	sudo make install
		\end{lstlisting}

		\item[7.] 配置动态链接,命令如下所示:
		\begin{lstlisting}
	sudo vi /etc/ld.so.conf.d/libc.conf
		\end{lstlisting}

		将文件内容写为下图中的内容:
		\sizedfic{0.5}{1.png}

		然后再输入如下命令:
		\begin{lstlisting}
	sudo ldconfig
		\end{lstlisting}

		\item[8.] 检查是否安装成功,命令如下所示:
		\begin{lstlisting}
	which libvirtd
	libvirted --version
	virsh
		\end{lstlisting}

		如果安装成功,将得到如下图的结果:
		\fic{2.png}
	\end{itemize}

\subsection{libvirt和libvirtd的配置}
\subsubsection{libvirt的配置文件}
	libvirt的相关配置文件都在/etc/libvirt/目录下,如下图所示:
	\fic{3.png}

	下面介绍其中几个重要的配置文件和目录:
	\begin{itemize}
		\item[1.] /etc/libvirt/libvirt.conf。这个文件用于配置一些常用的libvirt连接的别名,文件内容可以如下所示:
		\begin{lstlisting}
	uri_aliases = [
		"remote = qemu+ssh://root@192.168.93.201/system",
	]
		\end{lstlisting}

		文件中,将“remote”这个别名用于指代“qemu+ssh://root@192.168.93.201/system”这个libvirt连接。

		\item[2.] /etc/libvirt/libvirtd.conf。这个文件是libvirt的守护进程libvirtd的配置文件。
		文件中使用“配置项 = 值”这样的配对格式来配置libvirtd。\par
		例如,下面的几个配置项表示关闭TLS安全认证的连接、打开TCP连接、设置TCP监听的端口、TCP连接不使用认证授权方式以及设置UNIX domain socket的保存目录。如下所示:
		\begin{lstlisting}
	listen_tls = 0
	listen_tcp = 1
	tcp_port = "16666"
	auth_tcp = "none"
	unix_socket_dir = "/var/run/libvirt"
		\end{lstlisting}

		需要注意的是,这个文件被修改后,需要让libvirtd重新加载配置文件才会生效。
		如果想要让TCP、TLS等连接生效,需要在启动libvirtd时加上“--listen”参数,命令如下所示:
		\begin{lstlisting}
	libvirtd --listen
		\end{lstlisting}

		\item[3.] /etc/libvirt/qemu.conf。这个文件是QEMU驱动的配置文件。
		\item[4.] /etc/libvirt/qemu/目录。在这个目录下存放着使用QEMU驱动的域的配置文件。
	\end{itemize}

\subsubsection{libvirtd的配置}
	下面介绍以下几个 libvirtd命令行的参数:
	\interval
	\begin{longtable}{p{2.5cm}p{10cm}}
	\hline
	-d & 表示让libvirtd作为守护进程在后台运行。 \\
	\hline
	-f FILE & 指定libvirtd的配置文件为FILE。默认值为/etc/libvirt/libvirtd.conf\\
	\hline
	-l & 开启配置文件中配置的TCP/IP连接。 \\
	\hline
	-p FILE & 将libvirtd进程的PID写入到FILE文件中。默认值为/var/run/libvirt.pid \\ 
	\hline
	-t SECONDS & 设置对libvirtd连接的超时时间为SECONDS秒。 \\
	\hline
	-v & libvirtd运行时输出详细的输出信息。 \\
	\hline
	--version & 显示libvirtd程序的版本信息。 \\
	\hline
	\end{longtable}

\subsection{libvirt域的XML配置文件}
\subsubsection{CPU的配置}
	XML配置文件中使用vcpu标签表示客户机中vCPU的个数,使用features标签表示为客户机打开或关闭CPU或其他硬件的特性。
	具体例子如下所示:
	\begin{lstlisting}[language = XML]
	<domain>
		<!-- 设置客户机中vCPU的个数为2 -->
		<vcpu placement='static'>2</vcpu>
		<!-- 打开客户机的ACPI、APIC和PAE特性 -->
		<features>
			<acpi/>
			<apic/>
			<pae/>
		</features>
	</domain>
	\end{lstlisting}

	libvirt还提供了cputune标签来对CPU分配进行更多调节,具体例子如下所示:
	\begin{lstlisting}[language = XML]
	<domain>
		<cputune>
			<vcpupin vcpu="0" cpuset="1"/>
			<vcpupin vcpu="1" cpuset="2,3"/>
			<vcpupin vcpu="2" cpuset="4"/>
			<vcpupin vcpu="3" cpuset="5"/>
			<emulatorpin cpuset="1-3">
			<iothreadpin iothread="1" cpuset="5,6"/>
			<iothreadpin iothread="2" cpuset="7,8"/>
			<shares>2048</shares>
			<period>1000000</period>
			<quota>-1<quota>
			<emulator_period>1000000</emulator_period>
			<emulator_quota>-1</emulator_quota>
			<iothread_period>1000000</iothread_period>
			<iothread_quota>-1<iothread_quota>
			<vcpushed vcpus='0-4,^3' scheduler='fifo' priority='1'/>
			<iothreadsched iothreads='2' scheduler='batch'/>
		</cputune>
	</domain>
	\end{lstlisting}

	在此介绍以上例子中的标签:
	\interval
	\begin{longtable}{p{2.5cm}p{11cm}}
	\hline
	cputune & 在这个标签中可以设置cpu的参数 \\
	\hline
	vcpupin & 该标签表示了vCPU应该放置到宿主机哪个CPU上 \\
	\hline
	emulatorpin & 该标签表示了qemu emulator应该绑定到哪个宿主机CPU上 \\
	\hline
	iothreadpin & 该标签表示了IOThreads应该绑定到哪个宿主机CPU上 \\
	\hline
	shares & 该标签表示了客户机占用CPU时间的加权配额 \\
	\hline
	period & 该标签表示了vCPU的执行周期,在这个时间段内,vCPU不允许执行超过quota的时间。这个值的范围是[1000, 1000000] \\
	\hline
	quota & 该标签表示了vCPU的最大允许执行时间。如果quota为负,说明vCPU可以执行无限长的时间。\\
	\hline
	emulator\_period & 该标签表示了emulator的执行周期。在这个时间段内,emulator不能执行超过emulator\_quota的时间。 \\
	\hline
	emulator\_quota & 该标签表示了emulator的最大允许执行时间。如果emulator\_quota为负,说明emulator可以执行无限长的时间。\\
	\hline
	iothread\_period & 该标签表示了IOThreads的执行周期。在这个时间段内,IOThreads不能执行超过iothread\_quota的时间。 \\
	\hline
	iothread\_quota & 该标签表示了IOThreads的最大允许执行时间。如果iothread\_quota为负,说明IOThreads可以执行无限长的时间。\\
	\hline
	vcpushed & 该标签表示了vcpu调度算法的类型 \\
	\hline
	iothreadsched & 该标签表示了IOThread调度算法的类型 \\
	\hline
	\end{longtable}

\subsubsection{内存的配置}
	内存配置的例子如下所示:
	\begin{lstlisting}
	<domain>
		<maxMemory slots='16'></maxMemory>
		<memory unit='KiB'>524288</memory>
		<currentMemory unit='KiB'>524288</currentMemory>
	</domain>
	\end{lstlisting}

	在此介绍以上例子中的标签:
	\interval
	\begin{longtable}{p{2.5cm}p{10cm}}
	\hline
	memory & 该标签表示客户机启动时最大可使用的内存 \\
	\hline
	maxMemory & 该标签表示客户机运行时最大可使用的内存 \\
	\hline
	currentMemory & 该标签表示客户机实际被分配的内存 \\
	\hline
	\end{longtable}

\subsubsection{客户机启动的配置}
	首先看配置的例子:
	\begin{lstlisting}[language = XML]
	<os>
		<type>hvm</type>
		<loader readonly='yes' secure='no' type='rom'>/usr/lib/xen/boot/hvmloader</loader>
		<nvram template='/usr/share/OVMF/OVMF_VARS.fd'>/var/lib/libvirt/nvram/guest_VARS.fd</nvram>
		<boot dev='hd'/> <boot dev='cdrom'/>
		<bootmenu enable='yes' timeout='3000'/>
		<smbios mode='sysinfo'/>
		<bios useserial='yes' rebootTimeout='0'/>
	</os>
	\end{lstlisting}

	在此介绍以上例子中中的标签:
	\interval
	\begin{longtable}{p{2.5cm}p{10cm}}
	\hline
	type & 在虚拟机中启动的操作系统类型 \\
	\hline
	loader & 该标签指定了固件用于协助启动客户机 \\
	\hline
	nvram & 该标签指定了一个文件,用于存放客户机中不可修改的变量 \\
	\hline
	boot & 该标签指定了下一个要启动的设备 \\
	\hline
	smbios & 该标签指定了客户机SMBIOS信息可见的方式,有‘emulate’、‘host’、‘sysinfo’三种方式 \\
	\hline
	bootmenu & 该标签决定了是否在客户机启动时有启动菜单 \\
	\hline
	bios & useserial属性决定了用户能否在串口看到BIOS信息,rebootTimeout属性决定了启动过程中客户机多久后重新启动 \\
	\hline
	\end{longtable}

\subsubsection{网络的配置}
	libvirt中支持以下四种网络配置方式:
	\begin{itemize}
		\item[1.] 桥接方式的网络配置
		\item[2.] NAT方式的虚拟网络配置
		\item[3.] 用户模式网络的配置
		\item[4.] 网卡设备直接分配
	\end{itemize}

	使用桥接方式的网络的配置如下所示:
	\begin{lstlisting}[language = XML]
	<devices>
		<interface type='bridge'>
			<mac address='52:54:00:e9:e0:3b'/>
			<source bridge='br0'>
			<model type='virtio'>
			<address type='pci' domain='0x0000' bus='0x00' slot='0x03' function='0x0'/>
		</interface>
	</devices>
	\end{lstlisting}

	在此介绍以上例子中使用的标签:
	\interval
	\begin{longtable}{p{2.5cm}p{10cm}}
	\hline
	interface & 属性type='bridge'表示使用桥接方式使客户机获得网络 \\
	\hline
	max & 属性address用于配置客户机网卡的MAC地址 \\
	\hline
	source & 属性bridge='br0'表示使用宿主机中的br0来建立网桥 \\
	\hline
	model & 属性type='virtio'表示在客户机中使用virtio-net驱动的网卡设备 \\
	\hline
	address & 属性pci用于配置该网卡在客户机中的PCI设备编号为0000:00:03.0 \\
	\hline
	\end{longtable}

	使用NAT进行虚拟网络的配置的例子如下所示:
	\begin{lstlisting}[language = XML]
	<devices>
		<interface type='network'>
			<mac address='52:54:00:32:7d:f6'/>
			<source network='default'/>
			<address type='pci' domain='0x0000' bus='0x00' slot='0x03' function='0x0'/>
		</interface>
	</devices>
	\end{lstlisting}

	这里<interface type='network'>和<source network='default'/>表示使用NAT的方式。使用NAT必须保证宿主机中运行着DHCP和DNS服务器。\par

	使用用户模式的网络配置如下所示:
	\begin{lstlisting}[language = XML]
	<devices>
		<interface type='user'>
			<mac address="00:11:22:33:44:55"/>
		</interface>
	</devices>
	\end{lstlisting}

	这里<interface type='user'>表示该客户机的网络接口是用户模式网络,是完全有qemu-kvm软件模拟的一个网络协议栈。\par

	使用网卡设备直接分配的网络配置的例子如下:
	\begin{lstlisting}[language = XML]
	<devices>
		<interface type='hostdev'>
			<source>
				<address type='pci' domain='0x0000' bus='0x08' slot='0x10' function='0x0'/>
			</source>
			<mac address='52:54:00:6d:90:02'/>
		</interface>
	</devices>
	\end{lstlisting}

	这里<interface type='hostdev'>指定将网卡设备直接分配给客户机使用。以下代码直接将宿主机中的PCI 0000:08:00.0设备直接分配给客户机使用:
	\begin{lstlisting}
	<source>
		<address type='pci' domain='0x0000' bus='0x08' slot='0x10' function='0x0'/>
	</source>
	\end{lstlisting}

\subsubsection{存储的配置}
	首先来看一个关于客户机磁盘的配置例子:
	\begin{lstlisting}[language = XML]
	<devices>
		<disk type='file' device='disk'>
			<driver name='qemu' type='raw' cache='none'/>
			<source file='/var/lib/libvirt/images/ubuntu1604.img'/>
			<target dev='vda' bus='virtio'/>
			<address type='pci' domain='0x0000' bus='0x00' slot='0x05' function='0x0'/>
		</disk>
	</devices>
	\end{lstlisting}

	在此介绍上面例子中的标签:
	\interval
	\begin{longtable}{p{2cm}p{10cm}}
	\hline
	disk & 客户机磁盘配置的主标签,属性type表示磁盘使用哪种类型作为磁盘的来源,属性device表示让客户机如何使用该磁盘设备 \\
	\hline
	driver & 用于定义Hypervisor如何为该磁盘提供驱动,属性name制定了宿主机中使用的后端驱动名称,属性type指定了镜像文件的格式,属性cache表示在宿主机中打开该磁盘时使用的缓存方式 \\
	\hline
	source & 属性file指定了磁盘的来源 \\
	\hline
	target & 表示将磁盘暴露给客户时的总线类型和设备名称,属性dev制定了客户机中该磁盘设备的逻辑设备名称,属性bus指定了该磁盘设备被模拟挂载的总线类型 \\
	\hline
	address & 表示该磁盘设备在客户机中的PCI地址
	\end{longtable}

\subsubsection{其他配置简介}
	使用domain标签配置域,如下所示:
	\begin{lstlisting}[language = XML]
	<domain type='kvm'>
		<!-- ... -->
	</domain>
	\end{lstlisting}

	domain标签中有两个属性:
	\interval
	\begin{longtable}{p{1.5cm}p{10cm}}
	\hline
	type & 用于表示Hypervisor的类型,可选值有xen、kvm、qemu、lxc、kqemu和vmware \\
	\hline
	id & 用于标识在libvirt中运行的客户机,如果不设置id属性,libvirt会按顺序分配一个最小可用的ID \\
	\hline
	\end{longtable}

	还可以配置域的元数据,例子如下所示:
	\begin{lstlisting}[language = XML]
	<domain type='xen' id='3'>
		<name>fv0</name>
		<uuid>4dea22b31d52d8f32516782e98ab3fa0</uuid>
		<title>A short description - title - of the domain</title>
		<description>Some human readable description</description>
		<metadata>
			<app1:foo xmlns:app1="http://app1.org/app1/">..</app1:foo>
			<app2:bar xmlns:app2="http://app1.org/app2/">..</app2:bar>
		</metadata>
	</domain>
	\end{lstlisting}

	以上例子的标签介绍如下:
	\interval
	\begin{longtable}{p{1.5cm}p{10cm}}
	\hline
	name & 为客户机提供了名字 \\
	\hline
	uuid & 为客户机提供了uuid \\
	\hline
	title & 为域提供了简短的描述,不可以包换换行符 \\
	\hline
	description & 为客户机提供了描述 \\
	\hline
	metadata & 可以被应用程序用来存放自定义的元数据 \\
	\hline
	\end{longtable}

	可以使用emulator标签指定QEMU模拟器的绝对路径,如下所示:
	\begin{lstlisting}[language = XML]
	<device>
		<emulator>/usr/libexec/qemu-kvm</emulator>
	</devices>
	\end{lstlisting}

	可以使用controller标签来配置PCI控制器,相关例子如下所示:
	\begin{lstlisting}[language = XML]
	<domain>
		<controller type='usb' index='0'>
			<address type='pci' domain='0x0000' bus='0x00' slot='0x01' function='0x2'/>
		</controller>
		<controller type='usb' index='0'>
			<address type='pci' domain='0x0000' bus='0x00' slot='0x01' function='0x1'/>
		</controller>
	</domain>
	\end{lstlisting}

	在这个例子里指定了一个USB控制器和一个IDE控制器。

\clearpage

\section{virsh}
	virsh是完全在命令行文本模式下运行的用户态工具,用于管理虚拟化环境中的客户机和Hypervisor。

\subsection{virsh常用命令}
	在linux系统中可以通过“man virsh”命令查看virsh的帮助文档。以下介绍一些virsh的常用命令。

\subsubsection{域管理的命令}
	下面是域管理中常用的virsh命令:
	\begin{longtable}{p{6cm}p{7cm}}
	\hline
	list & 获取当前节点上所有域的列表 \\
	\hline
	domstate<ID or NAME or UUID> & 获取一个域的运行状态 \\
	\hline
	dominfo<ID> & 获取一个域的基本信息 \\
	\hline
	domid<NAME or UUID> & 根据域的名称或UUID返回域的ID值 \\
	\hline
	domname<ID or UUID> & 根据域的ID或UUID返回域的名称 \\
	\hline
	dommemstat<ID> & 获取一个域的内存使用情况的统计信息 \\
	\hline
	setmem<ID><mem-size> & 设置一个域的内存大小 \\
	\hline
	vcpuinfo<ID> & 获取一个域的vCPU的基本信息 \\
	\hline
	vcpupin<ID><vCPU><pCPU> & 将一个域的vCPU绑定到某一个物理CPU上运行 \\
	\hline
	setvcpus<ID><vCPU-num> & 设置一个域的vCPU个数 \\
	\hline
	vncdisplay<ID> & 获取一个域的VNC连接IP地址和端口 \\
	\hline
	create<dom.xml> & 根据域的XML配置文件创建一个域 \\
	\hline
	suspend<ID> & 暂停一个域 \\
	\hline
	resume<ID> & 唤醒一个域 \\
	\hline
	shutdown<ID> & 让一个域执行关机操作 \\
	\hline
	reboot<ID> & 让一个域重启 \\
	\hline
	reset<ID> & 强制重启一个域,相当于按电源“reset”按钮 \\
	\hline
	destroy<ID> & 立即删除一个域 \\
	\hline
	save<ID><file.img> & 保存一个运行中的域的状态到一个文件中 \\
	\hline
	restore<ID><file.img> & 从一个被保存的文件中恢复一个域的运行 \\
	\hline
	migrate<ID><dest\_url> & 将一个域迁移到另外一个目的地址 \\
	\hline
	dumpxml<ID> & 以XML格式转存出一个域的信息到标准输出中 \\
	\hline
	attach-device<ID><device.xml> & 向一个域添加XML文件中的设备 \\
	\hline
	detach-device<ID><device.xml> & 将XML文件中的设备从一个域中移除 \\
	\hline
	console<ID> & 连接到一个域的控制台 \\
	\hline
	\end{longtable}

\subsubsection{宿主机和Hypervisor的管理命令}
	下面是宿主机和Hypervisor管理中常用的virsh命令:
	\begin{longtable}{p{6cm}p{7cm}}
	\hline
	version & 显示libvirt和Hypervisor的版本信息 \\
	\hline
	sysinfo & 以XML格式打印宿主机系统的信息 \\
	\hline
	nodeinfo & 显示该节点的基本信息 \\
	\hline
	uri & 显示当前连接的URI \\
	\hline
	hostname & 显示当前节点的主机名 \\
	\hline
	capabilities & 显示该节点宿主机和客户机的架构和特性 \\
	\hline
	freecell & 显示当前MUMA单元的可用空闲内存 \\
	\hline
	nodememstats<cell> & 显示该节点的内存单元使用情况的统计 \\
	\hline
	connect<URI> & 连接到URI指示的Hypervisor \\
	\hline
	nodecpustats<cpu-num> & 显示该节点的某个CPU使用情况 \\
	\hline
	qemu-attach<pid> & 根据PID添加一个QEMU进程libvirt中 \\
	\hline
	qemu-monitor-command  & 向域的QEMU monitor发送一个命令 \\
	domain [--hmp] command & \\
	\hline
	\end{longtable}

\subsubsection{网络的管理命令}
	下面是网络管理中常用的virsh命令:
	\begin{longtable}{p{6cm}p{7cm}}
	\hline
	iface-list & 显示出物理主机的网络接口列表 \\
	\hline
	iface-mac<if-name> & 根据网络接口名称查询其对应的MAC地址 \\
	\hline
	iface-name<MAC> & 根据MAC地址查询其对应的网络接口名称 \\
	\hline
	iface-edit<if-name-or-uuid> & 编辑一个物理主机的网络接口的XML配置文件 \\
	\hline
	iface-dumpxml<if-name-or-uuid> & 以XML格式转存出一个网络接口的状态信息 \\
	\hline
	iface-destroy<if-name-or-uuid> & 关闭宿主机上一个物理网络接口 \\
	\hline
	net-list & 列出libvirt管理的虚拟网络 \\
	\hline
	net-info<net-name-or-uuid> & 根据名称查询一个虚拟网络的基本信息 \\
	\hline
	net-uuid<net-name> & 根据名称查询一个虚拟网络的UUID \\
	\hline
	net-name<net-UUID> & 根据UUID查询一个虚拟网络的名称 \\
	\hline
	net-create<net.xml> & 根据一个网络XML配置文件创建一个虚拟网络 \\
	\hline
	net-edit<net-name-or-uuid> & 编辑一个虚拟网络的XML配置文件 \\
	\hline
	net-dumpxml<net-name-or-uuid> & 转存出一个虚拟网络的XML格式的配置信息 \\
	\hline
	net-destroy<net-name-or-uuid> & 销毁一个虚拟网络 \\
	\hline 
	\end{longtable}

\subsubsection{存储池和存储卷的管理命令}
	下面是存储池和存储卷管理中常用的virsh命令:
	\begin{longtable}{p{6cm}p{7cm}}
	\hline
	pool-list & 显示出libvirt管理的存储池 \\
	\hline
	pool-info<pool-name> & 根据一个存储池名称查询其基本信息 \\
	\hline
	pool-uuid<pool-name> & 根据存储池名称查询其UUID \\
	\hline
	pool-create<pool.xml> & 根据XML配置文件来创建一个存储池 \\
	\hline
	pool-edit<pool-name-or-uuid> & 编辑一个存储池的XML配置文件 \\
	\hline
	pool-destroy<pool-name-or-uuid> & 关闭一个存储池 \\
	\hline
	pool-delete<pool-name-or-uuid> & 删除一个存储池 \\
	\hline
	vol-list<pool-name-or-uuid> & 查询一个存储池中存储卷的列表 \\
	\hline
	vol-name<vol-key-or-path> & 查询一个存储卷的名称 \\
	\hline
	vol-path --pool <pool><vol-name-or-key> & 查询一个存储卷的路径 \\
	\hline
	vol-create<vol.xml> & 根据XML配置文件来创建一个存储卷 \\
	\hline
	vol-clone<vol-name-path><name> & 克隆一个存储卷 \\
	\hline
	vol-delete<vol-name-or-key-or-path> & 删除一个存储卷 \\
	\hline
	\end{longtable}

\subsubsection{其他常用命令}
	以下是virsh其他方面的常用命令:
	\begin{longtable}{p{5cm}p{7cm}}
	\hline
	help & 显示出virsh的命令帮助文档 \\
	\hline
	pwd & 打印出当前的工作目录 \\
	\hline
	cd<your-dir> & 改变当前工作目录 \\
	\hline
	quit & 退出virsh \\
	\hline
	exit & 退出virsh \\
	\hline
	\end{longtable}

\clearpage

\section{创建一个虚拟机}
\subsection{制作虚拟机镜像}
	命令如下所示:
	\begin{lstlisting}[language = bash]
	qemu-img create -f qcow2 Ubuntu.qcow2 10G
	\end{lstlisting}

\subsection{编写客户机配置文件}
	根据第一节中的内容,可以编写libvirt域的配置文件,例子如下所示:
	\begin{lstlisting}[language = XML]
	<domain type='kvm'>
		<name>Ubuntu</name>
		<memory>1048576</memory>
		<currentMemory>1048576</currentMemory>
		<vcpu>4<vcpu>
		<os>
			<type arch='x86_64' machine='pc'>hvm</type>
			<boot dev='cdrom'/>
		</os>
		<features>
			<acpi/>
			<apic/>
			<pae/>
		</features>
		<clock offset='localtime'/>
		<on_poweroff>destroy</on_poweroff>
		<on_reboot>restart</on_reboot>
		<on_crash>destroy</on_crash>
		<devices>
			<emulator>/usr/local/bin/qemu-system-x86_64</emulator>
			<disk type='file' device='disk'>
				<driver name='qemu' type='qcow2'/>
				<source file='/home/pengsida/下载/Ubuntu.qcow2'/>
				<target dev='hda' bus='ide'/>
			</disk>
			<disk type='file' device='cdrom'>
				<source file='/home/pengsida/下载/ubuntu.iso'/>
				<target dev='hdb' bus='ide'/>
			</disk>
			<input type='mouse' bus='ps2'/>
			<graphics type='vnc' port='-1' autoport='yes' listen='0.0.0.0' keymap='en-us'/>
		</devices>
	</domain>
	\end{lstlisting}

\subsection{创建虚拟机}
	首先需要保证libvirtd守护进程是启动的,否则会报错。启动libvirtd守护进程的命令如下:
	\begin{lstlisting}[language = bash]
	libvirtd
	\end{lstlisting}

	然后还需要启动virtlogd服务,命令如下:
	\begin{lstlisting}
	sudo systemctl start virtlogd
	# 如果遇到Unit virlogd.service is masked,就输入如下命令:
	# sudo systemctl unmask virtlogd.service
	# 如果遇到Unit virlogd.socket is masked,就输入如下命令:
	# sudo systemctl unmask virtlogd.socket
	\end{lstlisting}

	然后使用如下命令创建一个虚拟机:
	\begin{lstlisting}[language = bash]
	# demo.xml是刚才创建的XML配置文件的文件名
	virsh define demo.xml
	virsh
	# ubuntu是创建的虚拟机的名字
	start ubuntu
	\end{lstlisting}

	使用以下命令可以通过vncviewer查看虚拟机:
	\begin{lstlisting}[language = bash]
	# ubuntu是刚刚创建的域的名字
	vncdisplay Ubuntu
	\end{lstlisting}

	如果terminal输出“:1”,可以根据这个参数查看虚拟机,命令如下所示:
	\begin{lstlisting}[language = bash]
	vncviewer :1
	\end{lstlisting}

	安装成功后,可以看到如下界面:
	\sizedfic{0.5}{5.png}
	
\clearpage

\section{建立到Hypervisor的连接}
	要使用libvirt API进行虚拟化管理,就必须先建立到Hypervisor的连接。\par
	在使用virsh工具时,可以使用“-c”参数来指定建立到某个URI上的连接,相关命令如下所示:
	\begin{lstlisting}[language = bash]
	sudo virsh -c URI
	# 如,virsh -c qemu:///system
	# 又如,virsh -c qemu+ssh://root@192.168.158.31/system
	\end{lstlisting}

\subsection{使用本地URI连接Hypervisor}
	使用本地URI可以连接本系统范围内的Hypervisor,本地URI的一般格式如下:
	\begin{lstlisting}
	driver[+transport]:///[path][?extral-param]
	\end{lstlisting}

	对其中的元素解释如下:
	\begin{longtable}{p{2.5cm}p{10cm}}
	\hline
	driver & 连接Hypervisor的驱动名称 \\
	\hline
	transport & 选择该连接所使用的传输方式 \\
	\hline
	path & 连接到服务器端上的某个路径 \\
	\hline
	?extral-param & 用于添加一些额外的参数 \\
	\hline
	\end{longtable}

	本地连接KVM的URI的例子如下:
	\begin{lstlisting}[language = bash]
	# 连接到本地的session实例,该连接仅能管理当前用户虚拟化资源
	qemu:///session

	# 以Unix domain socket的方式连接到本地的session实例,该连接仅能管理当前用户的虚拟化资源
	qemu+unix:///session

	# 连接到本地的system实例,该连接可以管理当前节点的所有虚拟化资源
	qemu:///system

	# 以Unix domain socket的方式连接到本地的system实例,该连接可以管理当前节点的所有虚拟化资源
	qemu:///system
	\end{lstlisting}

\subsection{使用远程URI连接Hypervisor}
	使用远程URI可以连接到网络上的Hypervisor,远程URI的例子如下:
	\begin{lstlisting}[language = bash]
	driver[+transport]://[user@][host][:port]/[path][?extral-param]
	\end{lstlisting}

	对其中的元素解释如下:
	\begin{longtable}{p{2.5cm}p{10cm}}
	\hline
	driver & 连接Hypervisor的驱动名称 \\
	\hline
	transport & 选择该连接所使用的传输方式,取值可以是ssh、tcp和libssh2 \\
	\hline
	user & 远程主机使用的用户名 \\
	\hline
	host & 远程主机的主机名或IP地址 \\
	\hline
	port & 连接远程主机的端口 \\
	\hline
	path & 连接到服务器端上的某个路径 \\
	\hline
	?extral-param & 用于添加一些额外的参数 \\
	\hline
	\end{longtable}

	远程连接KVM的URI的例子如下:
	\begin{lstlisting}[language = bash]
	# 通过ssh通道连接到远程节点的system实例,以最大权限管理远程节点上的虚拟化资源
	qemu+ssh://root@example.com/system

	# 通过ssh通道连接到远程节点的使用user用户的session实例,仅能对user用户的虚拟化资源进行管理
	qemu+ssh://user@example.com/session

	# 通过加密的TLS连接到远程节点的system实例,以最大权限管理远程节点上的虚拟化资源
	qemu://example.com/system

	# 通过加密的TCP连接到远程节点的system实例,以最大权限管理远程节点上的虚拟化资源
	qemu+tcp://example.com/system
	\end{lstlisting}

\subsection{连接到Hypervisor的例子}
	需要注意的是,如果要连接到某个节点上的Hypervisor,需要保证那个节点上的libvirtd守护进程正在执行,否则会报错。\par
	连接到当地Hypervisor的例子如下图所示:
	\sizedfic{0.7}{6.png}


\clearpage

\section{libvirt API}
\subsection{libvirt API的简介}
	libvirt API可以分为8个部分,接下来在每小节列出常用的API函数。
\subsubsection{连接Hypervisor的API}
		连接Hypervisor相关的API。有以下函数:
		\begin{longtable}{p{3.8cm}p{10.5cm}}
		\hline
		virConnectOpen & 建立一个连接,返回值是一个virConnectPtr对象,该对象代表到Hypervisor的一个连接 \\
		\hline
		virConnectOpenReadOnly & 建立一个只读的连接 \\
		\hline
		virConnectGetCapabilities & 返回对Hypervisor和驱动的功能的描述的XML格式的字符串 \\
		\hline
		virConnectListDomains & 返回一列域标识符,它们代表该Hypervisor上的活动域 \\
		\hline 
		\end{longtable}

		% \clearpage

\subsubsection{域管理的API}

		域管理的API。有如下函数:
		\begin{longtable}{p{15cm}}
		\hline
		根据域的id值到conn这个连接上去查找相应的域: virDomainPtrvirDomainLookupByID \\
		\hline
		根据域的名字去查找相应的域: virDomainLookupByName \\
		\hline
		根据域的UUID去查找相应的域: virDomainLookupByUUID  \\
		\hline
		查询域的信息: virDomainGetHostname virDomainGetInfo virDomainGetVcpus virDomainGetVcpusFlags virDomainGetCPUStats \\
		\hline
		控制域的生命周期: virDomainCreate virDomainSuspend virDomainResume virDomainDestroy virDomainMigrate \\
		\hline 
		\end{longtable}

\subsubsection{节点管理的API}
		节点管理的API。有如下函数:
		\begin{longtable}{p{13cm}}
		\hline
		virNodeGetInfo: 获取节点的物理硬件信息 \\
		\hline
		virNodeGetCPUStats: 获取节点上各个CPU的使用统计信息 \\
		\hline
		virNodeGetFreeMemory: 获取节点上可用的空闲内存大小 \\
		\hline
		virNodeSetMemoryParameters: 设置节点上的内存调度的参数 \\
		\hline
		virNodeSuspendForDuration: 让节点暂停运行一段时间 \\
		\hline
		\end{longtable}

\subsubsection{网络管理的API}
		网络管理的API。有如下函数:
		\begin{longtable}{p{13cm}}
		\hline
		virNetworkGetName: 获取网络的名称 \\
		\hline
		virNetworkGetBridgeName: 获取该网络中网桥的名称 \\
		\hline
		virNetworkGetUUID: 获取网络的UUID标识 \\
		\hline
		virNetworkGetXMLDesc: 获取网络的以XML格式的描述信息 \\
		\hline
		virNetworkIsActive: 查询网络是否正在使用 \\
		\hline
		virNetworkCreateXML: 根据提供的XML格式的字符串创建一个网络 \\
		\hline
		virNetworkDestroy: 销毁一个网络 \\
		\hline
		virNetworkFree: 回收一个网络 \\
		\hline
		virNetworkUpdate: 根据XML格式的网络配置来更新一个已存在的网络 \\
		\hline
		virInterfaceCreate: 创建一个网络接口 \\
		\hline
		virInterfaceFree: 释放一个网络接口 \\
		\hline
		virInterfaceDestroy: 销毁一个网络接口 \\
		\hline
		virInterfaceGetName: 获取网络接口的名称 \\
		\hline
		virInterfaceIsActive: 查询网络接口是否正在运行 \\
		\hline
		\end{longtable}

\subsubsection{存储卷管理的API}

		存储卷管理的API。有如下函数:
		\begin{longtable}{p{13cm}}
		\hline
		virStorageVolLookupByKey: 根据全局唯一的键值来获得一个存储卷的对象 \\
		\hline
		virStorageVolLookupByName: 根据名称来获得一个存储卷的对象 \\
		\hline
		virStorageVolLookupByPath: 根据节点上的路径来获取一个存储卷的对象 \\
		\hline
		virStorageVolGetInfo: 查询某个存储卷的使用情况 \\
		\hline
		virStorageVolGetPath: 获取存储卷的路径 \\
		\hline
		virStorageVolGetConnect: 查询存储卷的连接 \\
		\hline
		virStorageVolCreateXML: 根据XML配置文件来创建一个存储卷 \\
		\hline
		virStorageVolFree: 释放存储卷的句柄 \\
		\hline
		virStorageVolDelete: 删除一个存储卷 \\
		\hline
		virStorageVolResize: 调整存储卷的大小 \\
		\hline
		\end{longtable}

\subsubsection{存储池管理的API}
		存储池管理的API。有如下函数:
		\begin{longtable}{p{13cm}}
		\hline
		virStoragePoolLookupByName: 根据存储池的名称来获取一个存储池对象 \\
		\hline
		virStoragePoolLookupByVolume: 根据一个存储卷返回其对应的存储池对象 \\
		\hline
		virStoragePoolCreateXML: 根据XML配置文件来创建一个存储池 \\
		\hline
		virStoragePoolDefineXML: 根据XML配置文件静态地定义个存储池 \\
		\hline
		virStoragePoolCreate: 激活一个存储池 \\
		\hline
		virStoragePoolGetInfo: 获取存储池的信息 \\
		\hline
		virStoragePoolGetName: 获取存储池的名称 \\
		\hline
		virStoragePoolGetUUID: 获取存储池的UUID标识 \\
		\hline
		virStoragePoolIsActive: 查询存储池是否处于使用状态 \\
		\hline
		virStoragePoolFree: 释放存储池相关的内存 \\
		\hline
		virStoragePoolDestroy: 用于销毁一个存储池 \\
		\hline
		virStoragePoolDelete: 物理删除一个存储池资源 \\
		\hline
		\end{longtable}

\subsection{使用libvirt API的例子}
	下面用一个简单的例子介绍如何使用libvirt API,例子如下:
	\begin{lstlisting}
	#include<stdio.h>
	#include<stdlib.h>
	#include<libvirt/libvirt.h>

	int main(int argc, char* argv[])
	{
		virConnectPtr conn;
		conn = virConnectPtr("qemu:///system");
		if(conn == NULL)
		{
			fprintf(stderr, "Failed to open connection to qemu:///system");
			return 1;
		}
		else
			printf("Open connection successulfully");
		virConnectClose(conn);
		return 0;
	}
	\end{lstlisting}

	其实这个例子不是重点,这里想重点说明的是,编译这个程序的时候,需要在后面加上“-lvirt”参数,如下所示:
	\begin{lstlisting}[language = bash]
	# temp.c是刚刚那个例子的文件名字
	cc temp.c -lvirt
	\end{lstlisting}

	编译通过以后,就可以像普通程序一样使用执行文件了。

\clearpage

\section{自动化安装OpenStack}
	这里使用DevStack脚本来搭建OpenStack开发环境,有以下两个步骤:
	\begin{itemize}
		\item[1.] 下载DevStack的源代码,命令行如下:
		\begin{lstlisting}[language = bash]
	git clone git://github.com/openstack-dev/devstack.git
		\end{lstlisting}

		\item[2.] 在DevStack文件夹下创建local.conf文件,命令如下所示:
	\begin{lstlisting}[language = bash]
	sudo vi local.conf
	\end{lstlisting}

		然后写入如下内容:
		\begin{lstlisting}[language = bash]
	[[local|localrc]]
	#NOVA
	enable_service n-cell
		\end{lstlisting}

		\item[3.] 运行stack.sh脚本,命令行如下:
		\begin{lstlisting}[language = bash]
	# 注意,DevStack脚本所处的路径不能包含中文
	./stack.sh
		\end{lstlisting}
	\end{itemize}

	需要注意的是,安装成功后,terminal会输出很重要的信息,一定要记住,如下图所示:
	\fic{7.png}

	里面有dashboard的登陆地址,dashboard登陆的用户名和密码。如果丢失了这个信息,貌似是找不回来的。

\end{document}
