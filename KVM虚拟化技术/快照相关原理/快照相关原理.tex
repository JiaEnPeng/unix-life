% !TeX spellcheck = en_US
%% 字体:方正静蕾简体
%%		 方正粗宋
\documentclass[a4paper,left=1.5cm,right=1.5cm,11pt]{article}

\usepackage[utf8]{inputenc}
\usepackage{fontspec}
\usepackage{cite}
\usepackage{xeCJK}
\usepackage{indentfirst}
\usepackage{titlesec}
\usepackage{etoolbox}%
\makeatletter
\patchcmd{\ttlh@hang}{\parindent\z@}{\parindent\z@\leavevmode}{}{}%
\patchcmd{\ttlh@hang}{\noindent}{}{}{}%
\makeatother

\usepackage{hyperref}
\usepackage{longtable}
\usepackage{empheq}
\usepackage{graphicx}
\usepackage{float}
\usepackage{rotating}
\usepackage{subfigure}
\usepackage{tabu}
\usepackage{amsmath}
\usepackage{setspace}
\usepackage{amsfonts}
\usepackage{appendix}
\usepackage{listings}
\usepackage{xcolor}
\usepackage{geometry}
\setcounter{secnumdepth}{4}
%\titleformat*{\section}{\LARGE}
%\renewcommand\refname{参考文献}
%\titleformat{\chapter}{\centering\bfseries\huge}{}{0.7em}{}{}
\titleformat{\section}{\LARGE\bf}{\thesection}{1em}{}{}
\titleformat{\subsection}{\Large\bfseries}{\thesubsection}{1em}{}{}
\titleformat{\subsubsection}{\large\bfseries}{\thesubsubsection}{1em}{}{}
\renewcommand{\contentsname}{{ \centerline{目{  } 录}}}
\setCJKfamilyfont{cjkhwxk}{STXINGKA.TTF}
%\setCJKfamilyfont{cjkhwxk}{华文行楷}
%\setCJKfamilyfont{cjkfzcs}{方正粗宋简体}
%\newcommand*{\cjkfzcs}{\CJKfamily{cjkfzcs}}
\newcommand*{\cjkhwxk}{\CJKfamily{cjkhwxk}}
%\newfontfamily\wryh{Microsoft YaHei}
%\newfontfamily\hwzs{华文中宋}
%\newfontfamily\hwst{华文宋体}
%\newfontfamily\hwfs{华文仿宋}
%\newfontfamily\jljt{方正静蕾简体}
%\newfontfamily\hwxk{华文行楷}
\newcommand{\verylarge}{\fontsize{60pt}{\baselineskip}\selectfont}  
\newcommand{\chuhao}{\fontsize{44.9pt}{\baselineskip}\selectfont}  
\newcommand{\xiaochu}{\fontsize{38.5pt}{\baselineskip}\selectfont}  
\newcommand{\yihao}{\fontsize{27.8pt}{\baselineskip}\selectfont}  
\newcommand{\xiaoyi}{\fontsize{25.7pt}{\baselineskip}\selectfont}  
\newcommand{\erhao}{\fontsize{23.5pt}{\baselineskip}\selectfont}  
\newcommand{\xiaoerhao}{\fontsize{19.3pt}{\baselineskip}\selectfont} 
\newcommand{\sihao}{\fontsize{14pt}{\baselineskip}\selectfont}      % 字号设置  
\newcommand{\xiaosihao}{\fontsize{12pt}{\baselineskip}\selectfont}  % 字号设置  
\newcommand{\wuhao}{\fontsize{10.5pt}{\baselineskip}\selectfont}    % 字号设置  
\newcommand{\xiaowuhao}{\fontsize{9pt}{\baselineskip}\selectfont}   % 字号设置  
\newcommand{\liuhao}{\fontsize{7.875pt}{\baselineskip}\selectfont}  % 字号设置  
\newcommand{\qihao}{\fontsize{5.25pt}{\baselineskip}\selectfont}    % 字号设置 

\usepackage{diagbox}
\usepackage{multirow}
\boldmath
\XeTeXlinebreaklocale "zh"
\XeTeXlinebreakskip = 0pt plus 1pt minus 0.1pt
\definecolor{cred}{rgb}{0.8,0.8,0.8}
\definecolor{cgreen}{rgb}{0,0.3,0}
\definecolor{cpurple}{rgb}{0.5,0,0.35}
\definecolor{cdocblue}{rgb}{0,0,0.3}
\definecolor{cdark}{rgb}{0.95,1.0,1.0}
\lstset{
	language=bash,
	numbers=left,
	numberstyle=\tiny\color{black},
	showspaces=false,
	showstringspaces=false,
	basicstyle=\scriptsize,
	keywordstyle=\color{purple},
	commentstyle=\itshape\color{cgreen},
	stringstyle=\color{blue},
	frame=lines,
	% escapeinside=``,
	extendedchars=true, 
	xleftmargin=1em,
	xrightmargin=1em, 
	backgroundcolor=\color{cred},
	aboveskip=1em,
	breaklines=true,
	tabsize=4
} 

%\newfontfamily{\consolas}{Consolas}
%\newfontfamily{\monaco}{Monaco}
%\setmonofont[Mapping={}]{Consolas}	%英文引号之类的正常显示,相当于设置英文字体
%\setsansfont{Consolas} %设置英文字体 Monaco, Consolas,  Fantasque Sans Mono
%\setmainfont{Times New Roman}
%\setCJKmainfont{STZHONGS.TTF}
%\setmonofont{Consolas}
% \newfontfamily{\consolas}{YaHeiConsolas.ttf}
\newfontfamily{\monaco}{MONACO.TTF}
\setCJKmainfont{STZHONGS.TTF}
%\setmainfont{MONACO.TTF}
%\setsansfont{MONACO.TTF}

\newcommand{\fic}[1]{\begin{figure}[H]
		\center
		\includegraphics[width=0.8\textwidth]{#1}
	\end{figure}}
	
\newcommand{\sizedfic}[2]{\begin{figure}[H]
		\center
		\includegraphics[width=#1\textwidth]{#2}
	\end{figure}}

\newcommand{\codefile}[1]{\lstinputlisting{#1}}

\newcommand{\interval}{\vspace{0.5em}}

\newcommand{\tablestart}{
	\interval
	\begin{longtable}{p{2cm}p{10cm}}
	\hline}
\newcommand{\tableend}{
	\hline
	\end{longtable}
	\interval}

% 改变段间隔
\setlength{\parskip}{0.2em}
\linespread{1.1}

\usepackage{lastpage}
\usepackage{fancyhdr}
\pagestyle{fancy}
\lhead{\space \qquad \space}
\chead{快照相关原理 \qquad}
\rhead{\qquad\thepage/\pageref{LastPage}}

\begin{document}

\tableofcontents

\clearpage
\section{QCOW2 backing files和overlays}
    backing file的中文是后端镜像,overlay的中文是增量镜像。\par

    qemu中经常使用qcow2镜像,因为它具有copy-on-write特性。
    这种特性支持:当我们创建虚拟机时,可以先创建一个base-image,也就是backing file,也就是后端镜像。
    后端镜像中包含有完整的OS以及引导程序。随后可以根据base-image为模板创建多个虚拟机,
    也就是在backing file的基础上创建多个copy-on-write overlays镜像。\par

    不止如此,我们还可以在创建好的overlay镜像上再创建overlay镜像,如下图所示:
    \fic{1.png}

    在上图中,overlay-1的backing file是RootBase,overlay-1A的backing file是overlay-1,
    overlay-1B的backing file是overlay-1A。\par

    当然我们也可以在一个base-image上创建多个overlays镜像,如下图所示:
    \fic{2.png}

    以下是关于backing file的注意事项:
    \begin{lstlisting}
    一个overlay镜像创建以后,它的backing file是不允许被修改的,只能是read-only。
    \end{lstlisting}

\subsection{举个例子}
\subsubsection{创建增量镜像的命令}
    现在FedoraBase.img中刚刚装好了Fedora-17操作系统,现在我们可以如下命令将它变成后端镜像,并且在它的基础上制作增量镜像:
    \begin{lstlisting}
    qemu-img create -b FedoraBase.img -f qcow2 Fedora-guest-1.qcow2
    \end{lstlisting}

    有时候,一个虚拟机仅仅是刚刚装好操作系统是不够的,最好是还有完整的软件包更新,此时我们将Fedora-guest-1.qcow2中的
    虚拟机的软件包进行更新,然后在进行如下命令:
    \begin{lstlisting}
    qemu-img create -b Fedora-guest-1.qcow2 -f qcow2 Fedora-guest-with-updates-1.qcow2
    \end{lstlisting}

    此时Fedora-guest-1.qcow2也变成了一个read-only文件,而Fedora-guest-with-updates-1.qcow2作为一个增量镜像,
    它里面已经装好了带有完整软件包更新的操作系统。\par

\subsubsection{查看后端镜像的命令}

    可以通过如下命令查看虚拟机镜像的backing file:
    \begin{lstlisting}
    qemu-img info Fedora-guest-1.qcow2
    \end{lstlisting}

    可以通过如下命令查看虚拟机镜像的backing chain:
    \begin{lstlisting}
    qemu-img info --backing-chain Fedora-guest-with-updates-1.qcow2
    \end{lstlisting}

\clearpage

\section{快照相关术语}
\subsection{内部快照}
\subsubsection{内部磁盘快照}
    这个将存储snapshot时虚拟机磁盘的状态。
    虚拟机在执行快照之后的snapshot和delta都存放在原来的qcow2文件中。\par

    可以使用“qemu-img snapshot”命令对一台offline的虚拟机进行快照管理。
    关于它更多的描述可以查看《KVM基础功能》中的“qemu-img工具”一节。\par
    也可以在QEMU monitor中的“savevm”命令对一台online的虚拟机进行快照创建。
    关于它更多的描述可以查看《KVM高级功能》中的“QEMU监控器”一节。

\subsubsection{内部系统还原点}
    系统还原点指的是RAM state、device state和disk state。内部系统还原点存放在原来的qcow2镜像中。\par

    需要知道的是,只有在运行状态才能创建内部系统还原点。内部系统还原点的建立也是使用“savevm”命令。

\subsection{外部快照}
    当一个外部快照建立时,虚拟机被保存的状态将被存放在一个文件中,这个文件也就成为了read-only的后端镜像,
    而同时将产生一个新文件,作为一个增量镜像overlay来记录snapshot以后的delta。

\subsubsection{外部磁盘快照}
    虚拟机磁盘信息的快照将被存放在一个文件中,然后snapshot之后的delta将被一个新的qcow2文件记录。\par

\subsubsection{外部系统还原点}
    虚拟机的磁盘状态将被保存到一个文件中,内存和设备的状态将被保存到另外一个新的文件中。

\clearpage

\section{创建快照的命令}
\subsection{创建disk-only的内部快照}
    假设我们有一个虚拟机叫做“f17vm1”,可以通过如下命令创建一个内部快照,这个内部快照的名字叫做“snap1”,
    它的描述是“snap1-desc”:
    \begin{lstlisting}
    virsh snapshot-create-as f17vm1 snap1 snap1-desc
    \end{lstlisting}

    可以通过命令查看镜像的快照信息:
    \begin{lstlisting}
    # 方法一
    virsh snapshot-list f17vm1
    # 方法二
    qemu-img info f17vm1.qcow2
    \end{lstlisting}

\subsection{创建disk-only的外部快照}
    可以使用virsh创建外部快照。假设我们有一个“devstack”的虚拟机,它的磁盘名为“hda”,原先的磁盘为“devstack.qcow2”。
    一旦创建快照以后,“devstack.qcow2”文件就会变成backing file,而命令行中指定的“file”选项将作为overlay镜像,如下图所示:
    \fic{3.png}

\clearpage

\section{reverting to snapshots}
\subsection{内部快照的回滚}
    内部快照回滚很简单,如下所示:
    \begin{lstlisting}
    virsh snapshot-revert --domain f17vm1 snap1
    \end{lstlisting}

\subsection{外部快照的回滚}
    外部回滚需要将虚拟机的磁盘设备替换为某一版的外部快照文件:
    \begin{lstlisting}
    # 查看image chain
    qemu-img info --bacing-chain active.qcow2
    \end{lstlisting}

    然后将图中的“file”选项的值修改为某一版的外部快照文件的路径:
    \fic{6.png}

\clearpage

\section{增量镜像和后端镜像的合并}
    增量镜像和后端镜像合并的两种方式:
    \begin{itemize}
        \item[1.] blockcommit:将增量镜像融入后端镜像中。
        \item[2.] blockpull:将后端镜像融入增量镜像中。
    \end{itemize}

\subsection{blockcommit}
    blockcommit将增量镜像融入后端镜像中。\par
    假如存在镜像链base<-overlay1<-overlay2<-overlay3,
    将overlay2融入overlay1以后,原先指向overlay2的overlay3镜像将直接指向overlay1镜像。\par

    通过下图很容易理解上一段话:
    \fic{4.png}

    实现上图中的过程很简单,只要先将snap-3融入snap-2中,再把snap-2融入snap-1中就可以:
    \begin{lstlisting}
    virsh blockcommit --domain f17 vda --base sn2.qcow2 --top sn3.qcow2 --wait --verbose
    virsh blockcommit --domain f17 vda --base sn1.qcow2 --top sn2.qcow2 --wait --verbose
    \end{lstlisting}

\subsection{blockpull}
    blockpull将后端镜像融入增量镜像中。需要注意的是,这里的增量镜像必须是active的,也就是镜像链末端的增量镜像。\par

    假如存在镜像链base<-overlay1<-overlay2<-overlay3,
    将overlay1融入overlay2以后,原先指向overlay2的overlay3镜像将直接指向overlay1镜像。\par

    通过下图很容易理解上一段话:
    \fic{5.png}

    可以通过如下命令实现上述操作:
    \begin{lstlisting}
    # 将后端镜像融入增量镜像
    virsh blockpull --domain RootBase --path var/lib/libvirt/images/active.qcow2 --base /var/lib/libvirt/images/RootBase.qcow2 --wait --verbose
    # 清除原先快照的数据
    virsh snapshot-delete --domain RootBase Snap-3 --metadata
    virsh snapshot-delete --domain RootBase Snap-2 --metadata
    virsh snapshot-delete --domain RootBase Snap-1 --metadata
    \end{lstlisting}

\clearpage

\section{删除快照}
\subsection{删除内部快照}
    删除内部快照比较简单,命令如下:
    \begin{lstlisting}
    virsh snapshot-delete --domain f17vm --snapshotname snap6
    \end{lstlisting}

\subsection{删除外部快照}
    假设我们现在有一个image chain:base<-sn1<-sn2<-sn3。现在我们想删除外部快照sn2.。

\subsubsection{方式一}
    将sn2融入sn1中,然后将sn1作为sn3的后端镜像:
    \begin{lstlisting}
    qemu-img commit sn2.qcow2
    # -u代表非安全模式,假设新的后端镜像和旧的后端镜像完全相同
    qemu-img rebase -u -b sn1.qcow2 sn3.qcow2
    # 最后删除外部快照sn2
    rm sn2.qcow2
    \end{lstlisting}

\subsubsection{方式二}
    使用安全模式将sn1作为sn3的后端镜像:
    \begin{lstlisting}
    # 安全模式下,sn3原先的后端镜像sn2只要和sn1有一点不一样,都会被融入sn1中
    qemu-img rebase -b sn1.qcow2 sn3.qcow2
    # 最后删除外部快照sn2
    rm sn2.qcow2
    \end{lstlisting}

\end{document}