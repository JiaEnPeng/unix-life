% !TeX spellcheck = en_US
%% 字体:方正静蕾简体
%%		 方正粗宋
\documentclass[a4paper,left=2.5cm,right=2.5cm,11pt]{article}

\usepackage[utf8]{inputenc}
\usepackage{fontspec}
\usepackage{cite}
\usepackage{xeCJK}
\usepackage{indentfirst}
\usepackage{titlesec}
\usepackage{longtable}
\usepackage{graphicx}
\usepackage{float}
\usepackage{rotating}
\usepackage{subfigure}
\usepackage{tabu}
\usepackage{amsmath}
\usepackage{setspace}
\usepackage{amsfonts}
\usepackage{appendix}
\usepackage{listings}
\usepackage{xcolor}
\usepackage{geometry}
\setcounter{secnumdepth}{4}
\usepackage{mhchem}
\usepackage{multirow}
\usepackage{extarrows}
\usepackage{hyperref}
\titleformat*{\section}{\LARGE}
\renewcommand\refname{参考文献}
\renewcommand{\abstractname}{\sihao \cjkfzcs 摘{  }要}
%\titleformat{\chapter}{\centering\bfseries\huge\wryh}{}{0.7em}{}{}
%\titleformat{\section}{\LARGE\bf}{\thesection}{1em}{}{}
\titleformat{\subsection}{\Large\bfseries}{\thesubsection}{1em}{}{}
\titleformat{\subsubsection}{\large\bfseries}{\thesubsubsection}{1em}{}{}
\renewcommand{\contentsname}{{\cjkfzcs \centerline{目{  } 录}}}
\setCJKfamilyfont{cjkhwxk}{STXingkai}
\setCJKfamilyfont{cjkfzcs}{STSongti-SC-Regular}
% \setCJKfamilyfont{cjkhwxk}{华文行楷}
% \setCJKfamilyfont{cjkfzcs}{方正粗宋简体}
\newcommand*{\cjkfzcs}{\CJKfamily{cjkfzcs}}
\newcommand*{\cjkhwxk}{\CJKfamily{cjkhwxk}}
\newfontfamily\wryh{Microsoft YaHei}
\newfontfamily\hwzs{STZhongsong}
\newfontfamily\hwst{STSong}
\newfontfamily\hwfs{STFangsong}
\newfontfamily\jljt{MicrosoftYaHei}
\newfontfamily\hwxk{STXingkai}
% \newfontfamily\hwzs{华文中宋}
% \newfontfamily\hwst{华文宋体}
% \newfontfamily\hwfs{华文仿宋}
% \newfontfamily\jljt{方正静蕾简体}
% \newfontfamily\hwxk{华文行楷}
\newcommand{\verylarge}{\fontsize{60pt}{\baselineskip}\selectfont}  
\newcommand{\chuhao}{\fontsize{44.9pt}{\baselineskip}\selectfont}  
\newcommand{\xiaochu}{\fontsize{38.5pt}{\baselineskip}\selectfont}  
\newcommand{\yihao}{\fontsize{27.8pt}{\baselineskip}\selectfont}  
\newcommand{\xiaoyi}{\fontsize{25.7pt}{\baselineskip}\selectfont}  
\newcommand{\erhao}{\fontsize{23.5pt}{\baselineskip}\selectfont}  
\newcommand{\xiaoerhao}{\fontsize{19.3pt}{\baselineskip}\selectfont} 
\newcommand{\sihao}{\fontsize{14pt}{\baselineskip}\selectfont}      % 字号设置  
\newcommand{\xiaosihao}{\fontsize{12pt}{\baselineskip}\selectfont}  % 字号设置  
\newcommand{\wuhao}{\fontsize{10.5pt}{\baselineskip}\selectfont}    % 字号设置  
\newcommand{\xiaowuhao}{\fontsize{9pt}{\baselineskip}\selectfont}   % 字号设置  
\newcommand{\liuhao}{\fontsize{7.875pt}{\baselineskip}\selectfont}  % 字号设置  
\newcommand{\qihao}{\fontsize{5.25pt}{\baselineskip}\selectfont}    % 字号设置 

\usepackage{diagbox}
\usepackage{multirow}
\boldmath
\XeTeXlinebreaklocale "zh"
\XeTeXlinebreakskip = 0pt plus 1pt minus 0.1pt
\definecolor{cred}{rgb}{0.8,0.8,0.8}
\definecolor{cgreen}{rgb}{0,0.3,0}
\definecolor{cpurple}{rgb}{0.5,0,0.35}
\definecolor{cdocblue}{rgb}{0,0,0.3}
\definecolor{cdark}{rgb}{0.95,1.0,1.0}
\lstset{
	language=[x86masm]Assembler,
	numbers=left,
	numberstyle=\tiny\color{black},
	showspaces=false,
	showstringspaces=false,
	basicstyle=\scriptsize,
	keywordstyle=\color{purple},
	commentstyle=\itshape\color{cgreen},
	stringstyle=\color{blue},
	frame=lines,
	% escapeinside=``,
	extendedchars=true, 
	xleftmargin=1em,
	xrightmargin=1em, 
	backgroundcolor=\color{cred},
	aboveskip=1em,
	breaklines=true,
	tabsize=4
} 

\newfontfamily{\consolas}{Consolas}
\newfontfamily{\monaco}{Monaco}
\setmonofont[Mapping={}]{Consolas}	%英文引号之类的正常显示,相当于设置英文字体
\setsansfont{Consolas} %设置英文字体 Monaco, Consolas,  Fantasque Sans Mono
\setmainfont{Times New Roman}

\setCJKmainfont{华文中宋}


\newcommand{\fic}[1]{\begin{figure}[H]
		\center
		\includegraphics[width=0.8\textwidth]{#1}
	\end{figure}}
	
\newcommand{\sizedfic}[2]{\begin{figure}[H]
		\center
		\includegraphics[width=#1\textwidth]{#2}
	\end{figure}}

\newcommand{\codefile}[1]{\lstinputlisting{#1}}

\newcommand{\interval}{\vspace{0.5em}}

\newcommand{\tablestart}{
	\interval
	\begin{longtable}{p{2cm}p{10cm}}
	\hline}
\newcommand{\tableend}{
	\hline
	\end{longtable}
	\interval}

% 改变段间隔
\setlength{\parskip}{0.2em}
\linespread{1.1}

\usepackage{lastpage}
\usepackage{fancyhdr}
\pagestyle{fancy}
\lhead{\space \qquad \space}
\chead{8086汇编指令集 \qquad}
\rhead{\qquad\thepage/\pageref{LastPage}}
\begin{document}

\tableofcontents

\clearpage

\section{8086汇编指令集}
\subsection{aaa}
\subsection{adc}
\subsection{add}
\subsection{and}
	add指令有如下几种形式:
	\begin{lstlisting}
	add 寄存器, 数据
	add 寄存器, 寄存器
	add 寄存器, 内存单元
	add 内存单元, 寄存器
	\end{lstlisting}

\subsection{call}
\subsection{clc}
\subsection{cld}
	cld指令的功能是将DF标志置0。\par

	DF标志是方向标志位。当DF=0时,串处理指令操作后si,di递增。当DF=1时,串处理指令操作后si,di递减。

\subsection{cli}
	设置IF位为0,屏蔽外部中断。

\subsection{cmc}
\subsection{cmp}
	cmp指令是比较指令,指令格式为:“cmp 操作对象1, 操作对象2”,cmp的功能相当于减法指令,只是不保存结果,然后根据计算结果对标志寄存器进行设置。
	
\subsection{cmps}
\subsection{dec}
\subsection{div}
	div指令是除法指令,对它的介绍如下:
	\begin{itemize}
		\item[1.] 除数:有8位和16位两种,在一个寄存器或内存单元中。
		\item[2.] 被除数:如果除数为8位,那么被除数为16位,默认放在AX中。如果除数为16位,那么被除数为32位,默认在DX和AX中存放,DX存放高16位,AX存放低16位。
		\item[3.] 结果:如果除数为8位,那么AL存储商,AH存储余数。如果除数为16位,那么AX存储商,DX存储余数。
		\item[4.] div指令的格式如下:
		\begin{lstlisting}
	div reg
	div 内存单元
		\end{lstlisting}

		可以看出,div指令的操作对象不可以是段寄存器sreg,只能是寄存器reg。
	\end{itemize}

\subsection{dup}
	dup是一个操作符,和db、dw、dd这些指令配合使用。dup的使用格式如下:
	\begin{itemize}
		\item db 重复的次数 dup (重复的字节型数据)
		\item dw 重复的次数 dup (重复的字型数据)
		\item dd 重复的次数 dup (重复的双字型数据)
	\end{itemize}

	dup的使用如下例所示:
	\begin{lstlisting}
	db 3 dup (0)
	db 3 dup (0, 1, 2)
	db 3 dup ('abc', 'ABC')
	\end{lstlisting}

\subsection{esc}
\subsection{hlt}
	hlt指令使得程序停止运行,处理器进入暂停状态,不执行任何操作,不影响标志寄存器。\par

	如果在hlt指令之前做了cli指令,那么可屏蔽中断不能唤醒cpu。\par

	htl指令只能在实模式或ring0下执行。

\subsection{idiv}
\subsection{imul}
\subsection{inc}
\subsection{int}
\subsection{iret}
\subsection{ja}
\subsection{jb}
\subsection{jcxz}
\subsection{je}
\subsection{jmp}
\subsection{jna}
\subsection{jnb}
\subsection{jne}
\subsection{lgdt}
	lgdt指令用于将内存单元值加载到GDTR,语句如下:
	\begin{lstlisting}
	lgdt [address]
	\end{lstlisting}

\subsection{lock}
\subsection{lods}
	串传送指令lods,功能为:
	\begin{lstlisting}
	(ax)=((ds)*16+(si))
	\end{lstlisting}

\subsection{lodsb}
	串传送指令lodsb,功能为:
	\begin{lstlisting}
	(al)=((ds)*16+(si))
	(si)=(si)+1 如果DF=0
	(si)=(si)-1 如果DF=1
	\end{lstlisting}

\subsection{loop}
\subsection{mov}
	mov指令有如下几种形式:
	\begin{lstlisting}
	mov 寄存器, 数据
	mov 寄存器, 寄存器
	mov 寄存器, 内存单元
	mov 内存单元, 寄存器
	mov 段寄存器, 寄存器
	mov 寄存器, 段寄存器
	mov 段寄存器, 内存单元
	mov 内存单元, 段寄存器
	\end{lstlisting}

\subsection{movsb}
\subsection{movsw}
\subsection{mul}
	mul指令是乘法指令,乘法指令的介绍如下:
	\begin{itemize}
		\item 两个相乘的数:如果两个相乘的数都是8位,那么一个默认放在AH中,另一个放在8位寄存器或内存字节单元中。
			如果两个都是16位,那么一个默认放在AX中,另一个放在16位寄存器或内存字单元中。
		\item 结果:如果是8位乘法,结果默认放在AX中。如果是16位乘法,结果的高位默认放在DX中,低位放在AX中。
	\end{itemize}
	
	mul指令的格式如下:
	\begin{lstlisting}
	mul reg
	mul 内存单元
	\end{lstlisting}
	
\subsection{nop}
\subsection{not}
\subsection{or}
\subsection{push}
\subsection{pop}
\subsection{pushf}
\subsection{rcl}
\subsection{rcr}
\subsection{resb}
	程序中使用到的非初始化数据通常放在bss section中,bss代表未初始化存储空间。\par

	nasm使用resb来定义非初始化数据,相当于命令“db ?”。例子如下:
	\begin{lstlisting}
	resb 64 ; reserve 64 bytes
	\end{lstlisting}

\subsection{rep}
\subsection{repe}
\subsection{repne}
\subsection{ret}
\subsection{retf}
\subsection{rol}
\subsection{ror}
\subsection{sal}
\subsection{sar}
\subsection{sbb}
\subsection{scas}
\subsection{shl}
\subsection{shr}
\subsection{stc}
\subsection{std}
	std指令的功能是将DF标志置1。\par
	DF标志是方向标志位。当DF=0时,串处理指令操作后si,di递增。当DF=1时,串处理指令操作后si,di递减。

\subsection{sti}
	设置IF位为1,可以响应外部中断。

\subsection{stos}
\subsection{stosd}
	串传送指令stosd,功能为:
	\begin{lstlisting}
	((es)*16+(edi))=(eax)
	(si)=(si)+1 如果DF=0
	(si)=(si)-1 如果DF=1
	\end{lstlisting}

\subsection{sub}
	sub指令有如下几种形式:
	\begin{lstlisting}
	sub 寄存器, 数据
	sub 寄存器, 寄存器
	sub 寄存器, 内存单元
	sub 内存单元, 寄存器
	\end{lstlisting}
	
\subsection{test}
\subsection{wait}
\subsection{xchg}
\subsection{xor}

\end{document}
