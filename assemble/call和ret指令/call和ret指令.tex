% !TeX spellcheck = en_US
%% 字体:方正静蕾简体
%%		 方正粗宋
\documentclass[a4paper,left=2.5cm,right=2.5cm,11pt]{article}

\usepackage[utf8]{inputenc}
\usepackage{fontspec}
\usepackage{cite}
\usepackage{xeCJK}
\usepackage{indentfirst}
\usepackage{titlesec}
\usepackage{longtable}
\usepackage{graphicx}
\usepackage{float}
\usepackage{rotating}
\usepackage{subfigure}
\usepackage{tabu}
\usepackage{amsmath}
\usepackage{setspace}
\usepackage{amsfonts}
\usepackage{appendix}
\usepackage{listings}
\usepackage{xcolor}
\usepackage{geometry}
\setcounter{secnumdepth}{4}
\usepackage{mhchem}
\usepackage{multirow}
\usepackage{extarrows}
\usepackage{hyperref}
\titleformat*{\section}{\LARGE}
\renewcommand\refname{参考文献}
\renewcommand{\abstractname}{\sihao \cjkfzcs 摘{  }要}
%\titleformat{\chapter}{\centering\bfseries\huge\wryh}{}{0.7em}{}{}
%\titleformat{\section}{\LARGE\bf}{\thesection}{1em}{}{}
\titleformat{\subsection}{\Large\bfseries}{\thesubsection}{1em}{}{}
\titleformat{\subsubsection}{\large\bfseries}{\thesubsubsection}{1em}{}{}
\renewcommand{\contentsname}{{\cjkfzcs \centerline{目{  } 录}}}
\setCJKfamilyfont{cjkhwxk}{STXingkai}
\setCJKfamilyfont{cjkfzcs}{STSongti-SC-Regular}
% \setCJKfamilyfont{cjkhwxk}{华文行楷}
% \setCJKfamilyfont{cjkfzcs}{方正粗宋简体}
\newcommand*{\cjkfzcs}{\CJKfamily{cjkfzcs}}
\newcommand*{\cjkhwxk}{\CJKfamily{cjkhwxk}}
\newfontfamily\wryh{Microsoft YaHei}
\newfontfamily\hwzs{STZhongsong}
\newfontfamily\hwst{STSong}
\newfontfamily\hwfs{STFangsong}
\newfontfamily\jljt{MicrosoftYaHei}
\newfontfamily\hwxk{STXingkai}
% \newfontfamily\hwzs{华文中宋}
% \newfontfamily\hwst{华文宋体}
% \newfontfamily\hwfs{华文仿宋}
% \newfontfamily\jljt{方正静蕾简体}
% \newfontfamily\hwxk{华文行楷}
\newcommand{\verylarge}{\fontsize{60pt}{\baselineskip}\selectfont}  
\newcommand{\chuhao}{\fontsize{44.9pt}{\baselineskip}\selectfont}  
\newcommand{\xiaochu}{\fontsize{38.5pt}{\baselineskip}\selectfont}  
\newcommand{\yihao}{\fontsize{27.8pt}{\baselineskip}\selectfont}  
\newcommand{\xiaoyi}{\fontsize{25.7pt}{\baselineskip}\selectfont}  
\newcommand{\erhao}{\fontsize{23.5pt}{\baselineskip}\selectfont}  
\newcommand{\xiaoerhao}{\fontsize{19.3pt}{\baselineskip}\selectfont} 
\newcommand{\sihao}{\fontsize{14pt}{\baselineskip}\selectfont}      % 字号设置  
\newcommand{\xiaosihao}{\fontsize{12pt}{\baselineskip}\selectfont}  % 字号设置  
\newcommand{\wuhao}{\fontsize{10.5pt}{\baselineskip}\selectfont}    % 字号设置  
\newcommand{\xiaowuhao}{\fontsize{9pt}{\baselineskip}\selectfont}   % 字号设置  
\newcommand{\liuhao}{\fontsize{7.875pt}{\baselineskip}\selectfont}  % 字号设置  
\newcommand{\qihao}{\fontsize{5.25pt}{\baselineskip}\selectfont}    % 字号设置 

\usepackage{diagbox}
\usepackage{multirow}
\boldmath
\XeTeXlinebreaklocale "zh"
\XeTeXlinebreakskip = 0pt plus 1pt minus 0.1pt
\definecolor{cred}{rgb}{0.8,0.8,0.8}
\definecolor{cgreen}{rgb}{0,0.3,0}
\definecolor{cpurple}{rgb}{0.5,0,0.35}
\definecolor{cdocblue}{rgb}{0,0,0.3}
\definecolor{cdark}{rgb}{0.95,1.0,1.0}
\lstset{
	language=[x86masm]Assembler,
	numbers=left,
	numberstyle=\tiny\color{black},
	showspaces=false,
	showstringspaces=false,
	basicstyle=\scriptsize,
	keywordstyle=\color{purple},
	commentstyle=\itshape\color{cgreen},
	stringstyle=\color{blue},
	frame=lines,
	% escapeinside=``,
	extendedchars=true, 
	xleftmargin=1em,
	xrightmargin=1em, 
	backgroundcolor=\color{cred},
	aboveskip=1em,
	breaklines=true,
	tabsize=4
} 

\newfontfamily{\consolas}{Consolas}
\newfontfamily{\monaco}{Monaco}
\setmonofont[Mapping={}]{Consolas}	%英文引号之类的正常显示,相当于设置英文字体
\setsansfont{Consolas} %设置英文字体 Monaco, Consolas,  Fantasque Sans Mono
\setmainfont{Times New Roman}

\setCJKmainfont{华文中宋}


\newcommand{\fic}[1]{\begin{figure}[H]
		\center
		\includegraphics[width=0.8\textwidth]{#1}
	\end{figure}}
	
\newcommand{\sizedfic}[2]{\begin{figure}[H]
		\center
		\includegraphics[width=#1\textwidth]{#2}
	\end{figure}}

\newcommand{\codefile}[1]{\lstinputlisting{#1}}

\newcommand{\interval}{\vspace{0.5em}}

\newcommand{\tablestart}{
	\interval
	\begin{longtable}{p{2cm}p{10cm}}
	\hline}
\newcommand{\tableend}{
	\hline
	\end{longtable}
	\interval}

% 改变段间隔
\setlength{\parskip}{0.2em}
\linespread{1.1}

\usepackage{lastpage}
\usepackage{fancyhdr}
\pagestyle{fancy}
\lhead{\space \qquad \space}
\chead{call和ret指令 \qquad}
\rhead{\qquad\thepage/\pageref{LastPage}}
\begin{document}

\tableofcontents

\clearpage

\section{ret和retf}
	ret指令用栈中的数据,修改IP的内容,从而实现近转移。\par

	retf指令用栈中的数据,修改CS和IP的内容,从而实现远转移。\par

	CPU执行ret指令时进行以下两个步骤:
	\begin{lstlisting}
	(IP)=((ss)*16+(sp))
	(sp)=(sp)+2
	\end{lstlisting}

	所以ret指令相当于pop IP。\par

	CPU执行retf指令时进行以下四个步骤:
	\begin{lstlisting}
	(IP)=((ss)*16+(sp))
	(sp)=(sp)+2
	(CS)=((ss)*16+(sp))
	(sp)=(sp)+2
	\end{lstlisting}

	所以retf指令相当于pop IP;pop CS。

\section{call指令}
	CPU执行call指令时进行以下两个步骤:
	\begin{lstlisting}
	将当前的IP或CS和IP压入栈中
	转移
	\end{lstlisting}

\subsection{根据位移进行转移的call指令}
	“call 标号”的功能为:
	\begin{lstlisting}
	(sp)=(sp)-2
	((ss)*16+(sp))=(IP)
	(IP)=(IP)+16位位移
	\end{lstlisting}

	其中16位位移=“标号”处的地址-call指令后的第一个字节的地址。\par

	CPU执行“call 标号”时,相当于执行:
	\begin{lstlisting}
	push IP
	jmp near ptr 标号
	\end{lstlisting}

\subsection{转移的目的地址在指令中的call指令}
	“call far ptr 标号”实现的是段间转移,相当于进行如下操作:
	\begin{lstlisting}
	push CS
	push IP
	jmp far ptr 标号
	\end{lstlisting}

\subsection{转移的目的地址在寄存器中的call指令}
	“call 16位寄存器”相当于进行如下操作:
	\begin{lstlisting}
	push IP
	jmp 16位寄存器
	\end{lstlisting}

\subsection{转移的目的地址在内存中的call指令}
	“call word ptr 内存单元地址”相当于进行如下操作:
	\begin{lstlisting}
	push IP
	jmp word ptr 内存单元地址
	\end{lstlisting}

	“call dword ptr 内存单元地址”相当于进行如下操作:
	\begin{lstlisting}
	push CS
	push IP
	jmp dword ptr 内存单元地址
	\end{lstlisting}

\section{call和ret指令的配合使用}
	call和ret可以实现子程序的机制,子程序的框架如下:
	\begin{lstlisting}
	assume cs:code
	code segment
	main:
		; ...
		call sub1
		; ...
		mov ax, 4c00h
		int 21h
	sub1:
		; ...
		call sub2
		; ...
		ret
	sub2:
		; ...
		ret
	code ends
	end main
	\end{lstlisting}

\section{mul指令}
	mul指令是乘法指令,乘法指令的介绍如下:
	\begin{itemize}
		\item 两个相乘的数:如果两个相乘的数都是8位,那么一个默认放在AH中,另一个放在8位寄存器或内存字节单元中。
			如果两个都是16位,那么一个默认放在AX中,另一个放在16位寄存器或内存字单元中。
		\item 结果:如果是8位乘法,结果默认放在AX中。如果是16位乘法,结果的高位默认放在DX中,低位放在AX中。
	\end{itemize}
	
	mul指令的格式如下:
	\begin{lstlisting}
	mul reg
	mul 内存单元
	\end{lstlisting}

\end{document}
