% !TeX spellcheck = en_US
%% 字体:方正静蕾简体
%%		 方正粗宋
\documentclass[a4paper,left=2.5cm,right=2.5cm,11pt]{article}

\usepackage[utf8]{inputenc}
\usepackage{fontspec}
\usepackage{cite}
\usepackage{xeCJK}
\usepackage{indentfirst}
\usepackage{titlesec}
\usepackage{longtable}
\usepackage{graphicx}
\usepackage{float}
\usepackage{rotating}
\usepackage{subfigure}
\usepackage{tabu}
\usepackage{amsmath}
\usepackage{setspace}
\usepackage{amsfonts}
\usepackage{appendix}
\usepackage{listings}
\usepackage{xcolor}
\usepackage{geometry}
\setcounter{secnumdepth}{4}
\usepackage{mhchem}
\usepackage{multirow}
\usepackage{extarrows}
\usepackage{hyperref}
\titleformat*{\section}{\LARGE}
\renewcommand\refname{参考文献}
\renewcommand{\abstractname}{\sihao \cjkfzcs 摘{  }要}
%\titleformat{\chapter}{\centering\bfseries\huge\wryh}{}{0.7em}{}{}
%\titleformat{\section}{\LARGE\bf}{\thesection}{1em}{}{}
\titleformat{\subsection}{\Large\bfseries}{\thesubsection}{1em}{}{}
\titleformat{\subsubsection}{\large\bfseries}{\thesubsubsection}{1em}{}{}
\renewcommand{\contentsname}{{\cjkfzcs \centerline{目{  } 录}}}
\setCJKfamilyfont{cjkhwxk}{STXingkai}
\setCJKfamilyfont{cjkfzcs}{STSongti-SC-Regular}
% \setCJKfamilyfont{cjkhwxk}{华文行楷}
% \setCJKfamilyfont{cjkfzcs}{方正粗宋简体}
\newcommand*{\cjkfzcs}{\CJKfamily{cjkfzcs}}
\newcommand*{\cjkhwxk}{\CJKfamily{cjkhwxk}}
\newfontfamily\wryh{Microsoft YaHei}
\newfontfamily\hwzs{STZhongsong}
\newfontfamily\hwst{STSong}
\newfontfamily\hwfs{STFangsong}
\newfontfamily\jljt{MicrosoftYaHei}
\newfontfamily\hwxk{STXingkai}
% \newfontfamily\hwzs{华文中宋}
% \newfontfamily\hwst{华文宋体}
% \newfontfamily\hwfs{华文仿宋}
% \newfontfamily\jljt{方正静蕾简体}
% \newfontfamily\hwxk{华文行楷}
\newcommand{\verylarge}{\fontsize{60pt}{\baselineskip}\selectfont}  
\newcommand{\chuhao}{\fontsize{44.9pt}{\baselineskip}\selectfont}  
\newcommand{\xiaochu}{\fontsize{38.5pt}{\baselineskip}\selectfont}  
\newcommand{\yihao}{\fontsize{27.8pt}{\baselineskip}\selectfont}  
\newcommand{\xiaoyi}{\fontsize{25.7pt}{\baselineskip}\selectfont}  
\newcommand{\erhao}{\fontsize{23.5pt}{\baselineskip}\selectfont}  
\newcommand{\xiaoerhao}{\fontsize{19.3pt}{\baselineskip}\selectfont} 
\newcommand{\sihao}{\fontsize{14pt}{\baselineskip}\selectfont}      % 字号设置  
\newcommand{\xiaosihao}{\fontsize{12pt}{\baselineskip}\selectfont}  % 字号设置  
\newcommand{\wuhao}{\fontsize{10.5pt}{\baselineskip}\selectfont}    % 字号设置  
\newcommand{\xiaowuhao}{\fontsize{9pt}{\baselineskip}\selectfont}   % 字号设置  
\newcommand{\liuhao}{\fontsize{7.875pt}{\baselineskip}\selectfont}  % 字号设置  
\newcommand{\qihao}{\fontsize{5.25pt}{\baselineskip}\selectfont}    % 字号设置 

\usepackage{diagbox}
\usepackage{multirow}
\boldmath
\XeTeXlinebreaklocale "zh"
\XeTeXlinebreakskip = 0pt plus 1pt minus 0.1pt
\definecolor{cred}{rgb}{0.8,0.8,0.8}
\definecolor{cgreen}{rgb}{0,0.3,0}
\definecolor{cpurple}{rgb}{0.5,0,0.35}
\definecolor{cdocblue}{rgb}{0,0,0.3}
\definecolor{cdark}{rgb}{0.95,1.0,1.0}
\lstset{
	language=[x86masm]Assembler,
	numbers=left,
	numberstyle=\tiny\color{black},
	showspaces=false,
	showstringspaces=false,
	basicstyle=\scriptsize,
	keywordstyle=\color{purple},
	commentstyle=\itshape\color{cgreen},
	stringstyle=\color{blue},
	frame=lines,
	% escapeinside=``,
	extendedchars=true, 
	xleftmargin=1em,
	xrightmargin=1em, 
	backgroundcolor=\color{cred},
	aboveskip=1em,
	breaklines=true,
	tabsize=4
} 

\newfontfamily{\consolas}{Consolas}
\newfontfamily{\monaco}{Monaco}
\setmonofont[Mapping={}]{Consolas}	%英文引号之类的正常显示,相当于设置英文字体
\setsansfont{Consolas} %设置英文字体 Monaco, Consolas,  Fantasque Sans Mono
\setmainfont{Times New Roman}

\setCJKmainfont{华文中宋}


\newcommand{\fic}[1]{\begin{figure}[H]
		\center
		\includegraphics[width=0.8\textwidth]{#1}
	\end{figure}}
	
\newcommand{\sizedfic}[2]{\begin{figure}[H]
		\center
		\includegraphics[width=#1\textwidth]{#2}
	\end{figure}}

\newcommand{\codefile}[1]{\lstinputlisting{#1}}

% 改变段间隔
\setlength{\parskip}{0.2em}
\linespread{1.1}

\usepackage{lastpage}
\usepackage{fancyhdr}
\pagestyle{fancy}
\lhead{\space \qquad \space}
\chead{伪指令集 \qquad}
\rhead{\qquad\thepage/\pageref{LastPage}}
\begin{document}

\tableofcontents

\clearpage

\section{伪指令集}

\subsection{ALIGN}
	通过ALIGN伪指令设置一个域或变量的地址对齐方式,格式如下:
	\begin{lstlisting}
	ALIGN datatype
	\end{lstlisting}

	如果想让myVar对齐在双字地址边界上,如下操作:
	\begin{lstlisting}
	ALIGN DWORD
	myvar DWORD ?
	\end{lstlisting}

	以下是数据类型的地址对齐方式:
	\fic{1.png}

\subsection{LENGTHOF}
	LENGTHOF操作符返回数组元素的数目,例子如下:
	\begin{lstlisting}
	.data
	array WORD 30 DUP(0)

	.code
	mov ax, LENGTHOF array
	\end{lstlisting}

\subsection{OFFSET}
	使用OFFSET操作符可以获取结构变量中域的偏移地址,例子如下:
	\begin{lstlisting}
	Employee STRUCT
		IdNum BYTE "000000000"
		LastName BYTE 30 DUP(0)
		ALIGN WORD
		Years WORD 0
		ALIGN DWORD
		SalaryHistory DWORD 0, 0, 0, 0
	Employee ENDS

	mov edx, OFFSET worker.LastName	
	\end{lstlisting}

\subsection{SIZEOF}
	SIZEOF操作符返回LENGTHOF和TYPE的乘积,例子如下:
	\begin{lstlisting}
	Employee STRUCT
		IdNum BYTE "000000000"
		LastName BYTE 30 DUP(0)
		ALIGN WORD
		Years WORD 0
		ALIGN DWORD
		SalaryHistory DWORD 0, 0, 0, 0
	Employee ENDS

	.data
	worker Employee <>
	workers Employee 30 DUP(<>)

	.code
	mov ax, SIZEOF worker
	mov ax, SIZEOF workers
	\end{lstlisting}

\subsection{TYPE}
	TYPE操作符返回标志符的存储类型占用的字节数,例子如下:
	\begin{lstlisting}
	Employee STRUCT
		IdNum BYTE "000000000"
		LastName BYTE 30 DUP(0)
		ALIGN WORD
		Years WORD 0
		ALIGN DWORD
		SalaryHistory DWORD 0, 0, 0, 0
	Employee ENDS

	.data
	worker Employee <>

	.code
	mov ax, TYPE Employee
	mov ax, TYPE worker
	\end{lstlisting}

\subsection{定义结构体的伪指令}
	结构使用STRUCT和ENDS伪指令定义,格式如下:
	\begin{lstlisting}
	名字 STRUCT
		域的声明
	名字 ENDS
	\end{lstlisting}

\subsection{定义联合的伪指令}
	结构的每个域都有一个相对于结构第一个字节的偏移值,而联合中的所有域都是从同一偏移地址开始的,所以联合的大小等于其中最长的域的长度。
	联合使用UNION和ENDS伪指令声明。
	当联合不是某个结构的成员时,定义格式如下所示:
	\begin{lstlisting}
	unionname UNION
		union-fields
	unionname ENDS
	\end{lstlisting}

	当联合嵌套在结构中时,则格式如下:
	\begin{lstlisting}
	structname STRUCT
		; ...
		UNION unionname
			union-fields
		ENDS
	structname ENDS
	\end{lstlisting}

\end{document}
