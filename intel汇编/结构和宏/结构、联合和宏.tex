% !TeX spellcheck = en_US
%% 字体:方正静蕾简体
%%		 方正粗宋
\documentclass[a4paper,left=2.5cm,right=2.5cm,11pt]{article}

\usepackage[utf8]{inputenc}
\usepackage{fontspec}
\usepackage{cite}
\usepackage{xeCJK}
\usepackage{indentfirst}
\usepackage{titlesec}
\usepackage{longtable}
\usepackage{graphicx}
\usepackage{float}
\usepackage{rotating}
\usepackage{subfigure}
\usepackage{tabu}
\usepackage{amsmath}
\usepackage{setspace}
\usepackage{amsfonts}
\usepackage{appendix}
\usepackage{listings}
\usepackage{xcolor}
\usepackage{geometry}
\setcounter{secnumdepth}{4}
\usepackage{mhchem}
\usepackage{multirow}
\usepackage{extarrows}
\usepackage{hyperref}
\titleformat*{\section}{\LARGE}
\renewcommand\refname{参考文献}
\renewcommand{\abstractname}{\sihao \cjkfzcs 摘{  }要}
%\titleformat{\chapter}{\centering\bfseries\huge\wryh}{}{0.7em}{}{}
%\titleformat{\section}{\LARGE\bf}{\thesection}{1em}{}{}
\titleformat{\subsection}{\Large\bfseries}{\thesubsection}{1em}{}{}
\titleformat{\subsubsection}{\large\bfseries}{\thesubsubsection}{1em}{}{}
\renewcommand{\contentsname}{{\cjkfzcs \centerline{目{  } 录}}}
\setCJKfamilyfont{cjkhwxk}{STXingkai}
\setCJKfamilyfont{cjkfzcs}{STSongti-SC-Regular}
% \setCJKfamilyfont{cjkhwxk}{华文行楷}
% \setCJKfamilyfont{cjkfzcs}{方正粗宋简体}
\newcommand*{\cjkfzcs}{\CJKfamily{cjkfzcs}}
\newcommand*{\cjkhwxk}{\CJKfamily{cjkhwxk}}
\newfontfamily\wryh{Microsoft YaHei}
\newfontfamily\hwzs{STZhongsong}
\newfontfamily\hwst{STSong}
\newfontfamily\hwfs{STFangsong}
\newfontfamily\jljt{MicrosoftYaHei}
\newfontfamily\hwxk{STXingkai}
% \newfontfamily\hwzs{华文中宋}
% \newfontfamily\hwst{华文宋体}
% \newfontfamily\hwfs{华文仿宋}
% \newfontfamily\jljt{方正静蕾简体}
% \newfontfamily\hwxk{华文行楷}
\newcommand{\verylarge}{\fontsize{60pt}{\baselineskip}\selectfont}  
\newcommand{\chuhao}{\fontsize{44.9pt}{\baselineskip}\selectfont}  
\newcommand{\xiaochu}{\fontsize{38.5pt}{\baselineskip}\selectfont}  
\newcommand{\yihao}{\fontsize{27.8pt}{\baselineskip}\selectfont}  
\newcommand{\xiaoyi}{\fontsize{25.7pt}{\baselineskip}\selectfont}  
\newcommand{\erhao}{\fontsize{23.5pt}{\baselineskip}\selectfont}  
\newcommand{\xiaoerhao}{\fontsize{19.3pt}{\baselineskip}\selectfont} 
\newcommand{\sihao}{\fontsize{14pt}{\baselineskip}\selectfont}      % 字号设置  
\newcommand{\xiaosihao}{\fontsize{12pt}{\baselineskip}\selectfont}  % 字号设置  
\newcommand{\wuhao}{\fontsize{10.5pt}{\baselineskip}\selectfont}    % 字号设置  
\newcommand{\xiaowuhao}{\fontsize{9pt}{\baselineskip}\selectfont}   % 字号设置  
\newcommand{\liuhao}{\fontsize{7.875pt}{\baselineskip}\selectfont}  % 字号设置  
\newcommand{\qihao}{\fontsize{5.25pt}{\baselineskip}\selectfont}    % 字号设置 

\usepackage{diagbox}
\usepackage{multirow}
\boldmath
\XeTeXlinebreaklocale "zh"
\XeTeXlinebreakskip = 0pt plus 1pt minus 0.1pt
\definecolor{cred}{rgb}{0.8,0.8,0.8}
\definecolor{cgreen}{rgb}{0,0.3,0}
\definecolor{cpurple}{rgb}{0.5,0,0.35}
\definecolor{cdocblue}{rgb}{0,0,0.3}
\definecolor{cdark}{rgb}{0.95,1.0,1.0}
\lstset{
	language=[x86masm]Assembler,
	numbers=left,
	numberstyle=\tiny\color{black},
	showspaces=false,
	showstringspaces=false,
	basicstyle=\scriptsize,
	keywordstyle=\color{purple},
	commentstyle=\itshape\color{cgreen},
	stringstyle=\color{blue},
	frame=lines,
	% escapeinside=``,
	extendedchars=true, 
	xleftmargin=1em,
	xrightmargin=1em, 
	backgroundcolor=\color{cred},
	aboveskip=1em,
	breaklines=true,
	tabsize=4
} 

\newfontfamily{\consolas}{Consolas}
\newfontfamily{\monaco}{Monaco}
\setmonofont[Mapping={}]{Consolas}	%英文引号之类的正常显示,相当于设置英文字体
\setsansfont{Consolas} %设置英文字体 Monaco, Consolas,  Fantasque Sans Mono
\setmainfont{Times New Roman}

\setCJKmainfont{华文中宋}


\newcommand{\fic}[1]{\begin{figure}[H]
		\center
		\includegraphics[width=0.8\textwidth]{#1}
	\end{figure}}
	
\newcommand{\sizedfic}[2]{\begin{figure}[H]
		\center
		\includegraphics[width=#1\textwidth]{#2}
	\end{figure}}

\newcommand{\codefile}[1]{\lstinputlisting{#1}}

% 改变段间隔
\setlength{\parskip}{0.2em}
\linespread{1.1}

\usepackage{lastpage}
\usepackage{fancyhdr}
\pagestyle{fancy}
\lhead{\space \qquad \space}
\chead{结构、联合和宏 \qquad}
\rhead{\qquad\thepage/\pageref{LastPage}}
\begin{document}

\tableofcontents

\clearpage

\section{结构}
\subsection{定义结构}
	结构使用STRUCT和ENDS伪指令定义,格式如下:
	\begin{lstlisting}
	名字 STRUCT
		域的声明
	名字 ENDS
	\end{lstlisting}	

	如果结构的域有初始值,在定义结构变量时这些初始值就成了结构变量域的默认值。
	结构中可使用多种类型的初始值,有字符串、整数、数组。初始化数组时,可以使用DUP操作符初始化数组元素。还可以使用“?”,用于代表域未定义。
	以下是一个例子:
	\begin{lstlisting}
	Employee STRUCT
		IdNum BYTE "00000000"
		LastName BYTE 30 DUP(0)
		FirstName BYTE ?
		SalaryHistory DWORD 0, 0, 0, 0
	Employee ENDS
	\end{lstlisting}

\subsubsection{结构中域的对齐}
	通过ALIGN伪指令设置一个域或变量的地址对齐方式,格式如下:
	\begin{lstlisting}
	ALIGN datatype
	\end{lstlisting}

	如果想让myVar对齐在双字地址边界上,如下操作:
	\begin{lstlisting}
	ALIGN DWORD
	myvar DWORD ?
	\end{lstlisting}

	以下是数据类型的地址对齐方式:
	\fic{1.png}

	在结构中使用ALIGN伪指令使得成员years对齐在字边界上、成员SalaryHistory对齐在双字边界上,如下所示:
	\begin{lstlisting}
	Employee STRUCT
		IdNum BYTE "00000000"
		LastName BYTE 30 DUP(0)
		ALIGN WORD
		Years WORD 0
		ALIGN DWORD
		SalaryHistory DWORD 0,0,0,0
	Employee ENDS
	\end{lstlisting}

\subsection{声明结构变量}
	声明结构变量的格式如下:
	\begin{lstlisting}
	; identifier是标志符名
	; structureType是已经定义的结构
	identifier structureType <initializer-list>
	\end{lstlisting}

	初始值列表中的值按照从左到右的顺序对结构的相应成员依次赋值,可以插入逗号作为占位符跳过对结构中某些域的初始化,
	如下所示:
	\begin{lstlisting}
	point1 COORD <5,10>
	person3 Employee <,"dJones">
	\end{lstlisting}

	对于结构中数组类型的域,可以使用DUP操作符初始化某些或全部数组元素。如果初始化值比域短,那么剩余位置将用0填充。例子如下:
	\begin{lstlisting}
	person4 Employee <,,,2 DUP(20000)>
	\end{lstlisting}

	对于结构中字符串类型的域,如果初始化值比域短,那么剩余位置以空格填充,字符串的末尾不会自动插入空字符。

\subsection{结构中域成员的引用}
	汇编中对结构成员的引用和C语言类似,例子如下:
	\begin{lstlisting}
	Employee STRUCT
		IdNum BYTE "000000000"
		LastName BYTE 30 DUP(0)
		ALIGN WORD
		Years WORD 0
		ALIGN DWORD
		SalaryHistory DWORD 0, 0, 0, 0
	Employee ENDS

	.data
	worker Employee <>

	.code
	mov dx, worker.Years
	mov worker.SalaryHistory, 20000
	mov [woker.SalaryHistory+4], 30000
	\end{lstlisting}

\subsection{间接寻址结构数据}
	使用间接操作数饮用结构时要求使用PTR结构符,如下所示:
	\begin{lstlisting}
	mov esi, OFFSET worker
	mov ax, (Employee PTR [esi]).Years
	\end{lstlisting}

\subsection{变址寻址结构数据}
	可以使用变址操作数访问结构数组。假设department是一个包含5歌Employee对象的数组,可以使用下面语句访问其索引位置1处的Employee对象的Years域:
	\begin{lstlisting}
	.data
	department Employee 5 DUP(<>)

	.code
	mov esi, TYPE Employee
	mov department[esi].Years, 4
	\end{lstlisting}

\section{联合}
	结构的每个域都有一个相对于结构第一个字节的偏移值,而联合中的所有域都是从同一偏移地址开始的,所以联合的大小等于其中最长的域的长度。
	联合使用UNION和ENDS伪指令声明。
	当联合不是某个结构的成员时,定义格式如下所示:
	\begin{lstlisting}
	unionname UNION
		union-fields
	unionname ENDS
	\end{lstlisting}

	当联合嵌套在结构中时,则格式如下:
	\begin{lstlisting}
	structname STRUCT
		; ...
		UNION unionname
			union-fields
		ENDS
	structname ENDS
	\end{lstlisting}

	联合中所有域只能有一个初始化值,如下所示:
	\begin{lstlisting}
	Integer UNION
		D DWORD 1
		W WORD 5
		B BYTE 8
	Integer ENDS

	.data
	myInt Integer<>
	\end{lstlisting}

	myInt.D、myInt.W和myInt.B都等于1,而忽略了W和B的初始值。\par

	除了所有域只能有一个初始化值这一点之外,联合的语法和结构的语法完全相同。\par

	联合因为可以使用不同大小的操作数,所以在某些场合下具有灵活性,如下所示:
	\begin{lstlisting}
	.data
	val3 Integer <>

	.code
	mov val3.B, al
	mov val3.W, ax
	mov val3.D, eax
	\end{lstlisting}

	使用联合时同一时刻只能使用联合中的一种数据类型,以免造成数据的冲突。

\section{宏}
\subsection{宏的定义}
	可以使用MACRO和ENDM伪指令在源程序的任意位置定义宏,格式如下:
	\begin{lstlisting}
	macroname MACRO parameter-1, parameter-2...
		statement-list
	ENDM
	\end{lstlisting}

	MACRO和ENDM之间的语句只有在宏被调用的时候才会汇编。
	宏的定义中可以有任意多的参数,参数之间以逗号分隔。\par

	宏的参数没有类型,用于存放调用者传递给宏的文本参数,例子如下:
	\begin{lstlisting}
	mPutchar MACRO char
		push eax
		mov al, char
		call WriteChar
		pop eax
	ENDM
	\end{lstlisting}

\subsection{宏的调用}
	调用宏的格式如下:
	\begin{lstlisting}
	宏名称	参数1, 参数2, ...
	\end{lstlisting}

	实际参数的顺序必须与宏定义中的形式参数顺序相同,但实际参数的数目不一定非要与宏定义中形式参数的个数完全一致。
	如果传递的参数个数多于定义中形式参数的个数,汇编器会产生一个警告。
	如果传递的参数个数少于定义中形式参数的个数,那么未传递的参数为空。\par

	调用mPutchar的语句如下:
	\begin{lstlisting}
	mPutchar 'A'
	\end{lstlisting}

\subsection{宏的其他特性}
\subsubsection{必须参数}
	可以使用REQ修饰符来指定一个宏参数是必须的,如下所示:
	\begin{lstlisting}
	mPutchar MACRO char:REQ
		push eax
		mov al, char
		call WriteChar
		pop eax
	ENDM
	\end{lstlisting}

\subsubsection{宏的注释}
	宏定义内的普通注释行在每次宏展开的时候都会出现,如果不想让注释在宏展开的时候出现,需要在注释前使用双分号,如下所示:
	\begin{lstlisting}
	mPutchar MACRO char:REQ
		push eax		;; 提示:字符只能包含8个数据位
		mov al, char
		call WriteChar
		pop eax
	ENDM
	\end{lstlisting}

\subsection{本书附带的宏库}
	通过如下代码可以引用宏库:
	\begin{lstlisting}
	INCLUDE Macros.inc
	\end{lstlisting}

	宏库中的宏如下所示:
	\fic{2.png}

\end{document}
