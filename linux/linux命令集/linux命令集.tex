% !TeX spellcheck = en_US
%% 字体:方正静蕾简体
%%		 方正粗宋
\documentclass[a4paper,left=2.5cm,right=2.5cm,11pt]{article}

\usepackage[utf8]{inputenc}
\usepackage{fontspec}
\usepackage{cite}
\usepackage{xeCJK}
\usepackage{indentfirst}
\usepackage{titlesec}
\usepackage{longtable}
\usepackage{graphicx}
\usepackage{float}
\usepackage{rotating}
\usepackage{subfigure}
\usepackage{tabu}
\usepackage{amsmath}
\usepackage{setspace}
\usepackage{amsfonts}
\usepackage{appendix}
\usepackage{listings}
\usepackage{xcolor}
\usepackage{geometry}
\setcounter{secnumdepth}{4}
\usepackage{mhchem}
\usepackage{multirow}
\usepackage{extarrows}
\usepackage{hyperref}
\titleformat*{\section}{\LARGE}
\renewcommand\refname{参考文献}
\renewcommand{\abstractname}{\sihao \cjkfzcs 摘{  }要}
%\titleformat{\chapter}{\centering\bfseries\huge\wryh}{}{0.7em}{}{}
%\titleformat{\section}{\LARGE\bf}{\thesection}{1em}{}{}
\titleformat{\subsection}{\Large\bfseries}{\thesubsection}{1em}{}{}
\titleformat{\subsubsection}{\large\bfseries}{\thesubsubsection}{1em}{}{}
\renewcommand{\contentsname}{{\cjkfzcs \centerline{目{  } 录}}}
\setCJKfamilyfont{cjkhwxk}{STXingkai}
\setCJKfamilyfont{cjkfzcs}{STSongti-SC-Regular}
% \setCJKfamilyfont{cjkhwxk}{华文行楷}
% \setCJKfamilyfont{cjkfzcs}{方正粗宋简体}
\newcommand*{\cjkfzcs}{\CJKfamily{cjkfzcs}}
\newcommand*{\cjkhwxk}{\CJKfamily{cjkhwxk}}
\newfontfamily\wryh{Microsoft YaHei}
\newfontfamily\hwzs{STZhongsong}  
\newfontfamily\hwst{STSong}
\newfontfamily\hwfs{STFangsong}
\newfontfamily\jljt{MicrosoftYaHei}
\newfontfamily\hwxk{STXingkai}
% \newfontfamily\hwzs{华文中宋}
% \newfontfamily\hwst{华文宋体}
% \newfontfamily\hwfs{华文仿宋}
% \newfontfamily\jljt{方正静蕾简体}
% \newfontfamily\hwxk{华文行楷}
\newcommand{\verylarge}{\fontsize{60pt}{\baselineskip}\selectfont}  
\newcommand{\chuhao}{\fontsize{44.9pt}{\baselineskip}\selectfont}  
\newcommand{\xiaochu}{\fontsize{38.5pt}{\baselineskip}\selectfont}  
\newcommand{\yihao}{\fontsize{27.8pt}{\baselineskip}\selectfont}  
\newcommand{\xiaoyi}{\fontsize{25.7pt}{\baselineskip}\selectfont}  
\newcommand{\erhao}{\fontsize{23.5pt}{\baselineskip}\selectfont}  
\newcommand{\xiaoerhao}{\fontsize{19.3pt}{\baselineskip}\selectfont} 
\newcommand{\sihao}{\fontsize{14pt}{\baselineskip}\selectfont}      % 字号设置  
\newcommand{\xiaosihao}{\fontsize{12pt}{\baselineskip}\selectfont}  % 字号设置  
\newcommand{\wuhao}{\fontsize{10.5pt}{\baselineskip}\selectfont}    % 字号设置  
\newcommand{\xiaowuhao}{\fontsize{9pt}{\baselineskip}\selectfont}   % 字号设置  
\newcommand{\liuhao}{\fontsize{7.875pt}{\baselineskip}\selectfont}  % 字号设置  
\newcommand{\qihao}{\fontsize{5.25pt}{\baselineskip}\selectfont}    % 字号设置 

\usepackage{diagbox}
\usepackage{multirow}
\boldmath
\XeTeXlinebreaklocale "zh"
\XeTeXlinebreakskip = 0pt plus 1pt minus 0.1pt
\definecolor{cred}{rgb}{0.8,0.8,0.8}
\definecolor{cgreen}{rgb}{0,0.3,0}
\definecolor{cpurple}{rgb}{0.5,0,0.35}
\definecolor{cdocblue}{rgb}{0,0,0.3}
\definecolor{cdark}{rgb}{0.95,1.0,1.0}
\lstset{
	language=bash,
	numbers=left,
	numberstyle=\tiny\color{white},
	showspaces=false,
	showstringspaces=false,
	basicstyle=\scriptsize,
	keywordstyle=\color{purple},
	commentstyle=\color{cgreen},
	stringstyle=\color{blue},
	frame=lines,
	% escapeinside=``,
	extendedchars=true, 
	xleftmargin=0em,
	xrightmargin=0em, 
	backgroundcolor=\color{cred},
	aboveskip=1em,
	breaklines=true,
	tabsize=4
} 

\newfontfamily{\consolas}{Consolas}
\newfontfamily{\monaco}{Monaco}
\setmonofont[Mapping={}]{Consolas}	%英文引号之类的正常显示,相当于设置英文字体
\setsansfont{Consolas} %设置英文字体 Monaco, Consolas,  Fantasque Sans Mono
\setmainfont{Times New Roman}

\setCJKmainfont{华文中宋}


\newcommand{\fic}[1]{\begin{figure}[H]
		\center
		\includegraphics[width=0.8\textwidth]{#1}
	\end{figure}}
	
\newcommand{\sizedfic}[2]{\begin{figure}[H]
		\center
		\includegraphics[width=#1\textwidth]{#2}
	\end{figure}}

\newcommand{\codefile}[1]{\lstinputlisting{#1}}

\newcommand{\interval}{\vspace{0.5em}}

\newcommand{\tablestart}{
	\interval
	\begin{longtable}{p{2cm}p{10cm}}
	\hline}
\newcommand{\tableend}{
	\hline
	\end{longtable}
	\interval}

% 改变段间隔
\setlength{\parskip}{0.2em}
\linespread{1.1}

\usepackage{lastpage}
\usepackage{fancyhdr}
\pagestyle{fancy}
\lhead{\space \qquad \space}
\chead{linux命令集 \qquad}
\rhead{\qquad\thepage/\pageref{LastPage}}
\begin{document}

\tableofcontents

\clearpage

\section{chattr}
	chattr命令用来改变文件属性。这项指令可改变存放在ext2文件系统上的文件或目录属性,这些属性共有以下8种模式:
	\begin{lstlisting}
	a:让文件或目录仅供附加用途; 
	b:不更新文件或目录的最后存取时间; 
	c:将文件或目录压缩后存放; 
	d:将文件或目录排除在倾倒操作之外; 
	i:不得任意更动文件或目录; 
	s:保密性删除文件或目录; 
	S:即时更新文件或目录; 
	u:预防意外删除。
	\end{lstlisting}

	chattr参数选项如下:
	\begin{lstlisting}
	-R:递归处理,将指令目录下的所有文件及子目录一并处理; 
	-v<版本编号>:设置文件或目录版本; 
	-V:显示指令执行过程; 
	+<属性>:开启文件或目录的该项属性; 
	-<属性>:关闭文件或目录的该项属性; 
	=<属性>:指定文件或目录的该项属性。
	\end{lstlisting}

	使用例子如下:
	\begin{lstlisting}
	# 用chattr命令防止系统中某个关键文件被修改:
	chattr +i /etc/fstab
	\end{lstlisting}

\section{chkconfig}
	chkconfig命令检查、设置系统的各种服务。这是Red Hat公司遵循GPL规则所开发的程序,它可查询操作系统在每一个执行等级中会执行哪些系统服务,其中包括各类常驻服务。谨记chkconfig不是立即自动禁止或激活一个服务,它只是简单的改变了符号连接。\par

	chkconfig参数选项如下:
	\begin{lstlisting}
	--add:增加所指定的系统服务,让chkconfig指令得以管理它,并同时在系统启动的叙述文件内增加相关数据; 
	--del:删除所指定的系统服务,不再由chkconfig指令管理,并同时在系统启动的叙述文件内删除相关数据; 
	--level<等级代号>:指定读系统服务要在哪一个执行等级中开启或关毕。
	\end{lstlisting}

	等级代号列表如下:
	\begin{lstlisting}
	 等级0表示:表示关机 
	 等级1表示:单用户模式 
	 等级2表示:无网络连接的多用户命令行模式 
	 等级3表示:有网络连接的多用户命令行模式 
	 等级4表示:不可用 
	 等级5表示:带图形界面的多用户模式 
	 等级6表示:重新启动
	\end{lstlisting}

	chkconfig使用例子如下:
	\begin{lstlisting}
	chkconfig --list #列出所有的系统服务。 
	chkconfig --add httpd #增加httpd服务。 
	chkconfig --del httpd #删除httpd服务。 
	chkconfig --level httpd 2345 on #设置httpd在运行级别为2、3、4、5的情况下都是on(开启)的状态。 
	chkconfig --list #列出系统所有的服务启动情况。 
	chkconfig --list mysqld #列出mysqld服务设置情况。 
	chkconfig --level 35 mysqld on #设定mysqld在等级3和5为开机运行服务,--level 35表示操作只在等级3和5执行,on表示启动,off表示关闭。 
	chkconfig mysqld on #设定mysqld在各等级为on,“各等级”包括2、3、4、5等级。
	\end{lstlisting}

\subsection{增加一个服务}
	步骤如下:
	\begin{itemize}
		\item[1.] 服务脚本必须存放在/etc/ini.d/目录下。
		\item[2.] chkconfig --add servicename在chkconfig工具服务列表中增加此服务,此时服务会被在/etc/rc.d/rcN.d中赋予K/S入口了。
		\item[3.] chkconfig --level 35 mysqld on修改服务的默认启动等级。 
	\end{itemize}

\section{cpio}
	cpio命令主要是用来建立或者还原备份档的工具程序,cpio命令可以复制文件到归档包中,或者从归档包中复制文件。\par

	cpio参数选项如下:
	\begin{lstlisting}
	-0或--null:接受新增列控制字符,通常配合find指令的“-print0”参数使用; 
	-a或--rest-access-time:重新设置文件的存取时间; 
	-A或--append:附加到已存在的备份文档中,且这个备份文档必须存放在磁盘上,而不能放置于磁带机里; 
	-b或--awap:此参数的效果和同时指定“-ss”参数相同; 
	-B:将输入/输出的区块大小改成5210Bytes; 
	-c:使用旧ASCII备份格式; 
	-C<区块大小>或--io-size=<区块大小>:设置输入/输出的区块大小,单位是Byte; 
	-d或--make-directories:如有需要cpio会自行建立目录; 
	-E<范本文件>或--pattern-file=<范本文件>:指定范本文件,其内含有一个或多个范本样式,让cpio解开符合范本条件的文件,格式为每列一个范本样式; 
	-f或--nonmatching:让cpio解开所有不符合范本条件的文件; 
	-F<备份档>或--file=<备份档>:指定备份档的名称,用来取代标准输入或输出,也能借此通过网络使用另一台主机的保存设备存取备份档; 
	-H<备份格式>:指定备份时欲使用的文件格式; 
	-i或--extract:执行copy-in模式,还原备份档; 
	-l<备份档>:指定备份档的名称,用来取代标准输入,也能借此通过网络使用另一台主机的保存设备读取备份档; 
	-k:此参数将忽略不予处理,仅负责解决cpio不同版本间的兼容性问题; 
	-l或--link:以硬连接的方式取代复制文件,可在copy-pass模式下运用; 
	-L或--dereference:不建立符号连接,直接复制该连接所指向的原始文件; 
	-m或preserve-modification-time:不去更改文件的更改时间; 
	-M<回传信息>或--message=<回传信息>:设置更换保存媒体的信息; 
	-n或--numeric-uid-gid:使用“-tv”参数列出备份档的内容时,若再加上参数“-n”,则会以用户识别和群组识别码替代拥有者和群组名称列出文件清单; 
	-o或--create:执行copy-out模式,建立备份档; 
	-O<备份档>:指定备份档的名称,用来取代标准输出,也能借此通过网络使用另一台主机的保存设备存放备份档; 
	-p或--pass-through:执行copy-pass模式,略过备份步骤,直接将文件复制到目的目录; 
	-r或--rename:当有文件名称需要更改时,采用互动模式; 
	-R<拥有者><:/.><所属群组>或----owner<拥有者><:/.><所属群组> 在copy-in模式还原备份档,或copy-pass模式复制文件时,可指定这些备份,复制的文件的拥有者与所属群组; 
	-s或--swap-bytes:交换每队字节的内容; 
	-S或--swap-halfwords:交换每半个字节的内容; 
	-t或--list:将输入的内容呈现出来; 
	-u或--unconditional:置换所有文件,不论日期时间的新旧与否,皆不予询问而直接覆盖; 
	-v或--verbose:详细显示指令的执行过程; 
	-V或--dot:执行指令时。在每个文件的执行程序前面加上“.”号; 
	--block-size=<区块大小>:设置输入/输出的区块大小,假如设置数值为5,则区块大小为2500,若设置成10,则区块大小为5120,以此类推; 
	--force-local:强制将备份档存放在本地主机; 
	--help:在线帮助; 
	--no-absolute-filenames:使用相对路径建立文件名称; 
	--no-preserve-owner:不保留文件的拥有者,谁解开了备份档,那些文件就归谁所有; 
	-only-verify-crc:当备份档采用CRC备份格式时,可使用这项参数检查备份档内的每个文件是否正确无误; 
	--quiet:不显示复制了多少区块; 
	--sparse:倘若一个文件内含有大量的连续0字节,则将此文件存在稀疏文件; 
	--version:显示版本信息。
	\end{lstlisting}

	使用例子如下:
	\begin{lstlisting}
	# initrd.img是一个cpio归档文件,解压initrd.img文件
	cpio -i --make-directories < initrd.img
	\end{lstlisting}

\section{curl}
	curl命令是一个利用URL规则在命令行下工作的文件传输工具。它支持文件的上传和下载,所以是综合传输工具,但按传统,习惯称curl为下载工具。作为一款强力工具,curl支持包括HTTP、HTTPS、ftp等众多协议,还支持POST、cookies、认证、从指定偏移处下载部分文件、用户代理字符串、限速、文件大小、进度条等特征。做网页处理流程和数据检索自动化,curl可以祝一臂之力。\par

	curl参数选项如下:
	\begin{lstlisting}
	-a/--append 上传文件时,附加到目标文件 
	-A/--user-agent<string> 设置用户代理发送给服务器 
	-anyauth 可以使用“任何”身份验证方法 
	-b/--cookie cookie<name=string/file> 字符串或文件读取位置 
	--basic 使用HTTP基本验证 
	-B/--use-ascii 使用ASCII/文本传输 
	-c/--cookie-jar<file> 操作结束后把cookie写入到这个文件中 
	-C/--continue-at<offset> 断点续转 
	-d/--data<data> HTTP POST方式传送数据 
	--data-ascii<data> 以ascii的方式post数据 
	--data-binary<data> 以二进制的方式post数据 
	--negotiate 使用HTTP身份验证 
	--digest 使用数字身份验证 
	--disable-eprt 禁止使用EPRT或LPRT 
	--disable-epsv 禁止使用EPSV 
	-D/--dump-header<file> 把header信息写入到该文件中 
	--egd-file<file> 为随机数据(SSL)设置EGD socket路径 
	--tcp-nodelay<cert[:passwd]> 使用TCP_NODELAY选项 
	-e/--referer 来源网址 
	-E/--cert<cert[:passwd]> 客户端证书文件和密码 (SSL) 
	--cert-type<type> 证书文件类型 (DER/PEM/ENG) (SSL) 
	--key<key> 私钥文件名 (SSL) 
	--key-type<type> 私钥文件类型 (DER/PEM/ENG) (SSL) 
	--pass<pass> 私钥密码 (SSL) 
	--engine<eng> 加密引擎使用 (SSL). "--engine list" for list 
	--cacert<file> CA证书 (SSL) 
	--capath<directory> CA目录 (made using c_rehash) to verify peer against (SSL) 
	--ciphers<list> SSL密码 
	--compressed 要求返回是压缩的形势 (using deflate or gzip) 
	--connect-timeout<seconds> 设置最大请求时间 
	--create-dirs 建立本地目录的目录层次结构 
	--crlf 上传是把LF转变成CRLF 
	-f/--fail 连接失败时不显示http错误 
	--ftp-create-dirs 如果远程目录不存在,创建远程目录 
	--ftp-method [multicwd/nocwd/singlecwd] 控制CWD的使用 
	--ftp-pasv 使用 PASV/EPSV 代替端口 
	--ftp-skip-pasv-ip 使用PASV的时候,忽略该IP地址 
	--ftp-ssl 尝试用 SSL/TLS 来进行ftp数据传输 
	--ftp-ssl-reqd 要求用 SSL/TLS 来进行ftp数据传输 
	-F/--form<name=content> 模拟http表单提交数据 
	--form-string<name=string> 模拟http表单提交数据 
	-g/--globoff 禁用网址序列和范围使用{}和[] 
	-G/--get 以get的方式来发送数据 
	-H/--header<line> 自定义头信息传递给服务器 
	--ignore-content-length 忽略的HTTP头信息的长度 
	-i/--include 输出时包括protocol头信息 
	-I/--head 只显示请求头信息 
	-j/--junk-session-cookies 读取文件进忽略session cookie 
	--interface<interface> 使用指定网络接口/地址 
	--krb4<level> 使用指定安全级别的krb4 
	-k/--insecure 允许不使用证书到SSL站点 
	-K/--config 指定的配置文件读取 
	-l/--list-only 列出ftp目录下的文件名称 
	--limit-rate<rate> 设置传输速度 
	--local-port<NUM> 强制使用本地端口号 
	-m/--max-time<seconds> 设置最大传输时间 
	--max-redirs<num> 设置最大读取的目录数 
	--max-filesize<bytes> 设置最大下载的文件总量 
	-M/--manual 显示全手动 
	-n/--netrc 从netrc文件中读取用户名和密码 
	--netrc-optional 使用 .netrc 或者 URL来覆盖-n 
	--ntlm 使用 HTTP NTLM 身份验证 
	-N/--no-buffer 禁用缓冲输出 
	-o/--output 把输出写到该文件中 
	-O/--remote-name 把输出写到该文件中,保留远程文件的文件名 
	-p/--proxytunnel 使用HTTP代理 
	--proxy-anyauth 选择任一代理身份验证方法 
	--proxy-basic 在代理上使用基本身份验证 
	--proxy-digest 在代理上使用数字身份验证 
	--proxy-ntlm 在代理上使用ntlm身份验证 
	-P/--ftp-port 使用端口地址,而不是使用PASV 
	-q 作为第一个参数,关闭 .curlrc 
	-Q/--quote<cmd> 文件传输前,发送命令到服务器 
	-r/--range<range> 检索来自HTTP/1.1或FTP服务器字节范围 
	--range-file 读取(SSL)的随机文件 
	-R/--remote-time 在本地生成文件时,保留远程文件时间 
	--retry<num> 传输出现问题时,重试的次数 
	--retry-delay<seconds> 传输出现问题时,设置重试间隔时间 
	--retry-max-time<seconds> 传输出现问题时,设置最大重试时间 
	-s/--silent 静默模式。不输出任何东西 
	-S/--show-error 显示错误 
	--socks4<host[:port]> 用socks4代理给定主机和端口 
	--socks5<host[:port]> 用socks5代理给定主机和端口 
	--stderr<file> 
	-t/--telnet-option<OPT=val> Telnet选项设置 
	--trace<file> 对指定文件进行debug 
	--trace-ascii<file> Like --跟踪但没有hex输出 
	--trace-time 跟踪/详细输出时,添加时间戳 
	-T/--upload-file<file> 上传文件 
	--url<URL> Spet URL to work with 
	-u/--user<user[:passwd]> 设置服务器的用户和密码 
	-U/--proxy-user<user[:passwd]> 设置代理用户名和密码 
	-v 使用操作的过程更加详细
	-w/--write-out [format] 什么输出完成后 
	-x/--proxy<host[:port]> 在给定的端口上使用HTTP代理 
	-X/--request<command> 指定什么命令 
	-y/--speed-time 放弃限速所要的时间,默认为30 
	-Y/--speed-limit 停止传输速度的限制,速度时间
	\end{lstlisting}

	使用例子如下:
	\begin{lstlisting}
	curl http://man.linuxde.net/test.iso -o filename.iso --progress
	\end{lstlisting}

\section{cut}
	cut命令用来显示行中的指定部分,删除文件中指定字段。\par

	cut参数选项如下:
	\begin{lstlisting}
	-b:仅显示行中指定直接范围的内容; 
	-c:仅显示行中指定范围的字符; 
	-d:指定字段的分隔符,默认的字段分隔符为“TAB”; 
	-f:显示指定字段的内容; 
	-n:与“-b”选项连用,不分割多字节字符; 
	--complement:补足被选择的字节、字符或字段; 
	--out-delimiter=<字段分隔符>:指定输出内容是的字段分割符; 
	--help:显示指令的帮助信息; 
	--version:显示指令的版本信息。
	\end{lstlisting}

	使用例子如下:
	\begin{lstlisting}
	# text.txt内容如下:
	# No Name Mark Percent 
	# 01 tom 69 91 
	# 02 jack 71 87 
	# 03 alex 68 98
	
	# 打印第一个字段
	cut -f2 text.txt
	# 输出
	# Name
	# tom
	# jack
	# alex
	\end{lstlisting}

\section{egrep}
	egrep命令用于在文件内查找指定的字符串。egrep执行效果与grep -E相似,使用的语法及参数可参照grep指令,与grep的不同点在于解读字符串的方法。egrep是用extended regular expression语法来解读的,而grep则用basic regular expression 语法解读,extended regular expression比basic regular expression的表达更规范。

\section{file}
	file命令用来探测给定文件的类型。file命令对文件的检查分为文件系统、魔法幻数检查和语言检查3个过程。\par

	file参数选项如下:
	\begin{lstlisting}
	-b:列出辨识结果时,不显示文件名称; 
	-c:详细显示指令执行过程,便于排错或分析程序执行的情形; 
	-f<名称文件>:指定名称文件,其内容有一个或多个文件名称时,让file依序辨识这些文件,格式为每列一个文件名称; 
	-L:直接显示符号连接所指向的文件类别; 
	-m<魔法数字文件>:指定魔法数字文件; 
	-v:显示版本信息; 
	-z:尝试去解读压缩文件的内容。
	\end{lstlisting}

	使用例子如下:
	\begin{lstlisting}
	# 查看initrd.img文件类型
	file initrd.img
	\end{lstlisting}

\section{find}
	find命令用来在指定目录下查找文件。任何位于参数之前的字符串都将被视为欲查找的目录名。如果使用该命令时,不设置任何参数,则find命令将在当前目录下查找子目录与文件。并且将查找到的子目录和文件全部进行显示。\par

	find参数选项如下:
	\begin{lstlisting}
	-amin<分钟>:查找在指定时间曾被存取过的文件或目录,单位以分钟计算; 
	-anewer<参考文件或目录>:查找其存取时间较指定文件或目录的存取时间更接近现在的文件或目录; 
	-atime<24小时数>:查找在指定时间曾被存取过的文件或目录,单位以24小时计算; 
	-cmin<分钟>:查找在指定时间之时被更改过的文件或目录; 
	-cnewer<参考文件或目录>查找其更改时间较指定文件或目录的更改时间更接近现在的文件或目录; 
	-ctime<24小时数>:查找在指定时间之时被更改的文件或目录,单位以24小时计算; 
	-daystart:从本日开始计算时间; 
	-depth:从指定目录下最深层的子目录开始查找; 
	-expty:寻找文件大小为0 Byte的文件,或目录下没有任何子目录或文件的空目录; 
	-exec<执行指令>:假设find指令的回传值为True,就执行该指令; 
	-false:将find指令的回传值皆设为False; 
	-fls<列表文件>:此参数的效果和指定“-ls”参数类似,但会把结果保存为指定的列表文件; 
	-follow:排除符号连接; 
	-fprint<列表文件>:此参数的效果和指定“-print”参数类似,但会把结果保存成指定的列表文件; 
	-fprint0<列表文件>:此参数的效果和指定“-print0”参数类似,但会把结果保存成指定的列表文件; 
	-fprintf<列表文件><输出格式>:此参数的效果和指定“-printf”参数类似,但会把结果保存成指定的列表文件; 
	-fstype<文件系统类型>:只寻找该文件系统类型下的文件或目录; 
	-gid<群组识别码>:查找符合指定之群组识别码的文件或目录; 
	-group<群组名称>:查找符合指定之群组名称的文件或目录; 
	-help或——help:在线帮助; 
	-ilname<范本样式>:此参数的效果和指定“-lname”参数类似,但忽略字符大小写的差别; 
	-iname<范本样式>:此参数的效果和指定“-name”参数类似,但忽略字符大小写的差别; 
	-inum:查找符合指定的inode编号的文件或目录; 
	-ipath<范本样式>:此参数的效果和指定“-path”参数类似,但忽略字符大小写的差别; 
	-iregex<范本样式>:此参数的效果和指定“-regexe”参数类似,但忽略字符大小写的差别; 
	-links<连接数目>:查找符合指定的硬连接数目的文件或目录; 
	-iname<范本样式>:指定字符串作为寻找符号连接的范本样式; 
	-ls:假设find指令的回传值为True,就将文件或目录名称列出到标准输出; 
	-maxdepth<目录层级>:设置最大目录层级; 
	-mindepth<目录层级>:设置最小目录层级; 
	-mmin<分钟>:查找在指定时间曾被更改过的文件或目录,单位以分钟计算; 
	-mount:此参数的效果和指定“-xdev”相同; 
	-mtime<24小时数>:查找在指定时间曾被更改过的文件或目录,单位以24小时计算; 
	-name<范本样式>:指定字符串作为寻找文件或目录的范本样式; 
	-newer<参考文件或目录>:查找其更改时间较指定文件或目录的更改时间更接近现在的文件或目录; 
	-nogroup:找出不属于本地主机群组识别码的文件或目录; 
	-noleaf:不去考虑目录至少需拥有两个硬连接存在; 
	-nouser:找出不属于本地主机用户识别码的文件或目录; 
	-ok<执行指令>:此参数的效果和指定“-exec”类似,但在执行指令之前会先询问用户,若回答“y”或“Y”,则放弃执行命令; 
	-path<范本样式>:指定字符串作为寻找目录的范本样式; 
	-perm<权限数值>:查找符合指定的权限数值的文件或目录; 
	-print:假设find指令的回传值为True,就将文件或目录名称列出到标准输出。格式为每列一个名称,每个名称前皆有“./”字符串; 
	-print0:假设find指令的回传值为True,就将文件或目录名称列出到标准输出。格式为全部的名称皆在同一行; -printf<输出格式>:假设find指令的回传值为True,就将文件或目录名称列出到标准输出。格式可以自行指定; -prune:不寻找字符串作为寻找文件或目录的范本样式; -regex<范本样式>:指定字符串作为寻找文件或目录的范本样式; -size<文件大小>:查找符合指定的文件大小的文件; -true:将find指令的回传值皆设为True; -typ<文件类型>:只寻找符合指定的文件类型的文件; -uid<用户识别码>:查找符合指定的用户识别码的文件或目录; -used<日数>:查找文件或目录被更改之后在指定时间曾被存取过的文件或目录,单位以日计算; 
	-user<拥有者名称>:查找符和指定的拥有者名称的文件或目录; 
	-version或——version:显示版本信息; 
	-xdev:将范围局限在先行的文件系统中; 
	-xtype<文件类型>:此参数的效果和指定“-type”参数类似,差别在于它针对符号连接检查。
	\end{lstlisting}

	使用例子如下:
	\begin{lstlisting}
	# 在/home目录下查找以.txt结尾的文件名
	find /home -name "*.txt"
	\end{lstlisting}

\section{gunzip}
	gunzip命令用来解压缩文件。gunzip是个使用广泛的解压缩程序,它用于解开被gzip压缩过的文件,这些压缩文件预设最后的扩展名为.gz。事实上gunzip就是gzip的硬连接,因此不论是压缩或解压缩,都可通过gzip指令单独完成。\par

	gunzip参数选项如下;
	\begin{lstlisting}
	-a或——ascii:使用ASCII文字模式; 
	-c或--stdout或--to-stdout:把解压后的文件输出到标准输出设备; 
	-f或-force:强行解开压缩文件,不理会文件名称或硬连接是否存在以及该文件是否为符号连接; 
	-h或——help:在线帮助; 
	-l或——list:列出压缩文件的相关信息; 
	-L或——license:显示版本与版权信息; 
	-n或--no-name:解压缩时,若压缩文件内含有原来的文件名称及时间戳记,则将其忽略不予处理; 
	-N或——name:解压缩时,若压缩文件内含有原来的文件名称及时间戳记,则将其回存到解开的文件上; 
	-q或——quiet:不显示警告信息; 
	-r或——recursive:递归处理,将指定目录下的所有文件及子目录一并处理; 
	-S或<压缩字尾字符串>或----suffix<压缩字尾字符串>:更改压缩字尾字符串; 
	-t或——test:测试压缩文件是否正确无误; 
	-v或——verbose:显示指令执行过程; 
	-V或——version:显示版本信息;
	\end{lstlisting}

	使用例子如下:
	\begin{lstlisting}
	# 解压etc.zip.gz文件
	gunzip /opt/etc.zip.gz
	# 查看etc.zip.gz文件的压缩信息
	gunzip -l /opt/etc.zip.gz
	\end{lstlisting}

\section{grep}
	grep(global search regular expression(RE) and print out the line,全面搜索正则表达式并把行打印出来)是一种强大的文本搜索工具,它能使用正则表达式搜索文本,并把匹配的行打印出来。

	grep参数选项如下:
	\begin{lstlisting}
	-a 不要忽略二进制数据。 
	-A<显示列数> 除了显示符合范本样式的那一行之外,并显示该行之后的内容。 
	-b 在显示符合范本样式的那一行之外,并显示该行之前的内容。 
	-c 计算符合范本样式的列数。 
	-C<显示列数>或-<显示列数> 除了显示符合范本样式的那一列之外,并显示该列之前后的内容。 
	-d<进行动作> 当指定要查找的是目录而非文件时,必须使用这项参数,否则grep命令将回报信息并停止动作。 
	-e<范本样式> 指定字符串作为查找文件内容的范本样式。 
	-E 将范本样式为延伸的普通表示法来使用,意味着使用能使用扩展正则表达式。 
	-f<范本文件> 指定范本文件,其内容有一个或多个范本样式,让grep查找符合范本条件的文件内容,格式为每一列的范本样式。 
	-F 将范本样式视为固定字符串的列表。 
	-G 将范本样式视为普通的表示法来使用。 
	-h 在显示符合范本样式的那一列之前,不标示该列所属的文件名称。 
	-H 在显示符合范本样式的那一列之前,标示该列的文件名称。 
	-i 胡列字符大小写的差别。 
	-l 列出文件内容符合指定的范本样式的文件名称。 
	-L 列出文件内容不符合指定的范本样式的文件名称。 
	-n 在显示符合范本样式的那一列之前,标示出该列的编号。 
	-q 不显示任何信息。 
	-R/-r 此参数的效果和指定“-d recurse”参数相同。 
	-s 不显示错误信息。 
	-v 反转查找。 
	-w 只显示全字符合的列。 
	-x 只显示全列符合的列。 
	-y 此参数效果跟“-i”相同。 
	-o 只输出文件中匹配到的部分。
	\end{lstlisting}

\section{head}
	head命令用于显示文件的开头的内容。在默认情况下,head命令显示文件的头10行内容。\par

	head参数选项如下:
	\begin{lstlisting}
	-n<数字>:指定显示头部内容的行数; 
	-c<字符数>:指定显示头部内容的字符数; 
	-v:总是显示文件名的头信息; 
	-q:不显示文件名的头信息。
	\end{lstlisting}

\section{ip}
	ip命令用来显示或操纵Linux主机的路由、网络设备、策略路由和隧道,是Linux下较新的功能强大的网络配置工具。\par

	ip参数选项如下:
	\begin{lstlisting}
	-V:显示指令版本信息; 
	-s:输出更详细的信息; 
	-f:强制使用指定的协议族; 
	-4:指定使用的网络层协议是IPv4协议; 
	-6:指定使用的网络层协议是IPv6协议; 
	-0:输出信息每条记录输出一行,即使内容较多也不换行显示; 
	-r:显示主机时,不使用IP地址,而使用主机的域名。
	\end{lstlisting}

	ip参数如下:
	\begin{lstlisting}
	网络对象:指定要管理的网络对象; 
	具体操作:对指定的网络对象完成具体操作; 
	help:显示网络对象支持的操作命令的帮助信息。
	\end{lstlisting}

	使用例子如下:
	\begin{lstlisting}
	# 用ip命令显示网络设备的运行状态
	# link是网络对象,list是具体操作
	ip link list
	\end{lstlisting}

	网络对象如下:
	\begin{lstlisting}
	link:网络设备
	address:一个设备的协议(IP或者IPV6)地址
	neighbour:ARP或者NDISC缓冲区条目
	route:路由表条目
	rule:路由策略数据库中的规则
	maddress:多播地址
	mroute:多播路由缓冲区条目
	tunnel:IP上的通道
	netns:manage network namespaces.
	\end{lstlisting}

\section{mount}
	mount命令用于加载文件系统到指定的加载点。此命令的最常用于挂载cdrom,使我们可以访问cdrom中的数据,因为你将光盘插入cdrom中,Linux并不会自动挂载,必须使用Linux mount命令来手动完成挂载。\par

	mount参数选项如下:
	\begin{lstlisting}
	-V:显示程序版本; 
	-l:显示已加载的文件系统列表; 
	-h:显示帮助信息并退出; 
	-v:冗长模式,输出指令执行的详细信息; 
	-n:加载没有写入文件“/etc/mtab”中的文件系统; 
	-r:将文件系统加载为只读模式; 
	-o:选项用于指定挂载的分区有哪些特性;
	-a:加载文件“/etc/fstab”中描述的所有文件系统。
	\end{lstlisting}

	例子如下:
	\begin{lstlisting}
	# 在目录上挂载包含文件系统的文件
	# /dev/loop可以使得文件如同块设备一般被访问
	mount -o loop example.img /home/you/dir
	\end{lstlisting}

	在目录上挂载包含文件系统的文件一般需要两步:1.用一个循环设备节点连接文件。2.在目录上挂载该循环设备。
	这个挂载的实现除了使用上述的mount命令,还可以通过两条命令来完成:
	\begin{lstlisting}
	losetup /dev/loop0 example.img
	mount /dev/loop0 /home/you/dir
	\end{lstlisting}

\section{netstat}
	netstat命令用来打印Linux中网络系统的状态信息,可让你得知整个Linux系统的网络情况。\par

	netstat参数选项如下:
	\begin{lstlisting}
	-a或--all:显示所有连线中的Socket; 
	-A<网络类型>或--<网络类型>:列出该网络类型连线中的相关地址; 
	-c或--continuous:持续列出网络状态; 
	-C或--cache:显示路由器配置的快取信息; 
	-e或--extend:显示网络其他相关信息; 
	-F或--fib:显示FIB; 
	-g或--groups:显示多重广播功能群组组员名单; 
	-h或--help:在线帮助; 
	-i或--interfaces:显示网络界面信息表单; 
	-l或--listening:显示监控中的服务器的Socket; 
	-M或--masquerade:显示伪装的网络连线; 
	-n或--numeric:直接使用ip地址,而不通过域名服务器; 
	-N或--netlink或--symbolic:显示网络硬件外围设备的符号连接名称; 
	-o或--timers:显示计时器; 
	-p或--programs:显示正在使用Socket的程序识别码和程序名称; 
	-r或--route:显示Routing Table; 
	-s或--statistice:显示网络工作信息统计表; 
	-t或--tcp:显示TCP传输协议的连线状况; 
	-u或--udp:显示UDP传输协议的连线状况; 
	-v或--verbose:显示指令执行过程; 
	-V或--version:显示版本信息; 
	-w或--raw:显示RAW传输协议的连线状况; 
	-x或--unix:此参数的效果和指定"-A unix"参数相同; 
	--ip或--inet:此参数的效果和指定"-A inet"参数相同。
	\end{lstlisting}

\subsection{查看连接某服务端口最多的IP地址}
	\begin{lstlisting}
	netstat -ntu | grep :80 | awk '{print $5}' | cut -d: -f1 | awk '{++ip[$1]} END {for(i in ip) print ip[i],"\t",i}' | sort -nr
	\end{lstlisting}

\subsection{获得TCP各种状态列表}
	\begin{lstlisting}
	netstat -nt | grep -e 127.0.0.1 -e 0.0.0.0 -e ::: -v | awk '/^tcp/ {++state[$NF]} END {for(i in state) print i,"\t",state[i]}'
	\end{lstlisting}

\subsection{查看phpcgi进程数}
	\begin{lstlisting}
	netstat -anpo | grep "php-cgi" | wc -l
	\end{lstlisting}

\section{od}
	od命令用于输出文件的八进制、十六进制或其它格式编码的字节,通常用于显示或查看文件中不能直接显示在终端的字符。\par
	常见的文件为文本文件和二进制文件。此命令主要用来查看保存在二进制文件中的值。比如,程序可能输出大量的数据记录,每个数据是一个单精度浮点数。这些数据记录存放在一个文件中,如果想查看下这个数据,这时候od命令就派上用场了。在我看来,od命令主要用来格式化输出文件数据,即对文件中的数据进行无二义性的解释。不管是IEEE754格式的浮点数还是ASCII码,od命令都能按照需求输出它们的值。\par

	od参数选项如下:
	\begin{lstlisting}
	-a:此参数的效果和同时指定“-ta”参数相同; 
	-A:<字码基数>:选择以何种基数计算字码; 
	-b:此参数的效果和同时指定“-toc”参数相同; 
	-c:此参数的效果和同时指定“-tc”参数相同,select printable characters or backslash escapes 
	-d:此参数的效果和同时指定“-tu2”参数相同; 
	-f:此参数的效果和同时指定“-tfF”参数相同; 
	-h:此参数的效果和同时指定“-tx2”参数相同; 
	-i:此参数的效果和同时指定“-td2”参数相同; 
	-j<字符数目>或--skip-bytes=<字符数目>:略过设置的字符数目; 
	-l:此参数的效果和同时指定“-td4”参数相同; 
	-N<字符数目>或--read-bytes=<字符数目>:到设置的字符树目为止; 
	-o:此参数的效果和同时指定“-to2”参数相同; 
	-s<字符串字符数>或--strings=<字符串字符数>:只显示符合指定的字符数目的字符串; 
	-t<输出格式>或--format=<输出格式>:设置输出格式; 
	-v或--output-duplicates:输出时不省略重复的数据; 
	-w<每列字符数>或--width=<每列字符数>:设置每列的最大字符数; 
	-x:此参数的效果和同时指定“-h”参数相同; 
	--help:在线帮助; 
	--version:显示版本信息。
	\end{lstlisting}

	使用例子如下:
	\begin{lstlisting}
	# 使用单字节十进制进行解释
	od -t d1 tmp
	# 设置格式为十进制
	od -A d -c tmp
	# 每行输出两个字节
	od -w2 -c tmp
	\end{lstlisting}

\section{scp}
	scp命令用于在Linux下进行远程拷贝文件的命令,和它类似的命令有cp,不过cp只是在本机进行拷贝不能跨服务器,而且scp传输是加密的。可能会稍微影响一下速度。当你服务器硬盘变为只读read only system时,用scp可以帮你把文件移出来。另外,scp还非常不占资源,不会提高多少系统负荷,在这一点上,rsync就远远不及它了。虽然 rsync比scp会快一点,但当小文件众多的情况下,rsync会导致硬盘I/O非常高,而scp基本不影响系统正常使用。\par

	scp参数选项如下:
	\begin{lstlisting}
	-1:使用ssh协议版本1; 
	-2:使用ssh协议版本2; 
	-4:使用ipv4; 
	-6:使用ipv6; 
	-B:以批处理模式运行; 
	-C:使用压缩; 
	-F:指定ssh配置文件; 
	-l:指定宽带限制; 
	-o:指定使用的ssh选项; 
	-P:指定远程主机的端口号; 
	-p:保留文件的最后修改时间,最后访问时间和权限模式; 
	-q:不显示复制进度; 
	-r:以递归方式复制。
	\end{lstlisting}

	使用例子如下:
	\begin{lstlisting}
	# 从远程机器复制文件到本地目录
	scp pengsida@172.16.19.223:/home/pengsida/devstack/local.conf ~/local.conf
	# 上传本地文件到远程机器指定目录
	scp ~/local.conf pengsida@172.16.19.223:/home/pengsida/devstack/local.conf
	# 上传本地目录到远程机器指定目录
	scp -r ~/devstack pengsida@172.16.19.223:/home/pengsida/devstack
	\end{lstlisting}

\section{strings}
	strings命令在对象文件或二进制文件中查找可打印的字符串。字符串是4个或更多可打印字符的任意序列,以换行符或空字符结束。 strings命令对识别随机对象文件很有用。\par

	strings参数选项如下:
	\begin{lstlisting}
	-a --all:扫描整个文件而不是只扫描目标文件初始化和装载段 
	-f –print-file-name:在显示字符串前先显示文件名 
	-n –bytes=[number]:找到并且输出所有NUL终止符序列 
	- :设置显示的最少的字符数,默认是4个字符 
	-t --radix={o,d,x} :输出字符的位置,基于八进制,十进制或者十六进制 
	-o :类似--radix=o 
	-T --target= :指定二进制文件格式 -e --encoding={s,S,b,l,B,L} :选择字符大小和排列顺序:s = 7-bit, S = 8-bit, {b,l} = 16-bit, {B,L} = 32-bit
	\end{lstlisting}

	使用例子如下:
	\begin{lstlisting}
	# 列出ls中所有的ASCII文本
	strings /bin/ls
	\end{lstlisting}

\section{su}
	su命令用于切换当前用户身份到其他用户身份,变更时须输入所要变更的用户帐号与密码。\par

	su参数选项如下:
	\begin{lstlisting}
	-c<指令>或--command=<指令>:执行完指定的指令后,即恢复原来的身份; 
	-f或——fast:适用于csh与tsch,使shell不用去读取启动文件; 
	-l或——login:改变身份时,也同时变更工作目录,以及HOME,SHELL,USER,logname。此外,也会变更PATH变量; 
	-m,-p或--preserve-environment:变更身份时,不要变更环境变量; 
	-s或--shell=:指定要执行的shell; 
	--help:显示帮助; 
	--version;显示版本信息。
	\end{lstlisting}

\section{sudo}
	sudo命令用来以其他身份来执行命令,预设的身份为root。在/etc/sudoers中设置了可执行sudo指令的用户。若其未经授权的用户企图使用sudo,则会发出警告的邮件给管理员。用户使用sudo时,必须先输入密码,之后有5分钟的有效期限,超过期限则必须重新输入密码。\par

	sudo参数选项如下:
	\begin{lstlisting}
	-b:在后台执行指令; 
	-h:显示帮助; 
	-H:将HOME环境变量设为新身份的HOME环境变量; 
	-k:结束密码的有效期限,也就是下次再执行sudo时便需要输入密码;
	-l:列出目前用户可执行与无法执行的指令; 
	-p:改变询问密码的提示符号; 
	-s:执行指定的shell; 
	-u<用户>:以指定的用户作为新的身份。若不加上此参数,则预设以root作为新的身份; 
	-v:延长密码有效期限5分钟; 
	-V :显示版本信息。
	\end{lstlisting}

\section{tar}
	使用例子如下:
	\begin{lstlisting}
	# 压缩一个文件
	tar -zcvf devstack.tar.gz devstack
	# 解压缩一个文件
	tar -zcvf devstack.tar.gz
	\end{lstlisting}

\section{telnet}
	telnet命令用于登录远程主机,对远程主机进行管理。telnet因为采用明文传送报文,安全性不好,很多Linux服务器都不开放telnet服务,而改用更安全的ssh方式了。但仍然有很多别的系统可能采用了telnet方式来提供远程登录,因此弄清楚telnet客户端的使用方式仍是很有必要的。\par

	命令形式如下:
	\begin{lstlisting}
	telnet host [port]
	\end{lstlisting}

\section{wc}
	“wc -l”可以查看一个文件的行数。

\section{zcat}
	zcat命令用于不真正解压缩文件,就能显示压缩包中文件的内容的场合。\par

	zcat参数选项如下:
	\begin{lstlisting}
	-S:指定gzip格式的压缩包的后缀。当后缀不是标准压缩包后缀时使用此选项; 
	-c:将文件内容写到标注输出; 
	-d:执行解压缩操作; 
	-l:显示压缩包中文件的列表; 
	-L:显示软件许可信息; 
	-q:禁用警告信息; 
	-r:在目录上执行递归操作; 
	-t:测试压缩文件的完整性; 
	-V:显示指令的版本信息; 
	-l:更快的压缩速度; 
	-9:更高的压缩比。
	\end{lstlisting}

\end{document}
