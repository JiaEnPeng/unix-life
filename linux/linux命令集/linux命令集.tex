% !TeX spellcheck = en_US
%% 字体:方正静蕾简体
%%		 方正粗宋
\documentclass[a4paper,left=2.5cm,right=2.5cm,11pt]{article}

\usepackage[utf8]{inputenc}
\usepackage{fontspec}
\usepackage{cite}
\usepackage{xeCJK}
\usepackage{indentfirst}
\usepackage{titlesec}
\usepackage{longtable}
\usepackage{graphicx}
\usepackage{float}
\usepackage{rotating}
\usepackage{subfigure}
\usepackage{tabu}
\usepackage{amsmath}
\usepackage{setspace}
\usepackage{amsfonts}
\usepackage{appendix}
\usepackage{listings}
\usepackage{xcolor}
\usepackage{geometry}
\setcounter{secnumdepth}{4}
\usepackage{mhchem}
\usepackage{multirow}
\usepackage{extarrows}
\usepackage{hyperref}
\titleformat*{\section}{\LARGE}
\renewcommand\refname{参考文献}
\renewcommand{\abstractname}{\sihao \cjkfzcs 摘{  }要}
%\titleformat{\chapter}{\centering\bfseries\huge\wryh}{}{0.7em}{}{}
%\titleformat{\section}{\LARGE\bf}{\thesection}{1em}{}{}
\titleformat{\subsection}{\Large\bfseries}{\thesubsection}{1em}{}{}
\titleformat{\subsubsection}{\large\bfseries}{\thesubsubsection}{1em}{}{}
\renewcommand{\contentsname}{{\cjkfzcs \centerline{目{  } 录}}}
\setCJKfamilyfont{cjkhwxk}{STXingkai}
\setCJKfamilyfont{cjkfzcs}{STSongti-SC-Regular}
% \setCJKfamilyfont{cjkhwxk}{华文行楷}
% \setCJKfamilyfont{cjkfzcs}{方正粗宋简体}
\newcommand*{\cjkfzcs}{\CJKfamily{cjkfzcs}}
\newcommand*{\cjkhwxk}{\CJKfamily{cjkhwxk}}
\newfontfamily\wryh{Microsoft YaHei}
\newfontfamily\hwzs{STZhongsong}
\newfontfamily\hwst{STSong}
\newfontfamily\hwfs{STFangsong}
\newfontfamily\jljt{MicrosoftYaHei}
\newfontfamily\hwxk{STXingkai}
% \newfontfamily\hwzs{华文中宋}
% \newfontfamily\hwst{华文宋体}
% \newfontfamily\hwfs{华文仿宋}
% \newfontfamily\jljt{方正静蕾简体}
% \newfontfamily\hwxk{华文行楷}
\newcommand{\verylarge}{\fontsize{60pt}{\baselineskip}\selectfont}  
\newcommand{\chuhao}{\fontsize{44.9pt}{\baselineskip}\selectfont}  
\newcommand{\xiaochu}{\fontsize{38.5pt}{\baselineskip}\selectfont}  
\newcommand{\yihao}{\fontsize{27.8pt}{\baselineskip}\selectfont}  
\newcommand{\xiaoyi}{\fontsize{25.7pt}{\baselineskip}\selectfont}  
\newcommand{\erhao}{\fontsize{23.5pt}{\baselineskip}\selectfont}  
\newcommand{\xiaoerhao}{\fontsize{19.3pt}{\baselineskip}\selectfont} 
\newcommand{\sihao}{\fontsize{14pt}{\baselineskip}\selectfont}      % 字号设置  
\newcommand{\xiaosihao}{\fontsize{12pt}{\baselineskip}\selectfont}  % 字号设置  
\newcommand{\wuhao}{\fontsize{10.5pt}{\baselineskip}\selectfont}    % 字号设置  
\newcommand{\xiaowuhao}{\fontsize{9pt}{\baselineskip}\selectfont}   % 字号设置  
\newcommand{\liuhao}{\fontsize{7.875pt}{\baselineskip}\selectfont}  % 字号设置  
\newcommand{\qihao}{\fontsize{5.25pt}{\baselineskip}\selectfont}    % 字号设置 

\usepackage{diagbox}
\usepackage{multirow}
\boldmath
\XeTeXlinebreaklocale "zh"
\XeTeXlinebreakskip = 0pt plus 1pt minus 0.1pt
\definecolor{cred}{rgb}{0.8,0.8,0.8}
\definecolor{cgreen}{rgb}{0,0.3,0}
\definecolor{cpurple}{rgb}{0.5,0,0.35}
\definecolor{cdocblue}{rgb}{0,0,0.3}
\definecolor{cdark}{rgb}{0.95,1.0,1.0}
\lstset{
	language=bash,
	numbers=left,
	numberstyle=\tiny\color{white},
	showspaces=false,
	showstringspaces=false,
	basicstyle=\scriptsize,
	keywordstyle=\color{purple},
	commentstyle=\color{cgreen},
	stringstyle=\color{blue},
	frame=lines,
	% escapeinside=``,
	extendedchars=true, 
	xleftmargin=0em,
	xrightmargin=0em, 
	backgroundcolor=\color{cred},
	aboveskip=1em,
	breaklines=true,
	tabsize=4
} 

\newfontfamily{\consolas}{Consolas}
\newfontfamily{\monaco}{Monaco}
\setmonofont[Mapping={}]{Consolas}	%英文引号之类的正常显示,相当于设置英文字体
\setsansfont{Consolas} %设置英文字体 Monaco, Consolas,  Fantasque Sans Mono
\setmainfont{Times New Roman}

\setCJKmainfont{华文中宋}


\newcommand{\fic}[1]{\begin{figure}[H]
		\center
		\includegraphics[width=0.8\textwidth]{#1}
	\end{figure}}
	
\newcommand{\sizedfic}[2]{\begin{figure}[H]
		\center
		\includegraphics[width=#1\textwidth]{#2}
	\end{figure}}

\newcommand{\codefile}[1]{\lstinputlisting{#1}}

\newcommand{\interval}{\vspace{0.5em}}

\newcommand{\tablestart}{
	\interval
	\begin{longtable}{p{2cm}p{10cm}}
	\hline}
\newcommand{\tableend}{
	\hline
	\end{longtable}
	\interval}

% 改变段间隔
\setlength{\parskip}{0.2em}
\linespread{1.1}

\usepackage{lastpage}
\usepackage{fancyhdr}
\pagestyle{fancy}
\lhead{\space \qquad \space}
\chead{linux命令集 \qquad}
\rhead{\qquad\thepage/\pageref{LastPage}}
\begin{document}

\tableofcontents

\clearpage

\section{chkconfig}
	chkconfig命令检查、设置系统的各种服务。这是Red Hat公司遵循GPL规则所开发的程序,它可查询操作系统在每一个执行等级中会执行哪些系统服务,其中包括各类常驻服务。谨记chkconfig不是立即自动禁止或激活一个服务,它只是简单的改变了符号连接。\par

	chkconfig参数选项如下:
	\begin{lstlisting}
	--add:增加所指定的系统服务,让chkconfig指令得以管理它,并同时在系统启动的叙述文件内增加相关数据; 
	--del:删除所指定的系统服务,不再由chkconfig指令管理,并同时在系统启动的叙述文件内删除相关数据; 
	--level<等级代号>:指定读系统服务要在哪一个执行等级中开启或关毕。
	\end{lstlisting}

	等级代号列表如下:
	\begin{lstlisting}
	 等级0表示:表示关机 
	 等级1表示:单用户模式 
	 等级2表示:无网络连接的多用户命令行模式 
	 等级3表示:有网络连接的多用户命令行模式 
	 等级4表示:不可用 
	 等级5表示:带图形界面的多用户模式 
	 等级6表示:重新启动
	\end{lstlisting}

	chkconfig使用例子如下:
	\begin{lstlisting}
	chkconfig --list #列出所有的系统服务。 
	chkconfig --add httpd #增加httpd服务。 
	chkconfig --del httpd #删除httpd服务。 
	chkconfig --level httpd 2345 on #设置httpd在运行级别为2、3、4、5的情况下都是on(开启)的状态。 
	chkconfig --list #列出系统所有的服务启动情况。 
	chkconfig --list mysqld #列出mysqld服务设置情况。 
	chkconfig --level 35 mysqld on #设定mysqld在等级3和5为开机运行服务,--level 35表示操作只在等级3和5执行,on表示启动,off表示关闭。 
	chkconfig mysqld on #设定mysqld在各等级为on,“各等级”包括2、3、4、5等级。
	\end{lstlisting}

\subsection{增加一个服务}
	步骤如下:
	\begin{itemize}
		\item[1.] 服务脚本必须存放在/etc/ini.d/目录下。
		\item[2.] chkconfig --add servicename在chkconfig工具服务列表中增加此服务,此时服务会被在/etc/rc.d/rcN.d中赋予K/S入口了。
		\item[3.] chkconfig --level 35 mysqld on修改服务的默认启动等级。 
	\end{itemize}

\section{netstat}
	netstat命令用来打印Linux中网络系统的状态信息,可让你得知整个Linux系统的网络情况。\par

	netstat参数选项如下:
	\begin{lstlisting}
	-a或--all:显示所有连线中的Socket; 
	-A<网络类型>或--<网络类型>:列出该网络类型连线中的相关地址; 
	-c或--continuous:持续列出网络状态; 
	-C或--cache:显示路由器配置的快取信息; 
	-e或--extend:显示网络其他相关信息; 
	-F或--fib:显示FIB; 
	-g或--groups:显示多重广播功能群组组员名单; 
	-h或--help:在线帮助; 
	-i或--interfaces:显示网络界面信息表单; 
	-l或--listening:显示监控中的服务器的Socket; 
	-M或--masquerade:显示伪装的网络连线; 
	-n或--numeric:直接使用ip地址,而不通过域名服务器; 
	-N或--netlink或--symbolic:显示网络硬件外围设备的符号连接名称; 
	-o或--timers:显示计时器; 
	-p或--programs:显示正在使用Socket的程序识别码和程序名称; 
	-r或--route:显示Routing Table; 
	-s或--statistice:显示网络工作信息统计表; 
	-t或--tcp:显示TCP传输协议的连线状况; 
	-u或--udp:显示UDP传输协议的连线状况; 
	-v或--verbose:显示指令执行过程; 
	-V或--version:显示版本信息; 
	-w或--raw:显示RAW传输协议的连线状况; 
	-x或--unix:此参数的效果和指定"-A unix"参数相同; 
	--ip或--inet:此参数的效果和指定"-A inet"参数相同。
	\end{lstlisting}

\subsection{查看连接某服务端口最多的IP地址}
	\begin{lstlisting}
	netstat -ntu | grep :80 | awk '{print $5}' | cut -d: -f1 | awk '{++ip[$1]} END {for(i in ip) print ip[i],"\t",i}' | sort -nr
	\end{lstlisting}

\subsection{获得TCP各种状态列表}
	\begin{lstlisting}
	netstat -nt | grep -e 127.0.0.1 -e 0.0.0.0 -e ::: -v | awk '/^tcp/ {++state[$NF]} END {for(i in state) print i,"\t",state[i]}'
	\end{lstlisting}

\subsection{查看phpcgi进程数}
	\begin{lstlisting}
	netstat -anpo | grep "php-cgi" | wc -l
	\end{lstlisting}

\section{wc}
	“wc -l”可以查看一个文件的行数。

\end{document}
