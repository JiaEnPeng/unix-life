% !TeX spellcheck = en_US
%% 字体:方正静蕾简体
%%		 方正粗宋
\documentclass[a4paper,left=2.5cm,right=2.5cm,11pt]{article}

\usepackage[utf8]{inputenc}
\usepackage{fontspec}
\usepackage{cite}
\usepackage{xeCJK}
\usepackage{indentfirst}
\usepackage{titlesec}
\usepackage{longtable}
\usepackage{graphicx}
\usepackage{float}
\usepackage{rotating}
\usepackage{subfigure}
\usepackage{tabu}
\usepackage{amsmath}
\usepackage{setspace}
\usepackage{amsfonts}
\usepackage{appendix}
\usepackage{listings}
\usepackage{xcolor}
\usepackage{geometry}
\setcounter{secnumdepth}{4}
\usepackage{mhchem}
\usepackage{multirow}
\usepackage{extarrows}
\usepackage{hyperref}
\titleformat*{\section}{\LARGE}
\renewcommand\refname{参考文献}
\renewcommand{\abstractname}{\sihao \cjkfzcs 摘{  }要}
%\titleformat{\chapter}{\centering\bfseries\huge\wryh}{}{0.7em}{}{}
%\titleformat{\section}{\LARGE\bf}{\thesection}{1em}{}{}
\titleformat{\subsection}{\Large\bfseries}{\thesubsection}{1em}{}{}
\titleformat{\subsubsection}{\large\bfseries}{\thesubsubsection}{1em}{}{}
\renewcommand{\contentsname}{{\cjkfzcs \centerline{目{  } 录}}}
\setCJKfamilyfont{cjkhwxk}{STXingkai}
\setCJKfamilyfont{cjkfzcs}{STSongti-SC-Regular}
% \setCJKfamilyfont{cjkhwxk}{华文行楷}
% \setCJKfamilyfont{cjkfzcs}{方正粗宋简体}
\newcommand*{\cjkfzcs}{\CJKfamily{cjkfzcs}}
\newcommand*{\cjkhwxk}{\CJKfamily{cjkhwxk}}
\newfontfamily\wryh{Microsoft YaHei}
\newfontfamily\hwzs{STZhongsong}  
\newfontfamily\hwst{STSong}
\newfontfamily\hwfs{STFangsong}
\newfontfamily\jljt{MicrosoftYaHei}
\newfontfamily\hwxk{STXingkai}
% \newfontfamily\hwzs{华文中宋}
% \newfontfamily\hwst{华文宋体}
% \newfontfamily\hwfs{华文仿宋}
% \newfontfamily\jljt{方正静蕾简体}
% \newfontfamily\hwxk{华文行楷}
\newcommand{\verylarge}{\fontsize{60pt}{\baselineskip}\selectfont}  
\newcommand{\chuhao}{\fontsize{44.9pt}{\baselineskip}\selectfont}  
\newcommand{\xiaochu}{\fontsize{38.5pt}{\baselineskip}\selectfont}  
\newcommand{\yihao}{\fontsize{27.8pt}{\baselineskip}\selectfont}  
\newcommand{\xiaoyi}{\fontsize{25.7pt}{\baselineskip}\selectfont}  
\newcommand{\erhao}{\fontsize{23.5pt}{\baselineskip}\selectfont}  
\newcommand{\xiaoerhao}{\fontsize{19.3pt}{\baselineskip}\selectfont} 
\newcommand{\sihao}{\fontsize{14pt}{\baselineskip}\selectfont}      % 字号设置  
\newcommand{\xiaosihao}{\fontsize{12pt}{\baselineskip}\selectfont}  % 字号设置  
\newcommand{\wuhao}{\fontsize{10.5pt}{\baselineskip}\selectfont}    % 字号设置  
\newcommand{\xiaowuhao}{\fontsize{9pt}{\baselineskip}\selectfont}   % 字号设置  
\newcommand{\liuhao}{\fontsize{7.875pt}{\baselineskip}\selectfont}  % 字号设置  
\newcommand{\qihao}{\fontsize{5.25pt}{\baselineskip}\selectfont}    % 字号设置 

\usepackage{diagbox}
\usepackage{multirow}
\boldmath
\XeTeXlinebreaklocale "zh"
\XeTeXlinebreakskip = 0pt plus 1pt minus 0.1pt
\definecolor{cred}{rgb}{0.8,0.8,0.8}
\definecolor{cgreen}{rgb}{0,0.3,0}
\definecolor{cpurple}{rgb}{0.5,0,0.35}
\definecolor{cdocblue}{rgb}{0,0,0.3}
\definecolor{cdark}{rgb}{0.95,1.0,1.0}
\lstset{
	language=bash,
	numbers=left,
	numberstyle=\tiny\color{white},
	showspaces=false,
	showstringspaces=false,
	basicstyle=\scriptsize,
	keywordstyle=\color{purple},
	commentstyle=\color{cgreen},
	stringstyle=\color{blue},
	frame=lines,
	% escapeinside=``,
	extendedchars=true, 
	xleftmargin=0em,
	xrightmargin=0em, 
	backgroundcolor=\color{cred},
	aboveskip=1em,
	breaklines=true,
	tabsize=4
} 

\newfontfamily{\consolas}{Consolas}
\newfontfamily{\monaco}{Monaco}
\setmonofont[Mapping={}]{Consolas}	%英文引号之类的正常显示,相当于设置英文字体
\setsansfont{Consolas} %设置英文字体 Monaco, Consolas,  Fantasque Sans Mono
\setmainfont{Times New Roman}

\setCJKmainfont{华文中宋}


\newcommand{\fic}[1]{\begin{figure}[H]
		\center
		\includegraphics[width=0.8\textwidth]{#1}
	\end{figure}}
	
\newcommand{\sizedfic}[2]{\begin{figure}[H]
		\center
		\includegraphics[width=#1\textwidth]{#2}
	\end{figure}}

\newcommand{\codefile}[1]{\lstinputlisting{#1}}

\newcommand{\interval}{\vspace{0.5em}}

\newcommand{\tablestart}{
	\interval
	\begin{longtable}{p{2cm}p{10cm}}
	\hline}
\newcommand{\tableend}{
	\hline
	\end{longtable}
	\interval}

% 改变段间隔
\setlength{\parskip}{0.2em}
\linespread{1.1}

\usepackage{lastpage}
\usepackage{fancyhdr}
\pagestyle{fancy}
\lhead{\space \qquad \space}
\chead{linux命令集 \qquad}
\rhead{\qquad\thepage/\pageref{LastPage}}
\begin{document}

\tableofcontents

\clearpage

\section{brctl}
	命令参数:
	\begin{lstlisting}
Usage: brctl [commands]
commands:
	addbr           <bridge>                add bridge
	delbr           <bridge>                delete bridge
	addif           <bridge> <device>       add interface to bridge
	delif           <bridge> <device>       delete interface from bridge
	setageing       <bridge> <time>         set ageing time
	setbridgeprio   <bridge> <prio>         set bridge priority
	setfd           <bridge> <time>         set bridge forward delay
	sethello        <bridge> <time>         set hello time
	setmaxage       <bridge> <time>         set max message age
	sethashel       <bridge> <int>          set hash elasticity
	sethashmax      <bridge> <int>          set hash max
	setmclmc        <bridge> <int>          set multicast last member count
	setmcrouter     <bridge> <int>          set multicast router
	setmcsnoop      <bridge> <int>          set multicast snooping
	setmcsqc        <bridge> <int>          set multicast startup query count
	setmclmi        <bridge> <time>         set multicast last member interval
	setmcmi         <bridge> <time>         set multicast membership interval
	setmcqpi        <bridge> <time>         set multicast querier interval
	setmcqi         <bridge> <time>         set multicast query interval
	setmcqri        <bridge> <time>         set multicast query response interval
	setmcqri        <bridge> <time>         set multicast startup query interval
	setpathcost     <bridge> <port> <cost>  set path cost
	setportprio     <bridge> <port> <prio>  set port priority
	setportmcrouter <bridge> <port> <int>   set port multicast router
	show            [ <bridge> ]            show a list of bridges
	showmacs        <bridge>                show a list of mac addrs
	showstp         <bridge>                show bridge stp info
	stp             <bridge> {on|off}       turn stp on/off
	\end{lstlisting}

\section{chattr}
	chattr命令用来改变文件属性。这项指令可改变存放在ext2文件系统上的文件或目录属性,这些属性共有以下8种模式:
	\begin{lstlisting}
	a:让文件或目录仅供附加用途; 
	b:不更新文件或目录的最后存取时间; 
	c:将文件或目录压缩后存放; 
	d:将文件或目录排除在倾倒操作之外; 
	i:不得任意更动文件或目录; 
	s:保密性删除文件或目录; 
	S:即时更新文件或目录; 
	u:预防意外删除。
	\end{lstlisting}

	chattr参数选项如下:
	\begin{lstlisting}
	-R:递归处理,将指令目录下的所有文件及子目录一并处理; 
	-v<版本编号>:设置文件或目录版本; 
	-V:显示指令执行过程; 
	+<属性>:开启文件或目录的该项属性; 
	-<属性>:关闭文件或目录的该项属性; 
	=<属性>:指定文件或目录的该项属性。
	\end{lstlisting}

	使用例子如下:
	\begin{lstlisting}
	# 用chattr命令防止系统中某个关键文件被修改:
	chattr +i /etc/fstab
	\end{lstlisting}

\section{chgrp}
	更改文件所在的组。

\section{chkconfig}
	chkconfig命令检查、设置系统的各种服务。这是Red Hat公司遵循GPL规则所开发的程序,它可查询操作系统在每一个执行等级中会执行哪些系统服务,其中包括各类常驻服务。谨记chkconfig不是立即自动禁止或激活一个服务,它只是简单的改变了符号连接。\par

	chkconfig参数选项如下:
	\begin{lstlisting}
	--add:增加所指定的系统服务,让chkconfig指令得以管理它,并同时在系统启动的叙述文件内增加相关数据; 
	--del:删除所指定的系统服务,不再由chkconfig指令管理,并同时在系统启动的叙述文件内删除相关数据; 
	--level<等级代号>:指定读系统服务要在哪一个执行等级中开启或关毕。
	\end{lstlisting}

	等级代号列表如下:
	\begin{lstlisting}
	 等级0表示:表示关机 
	 等级1表示:单用户模式 
	 等级2表示:无网络连接的多用户命令行模式 
	 等级3表示:有网络连接的多用户命令行模式 
	 等级4表示:不可用 
	 等级5表示:带图形界面的多用户模式 
	 等级6表示:重新启动
	\end{lstlisting}

	chkconfig使用例子如下:
	\begin{lstlisting}
	chkconfig --list #列出所有的系统服务。 
	chkconfig --add httpd #增加httpd服务。 
	chkconfig --del httpd #删除httpd服务。 
	chkconfig --level httpd 2345 on #设置httpd在运行级别为2、3、4、5的情况下都是on(开启)的状态。 
	chkconfig --list #列出系统所有的服务启动情况。 
	chkconfig --list mysqld #列出mysqld服务设置情况。 
	chkconfig --level 35 mysqld on #设定mysqld在等级3和5为开机运行服务,--level 35表示操作只在等级3和5执行,on表示启动,off表示关闭。 
	chkconfig mysqld on #设定mysqld在各等级为on,“各等级”包括2、3、4、5等级。
	\end{lstlisting}

\subsection{增加一个服务}
	步骤如下:
	\begin{itemize}
		\item[1.] 服务脚本必须存放在/etc/ini.d/目录下。
		\item[2.] chkconfig --add servicename在chkconfig工具服务列表中增加此服务,此时服务会被在/etc/rc.d/rcN.d中赋予K/S入口了。
		\item[3.] chkconfig --level 35 mysqld on修改服务的默认启动等级。 
	\end{itemize}

\section{chmod}
	chmod命令用来变更文件或目录的权限。在UNIX系统家族里,文件或目录权限的控制分别以读取、写入、执行3种一般权限来区分,另有3种特殊权限可供运用。用户可以使用chmod指令去变更文件与目录的权限,设置方式采用文字或数字代号皆可。符号连接的权限无法变更,如果用户对符号连接修改权限,其改变会作用在被连接的原始文件。\par

	格式:
	\begin{lstlisting}
	chmod 权限模式 文件
	\end{lstlisting}

	权限如下:
	\begin{lstlisting}
	u User,即文件或目录的拥有者; 
	g Group,即文件或目录的所属群组; 
	o Other,除了文件或目录拥有者或所属群组之外,其他用户皆属于这个范围; 
	a All,即全部的用户,包含拥有者,所属群组以及其他用户; 
	r 读取权限,数字代号为“4”; 
	w 写入权限,数字代号为“2”; 
	x 执行或切换权限,数字代号为“1”; 
	- 不具任何权限,数字代号为“0”; 
	s 特殊功能说明:变更文件或目录的权限。
	\end{lstlisting}

	参数选项:
	\begin{lstlisting}
	-c或——changes:效果类似“-v”参数,但仅回报更改的部分; 
	-f或--quiet或——silent:不显示错误信息; 
	-R或——recursive:递归处理,将指令目录下的所有文件及子目录一并处理; 
	-v或——verbose:显示指令执行过程; 
	--reference=<参考文件或目录>:把指定文件或目录的所属群组全部设成和参考文件或目录的所属群组相同; 
	<权限范围>+<权限设置>:开启权限范围的文件或目录的该选项权限设置; 
	<权限范围>-<权限设置>:关闭权限范围的文件或目录的该选项权限设置; 
	<权限范围>=<权限设置>:指定权限范围的文件或目录的该选项权限设置;
	\end{lstlisting}

\section{chown}
	chown命令改变某个文件或目录的所有者和所属的组,该命令可以向某个用户授权,使该用户变成指定文件的所有者或者改变文件所属的组。用户可以是用户或者是用户D,用户组可以是组名或组id。文件名可以使由空格分开的文件列表,在文件名中可以包含通配符。\par
	只有文件主和超级用户才可以便用该命令。\par

	格式:
	\begin{lstlisting}
	# 如果省略组,就只改变文件所有者
	chown 用户:组 文件
	\end{lstlisting}

	参数选项如下:
	\begin{lstlisting}
	-c或——changes:效果类似“-v”参数,但仅回报更改的部分; 
	-f或--quite或——silent:不显示错误信息; 
	-h或--no-dereference:只对符号连接的文件作修改,而不更改其他任何相关文件; 
	-R或——recursive:递归处理,将指定目录下的所有文件及子目录一并处理; 
	-v或——version:显示指令执行过程; 
	--dereference:效果和“-h”参数相同; 
	--help:在线帮助; 
	--reference=<参考文件或目录>:把指定文件或目录的拥有者与所属群组全部设成和参考文件或目录的拥有者与所属群组相同; 
	--version:显示版本信息。
	\end{lstlisting}

\section{cpio}
	cpio命令主要是用来建立或者还原备份档的工具程序,cpio命令可以复制文件到归档包中,或者从归档包中复制文件。\par

	cpio参数选项如下:
	\begin{lstlisting}
	-0或--null:接受新增列控制字符,通常配合find指令的“-print0”参数使用; 
	-a或--rest-access-time:重新设置文件的存取时间; 
	-A或--append:附加到已存在的备份文档中,且这个备份文档必须存放在磁盘上,而不能放置于磁带机里; 
	-b或--awap:此参数的效果和同时指定“-ss”参数相同; 
	-B:将输入/输出的区块大小改成5210Bytes; 
	-c:使用旧ASCII备份格式; 
	-C<区块大小>或--io-size=<区块大小>:设置输入/输出的区块大小,单位是Byte; 
	-d或--make-directories:如有需要cpio会自行建立目录; 
	-E<范本文件>或--pattern-file=<范本文件>:指定范本文件,其内含有一个或多个范本样式,让cpio解开符合范本条件的文件,格式为每列一个范本样式; 
	-f或--nonmatching:让cpio解开所有不符合范本条件的文件; 
	-F<备份档>或--file=<备份档>:指定备份档的名称,用来取代标准输入或输出,也能借此通过网络使用另一台主机的保存设备存取备份档; 
	-H<备份格式>:指定备份时欲使用的文件格式; 
	-i或--extract:执行copy-in模式,还原备份档; 
	-l<备份档>:指定备份档的名称,用来取代标准输入,也能借此通过网络使用另一台主机的保存设备读取备份档; 
	-k:此参数将忽略不予处理,仅负责解决cpio不同版本间的兼容性问题; 
	-l或--link:以硬连接的方式取代复制文件,可在copy-pass模式下运用; 
	-L或--dereference:不建立符号连接,直接复制该连接所指向的原始文件; 
	-m或preserve-modification-time:不去更改文件的更改时间; 
	-M<回传信息>或--message=<回传信息>:设置更换保存媒体的信息; 
	-n或--numeric-uid-gid:使用“-tv”参数列出备份档的内容时,若再加上参数“-n”,则会以用户识别和群组识别码替代拥有者和群组名称列出文件清单; 
	-o或--create:执行copy-out模式,建立备份档; 
	-O<备份档>:指定备份档的名称,用来取代标准输出,也能借此通过网络使用另一台主机的保存设备存放备份档; 
	-p或--pass-through:执行copy-pass模式,略过备份步骤,直接将文件复制到目的目录; 
	-r或--rename:当有文件名称需要更改时,采用互动模式; 
	-R<拥有者><:/.><所属群组>或----owner<拥有者><:/.><所属群组> 在copy-in模式还原备份档,或copy-pass模式复制文件时,可指定这些备份,复制的文件的拥有者与所属群组; 
	-s或--swap-bytes:交换每队字节的内容; 
	-S或--swap-halfwords:交换每半个字节的内容; 
	-t或--list:将输入的内容呈现出来; 
	-u或--unconditional:置换所有文件,不论日期时间的新旧与否,皆不予询问而直接覆盖; 
	-v或--verbose:详细显示指令的执行过程; 
	-V或--dot:执行指令时。在每个文件的执行程序前面加上“.”号; 
	--block-size=<区块大小>:设置输入/输出的区块大小,假如设置数值为5,则区块大小为2500,若设置成10,则区块大小为5120,以此类推; 
	--force-local:强制将备份档存放在本地主机; 
	--help:在线帮助; 
	--no-absolute-filenames:使用相对路径建立文件名称; 
	--no-preserve-owner:不保留文件的拥有者,谁解开了备份档,那些文件就归谁所有; 
	-only-verify-crc:当备份档采用CRC备份格式时,可使用这项参数检查备份档内的每个文件是否正确无误; 
	--quiet:不显示复制了多少区块; 
	--sparse:倘若一个文件内含有大量的连续0字节,则将此文件存在稀疏文件; 
	--version:显示版本信息。
	\end{lstlisting}

	使用例子如下:
	\begin{lstlisting}
	# initrd.img是一个cpio归档文件,解压initrd.img文件
	cpio -i --make-directories < initrd.img
	\end{lstlisting}

\section{curl}
	curl命令是一个利用URL规则在命令行下工作的文件传输工具。它支持文件的上传和下载,所以是综合传输工具,但按传统,习惯称curl为下载工具。作为一款强力工具,curl支持包括HTTP、HTTPS、ftp等众多协议,还支持POST、cookies、认证、从指定偏移处下载部分文件、用户代理字符串、限速、文件大小、进度条等特征。做网页处理流程和数据检索自动化,curl可以祝一臂之力。\par

	curl参数选项如下:
	\begin{lstlisting}
	-a/--append 上传文件时,附加到目标文件 
	-A/--user-agent<string> 设置用户代理发送给服务器 
	-anyauth 可以使用“任何”身份验证方法 
	-b/--cookie cookie<name=string/file> 字符串或文件读取位置 
	--basic 使用HTTP基本验证 
	-B/--use-ascii 使用ASCII/文本传输 
	-c/--cookie-jar<file> 操作结束后把cookie写入到这个文件中 
	-C/--continue-at<offset> 断点续转 
	-d/--data<data> HTTP POST方式传送数据 
	--data-ascii<data> 以ascii的方式post数据 
	--data-binary<data> 以二进制的方式post数据 
	--negotiate 使用HTTP身份验证 
	--digest 使用数字身份验证 
	--disable-eprt 禁止使用EPRT或LPRT 
	--disable-epsv 禁止使用EPSV 
	-D/--dump-header<file> 把header信息写入到该文件中 
	--egd-file<file> 为随机数据(SSL)设置EGD socket路径 
	--tcp-nodelay<cert[:passwd]> 使用TCP_NODELAY选项 
	-e/--referer 来源网址 
	-E/--cert<cert[:passwd]> 客户端证书文件和密码 (SSL) 
	--cert-type<type> 证书文件类型 (DER/PEM/ENG) (SSL) 
	--key<key> 私钥文件名 (SSL) 
	--key-type<type> 私钥文件类型 (DER/PEM/ENG) (SSL) 
	--pass<pass> 私钥密码 (SSL) 
	--engine<eng> 加密引擎使用 (SSL). "--engine list" for list 
	--cacert<file> CA证书 (SSL) 
	--capath<directory> CA目录 (made using c_rehash) to verify peer against (SSL) 
	--ciphers<list> SSL密码 
	--compressed 要求返回是压缩的形势 (using deflate or gzip) 
	--connect-timeout<seconds> 设置最大请求时间 
	--create-dirs 建立本地目录的目录层次结构 
	--crlf 上传是把LF转变成CRLF 
	-f/--fail 连接失败时不显示http错误 
	--ftp-create-dirs 如果远程目录不存在,创建远程目录 
	--ftp-method [multicwd/nocwd/singlecwd] 控制CWD的使用 
	--ftp-pasv 使用 PASV/EPSV 代替端口 
	--ftp-skip-pasv-ip 使用PASV的时候,忽略该IP地址 
	--ftp-ssl 尝试用 SSL/TLS 来进行ftp数据传输 
	--ftp-ssl-reqd 要求用 SSL/TLS 来进行ftp数据传输 
	-F/--form<name=content> 模拟http表单提交数据 
	--form-string<name=string> 模拟http表单提交数据 
	-g/--globoff 禁用网址序列和范围使用{}和[] 
	-G/--get 以get的方式来发送数据 
	-H/--header<line> 自定义头信息传递给服务器 
	--ignore-content-length 忽略的HTTP头信息的长度 
	-i/--include 输出时包括protocol头信息 
	-I/--head 只显示请求头信息 
	-j/--junk-session-cookies 读取文件进忽略session cookie 
	--interface<interface> 使用指定网络接口/地址 
	--krb4<level> 使用指定安全级别的krb4 
	-k/--insecure 允许不使用证书到SSL站点 
	-K/--config 指定的配置文件读取 
	-l/--list-only 列出ftp目录下的文件名称 
	--limit-rate<rate> 设置传输速度 
	--local-port<NUM> 强制使用本地端口号 
	-m/--max-time<seconds> 设置最大传输时间 
	--max-redirs<num> 设置最大读取的目录数 
	--max-filesize<bytes> 设置最大下载的文件总量 
	-M/--manual 显示全手动 
	-n/--netrc 从netrc文件中读取用户名和密码 
	--netrc-optional 使用 .netrc 或者 URL来覆盖-n 
	--ntlm 使用 HTTP NTLM 身份验证 
	-N/--no-buffer 禁用缓冲输出 
	-o/--output 把输出写到该文件中 
	-O/--remote-name 把输出写到该文件中,保留远程文件的文件名 
	-p/--proxytunnel 使用HTTP代理 
	--proxy-anyauth 选择任一代理身份验证方法 
	--proxy-basic 在代理上使用基本身份验证 
	--proxy-digest 在代理上使用数字身份验证 
	--proxy-ntlm 在代理上使用ntlm身份验证 
	-P/--ftp-port 使用端口地址,而不是使用PASV 
	-q 作为第一个参数,关闭 .curlrc 
	-Q/--quote<cmd> 文件传输前,发送命令到服务器 
	-r/--range<range> 检索来自HTTP/1.1或FTP服务器字节范围 
	--range-file 读取(SSL)的随机文件 
	-R/--remote-time 在本地生成文件时,保留远程文件时间 
	--retry<num> 传输出现问题时,重试的次数 
	--retry-delay<seconds> 传输出现问题时,设置重试间隔时间 
	--retry-max-time<seconds> 传输出现问题时,设置最大重试时间 
	-s/--silent 静默模式。不输出任何东西 
	-S/--show-error 显示错误 
	--socks4<host[:port]> 用socks4代理给定主机和端口 
	--socks5<host[:port]> 用socks5代理给定主机和端口 
	--stderr<file> 
	-t/--telnet-option<OPT=val> Telnet选项设置 
	--trace<file> 对指定文件进行debug 
	--trace-ascii<file> Like --跟踪但没有hex输出 
	--trace-time 跟踪/详细输出时,添加时间戳 
	-T/--upload-file<file> 上传文件 
	--url<URL> Spet URL to work with 
	-u/--user<user[:passwd]> 设置服务器的用户和密码 
	-U/--proxy-user<user[:passwd]> 设置代理用户名和密码 
	-v 使用操作的过程更加详细
	-w/--write-out [format] 什么输出完成后 
	-x/--proxy<host[:port]> 在给定的端口上使用HTTP代理 
	-X/--request<command> 指定什么命令 
	-y/--speed-time 放弃限速所要的时间,默认为30 
	-Y/--speed-limit 停止传输速度的限制,速度时间
	\end{lstlisting}

	使用例子如下:
	\begin{lstlisting}
	curl http://man.linuxde.net/test.iso -o filename.iso --progress
	\end{lstlisting}

\section{cut}
	cut命令用来显示行中的指定部分,删除文件中指定字段。\par

	cut参数选项如下:
	\begin{lstlisting}
	-b:仅显示行中指定直接范围的内容; 
	-c:仅显示行中指定范围的字符; 
	-d:指定字段的分隔符,默认的字段分隔符为“TAB”; 
	-f:显示指定字段的内容; 
	-n:与“-b”选项连用,不分割多字节字符; 
	--complement:补足被选择的字节、字符或字段; 
	--out-delimiter=<字段分隔符>:指定输出内容是的字段分割符; 
	--help:显示指令的帮助信息; 
	--version:显示指令的版本信息。
	\end{lstlisting}

	使用例子如下:
	\begin{lstlisting}
	# text.txt内容如下:
	# No Name Mark Percent 
	# 01 tom 69 91 
	# 02 jack 71 87 
	# 03 alex 68 98
	
	# 打印第一个字段
	cut -f2 text.txt
	# 输出
	# Name
	# tom
	# jack
	# alex
	\end{lstlisting}

\section{debconf-set-selections}
	在debconf database中插入默认值。\par

	格式:
	\begin{lstlisting}
	debconf-set-selections [-vcu] [file]
	\end{lstlisting}

	参数选项如下:
	\begin{lstlisting}
	-v, --verbose     显示运行过程信息
	-c, --checkonly   只检查输入文件的格式
	-u, --unseen      在预置参数时,不设置'seen'标识
	\end{lstlisting}

	使用例子:
	\begin{lstlisting}
	# 命令行输入设置参数
	sudo debconf-set-selections <<< 'mysql-server-5.5 mysql-server/root_password password your_password'
	sudo debconf-set-selections <<< 'mysql-server-5.5 mysql-server/root_password_again password your_password'
	sudo apt-get -y install mysql-server

	# 先输入文件,再一次性添入设置参数
	echo "mysql-server-5.5 mysql-server/root_password password $MYSQL_ROOT_PASS" > /tmp/mysql.preseed
	echo "mysql-server-5.5 mysql-server/root_password_again password $MYSQL_ROOT_PASS" >> /tmp/mysql.preseed
	cat /tmp/mysql.preseed | sudo debconf-set-selections
	rm /tmp/mysql.preseed
	sudo apt-get install -y mysql-server
	\end{lstlisting}

\section{diff}
	diff命令在最简单的情况下,比较给定的两个文件的不同。如果使用“-”代替“文件”参数,则要比较的内容将来自标准输入。diff命令是以逐行的方式,比较文本文件的异同处。如果该命令指定进行目录的比较,则将会比较该目录中具有相同文件名的文件,而不会对其子目录文件进行任何比较操作。\par

	参数选项如下:
	\begin{lstlisting}
	-<行数>:指定要显示多少行的文本。此参数必须与-c或-u参数一并使用; 
	-a或——text:diff预设只会逐行比较文本文件; 
	-b或--ignore-space-change:不检查空格字符的不同; 
	-B或--ignore-blank-lines:不检查空白行; 
	-c:显示全部内容,并标出不同之处; 
	-C<行数>或--context<行数>:与执行“-c-<行数>”指令相同; 
	-d或——minimal:使用不同的演算法,以小的单位来做比较; 
	-D<巨集名称>或ifdef<巨集名称>:此参数的输出格式可用于前置处理器巨集; 
	-e或——ed:此参数的输出格式可用于ed的script文件; 
	-f或-forward-ed:输出的格式类似ed的script文件,但按照原来文件的顺序来显示不同处; 
	-H或--speed-large-files:比较大文件时,可加快速度; 
	-l<字符或字符串>或--ignore-matching-lines<字符或字符串>:若两个文件在某几行有所不同,而之际航同时都包含了选项中指定的字符或字符串,则不显示这两个文件的差异; 
	-i或--ignore-case:不检查大小写的不同; -l或——paginate:将结果交由pr程序来分页; -n或——rcs:将比较结果以RCS的格式来显示; 
	-N或--new-file:在比较目录时,若文件A仅出现在某个目录中,预设会显示:Only in目录,文件A 若使用-N参数,则diff会将文件A 与一个空白的文件比较; 
	-p:若比较的文件为C语言的程序码文件时,显示差异所在的函数名称; 
	-P或--unidirectional-new-file:与-N类似,但只有当第二个目录包含了第一个目录所没有的文件时,才会将这个文件与空白的文件做比较; 
	-q或--brief:仅显示有无差异,不显示详细的信息; 
	-r或——recursive:比较子目录中的文件; 
	-s或--report-identical-files:若没有发现任何差异,仍然显示信息; 
	-S<文件>或--starting-file<文件>:在比较目录时,从指定的文件开始比较; 
	-t或--expand-tabs:在输出时,将tab字符展开; 
	-T或--initial-tab:在每行前面加上tab字符以便对齐; 
	-u,-U<列数>或--unified=<列数>:以合并的方式来显示文件内容的不同; 
	-v或——version:显示版本信息; 
	-w或--ignore-all-space:忽略全部的空格字符; 
	-W<宽度>或--width<宽度>:在使用-y参数时,指定栏宽; 
	-x<文件名或目录>或--exclude<文件名或目录>:不比较选项中所指定的文件或目录; 
	-X<文件>或--exclude-from<文件>;您可以将文件或目录类型存成文本文件,然后在=<文件>中指定此文本文件; 
	-y或--side-by-side:以并列的方式显示文件的异同之处; 
	--help:显示帮助; 
	--left-column:在使用-y参数时,若两个文件某一行内容相同,则仅在左侧的栏位显示该行内容; 
	--suppress-common-lines:在使用-y参数时,仅显示不同之处。
	\end{lstlisting}

	使用例子:
	\begin{lstlisting}
	# 将目录/usr/li下的文件"test.txt"与当前目录下的文件"test.txt"进行比较,输入如下命令
	diff /usr/li test.txt #使用diff指令对文件进行比较

	# 上面的命令执行后,会将比较后的不同之处以指定的形式列出,如下所示
	n1 a n3,n4 
	n1,n2 d n3 
	n1,n2 c n3,n4 
	# 其中,字母"a"、"d"、"c"分别表示添加、删除及修改操作。而"n1"、"n2"表示在文件1中的行号,"n3"、"n4"表示在文件2中的行号。
	# 注意:以上说明指定了两个文件中不同处的行号及其相应的操作。在输出形式中,每一行后面将跟随受到影响的若干行。
	# 其中,以<开始的行属于文件1,以>开始的行属于文件2。
	\end{lstlisting}

\section{egrep}
	egrep命令用于在文件内查找指定的字符串。egrep执行效果与grep -E相似,使用的语法及参数可参照grep指令,与grep的不同点在于解读字符串的方法。egrep是用extended regular expression语法来解读的,而grep则用basic regular expression 语法解读,extended regular expression比basic regular expression的表达更规范。

\section{file}
	file命令用来探测给定文件的类型。file命令对文件的检查分为文件系统、魔法幻数检查和语言检查3个过程。\par

	file参数选项如下:
	\begin{lstlisting}
	-b:列出辨识结果时,不显示文件名称; 
	-c:详细显示指令执行过程,便于排错或分析程序执行的情形; 
	-f<名称文件>:指定名称文件,其内容有一个或多个文件名称时,让file依序辨识这些文件,格式为每列一个文件名称; 
	-L:直接显示符号连接所指向的文件类别; 
	-m<魔法数字文件>:指定魔法数字文件; 
	-v:显示版本信息; 
	-z:尝试去解读压缩文件的内容。
	\end{lstlisting}

	使用例子如下:
	\begin{lstlisting}
	# 查看initrd.img文件类型
	file initrd.img
	\end{lstlisting}

\section{find}
	find命令用来在指定目录下查找文件。任何位于参数之前的字符串都将被视为欲查找的目录名。如果使用该命令时,不设置任何参数,则find命令将在当前目录下查找子目录与文件。并且将查找到的子目录和文件全部进行显示。\par

	find参数选项如下:
	\begin{lstlisting}
	-amin<分钟>:查找在指定时间曾被存取过的文件或目录,单位以分钟计算; 
	-anewer<参考文件或目录>:查找其存取时间较指定文件或目录的存取时间更接近现在的文件或目录; 
	-atime<24小时数>:查找在指定时间曾被存取过的文件或目录,单位以24小时计算; 
	-cmin<分钟>:查找在指定时间之时被更改过的文件或目录; 
	-cnewer<参考文件或目录>查找其更改时间较指定文件或目录的更改时间更接近现在的文件或目录; 
	-ctime<24小时数>:查找在指定时间之时被更改的文件或目录,单位以24小时计算; 
	-daystart:从本日开始计算时间; 
	-depth:从指定目录下最深层的子目录开始查找; 
	-expty:寻找文件大小为0 Byte的文件,或目录下没有任何子目录或文件的空目录; 
	-exec<执行指令>:假设find指令的回传值为True,就执行该指令; 
	-false:将find指令的回传值皆设为False; 
	-fls<列表文件>:此参数的效果和指定“-ls”参数类似,但会把结果保存为指定的列表文件; 
	-follow:排除符号连接; 
	-fprint<列表文件>:此参数的效果和指定“-print”参数类似,但会把结果保存成指定的列表文件; 
	-fprint0<列表文件>:此参数的效果和指定“-print0”参数类似,但会把结果保存成指定的列表文件; 
	-fprintf<列表文件><输出格式>:此参数的效果和指定“-printf”参数类似,但会把结果保存成指定的列表文件; 
	-fstype<文件系统类型>:只寻找该文件系统类型下的文件或目录; 
	-gid<群组识别码>:查找符合指定之群组识别码的文件或目录; 
	-group<群组名称>:查找符合指定之群组名称的文件或目录; 
	-help或——help:在线帮助; 
	-ilname<范本样式>:此参数的效果和指定“-lname”参数类似,但忽略字符大小写的差别; 
	-iname<范本样式>:此参数的效果和指定“-name”参数类似,但忽略字符大小写的差别; 
	-inum:查找符合指定的inode编号的文件或目录; 
	-ipath<范本样式>:此参数的效果和指定“-path”参数类似,但忽略字符大小写的差别; 
	-iregex<范本样式>:此参数的效果和指定“-regexe”参数类似,但忽略字符大小写的差别; 
	-links<连接数目>:查找符合指定的硬连接数目的文件或目录; 
	-iname<范本样式>:指定字符串作为寻找符号连接的范本样式; 
	-ls:假设find指令的回传值为True,就将文件或目录名称列出到标准输出; 
	-maxdepth<目录层级>:设置最大目录层级; 
	-mindepth<目录层级>:设置最小目录层级; 
	-mmin<分钟>:查找在指定时间曾被更改过的文件或目录,单位以分钟计算; 
	-mount:此参数的效果和指定“-xdev”相同; 
	-mtime<24小时数>:查找在指定时间曾被更改过的文件或目录,单位以24小时计算; 
	-name<范本样式>:指定字符串作为寻找文件或目录的范本样式; 
	-newer<参考文件或目录>:查找其更改时间较指定文件或目录的更改时间更接近现在的文件或目录; 
	-nogroup:找出不属于本地主机群组识别码的文件或目录; 
	-noleaf:不去考虑目录至少需拥有两个硬连接存在; 
	-nouser:找出不属于本地主机用户识别码的文件或目录; 
	-ok<执行指令>:此参数的效果和指定“-exec”类似,但在执行指令之前会先询问用户,若回答“y”或“Y”,则放弃执行命令; 
	-path<范本样式>:指定字符串作为寻找目录的范本样式; 
	-perm<权限数值>:查找符合指定的权限数值的文件或目录; 
	-print:假设find指令的回传值为True,就将文件或目录名称列出到标准输出。格式为每列一个名称,每个名称前皆有“./”字符串; 
	-print0:假设find指令的回传值为True,就将文件或目录名称列出到标准输出。格式为全部的名称皆在同一行; -printf<输出格式>:假设find指令的回传值为True,就将文件或目录名称列出到标准输出。格式可以自行指定; -prune:不寻找字符串作为寻找文件或目录的范本样式; -regex<范本样式>:指定字符串作为寻找文件或目录的范本样式; -size<文件大小>:查找符合指定的文件大小的文件; -true:将find指令的回传值皆设为True; -typ<文件类型>:只寻找符合指定的文件类型的文件; -uid<用户识别码>:查找符合指定的用户识别码的文件或目录; -used<日数>:查找文件或目录被更改之后在指定时间曾被存取过的文件或目录,单位以日计算; 
	-user<拥有者名称>:查找符和指定的拥有者名称的文件或目录; 
	-version或——version:显示版本信息; 
	-xdev:将范围局限在先行的文件系统中; 
	-xtype<文件类型>:此参数的效果和指定“-type”参数类似,差别在于它针对符号连接检查。
	\end{lstlisting}

	使用例子如下:
	\begin{lstlisting}
	# 在/home目录下查找以.txt结尾的文件名
	find /home -name "*.txt"
	\end{lstlisting}

\section{getent}
	从管理数据库取得条目。

	用法: 
	\begin{lstlisting}
	getent [选项...] 数据库 [键 ...]
	\end{lstlisting}
	
	参数选项:
	\begin{lstlisting}
	-i, --no-idn               停用 IDN 编码
	-s, --service=组态       要使用的服务配置
	-?, --help                 给出此帮助列表
		--usage                给出简要的用法信息
	-V, --version              打印程序版本号
	\end{lstlisting}

	支持的数据库:
	\begin{lstlisting}
	ahosts ahostsv4 ahostsv6 aliases ethers group gshadow hosts initgroups
	netgroup networks passwd protocols rpc services shadow
	\end{lstlisting}

\section{grep}
	grep(global search regular expression(RE) and print out the line,全面搜索正则表达式并把行打印出来)是一种强大的文本搜索工具,它能使用正则表达式搜索文本,并把匹配的行打印出来。

	grep参数选项如下:
	\begin{lstlisting}
	-a 不要忽略二进制数据。 
	-A <显示列数> 除了显示符合范本样式的那一行之外,并显示该行之后的内容。 
	-b 在显示符合范本样式的那一行之外,并显示该行之前的内容。 
	-c 计算符合范本样式的列数。 
	-C <显示列数>或-<显示列数> 除了显示符合范本样式的那一列之外,并显示该列之前后的内容。 
	-d <进行动作> 当指定要查找的是目录而非文件时,必须使用这项参数,否则grep命令将回报信息并停止动作。 
	-e <范本样式> 指定字符串作为查找文件内容的范本样式。 
	-E 将范本样式为延伸的普通表示法来使用,意味着使用能使用扩展正则表达式。 
	-f <范本文件> 指定范本文件,其内容有一个或多个范本样式,让grep查找符合范本条件的文件内容,格式为每一列的范本样式。 
	-F 将范本样式视为固定字符串的列表。 
	-G 将范本样式视为普通的表示法来使用。 
	-h 在显示符合范本样式的那一列之前,不标示该列所属的文件名称。 
	-H 在显示符合范本样式的那一列之前,标示该列的文件名称。 
	-i 忽略字符大小写的差别。 
	-l 列出文件内容符合指定的范本样式的文件名称。 
	-L 列出文件内容不符合指定的范本样式的文件名称。 
	-n 在显示符合范本样式的那一列之前,标示出该列的编号。 
	-q 不显示任何信息。 
	-R/-r 此参数的效果和指定“-d recurse”参数相同。 
	-s 不显示错误信息。 
	-v 反转查找。 
	-w 只显示全字符合的列。 
	-x 只显示全列符合的列。 
	-y 此参数效果跟“-i”相同。 
	-o 只输出文件中匹配到的部分。
	\end{lstlisting}

\section{groupadd}
	groupadd命令用于创建一个新的工作组,新工作组的信息将被添加到系统文件中。\par

	格式:
	\begin{lstlisting}
	groupadd 组名
	\end{lstlisting}

	参数选项如下:
	\begin{lstlisting}
	-g:指定新建工作组的id; 
	-r:创建系统工作组,系统工作组的组ID小于500; 
	-K:覆盖配置文件“/ect/login.defs”; 
	-o:允许添加组ID号不唯一的工作组。
	\end{lstlisting}

\section{gunzip}
	gunzip命令用来解压缩文件。gunzip是个使用广泛的解压缩程序,它用于解开被gzip压缩过的文件,这些压缩文件预设最后的扩展名为.gz。事实上gunzip就是gzip的硬连接,因此不论是压缩或解压缩,都可通过gzip指令单独完成。\par

	gunzip参数选项如下;
	\begin{lstlisting}
	-a或——ascii:使用ASCII文字模式; 
	-c或--stdout或--to-stdout:把解压后的文件输出到标准输出设备; 
	-f或-force:强行解开压缩文件,不理会文件名称或硬连接是否存在以及该文件是否为符号连接; 
	-h或——help:在线帮助; 
	-l或——list:列出压缩文件的相关信息; 
	-L或——license:显示版本与版权信息; 
	-n或--no-name:解压缩时,若压缩文件内含有原来的文件名称及时间戳记,则将其忽略不予处理; 
	-N或——name:解压缩时,若压缩文件内含有原来的文件名称及时间戳记,则将其回存到解开的文件上; 
	-q或——quiet:不显示警告信息; 
	-r或——recursive:递归处理,将指定目录下的所有文件及子目录一并处理; 
	-S或<压缩字尾字符串>或----suffix<压缩字尾字符串>:更改压缩字尾字符串; 
	-t或——test:测试压缩文件是否正确无误; 
	-v或——verbose:显示指令执行过程; 
	-V或——version:显示版本信息;
	\end{lstlisting}

	使用例子如下:
	\begin{lstlisting}
	# 解压etc.zip.gz文件
	gunzip /opt/etc.zip.gz
	# 查看etc.zip.gz文件的压缩信息
	gunzip -l /opt/etc.zip.gz
	\end{lstlisting}

\section{head}
	head命令用于显示文件的开头的内容。在默认情况下,head命令显示文件的头10行内容。\par

	head参数选项如下:
	\begin{lstlisting}
	-n<数字>:指定显示头部内容的行数; 
	-c<字符数>:指定显示头部内容的字符数; 
	-v:总是显示文件名的头信息; 
	-q:不显示文件名的头信息。
	\end{lstlisting}

\section{insmod}
	insmod命令用于将给定的模块加载到内核中。
	Linux有许多功能是通过模块的方式,在需要时才载入kernel。
	如此可使kernel较为精简,进而提高效率,以及保有较大的弹性。这类可载入的模块,通常是设备驱动程序。\par

	选项:
	\begin{lstlisting}
	-f:不检查目前kernel版本与模块编译时的kernel版本是否一致,强制将模块载入; 
	-k:将模块设置为自动卸除; 
	-m:输出模块的载入信息; 
	-o <模块名称>:指定模块的名称,可使用模块文件的文件名; 
	-p:测试模块是否能正确地载入kernel; 
	-s:将所有信息记录在系统记录文件中; 
	-v:执行时显示详细的信息;
	-x:不要汇出模块的外部符号; 
	-X:汇出模块所有的外部符号,此为预设置。
	\end{lstlisting}

\section{install}
	install命令的作用是安装或升级软件或备份数据,它的使用权限是所有用户。
	install命令和cp命令类似,都可以将文件/目录拷贝到指定的地点。
	但是,install允许你控制目标文件的属性。
	install通常用于程序的makefile,使用它来将程序拷贝到目标(安装)目录。\par

	格式:
	\begin{lstlisting}
	install [OPTION]... [-T] SOURCE DEST 
	install [OPTION]... SOURCE... DIRECTORY 
	install [OPTION]... -t DIRECTORY SOURCE... 
	install [OPTION]... -d DIRECTORY...
	\end{lstlisting}

	参数选项如下:
	\begin{lstlisting}
	--backup[=CONTROL]:为每个已存在的目的地文件进行备份。 
	-b:类似 --backup,但不接受任何参数。 
	-c:(此选项不作处理)。 
	-d,--directory:所有参数都作为目录处理,而且会创建指定目录的所有主目录。 
	-D:创建<目的地>前的所有主目录,然后将<来源>复制至 <目的地>;在第一种使用格式中有用。 
	-g,--group=组:自行设定所属组,而不是进程目前的所属组。 
	-m,--mode=模式:自行设定权限模式 (像chmod),而不是rwxr-xr-x。 
	-o,--owner=所有者:自行设定所有者 (只适用于超级用户)。 
	-p,--preserve-timestamps:以<来源>文件的访问/修改时间作为相应的目的地文件的时间属性。 
	-s,--strip:用strip命令删除symbol table,只适用于第一及第二种使用格式。 
	-S,--suffix=后缀:自行指定备份文件的<后缀>。 
	-v,--verbose:处理每个文件/目录时印出名称。 
	--help:显示此帮助信息并离开。 
	--version:显示版本信息并离开。
	\end{lstlisting}

	使用例子如下:
	\begin{lstlisting}
	install -d a/b/c e/f # 类似于mkdir -p a/b/c e/f
	install a/e c # 类似于cp a/e c
	install -D x a/b/c # 类似于mkdir -p a/b && cp x a/b/c
	\end{lstlisting}

\section{ip}
	ip命令用来显示或操纵Linux主机的路由、网络设备、策略路由和隧道,是Linux下较新的功能强大的网络配置工具。\par

	ip参数选项如下:
	\begin{lstlisting}
	-V:显示指令版本信息; 
	-s:输出更详细的信息; 
	-f:强制使用指定的协议族; 
	-4:指定使用的网络层协议是IPv4协议; 
	-6:指定使用的网络层协议是IPv6协议; 
	-0:输出信息每条记录输出一行,即使内容较多也不换行显示; 
	-r:显示主机时,不使用IP地址,而使用主机的域名。
	\end{lstlisting}

	ip参数如下:
	\begin{lstlisting}
	网络对象:指定要管理的网络对象; 
	具体操作:对指定的网络对象完成具体操作; 
	help:显示网络对象支持的操作命令的帮助信息。
	\end{lstlisting}

	使用例子如下:
	\begin{lstlisting}
	# 用ip命令显示网络设备的运行状态
	# link是网络对象,list是具体操作
	ip link list
	\end{lstlisting}

	网络对象如下:
	\begin{lstlisting}
	link:网络设备
	address:一个设备的协议(IP或者IPV6)地址
	neighbour:ARP或者NDISC缓冲区条目
	route:路由表条目
	rule:路由策略数据库中的规则
	maddress:多播地址
	mroute:多播路由缓冲区条目
	tunnel:IP上的通道
	netns:manage network namespaces.
	\end{lstlisting}

	常用操作:
	\begin{lstlisting}
	ip link set [device] [动作与参数]
	参数:
		up|down :启动 (up) 或关闭 (down) 某个设备,其他参数使用预设的以太网路;
		address :如果这个装置可以更改 MAC 的话,用这个参数修改!
		name     :给予这个装置一个特殊的名字;
		mtu      :就是最大传输单元啊!
	\end{lstlisting}

\section{ln}
	ln命令用来为文件创件连接,连接类型分为硬连接和符号连接两种,默认的连接类型是硬连接。如果要创建符号连接必须使用"-s"选项。 \par
	注意:符号链接文件不是一个独立的文件,它的许多属性依赖于源文件,所以给符号链接文件设置存取权限是没有意义的。\par

	格式:
	\begin{lstlisting}
	ln 源文件 目标文件
	\end{lstlisting}

	ln参数选项如下:
	\begin{lstlisting}
	-b或--backup:删除,覆盖目标文件之前的备份; 
	-d或-F或——directory:建立目录的硬连接; 
	-f或——force:强行建立文件或目录的连接,不论文件或目录是否存在; 
	-i或——interactive:覆盖既有文件之前先询问用户; 
	-n或--no-dereference:把符号连接的目的目录视为一般文件; 
	-s或——symbolic:对源文件建立符号连接,而非硬连接; 
	-S<字尾备份字符串>或--suffix=<字尾备份字符串>:用"-b"参数备份目标文件后,备份文件的字尾会被加上一个备份字符串,预设的备份字符串是符号“~”,用户可通过“-S”参数来改变它; 
	-v或——verbose:显示指令执行过程; 
	-V<备份方式>或--version-control=<备份方式>:用“-b”参数备份目标文件后,备份文件的字尾会被加上一个备份字符串,这个字符串不仅可用“-S”参数变更,当使用“-V”参数<备份方式>指定不同备份方式时,也会产生不同字尾的备份字符串; 
	--help:在线帮助; 
	--version:显示版本信息。
	\end{lstlisting}

	使用例子如下:
	\begin{lstlisting}
	# 将目录/usr/mengqc/mub1下的文件m2.c链接到目录/usr/liu下的文件a2.c
	ln /mub1/m2.c /usr/liu/a2.c
	# 在执行ln命令之前,目录/usr/liu中不存在a2.c文件。执行ln之后,在/usr/liu目录中才有a2.c这一项,表明m2.c和a2.c链接起来(注意,二者在物理上是同一文件),利用ls -l命令可以看到链接数的变化
	\end{lstlisting}

\section{make}
	参数选项:
	\begin{lstlisting}
-b 忽略兼容性 

-B 无条件make所有目标 

-C dir或者--directory=DIR 
   在读取makefile文件前,先切换到"dir"目录下,即把dir当作为当前目录。如果存在多个-C选项,make的最终当前目录是第一个目录的相对路径,如“make -C /home/root -C src”,等价于"make -C /home/root/src" 

-d make执行时打印出所有的调试信息。包括:make认为那些需要重新生成的文件;那些需要比较它们的最后修改时间的文件,比较的结果;重新生成目标所要执行的命令;使用的隐含规则等。 

-e 或者--environment-override 
不允许在Makefile中对系统环境变量进行重新赋值 

-f filename 或者 --file=File 或者 --makefile=File 
使用指定的文件作为Makefile文件 

-i 或者 --ingore-errors 
忽略执行Makefile中命令时产生的错误,不退出make 

-h 或者 -help 
打印出帮助信息 

-k 或者 --keep-going 
执行命令遇到错误时不终止make的执行,make尽最大可能执行所有的命令,直到出现致命错误才终止 

-n 或者 --just-print 或者 --dry-run 
只打印出要执行的命令,但不执行命令 

-o filename 或者 --old-file=File 
指定文件“filename”不需要重建,即使相对于它的依赖已经过时,同时也不重建依赖于此文件的任何目标文件 

-O[type], --output-sync[=type]
	When running multiple jobs in parallel with -j, ensure the  output
	of  each  job  is collected together rather than interspersed with
	output from other jobs.  If type is not specified or is target the
	output from the entire recipe for each target is grouped together.
	If type is line the output from each command line within a  recipe
	is  grouped  together.   If  type is recurse output from an entire
	recursive make is grouped together.  If type is none  output  syn‐
	chronization is disabled.

-p 或者 --print-data-base 
命令执行之前,打印出make读取的Makfile的所有数据(包括规则和变量的值),同时打印出make的版本信息。如果只需要打印这些数据信息矶不执行命令,可以使用“make -qp”命令。查看make执行的隐含规则和预定义变量,可以使用命令“make -p-f /dev/null”。 

-q 或者 -question 
称为“询问模式”,不执行任何命令。make只是返回一个查询状态值,返回的状态值为0表示没有目标需要重建,1表示存在需要重建的目标,2表示有错误发生。 

-r 或者--no-builtin-rules 
忽略隐规则,使之不起作用。该选项不会取消make内嵌的预定义变量。 

-R 或者 --no-builtin-variabes 
取消make内嵌的预定义变量,不过我们可以在makefile中明确定义某些变量。注意,-R选项同时打开-r选项,因为没有预定义变量,隐含规则将失去意义(隐含规则是以内嵌的预定义变量为基础的) 

-s 或者 -silent 
执行但不显示所执行的命令 

-t 或者 -touch 
把所有目标文件的最后修改时间设置为当前系统时间 

-v 或者 -version 
打印出make的版本信息
	\end{lstlisting}

\section{mknod}
	mknod命令用于创建Linux中的字符设备文件和块设备文件。\par

	格式:
	\begin{lstlisting}
	mknod [options] 文件名 类型 主设备号 次设备号
	\end{lstlisting}

	选项:
	\begin{lstlisting}
	-Z:设置安全的上下文; 
	-m:设置权限模式; 
	-help:显示帮助信息; 
	--version:显示版本信息。
	\end{lstlisting}

\section{mount}
	mount命令用于加载文件系统到指定的加载点。此命令的最常用于挂载cdrom,使我们可以访问cdrom中的数据,因为你将光盘插入cdrom中,Linux并不会自动挂载,必须使用Linux mount命令来手动完成挂载。\par

	mount参数选项如下:
	\begin{lstlisting}
	-V:显示程序版本; 
	-l:显示已加载的文件系统列表; 
	-h:显示帮助信息并退出; 
	-v:冗长模式,输出指令执行的详细信息; 
	-n:加载没有写入文件“/etc/mtab”中的文件系统; 
	-r:将文件系统加载为只读模式; 
	-o:选项用于指定挂载的分区有哪些特性;
	-a:加载文件“/etc/fstab”中描述的所有文件系统。
	\end{lstlisting}

	例子如下:
	\begin{lstlisting}
	# 在目录上挂载包含文件系统的文件
	# /dev/loop可以使得文件如同块设备一般被访问
	mount -o loop example.img /home/you/dir
	\end{lstlisting}

	在目录上挂载包含文件系统的文件一般需要两步:1.用一个循环设备节点连接文件。2.在目录上挂载该循环设备。
	这个挂载的实现除了使用上述的mount命令,还可以通过两条命令来完成:
	\begin{lstlisting}
	losetup /dev/loop0 example.img
	mount /dev/loop0 /home/you/dir
	\end{lstlisting}

\section{netstat}
	netstat命令用来打印Linux中网络系统的状态信息,可让你得知整个Linux系统的网络情况。\par

	netstat参数选项如下:
	\begin{lstlisting}
	-a或--all:显示所有连线中的Socket; 
	-A<网络类型>或--<网络类型>:列出该网络类型连线中的相关地址; 
	-c或--continuous:持续列出网络状态; 
	-C或--cache:显示路由器配置的快取信息; 
	-e或--extend:显示网络其他相关信息; 
	-F或--fib:显示FIB; 
	-g或--groups:显示多重广播功能群组组员名单; 
	-h或--help:在线帮助; 
	-i或--interfaces:显示网络界面信息表单; 
	-l或--listening:显示监控中的服务器的Socket; 
	-M或--masquerade:显示伪装的网络连线; 
	-n或--numeric:直接使用ip地址,而不通过域名服务器; 
	-N或--netlink或--symbolic:显示网络硬件外围设备的符号连接名称; 
	-o或--timers:显示计时器; 
	-p或--programs:显示正在使用Socket的程序识别码和程序名称; 
	-r或--route:显示Routing Table; 
	-s或--statistice:显示网络工作信息统计表; 
	-t或--tcp:显示TCP传输协议的连线状况; 
	-u或--udp:显示UDP传输协议的连线状况; 
	-v或--verbose:显示指令执行过程; 
	-V或--version:显示版本信息; 
	-w或--raw:显示RAW传输协议的连线状况; 
	-x或--unix:此参数的效果和指定"-A unix"参数相同; 
	--ip或--inet:此参数的效果和指定"-A inet"参数相同。
	\end{lstlisting}

	

\subsection{查看连接某服务端口最多的IP地址}
	\begin{lstlisting}
	netstat -ntu | grep :80 | awk '{print $5}' | cut -d: -f1 | awk '{++ip[$1]} END {for(i in ip) print ip[i],"\t",i}' | sort -nr
	\end{lstlisting}

\subsection{获得TCP各种状态列表}
	\begin{lstlisting}
	netstat -nt | grep -e 127.0.0.1 -e 0.0.0.0 -e ::: -v | awk '/^tcp/ {++state[$NF]} END {for(i in state) print i,"\t",state[i]}'
	\end{lstlisting}

\subsection{查看phpcgi进程数}
	\begin{lstlisting}
	netstat -anpo | grep "php-cgi" | wc -l
	\end{lstlisting}

\section{od}
	od命令用于输出文件的八进制、十六进制或其它格式编码的字节,通常用于显示或查看文件中不能直接显示在终端的字符。\par
	常见的文件为文本文件和二进制文件。此命令主要用来查看保存在二进制文件中的值。比如,程序可能输出大量的数据记录,每个数据是一个单精度浮点数。这些数据记录存放在一个文件中,如果想查看下这个数据,这时候od命令就派上用场了。在我看来,od命令主要用来格式化输出文件数据,即对文件中的数据进行无二义性的解释。不管是IEEE754格式的浮点数还是ASCII码,od命令都能按照需求输出它们的值。\par

	od参数选项如下:
	\begin{lstlisting}
	-a:此参数的效果和同时指定“-ta”参数相同; 
	-A:<字码基数>:选择以何种基数计算字码; 
	-b:此参数的效果和同时指定“-toc”参数相同; 
	-c:此参数的效果和同时指定“-tc”参数相同,select printable characters or backslash escapes 
	-d:此参数的效果和同时指定“-tu2”参数相同; 
	-f:此参数的效果和同时指定“-tfF”参数相同; 
	-h:此参数的效果和同时指定“-tx2”参数相同; 
	-i:此参数的效果和同时指定“-td2”参数相同; 
	-j<字符数目>或--skip-bytes=<字符数目>:略过设置的字符数目; 
	-l:此参数的效果和同时指定“-td4”参数相同; 
	-N<字符数目>或--read-bytes=<字符数目>:到设置的字符树目为止; 
	-o:此参数的效果和同时指定“-to2”参数相同; 
	-s<字符串字符数>或--strings=<字符串字符数>:只显示符合指定的字符数目的字符串; 
	-t<输出格式>或--format=<输出格式>:设置输出格式; 
	-v或--output-duplicates:输出时不省略重复的数据; 
	-w<每列字符数>或--width=<每列字符数>:设置每列的最大字符数; 
	-x:此参数的效果和同时指定“-h”参数相同; 
	--help:在线帮助; 
	--version:显示版本信息。
	\end{lstlisting}

	使用例子如下:
	\begin{lstlisting}
	# 使用单字节十进制进行解释
	od -t d1 tmp
	# 设置格式为十进制
	od -A d -c tmp
	# 每行输出两个字节
	od -w2 -c tmp
	\end{lstlisting}

\section{pop}
	popd命令用于删除目录栈中的记录;如果popd命令不加任何参数,则会先删除目录栈最上面的记录,然后切换到删除过后的目录栈中的最上面的目录。\par

	格式:
	\begin{lstlisting}
	+N:将第N个目录删除(从左边数起,数字从0开始); 
	-N:将第N个目录删除(从右边数起,数字从0开始); 
	-n:将目录出栈时,不切换目录。
	\end{lstlisting}

\section{pushd}
	pushd命令是将目录加入命令堆叠中。如果指令没有指定目录名称,则会将当前的工作目录置入目录堆叠的最顶端。置入目录如果没有指定堆叠的位置,也会置入目录堆叠的最顶端,同时工作目录会自动切换到目录堆叠最顶端的目录去。\par

	格式:
	\begin{lstlisting}
	pushd 目录
	\end{lstlisting}

	参数选项如下:
	\begin{lstlisting}
	-n:只加入目录到堆叠中,不进行cd操作; 
	+n:删除从左到右的第n个目录,数字从0开始; 
	-n:删除从右到左的第n个目录,数字从0开始;
	\end{lstlisting}

\section{readlink}
	readlink是linux系统中一个常用工具,主要用来找出符号链接所指向的位置。\par

	参数选项:
	\begin{lstlisting}
	-f 选项:
	-f 选项可以递归跟随给出文件名的所有符号链接以标准化,除最后一个外所有组件必须存在。
	简单地说,就是一直跟随符号链接,直到直到非符号链接的文件位置,限制是最后必须存在一个非符号链接的文件。
	\end{lstlisting}

\section{scp}
	scp命令用于在Linux下进行远程拷贝文件的命令,和它类似的命令有cp,不过cp只是在本机进行拷贝不能跨服务器,而且scp传输是加密的。可能会稍微影响一下速度。当你服务器硬盘变为只读read only system时,用scp可以帮你把文件移出来。另外,scp还非常不占资源,不会提高多少系统负荷,在这一点上,rsync就远远不及它了。虽然 rsync比scp会快一点,但当小文件众多的情况下,rsync会导致硬盘I/O非常高,而scp基本不影响系统正常使用。\par

	scp参数选项如下:
	\begin{lstlisting}
	-1:使用ssh协议版本1; 
	-2:使用ssh协议版本2; 
	-4:使用ipv4; 
	-6:使用ipv6; 
	-B:以批处理模式运行; 
	-C:使用压缩; 
	-F:指定ssh配置文件; 
	-l:指定宽带限制; 
	-o:指定使用的ssh选项; 
	-P:指定远程主机的端口号; 
	-p:保留文件的最后修改时间,最后访问时间和权限模式; 
	-q:不显示复制进度; 
	-r:以递归方式复制。
	\end{lstlisting}

	使用例子如下:
	\begin{lstlisting}
	# 从远程机器复制文件到本地目录
	scp pengsida@172.16.19.223:/home/pengsida/devstack/local.conf ~/local.conf
	# 上传本地文件到远程机器指定目录
	scp ~/local.conf pengsida@172.16.19.223:/home/pengsida/devstack/local.conf
	# 上传本地目录到远程机器指定目录
	scp -r ~/devstack pengsida@172.16.19.223:/home/pengsida/devstack
	\end{lstlisting}

\section{screen}
	screen screen命令常用工具命令 Screen是一款由GNU计划开发的用于命令行终端切换的自由软件。用户可以通过该软件同时连接多个本地或远程的命令行会话,并在其间自由切换。GNU Screen可以看作是窗口管理器的命令行界面版本。它提供了统一的管理多个会话的界面和相应的功能。

\subsection{会话恢复}
	只要Screen本身没有终止,在其内部运行的会话都可以恢复。
	这一点对于远程登录的用户特别有用——即使网络连接中断,用户也不会失去对已经打开的命令行会话的控制。
	只要再次登录到主机上执行screen -r就可以恢复会话的运行。
	同样在暂时离开的时候,也可以执行分离命令detach,在保证里面的程序正常运行的情况下让Screen挂起(切换到后台)。
	这一点和图形界面下的VNC很相似。

\subsection{多窗口}
	在Screen环境下,所有的会话都独立的运行,并拥有各自的编号、输入、输出和窗口缓存。
	用户可以通过快捷键在不同的窗口下切换,并可以自由的重定向各个窗口的输入和输出。
	Screen实现了基本的文本操作,如复制粘贴等;还提供了类似滚动条的功能,可以查看窗口状况的历史记录。
	窗口还可以被分区和命名,还可以监视后台窗口的活动。 
	会话共享 Screen可以让一个或多个用户从不同终端多次登录一个会话,
	并共享会话的所有特性(比如可以看到完全相同的输出)。
	它同时提供了窗口访问权限的机制,可以对窗口进行密码保护。

\subsection{使用}
	格式:
	\begin{lstlisting}
	screen [-AmRvx -ls -wipe][-d <作业名称>][-h <行数>][-r <作业名称>][-s ][-S <作业名称>]
	\end{lstlisting}

	参数选项如下:
	\begin{lstlisting}
	-A  将所有的视窗都调整为目前终端机的大小。 
	-d <作业名称>  将指定的screen作业离线。 
	-h <行数>  指定视窗的缓冲区行数。 
	-m  即使目前已在作业中的screen作业,仍强制建立新的screen作业。 
	-r <作业名称>  恢复离线的screen作业。 
	-R  先试图恢复离线的作业。若找不到离线的作业,即建立新的screen作业。 
	-s  指定建立新视窗时,所要执行的shell。 
	-S <作业名称>  指定screen作业的名称。 
	-v  显示版本信息。 
	-x  恢复之前离线的screen作业。 
	-X   Execute <cmd> as a screen command in the specified session.
	-t title  Set title. (window's name).
	-ls或--list  显示目前所有的screen作业。 
	-wipe  检查目前所有的screen作业,并删除已经无法使用的screen作业。
	setenv [var [string]] Set the environment variable var to value string
	hardstatus string [string] 	根据字符串设置状态栏的样式
	-p number_or_name|-|=|+		 Preselect  a window. This is useful when you want to reattach to a specific window or you want to send a command via the "-X"  option to a specific window.
	logfile filename	 Defines the name the log files will get.
	log [on|off]	Start/stop writing output of the current window to its log file
	stuff [string]	Stuff the string string in the input  buffer  of  the  current  window. This  is like the "paste" command but with much less overhead.
	\end{lstlisting}

	使用例子如下:
	\begin{lstlisting}
	screen -S yourname -> 新建一个叫yourname的session 
	screen -ls -> 列出当前所有的session 
	screen -r yourname -> 回到yourname这个session 
	screen -d yourname -> 远程detach某个session 
	screen -d -r yourname -> 结束当前session并回到yourname这个session
	\end{lstlisting}

	在每个screen session下,所有命令都以ctrl+a开始:
	\begin{lstlisting}
	C-a ? -> 显示所有键绑定信息 
	C-a c -> 创建一个新的运行shell的窗口并切换到该窗口 
	C-a n -> Next,切换到下一个 window 
	C-a p -> Previous,切换到前一个 window 
	C-a 0..9 -> 切换到第 0..9 个 window 
	Ctrl+a [Space] -> 由视窗0循序切换到视窗9 
	C-a C-a -> 在两个最近使用的 window 间切换 
	C-a x -> 锁住当前的 window,需用用户密码解锁 
	C-a d -> detach,暂时离开当前session,将目前的 screen session (可能含有多个 windows) 丢到后台执行,并会回到还没进 screen 时的状态,此时在 screen session 里,每个 window 内运行的 process (无论是前台/后台)都在继续执行,即使 logout 也不影响。 
	C-a z -> 把当前session放到后台执行,用 shell 的 fg 命令则可回去。 
	C-a w -> 显示所有窗口列表 
	C-a t -> time,显示当前时间,和系统的 load 
	C-a k -> kill window,强行关闭当前的 window 
	C-a [ -> 进入 copy mode,在 copy mode 下可以回滚、搜索、复制就像用使用 vi 一样 
	C-b Backward,PageUp 
	C-f Forward,PageDown 
	H(大写) High,将光标移至左上角 
	L Low,将光标移至左下角 
	0 移到行首 
	$ 行末 \
	w forward one word,以字为单位往前移 
	b backward one word,以字为单位往后移 
	Space 第一次按为标记区起点,第二次按为终点 
	Esc 结束 copy mode 
	C-a ] -> paste,把刚刚在 copy mode 选定的内容贴上
	\end{lstlisting}

\subsection{设置hardstatus}
	具体字符串的意思可以在“man screen”中的“STRING ESCAPES”找到。\par

	参考网站:\par
	\url{https://havee.me/linux/2011-08/screen-status-bar.html}\par
	\url{http://www.gnu.org/software/screen/manual/screen.html#Message-Line}\par
	\url{https://www.zhangguangtong.cn/?p=66}

\section{set}
	格式:
	\begin{lstlisting}
	set: 用法: set [--abefhkmnptuvxBCHP] [-o 选项名] [--] [参数 ...]
	\end{lstlisting}

	选项:
	\begin{lstlisting}
	-a:标示已修改的变量,以供输出至环境变量。 
	-b:使被中止的后台程序立刻回报执行状态。 
	-C:转向所产生的文件无法覆盖已存在的文件。 
	-d:Shell预设会用杂凑表记忆使用过的指令,以加速指令的执行。使用-d参数可取消。 
	-e:若指令传回值不等于0,则立即退出shell。 
	-f:取消使用通配符。 
	-h:自动记录函数的所在位置。 
	-H Shell:可利用"!"加<指令编号>的方式来执行history中记录的指令。 
	-k:指令所给的参数都会被视为此指令的环境变量。 
	-l:记录for循环的变量名称。 
	-m:使用监视模式。 
	-n:只读取指令,而不实际执行。 
	-p:启动优先顺序模式。 
	-P:启动-P参数后,执行指令时,会以实际的文件或目录来取代符号连接。 
	-t:执行完随后的指令,即退出shell。 
	-u:当执行时使用到未定义过的变量,则显示错误信息。 
	-v:显示shell所读取的输入值。 
	-x:执行指令后,会先显示该指令及所下的参数。
	+<参数>  取消某个set曾启动的参数
	\end{lstlisting}

\section{stat}
	stat命令用于显示文件的状态信息。stat命令的输出信息比ls命令的输出信息要更详细。\par

	格式:
	\begin{lstlisting}
	stat 文件
	\end{lstlisting}

	stat参数选项如下:
	\begin{lstlisting}
	-L:支持符号连接; 
	-f:显示文件系统状态而非文件状态; 
	-t:以简洁方式输出信息; 
	--help:显示指令的帮助信息; 
	--version:显示指令的版本信息。
	\end{lstlisting}

	使用例子如下:
	\begin{lstlisting}
	[root@localhost ~]# ls -l myfile 
	-rw-r--r-- 1 root root 0 2010-10-09 myfile 

	[root@localhost ~]# stat myfile 
	
	[root@localhost ~]# stat -t myfile 
	\end{lstlisting}

\section{strings}
	strings命令在对象文件或二进制文件中查找可打印的字符串。字符串是4个或更多可打印字符的任意序列,以换行符或空字符结束。 strings命令对识别随机对象文件很有用。\par

	strings参数选项如下:
	\begin{lstlisting}
	-a --all:扫描整个文件而不是只扫描目标文件初始化和装载段 
	-f –print-file-name:在显示字符串前先显示文件名 
	-n –bytes=[number]:找到并且输出所有NUL终止符序列 
	- :设置显示的最少的字符数,默认是4个字符 
	-t --radix={o,d,x} :输出字符的位置,基于八进制,十进制或者十六进制 
	-o :类似--radix=o 
	-T --target= :指定二进制文件格式 -e --encoding={s,S,b,l,B,L} :选择字符大小和排列顺序:s = 7-bit, S = 8-bit, {b,l} = 16-bit, {B,L} = 32-bit
	\end{lstlisting}

	使用例子如下:
	\begin{lstlisting}
	# 列出ls中所有的ASCII文本
	strings /bin/ls
	\end{lstlisting}

\section{su}
	su命令用于切换当前用户身份到其他用户身份,变更时须输入所要变更的用户帐号与密码。\par

	su参数选项如下:
	\begin{lstlisting}
	-c<指令>或--command=<指令>:执行完指定的指令后,即恢复原来的身份; 
	-f或——fast:适用于csh与tsch,使shell不用去读取启动文件; 
	-l或——login:改变身份时,也同时变更工作目录,以及HOME,SHELL,USER,logname。此外,也会变更PATH变量; 
	-m,-p或--preserve-environment:变更身份时,不要变更环境变量; 
	-s或--shell=:指定要执行的shell; 
	--help:显示帮助; 
	--version;显示版本信息。
	\end{lstlisting}

\section{sudo}
	sudo命令用来以其他身份来执行命令,预设的身份为root。在/etc/sudoers中设置了可执行sudo指令的用户。若其未经授权的用户企图使用sudo,则会发出警告的邮件给管理员。用户使用sudo时,必须先输入密码,之后有5分钟的有效期限,超过期限则必须重新输入密码。\par

	sudo参数选项如下:
	\begin{lstlisting}
	-b:在后台执行指令; 
	-h:显示帮助; 
	-H:将HOME环境变量设为新身份的HOME环境变量; 
	-k:结束密码的有效期限,也就是下次再执行sudo时便需要输入密码;
	-l:列出目前用户可执行与无法执行的指令; 
	-p:改变询问密码的提示符号; 
	-s:执行指定的shell; 
	-u<用户>:以指定的用户作为新的身份。若不加上此参数,则预设以root作为新的身份; 
	-v:延长密码有效期限5分钟; 
	-V :显示版本信息。
	\end{lstlisting}

\section{sysctl}
	sysctl命令被用于在内核运行时动态地修改内核的运行参数,可用的内核参数在目录/proc/sys中。它包含一些TCP/ip堆栈和虚拟内存系统的高级选项, 这可以让有经验的管理员提高引人注目的系统性能。用sysctl可以读取设置超过五百个系统变量。\par

	参数选项如下:
	\begin{lstlisting}
	-n:打印值时不打印关键字; 
	-e:忽略未知关键字错误; 
	-N:仅打印名称; 
	-w:当改变sysctl设置时使用此项; 
	-p:从配置文件“/etc/sysctl.conf”加载内核参数设置; 
	-a:打印当前所有可用的内核参数变量和值; 
	-A:以表格方式打印当前所有可用的内核参数变量和值。
	\end{lstlisting}

	使用例子如下:
	\begin{lstlisting}
	# 读取一个变量
	sysctl vm.zone_reclaim_mode
	# 设置一个变量
	sysctl -w vm.zone_reclaim_mode=1
	\end{lstlisting}

	除了使用sysctl修改系统变量,也可以通过编辑/etc/sysctl.conf文件来修改系统变量,然后再使用以下命令:
	\begin{lstlisting}
	/sbin/sysctl -p
	\end{lstlisting}

\section{tar}
	解压的命令:
	\begin{lstlisting}
	tar xzvf temp.tar.gz
	\end{lstlisting}

\section{telnet}
	telnet命令用于登录远程主机,对远程主机进行管理。telnet因为采用明文传送报文,安全性不好,很多Linux服务器都不开放telnet服务,而改用更安全的ssh方式了。但仍然有很多别的系统可能采用了telnet方式来提供远程登录,因此弄清楚telnet客户端的使用方式仍是很有必要的。\par

	命令形式如下:
	\begin{lstlisting}
	telnet host [port]
	\end{lstlisting}

\section{trap}
	trap命令用于指定在接收到信号后将要采取的动作,常见的用途是在脚本程序被中断时完成清理工作。当shell接收到sigspec指定的信号时,arg参数(命令)将会被读取,并被执行。\par

	格式:
	\begin{lstlisting}
	trap commands signal
	\end{lstlisting}

\section{uname}
	uname命令用于打印当前系统相关信息(内核版本号、硬件架构、主机名称和操作系统类型等)。\par

	参数选项如下:
	\begin{lstlisting}
	-a或--all:显示全部的信息; 
	-m或--machine:显示电脑类型; 
	-n或-nodename:显示在网络上的主机名称; 
	-r或--release:显示操作系统的发行编号; 
	-s或--sysname:显示操作系统名称; 
	-v:显示操作系统的版本; 
	-p或--processor:输出处理器类型或"unknown"; 
	-i或--hardware-platform:输出硬件平台或"unknown"; 
	-o或--operating-system:输出操作系统名称; 
	--help:显示帮助; --version:显示版本信息。
	\end{lstlisting}

\section{wc}
	“wc -l”可以查看一个文件的行数。

\section{zcat}
	zcat命令用于不真正解压缩文件,就能显示压缩包中文件的内容的场合。\par

	zcat参数选项如下:
	\begin{lstlisting}
	-S:指定gzip格式的压缩包的后缀。当后缀不是标准压缩包后缀时使用此选项; 
	-c:将文件内容写到标注输出; 
	-d:执行解压缩操作; 
	-l:显示压缩包中文件的列表; 
	-L:显示软件许可信息; 
	-q:禁用警告信息; 
	-r:在目录上执行递归操作; 
	-t:测试压缩文件的完整性; 
	-V:显示指令的版本信息; 
	-l:更快的压缩速度; 
	-9:更高的压缩比。
	\end{lstlisting}

\end{document}
