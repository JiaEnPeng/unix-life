% !TeX spellcheck = en_US
%% 字体:方正静蕾简体
%%		 方正粗宋
\documentclass[a4paper,left=2.5cm,right=2.5cm,11pt]{article}

\usepackage[utf8]{inputenc}
\usepackage{fontspec}
\usepackage{cite}
\usepackage{xeCJK}
\usepackage{indentfirst}
\usepackage{titlesec}
\usepackage{longtable}
\usepackage{graphicx}
\usepackage{float}
\usepackage{rotating}
\usepackage{subfigure}
\usepackage{tabu}
\usepackage{amsmath}
\usepackage{setspace}
\usepackage{amsfonts}
\usepackage{appendix}
\usepackage{listings}
\usepackage{xcolor}
\usepackage{geometry}
\setcounter{secnumdepth}{4}
\usepackage{mhchem}
\usepackage{multirow}
\usepackage{extarrows}
\usepackage{hyperref}
\titleformat*{\section}{\LARGE}
\renewcommand\refname{参考文献}
\renewcommand{\abstractname}{\sihao \cjkfzcs 摘{  }要}
%\titleformat{\chapter}{\centering\bfseries\huge\wryh}{}{0.7em}{}{}
%\titleformat{\section}{\LARGE\bf}{\thesection}{1em}{}{}
\titleformat{\subsection}{\Large\bfseries}{\thesubsection}{1em}{}{}
\titleformat{\subsubsection}{\large\bfseries}{\thesubsubsection}{1em}{}{}
\renewcommand{\contentsname}{{\cjkfzcs \centerline{目{  } 录}}}
\setCJKfamilyfont{cjkhwxk}{STXingkai}
\setCJKfamilyfont{cjkfzcs}{STSongti-SC-Regular}
% \setCJKfamilyfont{cjkhwxk}{华文行楷}
% \setCJKfamilyfont{cjkfzcs}{方正粗宋简体}
\newcommand*{\cjkfzcs}{\CJKfamily{cjkfzcs}}
\newcommand*{\cjkhwxk}{\CJKfamily{cjkhwxk}}
\newfontfamily\wryh{Microsoft YaHei}
\newfontfamily\hwzs{STZhongsong}
\newfontfamily\hwst{STSong}
\newfontfamily\hwfs{STFangsong}
\newfontfamily\jljt{MicrosoftYaHei}
\newfontfamily\hwxk{STXingkai}
% \newfontfamily\hwzs{华文中宋}
% \newfontfamily\hwst{华文宋体}
% \newfontfamily\hwfs{华文仿宋}
% \newfontfamily\jljt{方正静蕾简体}
% \newfontfamily\hwxk{华文行楷}
\newcommand{\verylarge}{\fontsize{60pt}{\baselineskip}\selectfont}  
\newcommand{\chuhao}{\fontsize{44.9pt}{\baselineskip}\selectfont}  
\newcommand{\xiaochu}{\fontsize{38.5pt}{\baselineskip}\selectfont}  
\newcommand{\yihao}{\fontsize{27.8pt}{\baselineskip}\selectfont}  
\newcommand{\xiaoyi}{\fontsize{25.7pt}{\baselineskip}\selectfont}  
\newcommand{\erhao}{\fontsize{23.5pt}{\baselineskip}\selectfont}  
\newcommand{\xiaoerhao}{\fontsize{19.3pt}{\baselineskip}\selectfont} 
\newcommand{\sihao}{\fontsize{14pt}{\baselineskip}\selectfont}      % 字号设置  
\newcommand{\xiaosihao}{\fontsize{12pt}{\baselineskip}\selectfont}  % 字号设置  
\newcommand{\wuhao}{\fontsize{10.5pt}{\baselineskip}\selectfont}    % 字号设置  
\newcommand{\xiaowuhao}{\fontsize{9pt}{\baselineskip}\selectfont}   % 字号设置  
\newcommand{\liuhao}{\fontsize{7.875pt}{\baselineskip}\selectfont}  % 字号设置  
\newcommand{\qihao}{\fontsize{5.25pt}{\baselineskip}\selectfont}    % 字号设置 

\usepackage{diagbox}
\usepackage{multirow}
\boldmath
\XeTeXlinebreaklocale "zh"
\XeTeXlinebreakskip = 0pt plus 1pt minus 0.1pt
\definecolor{cred}{rgb}{0.8,0.8,0.8}
\definecolor{cgreen}{rgb}{0,0.3,0}
\definecolor{cpurple}{rgb}{0.5,0,0.35}
\definecolor{cdocblue}{rgb}{0,0,0.3}
\definecolor{cdark}{rgb}{0.95,1.0,1.0}
\lstset{
	language=bash,
	numbers=left,
	numberstyle=\tiny\color{white},
	showspaces=false,
	showstringspaces=false,
	basicstyle=\scriptsize,
	keywordstyle=\color{purple},
	commentstyle=\itshape\color{cgreen},
	stringstyle=\color{blue},
	frame=lines,
	% escapeinside=``,
	extendedchars=true, 
	xleftmargin=0em,
	xrightmargin=0em, 
	backgroundcolor=\color{cred},
	aboveskip=1em,
	breaklines=true,
	tabsize=4
} 

\newfontfamily{\consolas}{Consolas}
\newfontfamily{\monaco}{Monaco}
\setmonofont[Mapping={}]{Consolas}	%英文引号之类的正常显示,相当于设置英文字体
\setsansfont{Consolas} %设置英文字体 Monaco, Consolas,  Fantasque Sans Mono
\setmainfont{Times New Roman}

\setCJKmainfont{华文中宋}


\newcommand{\fic}[1]{\begin{figure}[H]
		\center
		\includegraphics[width=0.8\textwidth]{#1}
	\end{figure}}
	
\newcommand{\sizedfic}[2]{\begin{figure}[H]
		\center
		\includegraphics[width=#1\textwidth]{#2}
	\end{figure}}

\newcommand{\codefile}[1]{\lstinputlisting{#1}}

\newcommand{\interval}{\vspace{0.5em}}

\newcommand{\tablestart}{
	\interval
	\begin{longtable}{p{2cm}p{10cm}}
	\hline}
\newcommand{\tableend}{
	\hline
	\end{longtable}
	\interval}

% 改变段间隔
\setlength{\parskip}{0.2em}
\linespread{1.1}

\usepackage{lastpage}
\usepackage{fancyhdr}
\pagestyle{fancy}
\lhead{\space \qquad \space}
\chead{linux磁盘管理 \qquad}
\rhead{\qquad\thepage/\pageref{LastPage}}
\begin{document}

\tableofcontents

\clearpage

\section{linux磁盘管理}
\subsection{df命令查看系统磁盘容量}
	“df”命令可以查看已挂载磁盘的总容量、使用容量、剩余容量等,默认单位为k,如下图所示:
	\sizedfic{0.5}{1.png}

	“df -i”查看inodes使用情况,如下图所示:
	\sizedfic{0.5}{2.png}

	“df -h”可以使用合适的单位显示,如下图所示:
	\sizedfic{0.5}{3.png}

\subsection{du命令查看目录的大小}
	“du <dir>”可以查看某个目录以及它下面所有文件夹所占的空间,默认单位为k。\par

	“du -a <dir>”可以查看某个目录以及它下面所有文件夹、所有文件所占的空间。\par

	“du -h <dir>”可以查看某个目录以及它下面所有文件夹所占的空间,将使用合适的单位。\par

	“du -sh <dir>”可以查看目录所占的空间,将使用合适的单位。

\subsection{磁盘的分区和格式化}
\subsubsection{fdisk命令进行磁盘的分区}
	“fdisk -l”将直接列出系统中所有的磁盘设备以及分区表,如下图所示:
	\sizedfic{0.45}{4.png}

	"fdisk <device>"将对磁盘设备进行分区,然后进入交互界面。\par

	分区模式下常用的命令如下:
	\begin{lstlisting}
	n 建立一个新的分区
	w 保存操作
	q 退出
	d 删除一个分区
	p 打印分区情况
	\end{lstlisting}

	如果输入’n‘,将建立一个新的分区,它会提示’e‘(扩展分区)还是‘p’(主分区)。主分区和扩展分区的介绍如下:
	\begin{lstlisting}
	磁盘分区有三种形式:主分区、扩展分区和逻辑分区。

	主分区最多有四个。如果要在硬盘上安装操作系统,那么这个必须有一个主分区。
	主分区中不能再划分其他类型的分区,每个主分区相当于一个逻辑磁盘。

	扩展分区不能直接使用,必须将它划分为若干个逻辑分区才能使用。
	逻辑分区相当于一个逻辑磁盘,逻辑分区必须在扩展分区中划分。

	由主分区和逻辑分区构成的逻辑磁盘称为驱动器(Driver)或卷(Volume)。
	\end{lstlisting}

	使用“fdisk”进行磁盘分区的例子如下图所示:
	\sizedfic{0.65}{5.png}

\subsubsection{mkfs命令格式化磁盘分区}
	“mkfs -t filesystem <device>”将设备格式化为某一文件系统。\par

	使用“mkfs.ext2 <device>”可以将设备格式化为ext2的文件系统,“mkfs.ext3”、"mkfs.ext4"同理。\par

	使用“mkfs -t filesystem -b <count> <device>”可以在格式化时设定每个数据区块占用空间的大小。
	ext文件系统默认块大小为4096。一般来说不需要指定块的大小,否则容易出错。

\subsection{挂载/卸载磁盘}
\subsubsection{/etc/fstab文件}
	如果想永久性地挂载或卸载磁盘,需要更改/etc/fstab文件的内容。/etc/fstab文件的内容如下所示:
	\fic{6.png}

	对这个文件的介绍如下:
	\begin{lstlisting}
	第一列:分区的标识,可以写:分区的LABEL、分区的UUID或者分区名
	第二列:挂载点
	第三列:分区的格式
	第四列:mount的一些挂载参数
	第五列:分区是否被dump备份,如果是的话为1,不是为0
	第六列:开机时是否自检磁盘,1、2都表示检测,0表示不检测,/分区必须设为1
	\end{lstlisting}

	对挂载参数的介绍如下:
	\begin{lstlisting}
	async/sync:async表示系统每隔一段时间把内存数据写入磁盘,sync表示时时同步内存和磁盘中的数据
	auto/noauto:开机自动挂载/不自动挂载
	default:包含了rw,suid,dev,exec,auto,nouser,async
	ro:按只读权限挂载
	rw:按可读写权限挂载
	exec/noexec:按可执行/不可执行挂载
	user/nouser:允许/不允许root以外的其他用户挂载分区
	suid/nosuid:允许/不允许分区有suid属性
	usrquota:启动使用者磁盘配额模式
	grquota:启动群组磁盘配额模式
	\end{lstlisting}

\subsubsection{mount挂载磁盘}
	使用“mount <device> <dir>”命令把某个分区挂载到某个文件夹下。\par

	“mount -a”命令会把/etc/fstab文件中出现的所有磁盘分区都挂载上。\par

	“-t”选项用于指定挂载的分区类型,如果不指定将自动识别。\par

	“-o”选项用于指定挂载的分区有哪些特性,也就是/etc/fstab文件中第四列的那些,使用例子如下:
	\begin{lstlisting}
	mount -o ro,sync,noauto /dev/sdb1 /mnt
	\end{lstlisting}

\subsubsection{blkid查看块设备的属性}
	块设备的属性一般有UUID和TYPE,使用“blkid”可以获得所有分区的属性。\par

	在挂载分区的时候推荐使用UUID,以免分区名称混乱时出现错误。\par

\subsubsection{umount卸载磁盘}
	“umount”后面可以跟分区名,也可以跟挂载点,用于卸载分区。\par

	如果“umount”卸载失败时,有可能是当前目录处于要卸载的分区上,这时使用“-l”选项就可以轻松地解决问题。
	
\end{document}
