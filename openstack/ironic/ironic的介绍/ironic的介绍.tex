% !TeX spellcheck = en_US
%% 字体:方正静蕾简体
%%		 方正粗宋
\documentclass[a4paper,left=1.5cm,right=1.5cm,11pt]{article}

\usepackage[utf8]{inputenc}
\usepackage{fontspec}
\usepackage{cite}
\usepackage{xeCJK}
\usepackage{indentfirst}
\usepackage{titlesec}
\usepackage{etoolbox}%
\makeatletter
\patchcmd{\ttlh@hang}{\parindent\z@}{\parindent\z@\leavevmode}{}{}%
\patchcmd{\ttlh@hang}{\noindent}{}{}{}%
\makeatother
\usepackage{hyperref}
\usepackage{longtable}
\usepackage{empheq}
\usepackage{graphicx}
\usepackage{float}
\usepackage{rotating}
\usepackage{subfigure}
\usepackage{tabu}
\usepackage{amsmath}
\usepackage{setspace}
\usepackage{amsfonts}
\usepackage{appendix}
\usepackage{listings}
\usepackage{xcolor}
\usepackage{geometry}
\setcounter{secnumdepth}{4}
%\titleformat*{\section}{\LARGE}
%\renewcommand\refname{参考文献}
%\titleformat{\chapter}{\centering\bfseries\huge}{}{0.7em}{}{}
\titleformat{\section}{\LARGE\bf}{\thesection}{1em}{}{}
\titleformat{\subsection}{\Large\bfseries}{\thesubsection}{1em}{}{}
\titleformat{\subsubsection}{\large\bfseries}{\thesubsubsection}{1em}{}{}
\renewcommand{\contentsname}{{ \centerline{目{  } 录}}}
\setCJKfamilyfont{cjkhwxk}{STXINGKA.TTF}
%\setCJKfamilyfont{cjkhwxk}{华文行楷}
%\setCJKfamilyfont{cjkfzcs}{方正粗宋简体}
%\newcommand*{\cjkfzcs}{\CJKfamily{cjkfzcs}}
\newcommand*{\cjkhwxk}{\CJKfamily{cjkhwxk}}
%\newfontfamily\wryh{Microsoft YaHei}
%\newfontfamily\hwzs{华文中宋}
%\newfontfamily\hwst{华文宋体}
%\newfontfamily\hwfs{华文仿宋}
%\newfontfamily\jljt{方正静蕾简体}
%\newfontfamily\hwxk{华文行楷}
\newcommand{\verylarge}{\fontsize{60pt}{\baselineskip}\selectfont}  
\newcommand{\chuhao}{\fontsize{44.9pt}{\baselineskip}\selectfont}  
\newcommand{\xiaochu}{\fontsize{38.5pt}{\baselineskip}\selectfont}  
\newcommand{\yihao}{\fontsize{27.8pt}{\baselineskip}\selectfont}  
\newcommand{\xiaoyi}{\fontsize{25.7pt}{\baselineskip}\selectfont}  
\newcommand{\erhao}{\fontsize{23.5pt}{\baselineskip}\selectfont}  
\newcommand{\xiaoerhao}{\fontsize{19.3pt}{\baselineskip}\selectfont} 
\newcommand{\sihao}{\fontsize{14pt}{\baselineskip}\selectfont}      % 字号设置  
\newcommand{\xiaosihao}{\fontsize{12pt}{\baselineskip}\selectfont}  % 字号设置  
\newcommand{\wuhao}{\fontsize{10.5pt}{\baselineskip}\selectfont}    % 字号设置  
\newcommand{\xiaowuhao}{\fontsize{9pt}{\baselineskip}\selectfont}   % 字号设置  
\newcommand{\liuhao}{\fontsize{7.875pt}{\baselineskip}\selectfont}  % 字号设置  
\newcommand{\qihao}{\fontsize{5.25pt}{\baselineskip}\selectfont}    % 字号设置 

\usepackage{diagbox}
\usepackage{multirow}
\boldmath
\XeTeXlinebreaklocale "zh"
\XeTeXlinebreakskip = 0pt plus 1pt minus 0.1pt
\definecolor{cred}{rgb}{0.8,0.8,0.8}
\definecolor{cgreen}{rgb}{0,0.3,0}
\definecolor{cpurple}{rgb}{0.5,0,0.35}
\definecolor{cdocblue}{rgb}{0,0,0.3}
\definecolor{cdark}{rgb}{0.95,1.0,1.0}
\lstset{
	language=python,
	numbers=left,
	numberstyle=\tiny\color{black},
	showspaces=false,
	showstringspaces=false,
	basicstyle=\scriptsize,
	keywordstyle=\color{purple},
	commentstyle=\color{cgreen},
	stringstyle=\color{blue},
	frame=lines,
	% escapeinside=``,
	extendedchars=true, 
	xleftmargin=1em,
	xrightmargin=1em, 
	backgroundcolor=\color{cred},
	aboveskip=1em,
	breaklines=true,
	tabsize=4
} 

%\newfontfamily{\consolas}{Consolas}
%\newfontfamily{\monaco}{Monaco}
%\setmonofont[Mapping={}]{Consolas}	%英文引号之类的正常显示,相当于设置英文字体
%\setsansfont{Consolas} %设置英文字体 Monaco, Consolas,  Fantasque Sans Mono
%\setmainfont{Times New Roman}
%\setCJKmainfont{STZHONGS.TTF}
%\setmonofont{Consolas}
% \newfontfamily{\consolas}{YaHeiConsolas.ttf}
\newfontfamily{\monaco}{MONACO.TTF}
\setCJKmainfont{STZHONGS.TTF}
%\setmainfont{MONACO.TTF}
%\setsansfont{MONACO.TTF}

\newcommand{\fic}[1]{\begin{figure}[H]
		\center
		\includegraphics[width=0.8\textwidth]{#1}
	\end{figure}}
	
\newcommand{\sizedfic}[2]{\begin{figure}[H]
		\center
		\includegraphics[width=#1\textwidth]{#2}
	\end{figure}}

\newcommand{\codefile}[1]{\lstinputlisting{#1}}

\newcommand{\interval}{\vspace{0.5em}}

\newcommand{\tablestart}{
	\interval
	\begin{longtable}{p{2cm}p{10cm}}
	\hline}
\newcommand{\tableend}{
	\hline
	\end{longtable}
	\interval}

% 改变段间隔
\setlength{\parskip}{0.2em}
\linespread{1.1}

\usepackage{lastpage}
\usepackage{fancyhdr}
\pagestyle{fancy}
\lhead{\space \qquad \space}
\chead{ironic的介绍 \qquad}
\rhead{\qquad\thepage/\pageref{LastPage}}

\begin{document}

\tableofcontents

\clearpage

\section{ironic的功能}
	Ironic可以用于解决物理机的添加、删除、电源管理和安装部署。\par
	Ironic可以看成一组Hypervisor API的集合,其功能与libvirt类似,
	所以ironic可以看成一个hypervisor驱动来给Nova使用。

\section{ironic的架构}
\subsection{项目组成}
	\begin{itemize}
		\item ironic:包含ironic-api和ironic-conductor进程。
		\item python-ironicclient:python client and CLI。
		\item ironic-python-agent:一个运行在deployment ramdisk中的python程序,用于执行一系列部署动作。
		\item pyghmi:一个python的IPMI库,用于代替IPMItool。
		\item ironic-inspector:硬件自检工具。
		\item ironic-lib:ironic的通用库函数。
		\item ironic-webclient:web客户端。
		\item ironic-ui:ironic的horizon插件。
		\item bifrost:一套只运行ironic的Ansible脚本。
	\end{itemize}

\clearpage

\subsection{ironic与其他组件的调用关系}
	ironic与其他组件的调用关系如下图所示:
	\sizedfic{0.9}{1.png}

	从上图可以得到如下信息:
	\begin{itemize}
		\item[1.] Ironic API是一个RESTful API服务,nova compute服务通过Ironic API与Ironic进行交互。
		\item[2.] Ironic Conductor用于完成Ironic服务的绝大部分工作,与Ironic API通过RPC进行交互,负责与Neutron、Glance等组件进行交互。
		\item[3.] Drivers是真正管理物理机的模块,通过一系列的驱动来支持不同的硬件。
		\item[4.] Database用于存储资源信息。
		\item[5.] 消息队列。
	\end{itemize}

\section{ironic的关键技术}
	在安装操作系统时,我们需要存储介质来存储系统镜像、需要控制物理机来开关机,在网络部署环境中还需要预启动环境。

\subsection{预启动环境PXE}
	PXE是一种无盘启动技术,使计算机通过网络启动,采用Client/Server的网络模式。
	在计算机的启动过程中,客户端要求服务器分配IP地址,再用TFTP协议下载一个开机引导程序,从而引导安装放置于服务器中的操作系统。\par

	具体的安装操作系统的步骤如下:
	\begin{itemize}
		\item[1.] 客户机从自己的PXE网卡启动,向本网络中的DHCP服务器索取IP,并搜寻引导文件的位置。
		\item[2.] DHCP服务器返回分给客户机的IP和network bootstrap program的放置位置。
		\item[3.] 客户机想本网络中的TFTP服务器索取network bootstrap program。
		\item[4.] 客户机取得network bootstrap program之后执行该文件,并根据执行结果通过TFTP服务器加载内核和文件系统。
		\item[5.] 安装操作系统。
	\end{itemize}

	安装流程如下图所示:
	\sizedfic{0.7}{2.jpg}

\subsection{远程开机IPMI}
	IPMI是智能平台管理接口,IPMI通过网络连接到基板管理器(BMC)进行交流,不依赖BIOS或者操作系统,
	这使得在操作系统不响应或未加载的情况下仍然可以进行开关机、信息提取等操作。\par

	IPMI有如下功能:
	\begin{itemize}
		\item 远程电源控制
		\item 串口IP映射
		\item 支持健康关机
		\item 机箱环境监控(温度、风扇转速和CPU电压)
		\item 远程设备身份LED控制
		\item 系统事件日志
		\item 平台事件跟踪
		\item 数据记录
		\item 虚拟KVM会话
		\item 虚拟媒介
	\end{itemize}

	Ironic利用IPMI可以实现以下功能:
	\begin{itemize}
		\item[1.] 可以在客户机通电的情况下,对它进行远程管理:开机、关机和重启。
		\item[2.] 基于文本的控制台重定向,可以远程查看和修改BIOS设置,系统启动过程,登入系统等。
		\item[3.] 可以远程通过串口IP映射连接客户机,远程安装系统,查看系统启动故障等问题。
		\item[4.] 可以通过系统的串行端口进行访问。
		\item[5.] 记录故障日志和发送SNMP警报,访问系统事件日志和传感器状况。
	\end{itemize}

\subsection{存储技术iSCSI}
	iSCSI技术是一种新存储技术,该技术将现有SCSI接口与以太网络技术结合,使客户机可与使用IP网络的存储装置互相交换资料。
	iSCSI主要是透过TCP/IP的技术,将存储设备透过iSCSI target功能,做成可以提供磁盘的服务器端。再透过iSCSI initiator功能将客户机做成能够挂载使用远程存储装置的客户端,如此变成能够iSCSI设置来进行磁盘的应用了。\par

	也就是说,iSCSI这个架构将存储装置和主机分为了两个部分,分别是:
	\begin{itemize}
		\item iSCSI target,就是存储设备端,存放磁盘或RAID的设备。
		\item iSCSI initiator,就是能够使用target的用户端。想要连接到iSCSI target的客户机,必须安装iSCSI initiator的相关功能后才能够使用iSCSI target提供的磁盘。
	\end{itemize}

\section{ironic的API}
	这个网站记录了ironic的API:\href{https://developer.openstack.org/api-ref/baremetal/?expanded=list-api-versions-detail,show-v1-api-detail,create-node-detail,list-nodes-detail,list-nodes-detailed-detail,show-node-details-detail,update-node-detail,delete-node-detail,validate-node-detail,set-maintenance-flag-detail,clear-maintenance-flag-detail,set-boot-device-detail,get-boot-device-detail,get-supported-boot-devices-detail,inject-nmi-non-masking-interrupts-detail,node-state-summary-detail,change-node-power-state-detail,change-node-provision-state-detail,set-raid-config-detail,get-console-detail,start-stop-console-detail,list-methods-detail,call-a-method-detail,list-attached-vifs-of-a-node-detail,attach-a-vif-to-a-node-detail,detach-vif-from-a-node-detail,list-portgroups-detail,create-portgroup-detail,list-detailed-portgroups-detail,show-portgroup-details-detail,update-a-portgroup-detail,delete-portgroup-detail,list-portgroups-by-node-detail,list-detailed-portgroups-by-node-detail,list-ports-detail,create-port-detail,list-detailed-ports-detail,show-port-details-detail,update-a-port-detail,delete-port-detail,list-ports-by-node-detail,list-detailed-ports-by-node-detail,list-ports-by-portgroup-detail,list-detailed-ports-by-portgroup-detail,list-drivers-detail,show-driver-details-detail,show-driver-properties-detail,show-driver-logical-disk-properties-detail,id104-detail,id108-detail,list-chassis-with-details-detail,show-chassis-details-detail,update-chassis-detail,delete-chassis-detail,create-chassis-detail,list-chassis-detail,agent-lookup-detail,agent-heartbeat-detail}{ironic API}\par

	ironic API的使用需要搭配curl命令,使用例子可以查看这个网站:
	\url{https://developer.openstack.org/zh_CN/api-guide/quick-start/api-quick-start.html}

\section{ironic的安装与配置}
	这个网站记录了ironic的安装与配置:\par
	\url{https://doodu.gitbooks.io/openstack-ironic/content/an_zhuang_yu_pei_zhi.html}

\end{document}