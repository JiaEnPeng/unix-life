% !TeX spellcheck = en_US
%% 字体:方正静蕾简体
%%		 方正粗宋
\documentclass[a4paper,left=1.5cm,right=1.5cm,11pt]{article}

\usepackage[utf8]{inputenc}
\usepackage{fontspec}
\usepackage{cite}
\usepackage{xeCJK}
\usepackage{indentfirst}
\usepackage{titlesec}
\usepackage{etoolbox}%
\makeatletter
\patchcmd{\ttlh@hang}{\parindent\z@}{\parindent\z@\leavevmode}{}{}%
\patchcmd{\ttlh@hang}{\noindent}{}{}{}%
\makeatother
\usepackage{hyperref}
\usepackage{longtable}
\usepackage{empheq}
\usepackage{graphicx}
\usepackage{float}
\usepackage{rotating}
\usepackage{subfigure}
\usepackage{tabu}
\usepackage{amsmath}
\usepackage{setspace}
\usepackage{amsfonts}
\usepackage{appendix}
\usepackage{listings}
\usepackage{xcolor}
\usepackage{geometry}
\setcounter{secnumdepth}{4}
%\titleformat*{\section}{\LARGE}
%\renewcommand\refname{参考文献}
%\titleformat{\chapter}{\centering\bfseries\huge}{}{0.7em}{}{}
\titleformat{\section}{\LARGE\bf}{\thesection}{1em}{}{}
\titleformat{\subsection}{\Large\bfseries}{\thesubsection}{1em}{}{}
\titleformat{\subsubsection}{\large\bfseries}{\thesubsubsection}{1em}{}{}
\renewcommand{\contentsname}{{ \centerline{目{  } 录}}}
\setCJKfamilyfont{cjkhwxk}{STXINGKA.TTF}
%\setCJKfamilyfont{cjkhwxk}{华文行楷}
%\setCJKfamilyfont{cjkfzcs}{方正粗宋简体}
%\newcommand*{\cjkfzcs}{\CJKfamily{cjkfzcs}}
\newcommand*{\cjkhwxk}{\CJKfamily{cjkhwxk}}
%\newfontfamily\wryh{Microsoft YaHei}
%\newfontfamily\hwzs{华文中宋}
%\newfontfamily\hwst{华文宋体}
%\newfontfamily\hwfs{华文仿宋}
%\newfontfamily\jljt{方正静蕾简体}
%\newfontfamily\hwxk{华文行楷}
\newcommand{\verylarge}{\fontsize{60pt}{\baselineskip}\selectfont}  
\newcommand{\chuhao}{\fontsize{44.9pt}{\baselineskip}\selectfont}  
\newcommand{\xiaochu}{\fontsize{38.5pt}{\baselineskip}\selectfont}  
\newcommand{\yihao}{\fontsize{27.8pt}{\baselineskip}\selectfont}  
\newcommand{\xiaoyi}{\fontsize{25.7pt}{\baselineskip}\selectfont}  
\newcommand{\erhao}{\fontsize{23.5pt}{\baselineskip}\selectfont}  
\newcommand{\xiaoerhao}{\fontsize{19.3pt}{\baselineskip}\selectfont} 
\newcommand{\sihao}{\fontsize{14pt}{\baselineskip}\selectfont}      % 字号设置  
\newcommand{\xiaosihao}{\fontsize{12pt}{\baselineskip}\selectfont}  % 字号设置  
\newcommand{\wuhao}{\fontsize{10.5pt}{\baselineskip}\selectfont}    % 字号设置  
\newcommand{\xiaowuhao}{\fontsize{9pt}{\baselineskip}\selectfont}   % 字号设置  
\newcommand{\liuhao}{\fontsize{7.875pt}{\baselineskip}\selectfont}  % 字号设置  
\newcommand{\qihao}{\fontsize{5.25pt}{\baselineskip}\selectfont}    % 字号设置 

\usepackage{diagbox}
\usepackage{multirow}
\boldmath
\XeTeXlinebreaklocale "zh"
\XeTeXlinebreakskip = 0pt plus 1pt minus 0.1pt
\definecolor{cred}{rgb}{0.8,0.8,0.8}
\definecolor{cgreen}{rgb}{0,0.3,0}
\definecolor{cpurple}{rgb}{0.5,0,0.35}
\definecolor{cdocblue}{rgb}{0,0,0.3}
\definecolor{cdark}{rgb}{0.95,1.0,1.0}
\lstset{
	language=python,
	numbers=left,
	numberstyle=\tiny\color{black},
	showspaces=false,
	showstringspaces=false,
	basicstyle=\scriptsize,
	keywordstyle=\color{purple},
	commentstyle=\color{cgreen},
	stringstyle=\color{blue},
	frame=lines,
	% escapeinside=``,
	extendedchars=true, 
	xleftmargin=1em,
	xrightmargin=1em, 
	backgroundcolor=\color{cred},
	aboveskip=1em,
	breaklines=true,
	tabsize=4
} 

%\newfontfamily{\consolas}{Consolas}
%\newfontfamily{\monaco}{Monaco}
%\setmonofont[Mapping={}]{Consolas}	%英文引号之类的正常显示,相当于设置英文字体
%\setsansfont{Consolas} %设置英文字体 Monaco, Consolas,  Fantasque Sans Mono
%\setmainfont{Times New Roman}
%\setCJKmainfont{STZHONGS.TTF}
%\setmonofont{Consolas}
% \newfontfamily{\consolas}{YaHeiConsolas.ttf}
\newfontfamily{\monaco}{MONACO.TTF}
\setCJKmainfont{STZHONGS.TTF}
%\setmainfont{MONACO.TTF}
%\setsansfont{MONACO.TTF}

\newcommand{\fic}[1]{\begin{figure}[H]
		\center
		\includegraphics[width=0.8\textwidth]{#1}
	\end{figure}}
	
\newcommand{\sizedfic}[2]{\begin{figure}[H]
		\center
		\includegraphics[width=#1\textwidth]{#2}
	\end{figure}}

\newcommand{\codefile}[1]{\lstinputlisting{#1}}

\newcommand{\interval}{\vspace{0.5em}}

\newcommand{\tablestart}{
	\interval
	\begin{longtable}{p{2cm}p{10cm}}
	\hline}
\newcommand{\tableend}{
	\hline
	\end{longtable}
	\interval}

% 改变段间隔
\setlength{\parskip}{0.2em}
\linespread{1.1}

\usepackage{lastpage}
\usepackage{fancyhdr}
\pagestyle{fancy}
\lhead{\space \qquad \space}
\chead{使用devstack部署ironic \qquad}
\rhead{\qquad\thepage/\pageref{LastPage}}

\begin{document}

\tableofcontents

\clearpage

\section{使用devstack部署ironic}
\subsection{第一种方法}
	这里使用devstack在ubuntu14.04上部署ironic,步骤如下:
	\begin{itemize}
		\item[1.] 首先下载devstack:
		\begin{lstlisting}
	git clone http://git.trystack.cn/openstack-dev/devstack.git
	sudo ./devstack/tools/create-stack-user.sh
		\end{lstlisting}

		\item[2.] 进入devstack目录下,创建local.conf文件,文件内容如下:
		\begin{lstlisting}
[[local|localrc]]

GIT_BASE=http://git.trystack.cn
NOVNC_REPO=http://git.trystack.cn/kanaka/noVNC.git
SPICE_REPO=http://git.trystack.cn/git/spice/spice-html5.git

FORCE=yes

# Credentials
ADMIN_PASSWORD=p1111111
DATABASE_PASSWORD=p1111111
RABBIT_PASSWORD=p1111111
SERVICE_PASSWORD=p1111111
SERVICE_TOKEN=p1111111
SWIFT_HASH=p1111111
SWIFT_TEMPURL_KEY=p1111111

# Enable Ironic plugin
enable_plugin ironic http://git.trystack.cn/openstack/ironic

# Enable Neutron which is required by Ironic and disable nova-network.
disable_service n-net
disable_service n-novnc
enable_service q-svc
enable_service q-agt
enable_service q-dhcp
enable_service q-l3
enable_service q-meta
enable_service neutron

# Enable Swift for agent_* drivers
enable_service s-proxy
enable_service s-object
enable_service s-container
enable_service s-account

# Disable Horizon
disable_service horizon

# Disable Heat
disable_service heat h-api h-api-cfn h-api-cw h-eng

# Disable Cinder
disable_service cinder c-sch c-api c-vol

# Swift temp URL's are required for agent_* drivers.
SWIFT_ENABLE_TEMPURLS=True

# Create 3 virtual machines to pose as Ironic's baremetal nodes.
IRONIC_VM_COUNT=3
IRONIC_VM_SSH_PORT=22
IRONIC_BAREMETAL_BASIC_OPS=True
DEFAULT_INSTANCE_TYPE=baremetal

# Enable Ironic drivers.
IRONIC_ENABLED_DRIVERS=fake,agent_ssh,agent_ipmitool,pxe_ssh,pxe_ipmitool

# Change this to alter the default driver for nodes created by devstack.
# This driver should be in the enabled list above.
IRONIC_DEPLOY_DRIVER=agent_ipmitool

# The parameters below represent the minimum possible values to create
# functional nodes.
IRONIC_VM_SPECS_RAM=1280
IRONIC_VM_SPECS_DISK=10

# Size of the ephemeral partition in GB. Use 0 for no ephemeral partition.
IRONIC_VM_EPHEMERAL_DISK=0

# To build your own IPA ramdisk from source, set this to True
IRONIC_BUILD_DEPLOY_RAMDISK=False

VIRT_DRIVER=ironic

# By default, DevStack creates a 10.0.0.0/24 network for instances.
# If this overlaps with the hosts network, you may adjust with the
# following.
NETWORK_GATEWAY=10.1.0.1
FIXED_RANGE=10.1.0.0/24
FIXED_NETWORK_SIZE=256

# Log all output to files
LOGFILE=$HOME/temp/devstack.log
LOGDIR=$HOME/temp/logs
IRONIC_VM_LOG_DIR=$HOME/temp/ironic-bm-logs
		\end{lstlisting}

		\item[3.] 运行stack.sh:
		\begin{lstlisting}
	./stack.sh
		\end{lstlisting}
	\end{itemize}

\subsection{第二种方法}
	

\end{document}