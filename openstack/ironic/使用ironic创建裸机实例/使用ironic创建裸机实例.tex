% !TeX spellcheck = en_US
%% 字体:方正静蕾简体
%%		 方正粗宋
\documentclass[a4paper,left=1.5cm,right=1.5cm,11pt]{article}

\usepackage[utf8]{inputenc}
\usepackage{fontspec}
\usepackage{cite}
\usepackage{xeCJK}
\usepackage{indentfirst}
\usepackage{titlesec}
\usepackage{etoolbox}%
\makeatletter
\patchcmd{\ttlh@hang}{\parindent\z@}{\parindent\z@\leavevmode}{}{}%
\patchcmd{\ttlh@hang}{\noindent}{}{}{}%
\makeatother
\usepackage{hyperref}
\usepackage{longtable}
\usepackage{empheq}
\usepackage{graphicx}
\usepackage{float}
\usepackage{rotating}
\usepackage{subfigure}
\usepackage{tabu}
\usepackage{amsmath}
\usepackage{setspace}
\usepackage{amsfonts}
\usepackage{appendix}
\usepackage{listings}
\usepackage{xcolor}
\usepackage{geometry}
\setcounter{secnumdepth}{4}
%\titleformat*{\section}{\LARGE}
%\renewcommand\refname{参考文献}
%\titleformat{\chapter}{\centering\bfseries\huge}{}{0.7em}{}{}
\titleformat{\section}{\LARGE\bf}{\thesection}{1em}{}{}
\titleformat{\subsection}{\Large\bfseries}{\thesubsection}{1em}{}{}
\titleformat{\subsubsection}{\large\bfseries}{\thesubsubsection}{1em}{}{}
\renewcommand{\contentsname}{{ \centerline{目{  } 录}}}
\setCJKfamilyfont{cjkhwxk}{STXINGKA.TTF}
%\setCJKfamilyfont{cjkhwxk}{华文行楷}
%\setCJKfamilyfont{cjkfzcs}{方正粗宋简体}
%\newcommand*{\cjkfzcs}{\CJKfamily{cjkfzcs}}
\newcommand*{\cjkhwxk}{\CJKfamily{cjkhwxk}}
%\newfontfamily\wryh{Microsoft YaHei}
%\newfontfamily\hwzs{华文中宋}
%\newfontfamily\hwst{华文宋体}
%\newfontfamily\hwfs{华文仿宋}
%\newfontfamily\jljt{方正静蕾简体}
%\newfontfamily\hwxk{华文行楷}
\newcommand{\verylarge}{\fontsize{60pt}{\baselineskip}\selectfont}  
\newcommand{\chuhao}{\fontsize{44.9pt}{\baselineskip}\selectfont}  
\newcommand{\xiaochu}{\fontsize{38.5pt}{\baselineskip}\selectfont}  
\newcommand{\yihao}{\fontsize{27.8pt}{\baselineskip}\selectfont}  
\newcommand{\xiaoyi}{\fontsize{25.7pt}{\baselineskip}\selectfont}  
\newcommand{\erhao}{\fontsize{23.5pt}{\baselineskip}\selectfont}  
\newcommand{\xiaoerhao}{\fontsize{19.3pt}{\baselineskip}\selectfont} 
\newcommand{\sihao}{\fontsize{14pt}{\baselineskip}\selectfont}      % 字号设置  
\newcommand{\xiaosihao}{\fontsize{12pt}{\baselineskip}\selectfont}  % 字号设置  
\newcommand{\wuhao}{\fontsize{10.5pt}{\baselineskip}\selectfont}    % 字号设置  
\newcommand{\xiaowuhao}{\fontsize{9pt}{\baselineskip}\selectfont}   % 字号设置  
\newcommand{\liuhao}{\fontsize{7.875pt}{\baselineskip}\selectfont}  % 字号设置  
\newcommand{\qihao}{\fontsize{5.25pt}{\baselineskip}\selectfont}    % 字号设置 

\usepackage{diagbox}
\usepackage{multirow}
\boldmath
\XeTeXlinebreaklocale "zh"
\XeTeXlinebreakskip = 0pt plus 1pt minus 0.1pt
\definecolor{cred}{rgb}{0.8,0.8,0.8}
\definecolor{cgreen}{rgb}{0,0.3,0}
\definecolor{cpurple}{rgb}{0.5,0,0.35}
\definecolor{cdocblue}{rgb}{0,0,0.3}
\definecolor{cdark}{rgb}{0.95,1.0,1.0}
\lstset{
	language=bash,
	numbers=left,
	numberstyle=\tiny\color{black},
	showspaces=false,
	showstringspaces=false,
	basicstyle=\scriptsize,
	keywordstyle=\color{purple},
	commentstyle=\color{cgreen},
	stringstyle=\color{blue},
	frame=lines,
	% escapeinside=``,
	extendedchars=true, 
	xleftmargin=1em,
	xrightmargin=1em, 
	backgroundcolor=\color{cred},
	aboveskip=1em,
	breaklines=true,
	tabsize=4
} 

%\newfontfamily{\consolas}{Consolas}
%\newfontfamily{\monaco}{Monaco}
%\setmonofont[Mapping={}]{Consolas}	%英文引号之类的正常显示,相当于设置英文字体
%\setsansfont{Consolas} %设置英文字体 Monaco, Consolas,  Fantasque Sans Mono
%\setmainfont{Times New Roman}
%\setCJKmainfont{STZHONGS.TTF}
%\setmonofont{Consolas}
% \newfontfamily{\consolas}{YaHeiConsolas.ttf}
\newfontfamily{\monaco}{MONACO.TTF}
\setCJKmainfont{STZHONGS.TTF}
%\setmainfont{MONACO.TTF}
%\setsansfont{MONACO.TTF}

\newcommand{\fic}[1]{\begin{figure}[H]
		\center
		\includegraphics[width=0.8\textwidth]{#1}
	\end{figure}}
	
\newcommand{\sizedfic}[2]{\begin{figure}[H]
		\center
		\includegraphics[width=#1\textwidth]{#2}
	\end{figure}}

\newcommand{\codefile}[1]{\lstinputlisting{#1}}

\newcommand{\interval}{\vspace{0.5em}}

\newcommand{\tablestart}{
	\interval
	\begin{longtable}{p{2cm}p{10cm}}
	\hline}
\newcommand{\tableend}{
	\hline
	\end{longtable}
	\interval}

% 改变段间隔
\setlength{\parskip}{0.2em}
\linespread{1.1}

\usepackage{lastpage}
\usepackage{fancyhdr}
\pagestyle{fancy}
\lhead{\space \qquad \space}
\chead{使用ironic创建裸机实例 \qquad}
\rhead{\qquad\thepage/\pageref{LastPage}}

\begin{document}

\tableofcontents

\clearpage

\section{第一个例子}
\subsection{配置baremetal provisioning驱动}
	可以修改配置文件/etc/ironic/ironic.conf来设置openstack启用对应驱动:
	\begin{lstlisting}
	# 可以用逗号分隔来指定多个驱动
	enabled_drivers=pxe_ipmitool
	\end{lstlisting}

	修改后需要重启服务:
	\begin{lstlisting}
	systemctl restart openstack-ironic-conductor.service
	\end{lstlisting}

\subsection{上传镜像到glance服务器}
	openstack要实现部署裸机需要用到的镜像有5个。
	这5个镜像有两个是用作deploy,即被用来在安装操作系统前对裸机节点进行准备。
	有两镜像个用作系统的启动引导,还有一个就是系统镜像。\par

	命令如下:
	\begin{lstlisting}
	# 上传用于deploy的镜像
	glance image-create --name deploy_kernel --is-public true \
	--disk-format aki \
	--file deploy.kernel
	glance image-create --name deploy_initramfs --is-public true \
	--disk-format ari \
	--file deploy.initramfs

	# 上传用于boot镜像
	glance image-create --name boot_kernel --is-public true \
	--disk-format aki \
	--file boot.vmlinuz
	glance image-create --name boot_initrd --is-public true \
	--disk-format ari \
	--file boot.initrd

	# 上传系统镜像
	glance image-create --name NAME --is-public true \
	--disk-format qcow2 \
	--container-format bare \
	--property kernel_id=$boot_kernel_uuid \
	--property ramdisk_id=$boot_initrd_uuid \
	--property hypervisor_type=ironic \
	--file image.qcow2
	\end{lstlisting}

\subsection{把物理机注册为裸机节点}
	命令如下:
	\begin{lstlisting}
	# 创建新节点
	ironic node-create -d pxe_ipmitool

	# 创建逻辑名
	ironic node-update <node-uuid> add name=<node-name>

	# 可以通过下面的命令查看对于pxe_ipmitool,哪些驱动信息必须被添加
	ironic driver-properties pxe_ipmitool

	# 为主机添加IPMI驱动信息
	ironic node-update <node-uuid> add \ 
		driver_info/ipmi_username=<username> \ 
		driver_info/ipmi_password=<password> \ 
		driver_info/ipmi_address=<HOST-IP>

	# 添加用于deploy的镜像的uuid
	ironic node-update <node-uuid> add \ 
		driver_info/pxe_deploy_kernel=<deploy-kernel-uuid> \ 
		driver_info/pxe_deploy_ramdisk=<deploy-ramdisk-uuid>

	# 设置裸机硬件的规格
	ironic node-update <node-uuid> add \
		properties/cpus=4 \ 
		properties/memory_mb=98304 \ 
		properties/local_gb=80 \ 
		properties/cpu_arch=x86_64

	# 配置为本地引导(pxe初始化实施后的引导方式,flavor也需要设置)
	ironic node-update <node-uuid> add \
		properties/capabilities="boot_option:local"

	# 添加mac port(需要分配ip的所有网卡都要添加)
	ironic port-create -n <node-uuid> -a <mac-address>
	
	# 检验节点的设置
	ironic node-validate <node-uuid>
	\end{lstlisting}

\subsection{创建物理机需要的flavor}
	命令如下:
	\begin{lstlisting}
	nova flavor-create <flavor-name> auto 512 20 1
	nova flavor-key <flavor-name> set cpu_arch="x86_64"
	nova flavor-key <flavor-name> set capabilities:boot_option="local"
	\end{lstlisting}

	openstack通过指定实例的flavor来确定该实例生成在哪一个裸机中:
	\begin{lstlisting}
	ironic node-update $NODE_UUID add \ 
		properties/capabilities='profile:baremetal,boot_option:local'
	
	nova flavor-key $FLAVOR_NAME set capabilities:profile="baremetal"
	\end{lstlisting}

\section{第二个例子}
\subsection{创建虚拟机}

	\begin{lstlisting}
	sudo -E su pengsida -c '/opt/stack/ironic/devstack/tools/ironic/scripts/create-node.sh -n node-4 -c 1 -m 1280 -d 10 -a x86_64 -b brbm -e /usr/bin/qemu-system-x86_64 -E qemu -p 6233 -o 4 -f qcow2 -l /home/pengsida/temp/ironic-bm-logs'
	\end{lstlisting}

	\begin{lstlisting}
	# 设置虚拟机的存储池
	virsh pool-define-as --name default dir --target /var/lib/libvirt/images
	virsh pool-autostart default
	virsh pool-start default
	\end{lstlisting}

	\begin{lstlisting}
	domain_name="node-4"
	# 创建虚拟机使用的网桥
	brctl addbr br-$domain_name
	ip link set br-$domain_name up

	# 在网桥brbm上创建新的端口
	ovs-vsctl add-port brbm ovs-$domain_name -- set Interface ovs-$domain_name type=internal
	ip link set ovs-$domain_name up

	# 将ovs-node端口加入br-node这个网桥
	brctl addif br-$domain_name ovs-$domain_name

	# 创建虚拟机的磁盘镜像文件
	virsh vol-create-as default $domain_name.qcow2 11G --format qcow2 --prealloc-metadata
	virsh vol-path --pool default $domain_name.qcow2
	touch /var/lib/libvirt/images/$domain_name.qcow2
	chattr +C /var/lib/libvirt/images/$domain_name.qcow2
	/opt/stack/ironic/devstack/tools/ironic/scripts/configure-vm.py --bootdev network --name $domain_name --image /var/lib/libvirt/images/$domain_name.qcow2 --arch x86_64 --cpus 1 --memory 1310720 --libvirt-nic-driver virtio --bridge br-$domain_name --disk-format qcow2 --console-log /home/pengsida/temp/ironic-bm-logs/"$domain_name"_console.log --engine qemu --emulator /usr/bin/qemu-system-x86_64
	
	# 设置虚拟机的监听端口
	vbmc add $domain_name --port 6234
	# 启动虚拟机
	vbmc start $domain_name
	\end{lstlisting}
	
\subsection{注册物理机}
	\begin{lstlisting}
	sudo -E su pengsida -c '/opt/stack/ironic/devstack/tools/ironic/scripts/create-node.sh -n node-3 -c 1 -m 1280 -d 10 -a x86_64 -b brbm -e /usr/bin/qemu-system-x86_64 -E qemu -p 6233 -o 3 -f qcow2 -l /home/pengsida/temp/ironic-bm-logs'
	\end{lstlisting}

	\begin{lstlisting}
	# 注册物理机
	domain_name="node-4"
	node_uuid=`uuidgen`
	first_chassis_uuid=`openstack baremetal chassis list -f value | head -1 | cut -d" " -f1`
	ramdisk_uuid=`openstack image list | grep "\.initramfs" | cut -d"|" -f2 | cut -d" " -f2`
	kernel_uuid=`openstack image list | grep "\.kernel" | cut -d"|" -f2 | cut -d" " -f2`
	port_num=`vbmc list | grep $domain_name | egrep -o "| [0-9]+ |" | cut -d" " -f2`
	vcpu_num=`virsh dumpxml $domain_name | grep vcpu | cut -d">" -f2 | cut -d"<" -f1`
	memory=$(echo "`virsh dumpxml $domain_name | grep "memory" | cut -d">" -f2 | cut -d"<" -f1`/1024" | bc)
	arch=`virsh dumpxml $domain_name | grep arch | cut -d"'" -f2 | cut -d"'" -f1`
	disk_path=`virsh dumpxml $domain_name | grep "source file" | cut -d"'" -f2 | cut -d"'" -f1`
	disk_size=`qemu-img info $disk_path | grep "virtual size" | egrep -o "[0-9]+G" | cut -d"G" -f1`
	local_size=$[ $disk_size-1 ]
	driver_address=`openstack endpoint list | grep ironic | grep admin | cut -d"/" -f3 | cut -d":" -f1`
	domain_address=`virsh dumpxml $domain_name | grep 'mac address' | head -1 | cut -d"'" -f2`

	openstack flavor create --ephemeral 0 --ram $memory --disk $local_size --vcpus $vcpu_num $domain_name
	openstack flavor set $domain_name --property cpu_arch=$arch
	# 设置与domain关联
	nova flavor-key $domain_name set capabilities:profile="$domain_name"

	ironic node-create --uuid $node_uuid --chassis_uuid $first_chassis_uuid --driver agent_ipmitool --name $domain_name -p cpus=$vcpu_num -p memory_mb=$memory -p local_gb=$local_size -p cpu_arch=$arch -i ipmi_address=$driver_address -i ipmi_username=admin -i ipmi_password=password -i deploy_kernel=$kernel_uuid -i deploy_ramdisk=$ramdisk_uuid -i ipmi_port=$port_num
	ironic port-create --address $domain_address --node $node_uuid
	# 设置与flavor关联
	ironic node-update $NODE_UUID add properties/capabilities='profile:$domain_name'
	\end{lstlisting}

\end{document}