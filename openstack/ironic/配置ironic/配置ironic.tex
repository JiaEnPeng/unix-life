% !TeX spellcheck = en_US
%% 字体:方正静蕾简体
%%		 方正粗宋
\documentclass[a4paper,left=2.5cm,right=2.5cm,11pt]{article}

\usepackage[utf8]{inputenc}
\usepackage{fontspec}
\usepackage{cite}
\usepackage{xeCJK}
\usepackage{indentfirst}
\usepackage{titlesec}
\usepackage{longtable}
\usepackage{graphicx}
\usepackage{float}
\usepackage{rotating}
\usepackage{subfigure}
\usepackage{tabu}
\usepackage{amsmath}
\usepackage{setspace}
\usepackage{amsfonts}
\usepackage{appendix}
\usepackage{listings}
\usepackage{xcolor}
\usepackage{geometry}
\setcounter{secnumdepth}{4}
\usepackage{mhchem}
\usepackage{multirow}
\usepackage{extarrows}
\usepackage{hyperref}
\titleformat*{\section}{\LARGE}
\renewcommand\refname{参考文献}
\renewcommand{\abstractname}{\sihao \cjkfzcs 摘{  }要}
%\titleformat{\chapter}{\centering\bfseries\huge\wryh}{}{0.7em}{}{}
%\titleformat{\section}{\LARGE\bf}{\thesection}{1em}{}{}
\titleformat{\subsection}{\Large\bfseries}{\thesubsection}{1em}{}{}
\titleformat{\subsubsection}{\large\bfseries}{\thesubsubsection}{1em}{}{}
\renewcommand{\contentsname}{{\cjkfzcs \centerline{目{  } 录}}}
\setCJKfamilyfont{cjkhwxk}{STXingkai}
\setCJKfamilyfont{cjkfzcs}{STSongti-SC-Regular}
% \setCJKfamilyfont{cjkhwxk}{华文行楷}
% \setCJKfamilyfont{cjkfzcs}{方正粗宋简体}
\newcommand*{\cjkfzcs}{\CJKfamily{cjkfzcs}}
\newcommand*{\cjkhwxk}{\CJKfamily{cjkhwxk}}
\newfontfamily\wryh{Microsoft YaHei}
\newfontfamily\hwzs{STZhongsong}
\newfontfamily\hwst{STSong}
\newfontfamily\hwfs{STFangsong}
\newfontfamily\jljt{MicrosoftYaHei}
\newfontfamily\hwxk{STXingkai}
% \newfontfamily\hwzs{华文中宋}
% \newfontfamily\hwst{华文宋体}
% \newfontfamily\hwfs{华文仿宋}
% \newfontfamily\jljt{方正静蕾简体}
% \newfontfamily\hwxk{华文行楷}
\newcommand{\verylarge}{\fontsize{60pt}{\baselineskip}\selectfont}  
\newcommand{\chuhao}{\fontsize{44.9pt}{\baselineskip}\selectfont}  
\newcommand{\xiaochu}{\fontsize{38.5pt}{\baselineskip}\selectfont}  
\newcommand{\yihao}{\fontsize{27.8pt}{\baselineskip}\selectfont}  
\newcommand{\xiaoyi}{\fontsize{25.7pt}{\baselineskip}\selectfont}  
\newcommand{\erhao}{\fontsize{23.5pt}{\baselineskip}\selectfont}  
\newcommand{\xiaoerhao}{\fontsize{19.3pt}{\baselineskip}\selectfont} 
\newcommand{\sihao}{\fontsize{14pt}{\baselineskip}\selectfont}      % 字号设置  
\newcommand{\xiaosihao}{\fontsize{12pt}{\baselineskip}\selectfont}  % 字号设置  
\newcommand{\wuhao}{\fontsize{10.5pt}{\baselineskip}\selectfont}    % 字号设置  
\newcommand{\xiaowuhao}{\fontsize{9pt}{\baselineskip}\selectfont}   % 字号设置  
\newcommand{\liuhao}{\fontsize{7.875pt}{\baselineskip}\selectfont}  % 字号设置  
\newcommand{\qihao}{\fontsize{5.25pt}{\baselineskip}\selectfont}    % 字号设置 

\usepackage{diagbox}
\usepackage{multirow}
\boldmath
\XeTeXlinebreaklocale "zh"
\XeTeXlinebreakskip = 0pt plus 1pt minus 0.1pt
\definecolor{cred}{rgb}{0.8,0.8,0.8}
\definecolor{cgreen}{rgb}{0,0.3,0}
\definecolor{cpurple}{rgb}{0.5,0,0.35}
\definecolor{cdocblue}{rgb}{0,0,0.3}
\definecolor{cdark}{rgb}{0.95,1.0,1.0}
\lstset{
	language=[x86masm]Assembler,
	numbers=left,
	numberstyle=\tiny\color{black},
	showspaces=false,
	showstringspaces=false,
	basicstyle=\scriptsize,
	keywordstyle=\color{purple},
	commentstyle=\itshape\color{cgreen},
	stringstyle=\color{blue},
	frame=lines,
	% escapeinside=``,
	extendedchars=true, 
	xleftmargin=1em,
	xrightmargin=1em, 
	backgroundcolor=\color{cred},
	aboveskip=1em,
	breaklines=true,
	tabsize=4
} 

\newfontfamily{\consolas}{Consolas}
\newfontfamily{\monaco}{Monaco}
\setmonofont[Mapping={}]{Consolas}	%英文引号之类的正常显示,相当于设置英文字体
\setsansfont{Consolas} %设置英文字体 Monaco, Consolas,  Fantasque Sans Mono
\setmainfont{Times New Roman}

\setCJKmainfont{华文中宋}


\newcommand{\fic}[1]{\begin{figure}[H]
		\center
		\includegraphics[width=0.8\textwidth]{#1}
	\end{figure}}
	
\newcommand{\sizedfic}[2]{\begin{figure}[H]
		\center
		\includegraphics[width=#1\textwidth]{#2}
	\end{figure}}

\newcommand{\codefile}[1]{\lstinputlisting{#1}}

% 改变段间隔
\setlength{\parskip}{0.2em}
\linespread{1.1}

\usepackage{lastpage}
\usepackage{fancyhdr}
\pagestyle{fancy}
\lhead{\space \qquad \space}
\chead{配置ironic \qquad}
\rhead{\qquad\thepage/\pageref{LastPage}}

\begin{document}

\tableofcontents

\clearpage

\section{设置网络}

	设置网络的思路:
	\begin{itemize}
		\item[1.] 创建一个外部网络public和一个内部网络private。
		\item[2.] 创建内部网络子网池。
		\item[3.] 创建内部子网和外部子网。
		\item[4.] 创建一个路由器,将内部子网连接到路由器,再将外部网络设置为路由器的外部网关。
	\end{itemize}

	\begin{lstlisting}
	# 创建private网络
	openstack --os-cloud devstack-admin --os-region RegionOne network create --project 6eaff7ecc1694fcd94532bad5f09e17f private
	# 创建public网络
	openstack --os-cloud devstack-admin --os-region RegionOne network create public --external --default --provider-network-type flat --provider-physical-network public
	# 创建子网池
	openstack --os-cloud devstack-admin --os-region RegionOne subnet pool create shared-default-subnetpool --default-prefix-length 26 --pool-prefix 10.0.0.0/22 --share --default
	openstack --os-cloud devstack-admin --os-region RegionOne subnet pool create shared-default-subnetpool --default-prefix-length 64 --pool-prefix fd99:0295:1537::/56 --share --default
	# 创建private子网
	openstack --os-cloud devstack-admin --os-region RegionOne subnet create --project 6eaff7ecc1694fcd94532bad5f09e17f --ip-version 4 --gateway 10.1.0.1 --subnet-pool 700abf50-bd39-4e09-bbab-780d21cdde9b --network 056b7a85-726a-4d18-abef-59f2a43aa16d private-subnet
	openstack --os-cloud devstack-admin --os-region RegionOne subnet create --project 6eaff7ecc1694fcd94532bad5f09e17f --ip-version 6 --subnet-pool 254e2198-20c0-46d1-830a-ddfeaa14d3f1 --ipv6-ra-mode slaac --ipv6-address-mode slaac --network 056b7a85-726a-4d18-abef-59f2a43aa16d ipv6-private-subnet
	# 创建public子网
	openstack --os-cloud devstack-admin --os-region RegionOne subnet create --ip-version 4 --network 6b28596f-738f-4a0d-9864-0183801cdd49 --subnet-range 172.24.4.0/24 --no-dhcp public-subnet
	openstack --os-cloud devstack-admin --os-region RegionOne subnet create --ip-version 6 --gateway 2001:db8::2 --network 6b28596f-738f-4a0d-9864-0183801cdd49 --subnet-range 2001:db8::/64 --no-dhcp ipv6-public-subnet
	# 创建路由器
	openstack --os-cloud devstack-admin --os-region RegionOne router create --project 6eaff7ecc1694fcd94532bad5f09e17f router1
	# 将private subnet加入router
	openstack --os-cloud devstack-admin --os-region RegionOne router add subnet ff71a343-a4fb-4a6a-b7fc-f013566b31cc 26e240db-8cc9-4938-831a-5710d689426a
	openstack --os-cloud devstack-admin --os-region RegionOne router add subnet ff71a343-a4fb-4a6a-b7fc-f013566b31cc ebd2af8f-3450-49f4-abff-2d4577bc9b40
	# 设置外部网关,6b28596f-738f-4a0d-9864-0183801cdd49是public网络的uuid
	openstack --os-cloud devstack-admin --os-region RegionOne router set --external-gateway 6b28596f-738f-4a0d-9864-0183801cdd49 ff71a343-a4fb-4a6a-b7fc-f013566b31cc
	\end{lstlisting}

\section{配置ironic网络}

	\begin{lstlisting}
	# 获得内网的uuid
	nova network-list | grep private | cut -d"|" -f2 | cut -d" " -f2

	# 056b7a85-726a-4d18-abef-59f2a43aa16d是内网的uuid
	neutron port-create 056b7a85-726a-4d18-abef-59f2a43aa16d
	sudo ip netns exec qdhcp-056b7a85-726a-4d18-abef-59f2a43aa16d ip link list
	sudo ovs-vsctl get port tap20f9e951-ec tag
	sudo ip link show ovs-tap
	sudo ip link show brbm-tap
	sudo ip link add brbm-tap type veth peer name ovs-tap
	sudo ip link set dev brbm-tap up
	sudo ip link set dev ovs-tap up
	sudo ovs-vsctl -- --if-exists del-port ovs-tap -- add-port br-int ovs-tap tag=1
	sudo ovs-vsctl -- --if-exists del-port brbm-tap -- add-port brbm brbm-tap
	openstack port delete f44258e5-8805-460e-9de7-a7cc1af501a8

	# 建立网桥
	# 将内网设置为可共享的
	openstack network set 056b7a85-726a-4d18-abef-59f2a43aa16d --share

	# ff71a343-a4fb-4a6a-b7fc-f013566b31cc是router1的uuid
	openstack router show router1 -f value -c id
	# 获得网关地址
	sudo ip netns exec qrouter-ff71a343-a4fb-4a6a-b7fc-f013566b31cc ip -4 route get 8.8.8.8
	ip route replace 10.0.0.0/22 via 172.24.4.5

	# configure_ironic_networks
	iniset /etc/ironic/ironic.conf neutron cleaning_network private
	\end{lstlisting}

\section{启动ironic}

	\begin{lstlisting}
	# 启动ironic
	# run_process ir-api '/usr/local/bin/ironic-api --config-file=/etc/ironic/ironic.conf'
	screen -S stack -X screen -t ir-api
	screen -S stack -p ir-api -X logfile /home/pengsida/temp/logs/ir-api.log.2017-03-30-172117
	screen -S stack -p ir-api -X log on
	touch /home/pengsida/temp/logs/ir-api.log.2017-03-30-172117
	bash -c 'cd '\''/home/pengsida/temp/logs'\'' && ln -sf '\''ir-api.log.2017-03-30-172117'\'' ir-api.log'
	screen -S stack -p ir-api -X stuff '/usr/local/bin/ironic-api --config-file=/etc/ironic/ironic.conf & echo $! >/opt/stack/status/stack/ir-api.pid; fg || echo "ir-api failed to start. Exit code: $?" | tee "/opt/stack/status/stack/ir-api.failure"^M'

	# run_process ir-cond '/usr/local/bin/ironic-conductor --config-file=/etc/ironic/ironic.conf'
	screen -S stack -X screen -t ir-cond
	screen -S stack -p ir-cond -X logfile /home/pengsida/temp/logs/ir-cond.log.2017-03-30-172117
	screen -S stack -p ir-cond -X log on
	touch /home/pengsida/temp/logs/ir-cond.log.2017-03-30-172117
	bash -c 'cd '\''/home/pengsida/temp/logs'\'' && ln -sf '\''ir-cond.log.2017-03-30-172117'\'' ir-cond.log'
	screen -S stack -p ir-cond -X stuff '/usr/local/bin/ironic-conductor --config-file=/etc/ironic/ironic.conf & echo $! >/opt/stack/status/stack/ir-cond.pid; fg || echo "ir-cond failed to start. Exit code: $?" | tee "/opt/stack/status/stack/ir-cond.failure"^M'

	# restart_apache_server
	sudo service apache2 stop
	sudo service apache2 start

	# configure_ironic_ssh_keypair
	mkdir -p /opt/stack/data/ironic/ssh_keys
	ssh-keygen -q -t rsa -P '' -f /opt/stack/data/ironic/ssh_keys/ironic_key
	cat /opt/stack/data/ironic/ssh_keys/ironic_key.pub
	sort -u -o /home/pengsida/.ssh/authorized_keys /home/pengsida/.ssh/authorized_keys
	ssh -p 22 -o BatchMode=yes -o ConnectTimeout=15 -o StrictHostKeyChecking=no -i /opt/stack/data/ironic/ssh_keys/ironic_key pengsida@10.250.1.3 exit
	\end{lstlisting}

\section{创建物理机需要的镜像}
	
	\begin{lstlisting}
	# upload_baremetal_ironic_deploy
	openstack image create ir-deploy-agent_ipmitool.kernel --public --disk-format=aki --container-format=aki
	openstack image create ir-deploy-agent_ipmitool.initramfs --public --disk-format=ari --container-format=ari
	\end{lstlisting}

\section{创建物理机需要的flavor}
	\begin{lstlisting}
	openstack flavor create --ephemeral 0 --ram 1280 --disk 10 --vcpus 1 baremetal
	openstack flavor set baremetal --property cpu_arch=x86_64	
	\end{lstlisting}

\section{重启nova服务}
	\begin{lstlisting}
	# stop_nova_compute
	# stop_process n-cpu
	cat /opt/stack/status/stack/n-cpu.pid | pkill -g
	screen -S stack -p n-cpu -X kill
	
	# start_nova_compute
	# run_process n-cpu '/usr/local/bin/nova-compute --config-file /etc/nova/nova.conf'
	screen -S stack -X screen -t n-cpu
	screen -S stack -p n-cpu -X logfile /home/pengsida/temp/logs/n-cpu.log.2017-03-30-172117
	screen -S stack -p n-cpu -X log on
	touch /home/pengsida/temp/logs/n-cpu.log.2017-03-30-172117
	bash -c 'cd '\''/home/pengsida/temp/logs'\'' && ln -sf '\''n-cpu.log.2017-03-30-172117'\'' n-cpu.log'
	screen -S stack -p n-cpu -X stuff '/usr/local/bin/nova-compute --config-file /etc/nova/nova.conf & echo $! >/opt/stack/status/stack/n-cpu.pid; fg || echo "n-cpu failed to start. Exit code: $?" | tee "/opt/stack/status/stack/n-cpu.failure"^M'	
	\end{lstlisting}

\section{配置tftpd}
	\begin{lstlisting}
	# configure_tftpd
	sudo service tftpd-hpa stop
	sudo tee /etc/init/tftpd-hpa.override
	sudo cp /opt/stack/ironic/devstack/tools/ironic/templates/tftpd-xinetd.template /etc/xinetd.d/tftp
	sudo sed -e 's|%TFTPBOOT_DIR%|/opt/stack/data/ironic/tftpboot|g' -i /etc/xinetd.d/tftp
	chmod -R 0755 /opt/stack/data/ironic/tftpboot
	
	sudo service xinetd restart
	\end{lstlisting}

\section{配置iptable}
	\begin{lstlisting}
	# configure_iptables
	sudo modprobe nf_conntrack_tftp
	sudo modprobe nf_nat_tftp
	sudo iptables -I INPUT -p udp --dport 67:68 --sport 67:68 -j ACCEPT
	sudo iptables -I INPUT -d 10.250.1.3 -p udp --dport 69 -j ACCEPT
	sudo iptables -I INPUT -d 10.250.1.3 -p tcp --dport 6385 -j ACCEPT
	sudo iptables -I INPUT -d 10.250.1.3 -p tcp --dport 8080 -j ACCEPT
	sudo iptables -I INPUT -d 10.250.1.3 -p tcp --dport 9292 -j ACCEPT
	sudo iptables -I INPUT -d 10.250.1.3 -p tcp --dport 3928 -j ACCEPT
	\end{lstlisting}
	
\end{document}