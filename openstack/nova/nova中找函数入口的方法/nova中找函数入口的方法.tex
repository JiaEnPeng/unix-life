% !TeX spellcheck = en_US
%% 字体:方正静蕾简体
%%		 方正粗宋
\documentclass[a4paper,left=1.5cm,right=1.5cm,11pt]{article}

\usepackage[utf8]{inputenc}
\usepackage{fontspec}
\usepackage{cite}
\usepackage{xeCJK}
\usepackage{indentfirst}
\usepackage{titlesec}
\usepackage{etoolbox}%
\makeatletter
\patchcmd{\ttlh@hang}{\parindent\z@}{\parindent\z@\leavevmode}{}{}%
\patchcmd{\ttlh@hang}{\noindent}{}{}{}%
\makeatother

\usepackage{longtable}
\usepackage{empheq}
\usepackage{graphicx}
\usepackage{float}
\usepackage{rotating}
\usepackage{subfigure}
\usepackage{tabu}
\usepackage{amsmath}
\usepackage{setspace}
\usepackage{amsfonts}
\usepackage{appendix}
\usepackage{listings}
\usepackage{xcolor}
\usepackage{geometry}
\setcounter{secnumdepth}{4}
%\titleformat*{\section}{\LARGE}
%\renewcommand\refname{参考文献}
%\titleformat{\chapter}{\centering\bfseries\huge}{}{0.7em}{}{}
\titleformat{\section}{\LARGE\bf}{\thesection}{1em}{}{}
\titleformat{\subsection}{\Large\bfseries}{\thesubsection}{1em}{}{}
\titleformat{\subsubsection}{\large\bfseries}{\thesubsubsection}{1em}{}{}
\renewcommand{\contentsname}{{ \centerline{目{  } 录}}}
\setCJKfamilyfont{cjkhwxk}{STXINGKA.TTF}
%\setCJKfamilyfont{cjkhwxk}{华文行楷}
%\setCJKfamilyfont{cjkfzcs}{方正粗宋简体}
%\newcommand*{\cjkfzcs}{\CJKfamily{cjkfzcs}}
\newcommand*{\cjkhwxk}{\CJKfamily{cjkhwxk}}
%\newfontfamily\wryh{Microsoft YaHei}
%\newfontfamily\hwzs{华文中宋}
%\newfontfamily\hwst{华文宋体}
%\newfontfamily\hwfs{华文仿宋}
%\newfontfamily\jljt{方正静蕾简体}
%\newfontfamily\hwxk{华文行楷}
\newcommand{\verylarge}{\fontsize{60pt}{\baselineskip}\selectfont}  
\newcommand{\chuhao}{\fontsize{44.9pt}{\baselineskip}\selectfont}  
\newcommand{\xiaochu}{\fontsize{38.5pt}{\baselineskip}\selectfont}  
\newcommand{\yihao}{\fontsize{27.8pt}{\baselineskip}\selectfont}  
\newcommand{\xiaoyi}{\fontsize{25.7pt}{\baselineskip}\selectfont}  
\newcommand{\erhao}{\fontsize{23.5pt}{\baselineskip}\selectfont}  
\newcommand{\xiaoerhao}{\fontsize{19.3pt}{\baselineskip}\selectfont} 
\newcommand{\sihao}{\fontsize{14pt}{\baselineskip}\selectfont}      % 字号设置  
\newcommand{\xiaosihao}{\fontsize{12pt}{\baselineskip}\selectfont}  % 字号设置  
\newcommand{\wuhao}{\fontsize{10.5pt}{\baselineskip}\selectfont}    % 字号设置  
\newcommand{\xiaowuhao}{\fontsize{9pt}{\baselineskip}\selectfont}   % 字号设置  
\newcommand{\liuhao}{\fontsize{7.875pt}{\baselineskip}\selectfont}  % 字号设置  
\newcommand{\qihao}{\fontsize{5.25pt}{\baselineskip}\selectfont}    % 字号设置 

\usepackage{diagbox}
\usepackage{multirow}
\boldmath
\XeTeXlinebreaklocale "zh"
\XeTeXlinebreakskip = 0pt plus 1pt minus 0.1pt
\definecolor{cred}{rgb}{0.8,0.8,0.8}
\definecolor{cgreen}{rgb}{0,0.3,0}
\definecolor{cpurple}{rgb}{0.5,0,0.35}
\definecolor{cdocblue}{rgb}{0,0,0.3}
\definecolor{cdark}{rgb}{0.95,1.0,1.0}
\lstset{
	language=bash,
	numbers=left,
	numberstyle=\tiny\color{black},
	showspaces=false,
	showstringspaces=false,
	basicstyle=\scriptsize,
	keywordstyle=\color{purple},
	commentstyle=\itshape\color{cgreen},
	stringstyle=\color{blue},
	frame=lines,
	% escapeinside=``,
	extendedchars=true, 
	xleftmargin=1em,
	xrightmargin=1em, 
	backgroundcolor=\color{cred},
	aboveskip=1em,
	breaklines=true,
	tabsize=4
} 

%\newfontfamily{\consolas}{Consolas}
%\newfontfamily{\monaco}{Monaco}
%\setmonofont[Mapping={}]{Consolas}	%英文引号之类的正常显示,相当于设置英文字体
%\setsansfont{Consolas} %设置英文字体 Monaco, Consolas,  Fantasque Sans Mono
%\setmainfont{Times New Roman}
%\setCJKmainfont{STZHONGS.TTF}
%\setmonofont{Consolas}
% \newfontfamily{\consolas}{YaHeiConsolas.ttf}
\newfontfamily{\monaco}{MONACO.TTF}
\setCJKmainfont{STZHONGS.TTF}
%\setmainfont{MONACO.TTF}
%\setsansfont{MONACO.TTF}

\newcommand{\fic}[1]{\begin{figure}[H]
		\center
		\includegraphics[width=0.8\textwidth]{#1}
	\end{figure}}
	
\newcommand{\sizedfic}[2]{\begin{figure}[H]
		\center
		\includegraphics[width=#1\textwidth]{#2}
	\end{figure}}

\newcommand{\codefile}[1]{\lstinputlisting{#1}}

\newcommand{\interval}{\vspace{0.5em}}

\newcommand{\tablestart}{
	\interval
	\begin{longtable}{p{2cm}p{10cm}}
	\hline}
\newcommand{\tableend}{
	\hline
	\end{longtable}
	\interval}

% 改变段间隔
\setlength{\parskip}{0.2em}
\linespread{1.1}

\usepackage{lastpage}
\usepackage{fancyhdr}
\pagestyle{fancy}
\lhead{\space \qquad \space}
\chead{resize虚拟机的流程 \qquad}
\rhead{\qquad\thepage/\pageref{LastPage}}

\begin{document}

\tableofcontents

\clearpage

\section{分析resize流程前的必要知识}
\subsection{nova中的RPC机制}
\subsection{重要的数据类型}
\subsubsection{req}
\subsubsection{context}
	\begin{lstlisting}
	# 根据req创建环境上下文context
	# context是nova/context.py中的RequestContext类
	context = req.environ["nova.context"]
	\end{lstlisting}

\subsubsection{instance}
	\begin{lstlisting}
	# 根据req和instance_id创建instance
	# instance是nova/context/instance.py中的Instance类
	instance = self._get_server(context, req, instance_id)	


	instance_type
	flavor_id
	deltas
	quotas
	vm_state
	\end{lstlisting}

\section{nova-api阶段}
	入口函数为:
	\begin{lstlisting}
	# 这个函数在nova/api/openstack/compute/servers.py
	def _resize(self, req, instance_id, flavor_id, **kwargs):
		...
        try:
			# compute_api是nova/compute/api.py中的API类
            self.compute_api.resize(context, instance, flavor_id, **kwargs)
        ...
	\end{lstlisting}

	进一步看API.resize()函数:
	\begin{lstlisting}
	# nova/compute/api.py API.resize()
	def resize(self, context, instance, flavor_id=None,
               **extra_instance_updates):
		...

		# filter_properties与选择本地扩容或选择异地扩容有关
        filter_properties = {'ignore_hosts': []}

        if not CONF.allow_resize_to_same_host:
            filter_properties['ignore_hosts'].append(instance['host'])

        if (not flavor_id and not CONF.allow_migrate_to_same_host):
            filter_properties['ignore_hosts'].append(instance['host'])
		
		...

		# scheduler_hint挺重要的,是nova-scheduler的参数
        scheduler_hint = {'filter_properties': filter_properties}
        self.compute_task_api.resize_instance(context, instance,
                extra_instance_updates, scheduler_hint=scheduler_hint,
                flavor=new_instance_type,
                reservations=quotas.reservations or [])
	\end{lstlisting}

	\begin{lstlisting}
	# nova/conductor/api.py ComputeTaskAPI.resize_instance()
	def resize_instance(self, context, instance, extra_instance_updates,
                        scheduler_hint, flavor, reservations):
        self.conductor_compute_rpcapi.migrate_server(
            context, instance, scheduler_hint, False, False, flavor,
            None, None, reservations)
	\end{lstlisting}

	\begin{lstlisting}
	# nova/conductor/rpcapi.py ComputeTaskAPI.migrate_server()
	def migrate_server(self, context, instance, scheduler_hint, live, rebuild,
                  flavor, block_migration, disk_over_commit,
                  reservations=None):
		...
		cctxt = self.client.prepare(version=version)
        return cctxt.call(context, 'migrate_server',
                          instance=instance, scheduler_hint=scheduler_hint,
                          live=live, rebuild=rebuild, flavor=flavor_p,
                          block_migration=block_migration,
                          disk_over_commit=disk_over_commit,
                          reservations=reservations)
	\end{lstlisting}

\section{nova-conductor部分}

	\begin{lstlisting}
	# nova/conductor/manager.py ComputeTaskManager.migrate_server()
	def migrate_server(self, context, instance, scheduler_hint, live, rebuild,
            flavor, block_migration, disk_over_commit, reservations=None):
        ...
        if live and not rebuild and not flavor:
            self._live_migrate(context, instance, scheduler_hint,
                               block_migration, disk_over_commit)
        elif not live and not rebuild and flavor:
            ...
            with compute_utils.EventReporter(context, 'cold_migrate',
                                             instance_uuid):
                self._cold_migrate(context, instance, flavor,
                                   scheduler_hint['filter_properties'],
                                   reservations)
        ...
	\end{lstlisting}

\section{冷迁移}

\subsection{冷迁移中的nova-conductor部分}

	\begin{lstlisting}
	# nova/conductor/manager.py ComputeTaskManager._cold_migrate()
	def _cold_migrate(self, context, instance, flavor, filter_properties,
                      reservations):
        ...
        try:
			...
			# 选择目的主机
            hosts = self.scheduler_client.select_destinations(
                    context, request_spec, filter_properties)
            host_state = hosts[0]
        ...

        try:
            ...
            (host, node) = (host_state['host'], host_state['nodename'])
            self.compute_rpcapi.prep_resize(
                context, image, instance,
                flavor, host,
                reservations, request_spec=request_spec,
                filter_properties=filter_properties, node=node)
        ...
	\end{lstlisting}

	\begin{lstlisting}
	# nova/compute/rpcapi.py ComputeAPI.prep_resize()
	def prep_resize(self, ctxt, image, instance, instance_type, host,
                    reservations=None, request_spec=None,
                    filter_properties=None, node=None):
        ...
		cctxt = self.client.prepare(server=host, version=version)
        cctxt.cast(ctxt, 'prep_resize',
                   instance=instance,
                   instance_type=instance_type_p,
                   image=image_p, reservations=reservations,
                   request_spec=request_spec,
                   filter_properties=filter_properties,
                   node=node)
	\end{lstlisting}

\subsection{冷迁移中的nova-compute部分}
\subsubsection{目的主机上的操作:prep\_resize}
	\begin{lstlisting}
	# nova/compute/manager.py ComputeManager.prep_resize()
	def prep_resize(self, context, image, instance, instance_type,
                    reservations, request_spec, filter_properties, node):
        ...
        with self._error_out_instance_on_exception(context, instance,
                                                   quotas=quotas):
            ...
            try:
                self._prep_resize(context, image, instance,
                                  instance_type, quotas,
                                  request_spec, filter_properties,
                                  node)
			...
	\end{lstlisting}

	\begin{lstlisting}
	# nova/compute/manager.py ComputeManager._prep_resize()
	def _prep_resize(self, context, image, instance, instance_type,
            quotas, request_spec, filter_properties, node):

        ...
        with rt.resize_claim(context, instance, instance_type,
                             image_meta=image, limits=limits) as claim:
			...
            self.compute_rpcapi.resize_instance(
                    context, instance, claim.migration, image,
                    instance_type, quotas.reservations)
	\end{lstlisting}

	\begin{lstlisting}
	# nova/compute/rpcapi.py ComputeAPI.resize_instance()
	def resize_instance(self, ctxt, instance, migration, image, instance_type,
                        reservations=None):
        ...
        cctxt = self.client.prepare(server=_compute_host(None, instance),
                version=version)
        cctxt.cast(ctxt, 'resize_instance',
                   instance=instance, migration=migration,
                   image=image, reservations=reservations,
                   instance_type=instance_type_p)
	\end{lstlisting}

\subsubsection{源主机的操作:resize\_instance}
	\begin{lstlisting}
	# nova/compute/manager.py ComputeManager.resize_instance()
	def resize_instance(self, context, instance, image,
                        reservations, migration, instance_type,
                        clean_shutdown=True):
		...
        with self._error_out_instance_on_exception(context, instance,
                                                   quotas=quotas):
			...
			# 获得虚拟机块设备的信息
			block_device_info = self._get_instance_block_device_info(
                                context, instance, bdms=bdms)
			
			# 关闭虚拟机并迁移虚拟机的增量文件
            disk_info = self.driver.migrate_disk_and_power_off(
                    context, instance, migration.dest_host,
                    instance_type, network_info,
                    block_device_info,
                    timeout, retry_interval)
			...
            self.compute_rpcapi.finish_resize(context, instance,
                    migration, image, disk_info,
                    migration.dest_compute, reservations=quotas.reservations)
            ...
	\end{lstlisting}

	migrate\_disk\_and\_power\_off()是源主机上将虚拟机迁移给目的主机的实现函数,主要利用了libvirt API。
	这个函数的分析在《nova调用libvirt》中的“nova扩容时对libvirt的调用”一节。

	\begin{lstlisting}
	# nova/compute/rpcapi.py(690) ComputeAPI.finish_resize()
	def finish_resize(self, ctxt, instance, migration, image, disk_info,
            host, reservations=None):
		...
        cctxt = self.client.prepare(server=host, version=version)
        cctxt.cast(ctxt, 'finish_resize',
                   instance=instance, migration=migration,
                   image=image, disk_info=disk_info, reservations=reservations)
	\end{lstlisting}

\subsubsection{目的主机上的操作:finish\_resize}
	\begin{lstlisting}
	# nova/compute/manager.py ComputeManager.finish_resize()
	def finish_resize(self, context, disk_info, image, instance,
                      reservations, migration):
        quotas = quotas_obj.Quotas.from_reservations(context,
                                                     reservations,
                                                     instance=instance)
        try:
            self._finish_resize(context, instance, migration,
                                disk_info, image)
        ...
	\end{lstlisting}

	\begin{lstlisting}
	# nova/compute/manager.py ComputeManager._finish_resize()
	def _finish_resize(self, context, instance, migration, disk_info,
                       image):
        ...
        try:
            self.driver.finish_migration(context, migration, instance,
                                         disk_info,
                                         network_info,
                                         image, resize_instance,
                                         block_device_info, power_on)
        ...
	\end{lstlisting}

\end{document}