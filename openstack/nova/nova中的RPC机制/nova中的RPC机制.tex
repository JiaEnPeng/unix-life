% !TeX spellcheck = en_US
%% 字体:方正静蕾简体
%%		 方正粗宋
\documentclass[a4paper,left=2.5cm,right=2.5cm,11pt]{article}

\usepackage[utf8]{inputenc}
\usepackage{fontspec}
\usepackage{cite}
\usepackage{xeCJK}
\usepackage{indentfirst}
\usepackage{titlesec}
\usepackage{longtable}
\usepackage{graphicx}
\usepackage{float}
\usepackage{rotating}
\usepackage{subfigure}
\usepackage{tabu}
\usepackage{amsmath}
\usepackage{setspace}
\usepackage{amsfonts}
\usepackage{appendix}
\usepackage{listings}
\usepackage{xcolor}
\usepackage{geometry}
\setcounter{secnumdepth}{4}
\usepackage{mhchem}
\usepackage{multirow}
\usepackage{extarrows}
\usepackage{hyperref}
\titleformat*{\section}{\LARGE}
\renewcommand\refname{参考文献}
\renewcommand{\abstractname}{\sihao \cjkfzcs 摘{  }要}
%\titleformat{\chapter}{\centering\bfseries\huge\wryh}{}{0.7em}{}{}
%\titleformat{\section}{\LARGE\bf}{\thesection}{1em}{}{}
\titleformat{\subsection}{\Large\bfseries}{\thesubsection}{1em}{}{}
\titleformat{\subsubsection}{\large\bfseries}{\thesubsubsection}{1em}{}{}
\renewcommand{\contentsname}{{\cjkfzcs \centerline{目{  } 录}}}
\setCJKfamilyfont{cjkhwxk}{STXingkai}
\setCJKfamilyfont{cjkfzcs}{STSongti-SC-Regular}
% \setCJKfamilyfont{cjkhwxk}{华文行楷}
% \setCJKfamilyfont{cjkfzcs}{方正粗宋简体}
\newcommand*{\cjkfzcs}{\CJKfamily{cjkfzcs}}
\newcommand*{\cjkhwxk}{\CJKfamily{cjkhwxk}}
\newfontfamily\wryh{Microsoft YaHei}
\newfontfamily\hwzs{STZhongsong}
\newfontfamily\hwst{STSong}
\newfontfamily\hwfs{STFangsong}
\newfontfamily\jljt{MicrosoftYaHei}
\newfontfamily\hwxk{STXingkai}
% \newfontfamily\hwzs{华文中宋}
% \newfontfamily\hwst{华文宋体}
% \newfontfamily\hwfs{华文仿宋}
% \newfontfamily\jljt{方正静蕾简体}
% \newfontfamily\hwxk{华文行楷}
\newcommand{\verylarge}{\fontsize{60pt}{\baselineskip}\selectfont}  
\newcommand{\chuhao}{\fontsize{44.9pt}{\baselineskip}\selectfont}  
\newcommand{\xiaochu}{\fontsize{38.5pt}{\baselineskip}\selectfont}  
\newcommand{\yihao}{\fontsize{27.8pt}{\baselineskip}\selectfont}  
\newcommand{\xiaoyi}{\fontsize{25.7pt}{\baselineskip}\selectfont}  
\newcommand{\erhao}{\fontsize{23.5pt}{\baselineskip}\selectfont}  
\newcommand{\xiaoerhao}{\fontsize{19.3pt}{\baselineskip}\selectfont} 
\newcommand{\sihao}{\fontsize{14pt}{\baselineskip}\selectfont}      % 字号设置  
\newcommand{\xiaosihao}{\fontsize{12pt}{\baselineskip}\selectfont}  % 字号设置  
\newcommand{\wuhao}{\fontsize{10.5pt}{\baselineskip}\selectfont}    % 字号设置  
\newcommand{\xiaowuhao}{\fontsize{9pt}{\baselineskip}\selectfont}   % 字号设置  
\newcommand{\liuhao}{\fontsize{7.875pt}{\baselineskip}\selectfont}  % 字号设置  
\newcommand{\qihao}{\fontsize{5.25pt}{\baselineskip}\selectfont}    % 字号设置 

\usepackage{diagbox}
\usepackage{multirow}
\boldmath
\XeTeXlinebreaklocale "zh"
\XeTeXlinebreakskip = 0pt plus 1pt minus 0.1pt
\definecolor{cred}{rgb}{0.8,0.8,0.8}
\definecolor{cgreen}{rgb}{0,0.3,0}
\definecolor{cpurple}{rgb}{0.5,0,0.35}
\definecolor{cdocblue}{rgb}{0,0,0.3}
\definecolor{cdark}{rgb}{0.95,1.0,1.0}
\lstset{
	language=python,
	numbers=left,
	numberstyle=\tiny\color{white},
	showspaces=false,
	showstringspaces=false,
	basicstyle=\scriptsize,
	keywordstyle=\color{purple},
	commentstyle=\itshape\color{cgreen},
	stringstyle=\color{blue},
	frame=lines,
	% escapeinside=``,
	extendedchars=true, 
	xleftmargin=0em,
	xrightmargin=0em, 
	backgroundcolor=\color{cred},
	aboveskip=1em,
	breaklines=true,
	tabsize=4
} 

\newfontfamily{\consolas}{Consolas}
\newfontfamily{\monaco}{Monaco}
\setmonofont[Mapping={}]{Consolas}	%英文引号之类的正常显示,相当于设置英文字体
\setsansfont{Consolas} %设置英文字体 Monaco, Consolas,  Fantasque Sans Mono
\setmainfont{Times New Roman}

\setCJKmainfont{华文中宋}


\newcommand{\fic}[1]{\begin{figure}[H]
		\center
		\includegraphics[width=0.8\textwidth]{#1}
	\end{figure}}
	
\newcommand{\sizedfic}[2]{\begin{figure}[H]
		\center
		\includegraphics[width=#1\textwidth]{#2}
	\end{figure}}

\newcommand{\codefile}[1]{\lstinputlisting{#1}}

\newcommand{\interval}{\vspace{0.5em}}

\newcommand{\tablestart}{
	\interval
	\begin{longtable}{p{2cm}p{10cm}}
	\hline}
\newcommand{\tableend}{
	\hline
	\end{longtable}
	\interval}

% 改变段间隔
\setlength{\parskip}{0.2em}
\linespread{1.1}

\usepackage{lastpage}
\usepackage{fancyhdr}
\pagestyle{fancy}
\lhead{\space \qquad \space}
\chead{nova中的RPC机制 \qquad}
\rhead{\qquad\thepage/\pageref{LastPage}}

\begin{document}

\tableofcontents

\clearpage

\section{nova中的RPC机制}
\subsection{基础部分}
\subsubsection{rpc.py}
    \begin{lstlisting}
    def get_client(target, version_cap=None, serializer=None):
        assert TRANSPORT is not None
        serializer = RequestContextSerializer(serializer)
        return messaging.RPCClient(TRANSPORT,
                                target,
                                version_cap=version_cap,
                                serializer=serializer)


    def get_server(target, endpoints, serializer=None):
        assert TRANSPORT is not None
        serializer = RequestContextSerializer(serializer)
        return messaging.get_rpc_server(TRANSPORT,
                                        target,
                                        endpoints,
                                        executor='eventlet',
                                        serializer=serializer)
    \end{lstlisting}

\subsubsection{olso.messaging}
   \begin{lstlisting}
class Target(object):

    """Identifies the destination of messages.

    A Target encapsulates all the information to identify where a message
    should be sent or what messages a server is listening for.

    Different subsets of the information encapsulated in a Target object is
    relevant to various aspects of the API:

      creating a server:
        topic and server is required; exchange is optional
      an endpoint's target:
        namespace and version is optional
      client sending a message:
        topic is required, all other attributes optional

    Its attributes are:

    :param exchange: A scope for topics. Leave unspecified to default to the
      control_exchange configuration option.
    :type exchange: str
    :param topic: A name which identifies the set of interfaces exposed by a
      server. Multiple servers may listen on a topic and messages will be
      dispatched to one of the servers in a round-robin fashion.
    :type topic: str
    :param namespace: Identifies a particular interface (i.e. set of methods)
      exposed by a server. The default interface has no namespace identifier
      and is referred to as the null namespace.
    :type namespace: str
    :param version: Interfaces have a major.minor version number associated
      with them. A minor number increment indicates a backwards compatible
      change and an incompatible change is indicated by a major number bump.
      Servers may implement multiple major versions and clients may require
      indicate that their message requires a particular minimum minor version.
    :type version: str
    :param server: Clients can request that a message be directed to a specific
      server, rather than just one of a pool of servers listening on the topic.
    :type server: str
    :param fanout: Clients may request that a message be directed to all
      servers listening on a topic by setting fanout to ``True``, rather than
      just one of them.
    :type fanout: bool
    """

    def __init__(self, exchange=None, topic=None, namespace=None,
                 version=None, server=None, fanout=None):
        self.exchange = exchange
        self.topic = topic
        self.namespace = namespace
        self.version = version
        self.server = server
        self.fanout = fanout

class RPCClient(object):
    def prepare(self, exchange=_marker, topic=_marker, namespace=_marker,
                version=_marker, server=_marker, fanout=_marker,
                timeout=_marker, version_cap=_marker, retry=_marker):
        """Prepare a method invocation context.

        Use this method to override client properties for an individual method
        invocation. For example::

            def test(self, ctxt, arg):
                cctxt = self.prepare(version='2.5')
                return cctxt.call(ctxt, 'test', arg=arg)

        :param exchange: see Target.exchange
        :type exchange: str
        :param topic: see Target.topic
        :type topic: str
        :param namespace: see Target.namespace
        :type namespace: str
        :param version: requirement the server must support, see Target.version
        :type version: str
        :param server: send to a specific server, see Target.server
        :type server: str
        :param fanout: send to all servers on topic, see Target.fanout
        :type fanout: bool
        :param timeout: an optional default timeout (in seconds) for call()s
        :type timeout: int or float
        :param version_cap: raise a RPCVersionCapError version exceeds this cap
        :type version_cap: str
        :param retry: an optional connection retries configuration
                      None or -1 means to retry forever
                      0 means no retry
                      N means N retries
        :type retry: int
        """
        return _CallContext._prepare(self,
                                     exchange, topic, namespace,
                                     version, server, fanout,
                                     timeout, version_cap, retry)

class _CallContext(object):

    _marker = object()

    def __init__(self, transport, target, serializer,
                 timeout=None, version_cap=None, retry=None):
        self.conf = transport.conf

        self.transport = transport
        self.target = target
        self.serializer = serializer
        self.timeout = timeout
        self.retry = retry
        self.version_cap = version_cap

        super(_CallContext, self).__init__()

    def cast(self, ctxt, method, **kwargs):
        """Invoke a method and return immediately. See RPCClient.cast()."""
        msg = self._make_message(ctxt, method, kwargs)
        ctxt = self.serializer.serialize_context(ctxt)

        if self.version_cap:
            self._check_version_cap(msg.get('version'))
        try:
            self.transport._send(self.target, ctxt, msg, retry=self.retry)
        except driver_base.TransportDriverError as ex:
            raise ClientSendError(self.target, ex)

    def call(self, ctxt, method, **kwargs):
        """Invoke a method and wait for a reply. See RPCClient.call()."""
        msg = self._make_message(ctxt, method, kwargs)
        msg_ctxt = self.serializer.serialize_context(ctxt)

        timeout = self.timeout
        if self.timeout is None:
            timeout = self.conf.rpc_response_timeout

        if self.version_cap:
            self._check_version_cap(msg.get('version'))

        try:
            result = self.transport._send(self.target, msg_ctxt, msg,
                                          wait_for_reply=True, timeout=timeout,
                                          retry=self.retry)
        except driver_base.TransportDriverError as ex:
            raise ClientSendError(self.target, ex)
        return self.serializer.deserialize_entity(ctxt, result)

    @classmethod
    def _prepare(cls, base,
                 exchange=_marker, topic=_marker, namespace=_marker,
                 version=_marker, server=_marker, fanout=_marker,
                 timeout=_marker, version_cap=_marker, retry=_marker):
        """Prepare a method invocation context. See RPCClient.prepare()."""
        kwargs = dict(
            exchange=exchange,
            topic=topic,
            namespace=namespace,
            version=version,
            server=server,
            fanout=fanout)
        kwargs = dict([(k, v) for k, v in kwargs.items()
                       if v is not cls._marker])
        target = base.target(**kwargs)

        if timeout is cls._marker:
            timeout = base.timeout
        if retry is cls._marker:
            retry = base.retry
        if version_cap is cls._marker:
            version_cap = base.version_cap

        return _CallContext(base.transport, target,
                            base.serializer,
                            timeout, version_cap, retry)

    def prepare(self, exchange=_marker, topic=_marker, namespace=_marker,
                version=_marker, server=_marker, fanout=_marker,
                timeout=_marker, version_cap=_marker, retry=_marker):
        """Prepare a method invocation context. See RPCClient.prepare()."""
        return self._prepare(self,
                             exchange, topic, namespace,
                             version, server, fanout,
                             timeout, version_cap, retry)

    \end{lstlisting}

\subsubsection{Manager类}
    \begin{lstlisting}
class Manager(base.Base, periodic_task.PeriodicTasks):

    def __init__(self, host=None, db_driver=None, service_name='undefined'):
        if not host:
            host = CONF.host
        self.host = host
        self.service_name = service_name
        self.notifier = rpc.get_notifier(self.service_name, self.host)
        ...
    \end{lstlisting}

\subsection{nova-compute部分}
\subsubsection{computeAPI类}
    \begin{lstlisting}
class ComputeAPI(object):
    def __init__(self):
        super(ComputeAPI, self).__init__()
        target = messaging.Target(topic=CONF.compute_topic, version='3.0')
        ...
        self.client = self.get_client(target, version_cap, serializer)

    def get_client(self, target, version_cap, serializer):
        return rpc.get_client(target,
                              version_cap=version_cap,
                              serializer=serializer)
    \end{lstlisting}

    举个例子:
    \begin{lstlisting}
    def attach_interface(self, ctxt, instance, network_id, port_id,
                         requested_ip):
        version = '3.17'
        cctxt = self.client.prepare(server=_compute_host(None, instance),
                version=version)
        return cctxt.call(ctxt, 'attach_interface',
                          instance=instance, network_id=network_id,
                          port_id=port_id, requested_ip=requested_ip)
    \end{lstlisting}

\subsubsection{ComputeManager类}
    \begin{lstlisting}
class ComputeManager(manager.Manager):
    
    def __init__(self, compute_driver=None, *args, **kwargs):
        ...
        super(ComputeManager, self).__init__(service_name="compute",
                                             *args, **kwargs)
        ...
    \end{lstlisting}

\subsubsection{nova-compute部分总结}
    ComputeManager类接受ComputeAPI类的RPC请求,主机信息由RPC.cast()或RPC.call()中的server决定。

\subsection{nova-conductor部分}
\subsubsection{ComputeTaskAPI类}
    \begin{lstlisting}
class ComputeTaskAPI(object):
    def __init__(self):
        super(ComputeTaskAPI, self).__init__()
        target = messaging.Target(topic=CONF.conductor.topic,
                                  namespace='compute_task',
                                  version='1.0')
        serializer = objects_base.NovaObjectSerializer()
        self.client = rpc.get_client(target, serializer=serializer)
    \end{lstlisting}
\subsubsection{ComputeTaskManager类}
    \begin{lstlisting}
class ComputeTaskManager(base.Base):
    target = messaging.Target(namespace='compute_task', version='1.9')

    def __init__(self):
        super(ComputeTaskManager, self).__init__()
        self.compute_rpcapi = compute_rpcapi.ComputeAPI()
        self.image_api = image.API()
        self.scheduler_client = scheduler_client.SchedulerClient()
    \end{lstlisting}

\subsubsection{nova-conductor部分总结}
    ComputeTaskManager类接受ComputeTaskAPI类的RPC请求,主机信息由RPC.cast()或RPC.call()中的server决定。

\subsection{nova-scheduler部分}
\subsubsection{SchedulerAPI类}
    \begin{lstlisting}
class SchedulerAPI(object):
    
    def __init__(self):
        super(SchedulerAPI, self).__init__()
        target = messaging.Target(topic=CONF.scheduler_topic, version='3.0')
        version_cap = self.VERSION_ALIASES.get(CONF.upgrade_levels.scheduler,
                                               CONF.upgrade_levels.scheduler)
        serializer = objects_base.NovaObjectSerializer()
        self.client = rpc.get_client(target, version_cap=version_cap,
                                     serializer=serializer)

    def select_destinations(self, ctxt, request_spec, filter_properties):
        cctxt = self.client.prepare()
        return cctxt.call(ctxt, 'select_destinations',
            request_spec=request_spec, filter_properties=filter_properties)
    \end{lstlisting}

\subsubsection{SchedulerManager类}
    \begin{lstlisting}
class SchedulerManager(manager.Manager):
    """Chooses a host to run instances on."""

    target = messaging.Target(version='3.0')

    def __init__(self, scheduler_driver=None, *args, **kwargs):
        if not scheduler_driver:
            scheduler_driver = CONF.scheduler_driver
        self.driver = importutils.import_object(scheduler_driver)
        self.compute_rpcapi = compute_rpcapi.ComputeAPI()
        super(SchedulerManager, self).__init__(service_name='scheduler',
                                               *args, **kwargs)
    \end{lstlisting}

\subsubsection{nova-scheduler部分总结}
    ComputeManager类接受ComputeAPI类的RPC请求,主机信息由RPC.cast()或RPC.call()中的server决定。

\end{document}