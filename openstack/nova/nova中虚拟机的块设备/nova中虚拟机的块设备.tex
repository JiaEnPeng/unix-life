% !TeX spellcheck = en_US
%% 字体:方正静蕾简体
%%		 方正粗宋
\documentclass[a4paper,left=1.5cm,right=1.5cm,11pt]{article}

\usepackage[utf8]{inputenc}
\usepackage{fontspec}
\usepackage{cite}
\usepackage{xeCJK}
\usepackage{indentfirst}
\usepackage{titlesec}
\usepackage{etoolbox}%
\makeatletter
\patchcmd{\ttlh@hang}{\parindent\z@}{\parindent\z@\leavevmode}{}{}%
\patchcmd{\ttlh@hang}{\noindent}{}{}{}%
\makeatother

\usepackage{hyperref}
\usepackage{longtable}
\usepackage{empheq}
\usepackage{graphicx}
\usepackage{float}
\usepackage{rotating}
\usepackage{subfigure}
\usepackage{tabu}
\usepackage{amsmath}
\usepackage{setspace}
\usepackage{amsfonts}
\usepackage{appendix}
\usepackage{listings}
\usepackage{xcolor}
\usepackage{geometry}
\setcounter{secnumdepth}{4}
%\titleformat*{\section}{\LARGE}
%\renewcommand\refname{参考文献}
%\titleformat{\chapter}{\centering\bfseries\huge}{}{0.7em}{}{}
\titleformat{\section}{\LARGE\bf}{\thesection}{1em}{}{}
\titleformat{\subsection}{\Large\bfseries}{\thesubsection}{1em}{}{}
\titleformat{\subsubsection}{\large\bfseries}{\thesubsubsection}{1em}{}{}
\renewcommand{\contentsname}{{ \centerline{目{  } 录}}}
\setCJKfamilyfont{cjkhwxk}{STXINGKA.TTF}
%\setCJKfamilyfont{cjkhwxk}{华文行楷}
%\setCJKfamilyfont{cjkfzcs}{方正粗宋简体}
%\newcommand*{\cjkfzcs}{\CJKfamily{cjkfzcs}}
\newcommand*{\cjkhwxk}{\CJKfamily{cjkhwxk}}
%\newfontfamily\wryh{Microsoft YaHei}
%\newfontfamily\hwzs{华文中宋}
%\newfontfamily\hwst{华文宋体}
%\newfontfamily\hwfs{华文仿宋}
%\newfontfamily\jljt{方正静蕾简体}
%\newfontfamily\hwxk{华文行楷}
\newcommand{\verylarge}{\fontsize{60pt}{\baselineskip}\selectfont}  
\newcommand{\chuhao}{\fontsize{44.9pt}{\baselineskip}\selectfont}  
\newcommand{\xiaochu}{\fontsize{38.5pt}{\baselineskip}\selectfont}  
\newcommand{\yihao}{\fontsize{27.8pt}{\baselineskip}\selectfont}  
\newcommand{\xiaoyi}{\fontsize{25.7pt}{\baselineskip}\selectfont}  
\newcommand{\erhao}{\fontsize{23.5pt}{\baselineskip}\selectfont}  
\newcommand{\xiaoerhao}{\fontsize{19.3pt}{\baselineskip}\selectfont} 
\newcommand{\sihao}{\fontsize{14pt}{\baselineskip}\selectfont}      % 字号设置  
\newcommand{\xiaosihao}{\fontsize{12pt}{\baselineskip}\selectfont}  % 字号设置  
\newcommand{\wuhao}{\fontsize{10.5pt}{\baselineskip}\selectfont}    % 字号设置  
\newcommand{\xiaowuhao}{\fontsize{9pt}{\baselineskip}\selectfont}   % 字号设置  
\newcommand{\liuhao}{\fontsize{7.875pt}{\baselineskip}\selectfont}  % 字号设置  
\newcommand{\qihao}{\fontsize{5.25pt}{\baselineskip}\selectfont}    % 字号设置 

\usepackage{diagbox}
\usepackage{multirow}
\boldmath
\XeTeXlinebreaklocale "zh"
\XeTeXlinebreakskip = 0pt plus 1pt minus 0.1pt
\definecolor{cred}{rgb}{0.8,0.8,0.8}
\definecolor{cgreen}{rgb}{0,0.3,0}
\definecolor{cpurple}{rgb}{0.5,0,0.35}
\definecolor{cdocblue}{rgb}{0,0,0.3}
\definecolor{cdark}{rgb}{0.95,1.0,1.0}
\lstset{
	language=bash,
	numbers=left,
	numberstyle=\tiny\color{black},
	showspaces=false,
	showstringspaces=false,
	basicstyle=\scriptsize,
	keywordstyle=\color{purple},
	commentstyle=\itshape\color{cgreen},
	stringstyle=\color{blue},
	frame=lines,
	% escapeinside=``,
	extendedchars=true, 
	xleftmargin=1em,
	xrightmargin=1em, 
	backgroundcolor=\color{cred},
	aboveskip=1em,
	breaklines=true,
	tabsize=4
} 

%\newfontfamily{\consolas}{Consolas}
%\newfontfamily{\monaco}{Monaco}
%\setmonofont[Mapping={}]{Consolas}	%英文引号之类的正常显示,相当于设置英文字体
%\setsansfont{Consolas} %设置英文字体 Monaco, Consolas,  Fantasque Sans Mono
%\setmainfont{Times New Roman}
%\setCJKmainfont{STZHONGS.TTF}
%\setmonofont{Consolas}
% \newfontfamily{\consolas}{YaHeiConsolas.ttf}
\newfontfamily{\monaco}{MONACO.TTF}
\setCJKmainfont{STZHONGS.TTF}
%\setmainfont{MONACO.TTF}
%\setsansfont{MONACO.TTF}

\newcommand{\fic}[1]{\begin{figure}[H]
		\center
		\includegraphics[width=0.8\textwidth]{#1}
	\end{figure}}
	
\newcommand{\sizedfic}[2]{\begin{figure}[H]
		\center
		\includegraphics[width=#1\textwidth]{#2}
	\end{figure}}

\newcommand{\codefile}[1]{\lstinputlisting{#1}}

\newcommand{\interval}{\vspace{0.5em}}

\newcommand{\tablestart}{
	\interval
	\begin{longtable}{p{2cm}p{10cm}}
	\hline}
\newcommand{\tableend}{
	\hline
	\end{longtable}
	\interval}

% 改变段间隔
\setlength{\parskip}{0.2em}
\linespread{1.1}

\usepackage{lastpage}
\usepackage{fancyhdr}
\pagestyle{fancy}
\lhead{\space \qquad \space}
\chead{nova中虚拟机的块设备 \qquad}
\rhead{\qquad\thepage/\pageref{LastPage}}

\begin{document}

\tableofcontents

\clearpage
\section{虚拟机的本地磁盘空间}
\subsection{虚拟机的三个分区}
    一个虚拟机有root分区、swap分区和ephemeral分区:
    \begin{itemize}
        \item[1.] root disk:根分区,提供boot loader。
        \item[2.] swap disk:交换分区,用于内存耗尽时将物理内存中一部分空间挪到swap分区中,从而释放一些物理内存空间。
        \item[3.] ephemeral disk:狭义上的ephemeral disk,指的是根据资源状况提供额外的临时存储。
    \end{itemize}

    广义上的ephemeral disk指的是root分区、swap分区和ephemeral分区,这个ephemeral disk随着instance的生命周期创建和消亡。\par

    nova flavor中定义了一个虚拟机的三个分区,Disk指的是root分区,Ephemeral指的是ephemeral分区,Swap指的是swap分区,如下图所示:
    \fic{1.png}

\subsection{nova compute节点上的磁盘镜像文件}
    虚拟机的root分区、ephemeral分区、swap分区在nova compute节点上都有一个磁盘镜像文件,默认存放在/var/lib/nova/instances/vm-uuid目录中,如下图所示:
    \fic{2.png}

    上图中,disk对应root分区,disk.local对应ephemeral分区,disk.swap对应swap分区。\par

    这些磁盘文件都是qcow2格式,都是overlay文件,都有对应的backing file,这些backing file默认是raw格式,可以通过如下命令查看它们后端镜像的位置:
    \begin{lstlisting}
    qemu-img info disk
    qemu-img info disk.local
    qemu-img info disk.swap
    \end{lstlisting}

    libvirt中的xml配置文件定义了磁盘的镜像文件和磁盘的对应关系:
    \begin{lstlisting}
    <disk type="file" device="disk">
      <driver name="qemu" type="qcow2" cache="none"/>
      <source file="/var/lib/nova/instances/eddc46a8-e026-4b2c-af51-dfaa436fcc7b/disk"/>
      <target bus="virtio" dev="vda"/>
    </disk>
    <disk type="file" device="disk">
      <driver name="qemu" type="qcow2" cache="none"/>
      <source file="/var/lib/nova/instances/eddc46a8-e026-4b2c-af51-dfaa436fcc7b/disk.local"/>
      <target bus="virtio" dev="vdb"/>
    </disk>
    <disk type="file" device="disk">
      <driver name="qemu" type="qcow2" cache="none"/>
      <source file="/var/lib/nova/instances/eddc46a8-e026-4b2c-af51-dfaa436fcc7b/disk.swap"/>
      <target bus="virtio" dev="vdc"/>
    </disk>
    \end{lstlisting}

\section{nova虚拟机的block\_device\_info数据结构}
    block\_device\_info记录了虚拟机的所有磁盘和被附加的所有卷:
    \begin{lstlisting}
block_device_info = { 
            'root_device_name': "/dev/sda", # root分区
            'swap': {          # swap 分区
               'device_name': "/dev/sdb", 
               'swap_size': 5, 
            } 
            'ephemerals': [    # ephemerals 分区,可以有多个
               {'num': 0, 
                'virtual_name': 'eph0', 
                'device_name': "/dev/sdc", 
                'size': 5 }, 
               {'num': 1, 
                'virtual_name': 'eph1', 
                'device_name': "/dev/sdd", 
                'size': 5 }, 
               {'num': 2, 
                'virtual_name': 'eph2', 
                'device_name': "/dev/sde", 
                'size': 5 }, 
               ... 
            ], 
            'block_device_mapping': [ # block devices mapping,可以有多个
               {'cinfo': {....some cinder volume data....}, 
                'mount_device': "/dev/sdf", 
                'delete_on_termination': True }, 
               {'cinfo': {....some cinder volume data....}, 
                'mount_device': "/dev/sdg", 
                'delete_on_termination': True }, 
               {'cinfo': {....some cinder volume data....}, 
                'mount_device': "/dev/sdh", 
                'delete_on_termination': True }, 
               ... 
            ], 
     }
    \end{lstlisting}

\end{document}