% !TeX spellcheck = en_US
%% 字体:方正静蕾简体
%%		 方正粗宋
\documentclass[a4paper,left=1.5cm,right=1.5cm,11pt]{article}

\usepackage[utf8]{inputenc}
\usepackage{fontspec}
\usepackage{cite}
\usepackage{xeCJK}
\usepackage{indentfirst}
\usepackage{titlesec}
\usepackage{etoolbox}%
\makeatletter
\patchcmd{\ttlh@hang}{\parindent\z@}{\parindent\z@\leavevmode}{}{}%
\patchcmd{\ttlh@hang}{\noindent}{}{}{}%
\makeatother

\usepackage{hyperref}
\usepackage{longtable}
\usepackage{empheq}
\usepackage{graphicx}
\usepackage{float}
\usepackage{rotating}
\usepackage{subfigure}
\usepackage{tabu}
\usepackage{amsmath}
\usepackage{setspace}
\usepackage{amsfonts}
\usepackage{appendix}
\usepackage{listings}
\usepackage{xcolor}
\usepackage{geometry}
\setcounter{secnumdepth}{4}
%\titleformat*{\section}{\LARGE}
%\renewcommand\refname{参考文献}
%\titleformat{\chapter}{\centering\bfseries\huge}{}{0.7em}{}{}
\titleformat{\section}{\LARGE\bf}{\thesection}{1em}{}{}
\titleformat{\subsection}{\Large\bfseries}{\thesubsection}{1em}{}{}
\titleformat{\subsubsection}{\large\bfseries}{\thesubsubsection}{1em}{}{}
\renewcommand{\contentsname}{{ \centerline{目{  } 录}}}
\setCJKfamilyfont{cjkhwxk}{STXINGKA.TTF}
%\setCJKfamilyfont{cjkhwxk}{华文行楷}
%\setCJKfamilyfont{cjkfzcs}{方正粗宋简体}
%\newcommand*{\cjkfzcs}{\CJKfamily{cjkfzcs}}
\newcommand*{\cjkhwxk}{\CJKfamily{cjkhwxk}}
%\newfontfamily\wryh{Microsoft YaHei}
%\newfontfamily\hwzs{华文中宋}
%\newfontfamily\hwst{华文宋体}
%\newfontfamily\hwfs{华文仿宋}
%\newfontfamily\jljt{方正静蕾简体}
%\newfontfamily\hwxk{华文行楷}
\newcommand{\verylarge}{\fontsize{60pt}{\baselineskip}\selectfont}  
\newcommand{\chuhao}{\fontsize{44.9pt}{\baselineskip}\selectfont}  
\newcommand{\xiaochu}{\fontsize{38.5pt}{\baselineskip}\selectfont}  
\newcommand{\yihao}{\fontsize{27.8pt}{\baselineskip}\selectfont}  
\newcommand{\xiaoyi}{\fontsize{25.7pt}{\baselineskip}\selectfont}  
\newcommand{\erhao}{\fontsize{23.5pt}{\baselineskip}\selectfont}  
\newcommand{\xiaoerhao}{\fontsize{19.3pt}{\baselineskip}\selectfont} 
\newcommand{\sihao}{\fontsize{14pt}{\baselineskip}\selectfont}      % 字号设置  
\newcommand{\xiaosihao}{\fontsize{12pt}{\baselineskip}\selectfont}  % 字号设置  
\newcommand{\wuhao}{\fontsize{10.5pt}{\baselineskip}\selectfont}    % 字号设置  
\newcommand{\xiaowuhao}{\fontsize{9pt}{\baselineskip}\selectfont}   % 字号设置  
\newcommand{\liuhao}{\fontsize{7.875pt}{\baselineskip}\selectfont}  % 字号设置  
\newcommand{\qihao}{\fontsize{5.25pt}{\baselineskip}\selectfont}    % 字号设置 

\usepackage{diagbox}
\usepackage{multirow}
\boldmath
\XeTeXlinebreaklocale "zh"
\XeTeXlinebreakskip = 0pt plus 1pt minus 0.1pt
\definecolor{cred}{rgb}{0.8,0.8,0.8}
\definecolor{cgreen}{rgb}{0,0.3,0}
\definecolor{cpurple}{rgb}{0.5,0,0.35}
\definecolor{cdocblue}{rgb}{0,0,0.3}
\definecolor{cdark}{rgb}{0.95,1.0,1.0}
\lstset{
	language=python,
	numbers=left,
	numberstyle=\tiny\color{black},
	showspaces=false,
	showstringspaces=false,
	basicstyle=\scriptsize,
	keywordstyle=\color{purple},
	commentstyle=\color{cgreen},
	stringstyle=\color{blue},
	frame=lines,
	% escapeinside=``,
	extendedchars=true, 
	xleftmargin=1em,
	xrightmargin=1em, 
	backgroundcolor=\color{cred},
	aboveskip=1em,
	breaklines=true,
	tabsize=4
} 

%\newfontfamily{\consolas}{Consolas}
%\newfontfamily{\monaco}{Monaco}
%\setmonofont[Mapping={}]{Consolas}	%英文引号之类的正常显示,相当于设置英文字体
%\setsansfont{Consolas} %设置英文字体 Monaco, Consolas,  Fantasque Sans Mono
%\setmainfont{Times New Roman}
%\setCJKmainfont{STZHONGS.TTF}
%\setmonofont{Consolas}
% \newfontfamily{\consolas}{YaHeiConsolas.ttf}
\newfontfamily{\monaco}{MONACO.TTF}
\setCJKmainfont{STZHONGS.TTF}
%\setmainfont{MONACO.TTF}
%\setsansfont{MONACO.TTF}

\newcommand{\fic}[1]{\begin{figure}[H]
		\center
		\includegraphics[width=0.8\textwidth]{#1}
	\end{figure}}
	
\newcommand{\sizedfic}[2]{\begin{figure}[H]
		\center
		\includegraphics[width=#1\textwidth]{#2}
	\end{figure}}

\newcommand{\codefile}[1]{\lstinputlisting{#1}}

\newcommand{\interval}{\vspace{0.5em}}

\newcommand{\tablestart}{
	\interval
	\begin{longtable}{p{2cm}p{10cm}}
	\hline}
\newcommand{\tableend}{
	\hline
	\end{longtable}
	\interval}

% 改变段间隔
\setlength{\parskip}{0.2em}
\linespread{1.1}

\usepackage{lastpage}
\usepackage{fancyhdr}
\pagestyle{fancy}
\lhead{\space \qquad \space}
\chead{snapshot优化方案 \qquad}
\rhead{\qquad\thepage/\pageref{LastPage}}

\begin{document}

\tableofcontents

\clearpage

\section{前言}
    在提出snapshot优化方案之前,我先分析一下snapshot的流程,以便之后优化快照时思路更加清晰。

\section{snapshot的流程}

\subsection{快照入口函数}
    \begin{lstlisting}
    # nova/api/openstack/compute/servers.py Controller._action_create_image()
    def _action_create_image(self, req, id, body):
        # id是instance id
        context = req.environ['nova.context']
        # body == {u'createImage': {u'name': u'snap1', u'metadata': {}}}
        entity = body.get("createImage", {})

        image_name = entity.get("name")

        ...

        props = {}
        metadata = entity.get('metadata', {})
        
        try:
            props.update(metadata)
        ...
        # 根据context和req得到Instance类
        instance = self._get_server(context, req, id)

        bdms = objects.BlockDeviceMappingList.get_by_instance_uuid(
                    context, instance.uuid)

        try:
            # 判断root分区是否是volume
            if self.compute_api.is_volume_backed_instance(context, instance,
                                                          bdms):
                img = instance['image_ref']
                if not img:
                    properties = bdms.root_metadata(
                            context, self.compute_api.image_api,
                            self.compute_api.volume_api)
                    image_meta = {'properties': properties}
                else:
                    image_meta = self.compute_api.image_api.get(context, img)
                
                # Snapshot the given volume-backed instance
                image = self.compute_api.snapshot_volume_backed(
                                                       context,
                                                       instance,
                                                       image_meta,
                                                       image_name,
                                                       extra_properties=props)
            else:
                # 做快照的正常流程
                image = self.compute_api.snapshot(context,
                                                  instance,
                                                  image_name,
                                                  extra_properties=props)
        ...

        # 以下代码的功能:build location of newly-created image entity
        image_id = str(image['id'])
        url_prefix = self._view_builder._update_glance_link_prefix(
                req.application_url)
        image_ref = os.path.join(url_prefix,
                                 context.project_id,
                                 'images',
                                 image_id)

        resp = webob.Response(status_int=202)
        resp.headers['Location'] = image_ref
        return resp
    \end{lstlisting}

\subsection{使用RPC与nova-compute服务通信}

    \begin{lstlisting}
    # nova/compute/api.py API.snapshot()
    def snapshot(self, context, instance, name, extra_properties=None):
        # instance: nova.db.sqlalchemy.models.Instance
        # name: name of the snapshot
        # extra_properties: dict of extra image properties to include
        #                   when creating the image.
        # returns: A dict containing image metadata

        # 函数功能:Create new image entry in the image service.  
        #          This new image will be reserved for the compute 
        #          manager to upload a snapshot or backup.
        image_meta = self._create_image(context, instance, name,
                                        'snapshot',
                                        extra_properties=extra_properties)

        # 更改instance的task_state
        instance.task_state = task_states.IMAGE_SNAPSHOT_PENDING
        # 调用Instance.save()方法更改数据库
        instance.save(expected_task_state=[None])

        self.compute_rpcapi.snapshot_instance(context, instance,
                                              image_meta['id'])

        return image_meta
    \end{lstlisting}

    \begin{lstlisting}
    # nova/compute/rpcapi.py ComputeAPI.snapshot_instance()
    def snapshot_instance(self, ctxt, instance, image_id):
        # server: the destination host for a message.
        # server == instance['host']
        cctxt = self.client.prepare(server=_compute_host(None, instance),
                version=version)
        cctxt.cast(ctxt, 'snapshot_instance',
                   instance=instance,
                   image_id=image_id)
    \end{lstlisting}

\subsection{nova-compute中的快照动作}
    \begin{lstlisting}
    # nova/compute/manager.py ComputeManager.snapshot_instance()
    def snapshot_instance(self, context, image_id, instance):
        # context: security context
        # instance: a nova.objects.instance.Instance object
        # image_id: glance.db.sqlalchemy.models.Image.Id
        try:
            # 修改instance的task_state
            instance.task_state = task_states.IMAGE_SNAPSHOT
            # 调用Instance.save()方法更改数据库
            instance.save(
                        expected_task_state=task_states.IMAGE_SNAPSHOT_PENDING)
        ...

        self._snapshot_instance(context, image_id, instance,
                                task_states.IMAGE_SNAPSHOT)
    \end{lstlisting}

    \_snapshot\_instance()函数如下:

    \begin{lstlisting}
    # nova/compute/manager.py ComputeManager._snapshot_instance()
    def _snapshot_instance(self, context, image_id, instance,
                           expected_task_state):
        # self.driver.get_info(instance)["state"]
        current_power_state = self._get_power_state(context, instance)
        try:
            # 修改instance的power_state
            instance.power_state = current_power_state
            instance.save()
            if instance.power_state != power_state.RUNNING:
                state = instance.power_state
                running = power_state.RUNNING
            def update_task_state(task_state,
                                  expected_state=expected_task_state):
                instance.task_state = task_state
                instance.save(expected_task_state=expected_state)
            # 调用LibvirtDriver.snapshot()函数实现快照功能
            self.driver.snapshot(context, instance, image_id,
                                 update_task_state)
            # 记录instance状态
            instance.task_state = None
            instance.save(expected_task_state=task_states.IMAGE_UPLOADING)
        ...
    \end{lstlisting}

\clearpage

\subsection{实现快照功能的核心函数}

    \begin{lstlisting}
    def snapshot(self, context, instance, image_id, update_task_state):
        try:
            # 调用virConnect.lookupByName()返回virDomain对象
            virt_dom = self._lookup_by_name(instance['name'])
        ...

        base_image_ref = instance['image_ref']

        # 得到instance的image的相关数据
        base = compute_utils.get_image_metadata(
            context, self._image_api, base_image_ref, instance)
        
        # Retrieves the information record for a single disk image by image_id
        snapshot = self._image_api.get(context, image_id)

        # 通过virtDomain.XMLDesc(0)得到xml配置文件,从而得到instance的磁盘路径
        disk_path = libvirt_utils.find_disk(virt_dom)
        # 使用“qemu-img info disk_path”获得磁盘信息
        source_format = libvirt_utils.get_disk_type(disk_path)

        image_format = CONF.libvirt.snapshot_image_format or source_format

        # NOTE(bfilippov): save lvm and rbd as raw
        if image_format == 'lvm' or image_format == 'rbd':
            image_format = 'raw'
        
        # metadata = {'is_public': False,
        #             'status': 'active',
        #             'name': snp_name,
        #             'properties': {
        #                            'kernel_id': instance['kernel_id'],
        #                            'image_location': 'snapshot',
        #                            'image_state': 'available',
        #                            'owner_id': instance['project_id'],
        #                            'ramdisk_id': instance['ramdisk_id'],
        #                            }
        #             }
        metadata = self._create_snapshot_metadata(base,
                                                  instance,
                                                  image_format,
                                                  snapshot['name'])
        
        # 获得快照的名称
        snapshot_name = uuid.uuid4().hex

        # LIBVIRT_POWER_STATE = {
        #     VIR_DOMAIN_NOSTATE: power_state.NOSTATE,
        #     VIR_DOMAIN_RUNNING: power_state.RUNNING,
        #     VIR_DOMAIN_BLOCKED: power_state.RUNNING,
        #     VIR_DOMAIN_PAUSED: power_state.PAUSED,
        #     VIR_DOMAIN_SHUTDOWN: power_state.SHUTDOWN,
        #     VIR_DOMAIN_SHUTOFF: power_state.SHUTDOWN,
        #     VIR_DOMAIN_CRASHED: power_state.CRASHED,
        #     VIR_DOMAIN_PMSUSPENDED: power_state.SUSPENDED,
        # }
        # 调用virDomain.info()返回[state, maxMemory, memory, nbVirtCPU, cpuTime]
        state = LIBVIRT_POWER_STATE[virt_dom.info()[0]]

        # 动态快照要求QEMU 1.3 and Libvirt 1.0.0
        # Instances with LVM encrypted ephemeral storage只支持静态快照
        if (self._has_min_version(MIN_LIBVIRT_LIVESNAPSHOT_VERSION,
                                  MIN_QEMU_LIVESNAPSHOT_VERSION,
                                  REQ_HYPERVISOR_LIVESNAPSHOT)
             and source_format not in ('lvm', 'rbd')
             and not CONF.ephemeral_storage_encryption.enabled):
            live_snapshot = True
            try:
                # 终止虚拟机磁盘上的active block job
                virt_dom.blockJobAbort(disk_path, 0)
            ...
        else:
            live_snapshot = False

        if state == power_state.SHUTDOWN:
            live_snapshot = False

        # virDomain.managedSave() does not work for LXC
        # 如果是静态快照,需要执行virDomain.managedSave()函数
        if CONF.libvirt.virt_type != 'lxc' and not live_snapshot:
            if state == power_state.RUNNING or state == power_state.PAUSED:
                # 卸载虚拟机的pci设备
                self._detach_pci_devices(virt_dom,
                    pci_manager.get_instance_pci_devs(instance))
                # 关闭SR-IOV端口
                self._detach_sriov_ports(context, instance, virt_dom)
                # This method will suspend a domain and save its memory contents to a file on disk.
                virt_dom.managedSave(0)
            
        # 返回Qcow2类
        # snapshot_backend是nova.virt.libvirt.imagebackend.Qcow2 object
        snapshot_backend = self.image_backend.snapshot(instance,
                disk_path,
                image_type=source_format)

        ...

        update_task_state(task_state=task_states.IMAGE_PENDING_UPLOAD)
        # snapshot_directory == /var/lib/nova/instances/snapshots
        snapshot_directory = CONF.libvirt.snapshots_directory
        # 确保有这个目录存在
        fileutils.ensure_tree(snapshot_directory)
        with utils.tempdir(dir=snapshot_directory) as tmpdir:
            try:
                # 得到快照路径
                out_path = os.path.join(tmpdir, snapshot_name)
                if live_snapshot:
                    os.chmod(tmpdir, 0o701)
                    # 动态快照的步骤如下:
                    # disk_path为/var/lib/nova/instances/vm-uuid/disk
                    # out_path为/var/lib/nova/instances/snapshots/snapshot_name
                    # 首先调用“qemu-img create -f qcow2 -o backing_file=disk_path的backing_file,size=disk_path的virtual_size out_path.delta”创建镜像
                    # 然后调用domain.blockRebase(disk_path, disk_delta, 0,
                    #          libvirt.VIR_DOMAIN_BLOCK_REBASE_COPY |
                    #          libvirt.VIR_DOMAIN_BLOCK_REBASE_REUSE_EXT |
                    #          libvirt.VIR_DOMAIN_BLOCK_REBASE_SHALLOW)函数将disk_path的内容拷贝给out_path.delta
                    # 然后使用“qemu-img convert”命令将out_path.delta拷贝到out_path
                    # 这条命令在拷贝过程中,会先把后端镜像和增量镜像合并,然后在拷贝到另外一个镜像文件中,所以会比较久
                    self._live_snapshot(virt_dom, disk_path, out_path,
                                        image_format)
                else:
                    # 使用“qemu-img convert”命令将虚拟机磁盘拷贝到out_path
                    # 这条命令在拷贝过程中,会先把后端镜像和增量镜像合并,然后在拷贝到另外一个镜像文件中,所以会比较久
                    # 如果不想新建立一个镜像,只是想做个快照,那么这是可以改进的一点
                    snapshot_backend.snapshot_extract(out_path, image_format)
            finally:
                new_dom = None
                # NOTE(dkang): because previous managedSave is not called
                #              for LXC, _create_domain must not be called.
                if CONF.libvirt.virt_type != 'lxc' and not live_snapshot:
                    if state == power_state.RUNNING:
                        # 这种情况new_dom == virt_dom
                        new_dom = self._create_domain(domain=virt_dom)
                    elif state == power_state.PAUSED:
                        # 这种情况new_dom == virt_dom.createWithFlags(libvirt.VIR_DOMAIN_START_PAUSED)
                        new_dom = self._create_domain(domain=virt_dom,
                                launch_flags=libvirt.VIR_DOMAIN_START_PAUSED)
                    if new_dom is not None:
                        # 安装原有的pci设备
                        self._attach_pci_devices(new_dom,
                            pci_manager.get_instance_pci_devs(instance))
                        # 开启SR-IOV端口
                        self._attach_sriov_ports(context, instance, new_dom)

            # Upload that image to the image service
            with libvirt_utils.file_open(out_path) as image_file:
                self._image_api.update(context,
                                       image_id,
                                       metadata,
                                       image_file)
    \end{lstlisting}

\clearpage

\section{snapshot优化方案}
\subsection{最基本的情况}
\subsubsection{最基本的情况的描述}
    假设现在我们想迅速地备份一个虚拟机,而且现在这个虚拟机的镜像链只有两个,也就是只有一个当初创建虚拟机用的后端镜像和一个增量文件。\par

    在这里我说一下我们想实现哪些功能:
    \begin{itemize}
        \item[1.] 虚拟机原先有一个磁盘镜像,它的路径是disk\_path,现在我们要把它作为后端镜像,并且在它的基础上创建一块增量镜像。
        \item[2.] 将后端镜像上传到镜像服务器上,以备之后根据这块镜像创建虚拟机。
    \end{itemize}

    原先我们horizon上已经有一个创建快照的按键,现在我希望把这个按键名称改为“创建完整的虚拟机镜像”。
    然后在添加一个按键,名称为“迅速备份虚拟机”。\par

    因为从网页命令到底层函数的消息的传递我不了解,这里就不提,而是把自己在底层函数的开发方案的实现方法提出来。
    这里的修改方法是根据原先的snapshot流程更改的,如果不熟悉snapshot流程,可以参看上一节。\par

    当然不局限于我这种实现方法,如果有更好的实现方法,当然可以把我这个方案完全推翻。

\subsubsection{入口函数的改动}
    我的想法是新的快照动作与原先的快照动作共用一个入口函数,只是传入的body参数改一下。\par

    本来传入的body参数是:
    \begin{lstlisting}
    body == {u'createImage': {u'name': u'snap1', u'metadata': {}}}
    \end{lstlisting}

    现在新的快照动作对应的body参数可以是:
    \begin{lstlisting}
    body == {u'snapshot': {u'name': u'snap1', u'metadata': {}}}
    \end{lstlisting}

    \begin{lstlisting}
    # nova/api/openstack/compute/servers.py Controller._action_create_image()
    def _action_create_image(self, req, id, body):
        # id是instance id
        context = req.environ['nova.context']
        # body == {u'createImage': {u'name': u'snap1', u'metadata': {}}}
        entity = body.get("createImage", {})
        
        # 新增加的代码
        action = "createImage"
        if not entity:
            entity = body.get("snapshot", {})
            action = "snapshot"

        image_name = entity.get("name")

        # 更改的代码
        if not image_name:
            msg = _("%(action) entity requires name attribute") % {"action":action}
            raise exc.HTTPBadRequest(explanation=msg)

        props = {}
        metadata = entity.get('metadata', {})
        common.check_img_metadata_properties_quota(context, metadata)
        try:
            props.update(metadata)
        except ValueError:
            msg = _("Invalid metadata")
            raise exc.HTTPBadRequest(explanation=msg)
        
        instance = self._get_server(context, req, id)

        bdms = objects.BlockDeviceMappingList.get_by_instance_uuid(
                    context, instance.uuid)

        try:
            # 因为对linux的device mapping机制不了解,这里先不动
            if self.compute_api.is_volume_backed_instance(context, instance,
                                                          bdms):
                img = instance['image_ref']
                if not img:
                    properties = bdms.root_metadata(
                            context, self.compute_api.image_api,
                            self.compute_api.volume_api)
                    image_meta = {'properties': properties}
                else:
                    image_meta = self.compute_api.image_api.get(context, img)
                
                # Snapshot the given volume-backed instance
                image = self.compute_api.snapshot_volume_backed(
                                                       context,
                                                       instance,
                                                       image_meta,
                                                       image_name,
                                                       extra_properties=props)
            else:
                # 做快照的正常流程
                # 为这个函数添加一个action的参数
                image = self.compute_api.snapshot(context,
                                                  instance,
                                                  image_name,
                                                  extra_properties=props,
                                                  action=action)
        except exception.InstanceInvalidState as state_error:
            # 修改了这里的代码
            common.raise_http_conflict_for_instance_invalid_state(state_error, action)
        except exception.Invalid as err:
            raise exc.HTTPBadRequest(explanation=err.format_message())

        image_id = str(image['id'])
        url_prefix = self._view_builder._update_glance_link_prefix(
                req.application_url)
        image_ref = os.path.join(url_prefix,
                                 context.project_id,
                                 'images',
                                 image_id)

        resp = webob.Response(status_int=202)
        resp.headers['Location'] = image_ref
        return resp
    \end{lstlisting}

\subsubsection{修改RPC与nova-compute服务通信部分的代码}
    在上面的\_action\_create\_image()函数中调用compute\_api.snapshot()函数时,我们增加了一个action参数,对它的修改如下:
    \begin{lstlisting}
    # 之所以将action参数设置为关键字参数,是怕有其他地方调用snapshot,关键字参数能把影响降到最小
    def snapshot(self, context, instance, name, extra_properties=None, action="createImage"):
        image_meta = self._create_image(context, instance, name,
                                        'snapshot',
                                        extra_properties=extra_properties)
        instance.task_state = task_states.IMAGE_SNAPSHOT_PENDING
        instance.sace(excepted_task_state=[None])

        # 仍然是多传递一个参数action
        self.compute_rpcapi.snapshot_instance(context, instance, image_meta['id'], action=action)

        return image_meta
    \end{lstlisting}

    相应地修改compute.rpcapi.py文件中的snapshot\_instance()函数:
    \begin{lstlisting}
    # 增加一个关键字参数action
    def snapshot_instance(self, ctxt, instance, image_id, action="createImage"):
        version = '3.0'
        cctxt = self.client.prepare(server=_compute_host(None, instance),
                version=version)
        cctxt.cast(ctxt, 'snapshot_instance', instance=instance, image_id=image_id, action=action)
    \end{lstlisting}

    随后在nova-compute服务中修改相应的函数。

\subsubsection{在nova-compute中修改相应的快照动作}
    在compute.manager.py文件中添加snapshot\_instance函数,只是增加一个关键字参数:
    \begin{lstlisting}
    def snapshot_instance(self, context, image_id, instance, action="createImage"):
        try:
            instance.task_state = task_states.IMAGE_SNAPSHOT
            instance.save(exceptd_task_state=task_states.IMAGE_SNAPSHOT_PENDING)
        except exception.InstanceNotFound:
            LOG.debug("Instance not found, could not set state %s for instance", task_states.IMAGE_SNAPSHOT, instance=instance)
            return
        except exception.UnexpectedDeletingTaskStateError:
            LOG.debug("Instance being deleted, snapshot cannot continue", instance=instance)
            return
        
        # 仍然是多传递一个参数action
        self._snapshot_instance(context, image_id, instance, task_states.IMAGE_SNAPSHOT, action=action)
    \end{lstlisting}

    仍然是将\_snapshot\_instance()函数多增加一个关键字参数action,
    同时因为这个函数已经开始调用LibvirtDriver类中的函数,
    在这里开始我觉得就可以和原先的快照流程分开了。\par

    也就是在LibvirtDriver类中新增加一个snapshot\_overlay函数,
    然后在\_snapshot\_instance函数根据action参数选择跳转到snapshot()函数还是跳转到snapshot\_overlay函数。

    \begin{lstlisting}
    def _snapshot_instance(self, context, image_id, instance, exceptd_task_state, action="createImage"):
        context = context.elevated()

        current_power_state = self._get_power_state(context, instance)
        try:
            instance.power_state = current_power_state
            instance.save()

            LOG.audit(_('instance snapshotting'), context=context, instance=instance)

            if instance.power_state != power_state.RUNNING:
                state = instance.power_state
                running = power_state.RUNNING
                LOG.warn(_('trying to snapshot a non-running instance:: '
                        '(state: %(state)s expected: %(running)s)'),
                        {'state': state, 'running': running},
                        instance=instance)

            self._notify_about_instance_usage(context, instance, "snapshot.start")
        
            def update_task_state(task_state, excepted_state=expected_task_state):
                instance.task_state = task_state
                instance.save(expected_task_state=expected_state)
            
            # 修改的代码:根据action选择LibvirtDriver中的参数
            if action == "createImage":
                self.driver.snapshot(context, instance, image_id, update_task_state)
            else:
                self.driver.snapshot_overlay(context, instance, image_id, update_task_state)
            
            instance.task_state = None
            instance.save(expected_task_state=task_states.IMAGE_UPLOADING)

            self._notify_about_instance_usage(context, instance, "snapshot.end")
        except (exception.InstanceNotFound,
                exception.UnexpectedDeletingTaskStateError):
            msg = 'Instance disappeared during snapshot'
            LOG.debug(msg, instance=instance)
            try:
                iamge_service = glance.get_default_image_service()
                image = image_service.show(context, image_id)
                if image['status'] != 'active':
                    image_service.delete(context, image_id)
            except Exception:
                LOG.warning(_("Error while trying to clean up image %s"),
                            image_id, instance=instance)
        except exception.ImageNotFound:
            instance.task_state = None
            instance.save()
            msg = _("Image not found during snapshot")
            LOG.warn(msg, instance=instance)
    \end{lstlisting}

\subsubsection{新增加一个核心函数snapshot\_overlay}
    根据上一节的描述我们知道,我们需要在LibvirtDriver类中增加一个snapshot\_overlay函数,用于实现我们想实现的功能。\par

    在这里我再说一下我们想实现哪些功能:
    \begin{itemize}
        \item[1.] 虚拟机原先有一个磁盘镜像,它的路径是disk\_path,现在我们要把它作为后端镜像,并且在它的基础上创建一块增量镜像。
        \item[2.] 将后端镜像上传到镜像服务器上,以备之后根据这块镜像创建虚拟机。
    \end{itemize}
    
    针对功能一的实现细节,我有以下的想法:
    \begin{itemize}
        \item[1.] 首先将虚拟机磁盘镜像改名,加个后缀”.base“,它的路径也就变成了disk\_path.base。
        \item[2.] 然后根据这个镜像文件创建增量文件,这个增量文件的名字和原先磁盘镜像的名字相同,所以它的路径也就是disk\_path。
        \item[3.] 重启虚拟机,虚拟机就会开始使用这个增量文件。
    \end{itemize}

    针对可能的疑问,我这里作出回答:
    \begin{lstlisting}
    问:虚拟机运行时,将磁盘镜像更名是否会影响它的运行?
    答:我在kvm中进行过操作,发现是不会影响虚拟机的运行的。

    问:虚拟机磁盘镜像更名后,虚拟机的数据改动是否会保存?
    答:会保存,而且是保存到原先的镜像文件中。

    问:根据原先的虚拟机镜像创建增量镜像后,虚拟机的数据改动是否会保存到增量镜像中?
    答:数据的改动会保存到当前的后端镜像中,也保存到增量镜像中。
        不过需要知道的是,当虚拟机重启后,虚拟机开始使用这个增量镜像,所以虚拟机的数据改动会保存到增量镜像中,不会再保存到原先的镜像文件中。
    \end{lstlisting}

    针对功能二我有想补充的话:
    \begin{lstlisting}
    当后端镜像传到镜像服务器以后,我们应该知道一个事实,这个后端镜像也是有自己的backing file的。如果想要根据这个后端镜像创建原先的虚拟机,必须保证它的backing file是存在的。
    \end{lstlisting}

    新增加snapshot\_overlay函数如下:
    \begin{lstlisting}
    def snapshot_overlay(self, context, instance, image_id, update_task_state):
        try:
            virt_dom = self._lookup_by_name(instance['name'])
        except exception.InstanceNotFound:
            raise exception.InstanceNotRunning(instance_id=instance['uuid'])

        base_image_ref = instance['image_ref']

        base = compute_utils.get_image_metadata(context, self._image_api, base_iamge_ref, instance)
        
        snapshot = self._image_api.get(context, image_id)

        disk_path = libvirt_utils.find_disk(virt_dom)
        source_format = libvirt_utils.get_disk_type(disk_path)

        image_format = CONF.libvirt.snapshot_image_format or source_format

        if image_format == 'lvm' or image_format == 'rbd':
            image_format = 'raw'

        metadata = self._create_snapshot_metadata(base,
                                                  instance,
                                                  image_format,
                                                  snapshot['name'])

        snapshot_name = uuid.uuid4().hex

        state = LIBVIRT_POWER_STATE[virt_dom.info()[0]]

        if(self._has_min_version(MIN_LIBVIRT_LIVESNAPSHOT_VERSION,
                                 MIN_QEMU_LIVESNAPSHOT_VERSION,
                                 REQ_HYPERVISOR_LIVESNAPSHOT)
            and source_format not in ('lvm', 'rbd')
            and not CONF.ephemeral_storage_encryption.enabled):
            live_snapshot = True

            try:
                virt_dom.blockJobAbort(disk_path, 0)
            except libvirt.libvirtError as ex:
                error_code = ex.get_error_code()
                if error_code == libvirt.VIR_ERR_CONFIG_UNSUPPORTED:
                    live_snapshot = False
                else:
                    pass
        else:
            live_snapshot = False

        if state == power_state.SHUTDOWN:
            live_snapshot = False

        if CONF.libvirt.virt_type != 'lxc' and not live_snapshot:
            if state == power_state.RUNNING or state == power_state.PAUSED:
                self._detach_pci_devices(virt_dom, pci_manager.get_instance_pci_devs(instance))
                self._detach_sriov_ports(context, instance, virt_dom)
                virt_dom.managedSave(0)

        snapshot_backend = self.image_backend.snapshot(instance,
                                disk_path,
                                image_type=source_format)

        if live_snapshot:
            LOG.info(_LI("Beginning live snapshot process"),
                     instance=instance)
        else:
            LOG.info(_LI("Beginning cold snapshot process"),
                     instance=instance)

        update_task_state(task_state=task_states.IMAGE_PENDING_UPLOAD)
        snapshot_directory = CONF.libvirt.snapshots_directory
        fileutils.ensure_tree(snapshot_directory)
        with utils.tempdir(dir=snapshot_directory) as tmpdir:
            try:
                out_path = os.path.join(tmpdir, snapshot_name)
                if live_snapshot:
                    # openstack貌似不支持动态创建快照,所以不修改这里的代码
                    os.chmod(tmpdir, 0o701)
                    self._live_snapshot(virt_dom, disk_path, out_path, image_format)
                else:
                    # 修改的代码
                    # 首先将虚拟机磁盘镜像改名,加个后缀“.base”
                    disk_path_base = disk_path + '.base'
                    out_path = disk_path
                    utils.execute('mv', disk_path, disk_path_base)
                    disk_path = disk_path_base
                    # 然后在原磁盘文件上创建增量文件
                    libvirt_utils.create_overlay(disk_path, out_path, image_format)
                    # 强制重启虚拟机
                    virt_dom.reset()
            finally:
                new_dom = None
                if CONF.libvirt.type != 'lxc' and not live_snapshot:
                    if state == power_state.RUNNING:
                        new_dom = self._create_domain(domain=virt_dom)
                    elif state == power_state.PAUSED:
                        new_dom = self._create_domain(domain=virt_dom,
                                launch_flags=libvirt.VIR_DOMAIN_START_PAUSED)
                    if new_dom is not None:
                        self._attach_pci_devices(new_dom, pci_manager.get_instance_pci_devs(instance))
                        self._attach_sriov_ports(context, instance, new_dom)
                LOG.info(_LI("Snapshot extracted, beginning image upload"), instance=instance)
            
            update_task_state(task_state=task_states.IMAGE_UPLOADING, 
                              expected_state=task_states.IMAGE_PENDING_UPLOAD)
            
            # 将原先的磁盘文件上传到镜像服务器
            with libvirt_utils.file_open(disk_path) as image_file:
                self._image_api.update(context,
                                       image_id,
                                       metadata,
                                       image_file)
                LOG.info(_LI("Snapshot image upload complete"),
                         instance=instance) 
    \end{lstlisting}

\subsubsection{在virt/libvirt/utils.py创建新函数}
    \begin{lstlisting}
    def create_overlay(disk_path, out_path, dest_format):
        if dest_fmt != 'qcow2':
            raise

        qemu_img_cmd = ('qemu-img', 'create', '-b', disk_path, '-f', dest_fmt, out_path)
        execute(*qemu_img_cmd)
    \end{lstlisting}

\clearpage

\subsection{镜像链为3的情况}
\subsubsection{镜像链为3的情况的描述}
    原先最基本的情况是:虚拟机的镜像链只有两个,也就是只有一个当初创建虚拟机用的后端镜像和一个增量文件。\par

    现在的情况是:我们已经为备份过一次虚拟机,也就是现在虚拟机的镜像链为3,有最初创建虚拟机用的后端镜像,一个备份用的镜像和一个最新的增量文件,
    也就是base <-- overlay1 <-- overlay2。\par

    如果再备份一次虚拟机的话,虚拟机的镜像链就会变成4,为了保持虚拟机的性能,我们决定在备份虚拟机时,能够将overlay1和overlay2合并,保持镜像链为3。

\subsubsection{核心函数的修改}
    这里的修改方法参考了《快照相关原理》的第六节“删除快照”。\par

    当镜像链为4时,镜像链为:base <- overlay1 <- overlay2 <- overlay3。
    为了保持镜像链为3,我们现在需要先把overlay2的内容融入overlay1中,然后把overlay3的后端镜像变为overlay1。\par

    修改后的核心函数如下:

    \begin{lstlisting}
    def snapshot_overlay(self, context, instance, image_id, update_task_state):
        try:
            virt_dom = self._lookup_by_name(instance['name'])
        except exception.InstanceNotFound:
            raise exception.InstanceNotRunning(instance_id=instance['uuid'])

        base_image_ref = instance['image_ref']
        base = compute_utils.get_image_metadata(context, self._image_api, base_iamge_ref, instance)
        snapshot = self._image_api.get(context, image_id)
        disk_path = libvirt_utils.find_disk(virt_dom)
        source_format = libvirt_utils.get_disk_type(disk_path)

        image_format = CONF.libvirt.snapshot_image_format or source_format

        if image_format == 'lvm' or image_format == 'rbd':
            image_format = 'raw'

        metadata = self._create_snapshot_metadata(base,
                                                  instance,
                                                  image_format,
                                                  snapshot['name'])

        snapshot_name = uuid.uuid4().hex

        state = LIBVIRT_POWER_STATE[virt_dom.info()[0]]

        if(self._has_min_version(MIN_LIBVIRT_LIVESNAPSHOT_VERSION,
                                 MIN_QEMU_LIVESNAPSHOT_VERSION,
                                 REQ_HYPERVISOR_LIVESNAPSHOT)
            and source_format not in ('lvm', 'rbd')
            and not CONF.ephemeral_storage_encryption.enabled):
            live_snapshot = True

            try:
                virt_dom.blockJobAbort(disk_path, 0)
            except libvirt.libvirtError as ex:
                error_code = ex.get_error_code()
                if error_code == libvirt.VIR_ERR_CONFIG_UNSUPPORTED:
                    live_snapshot = False
                else:
                    pass
        else:
            live_snapshot = False

        if state == power_state.SHUTDOWN:
            live_snapshot = False

        if CONF.libvirt.virt_type != 'lxc' and not live_snapshot:
            if state == power_state.RUNNING or state == power_state.PAUSED:
                self._detach_pci_devices(virt_dom, pci_manager.get_instance_pci_devs(instance))
                self._detach_sriov_ports(context, instance, virt_dom)
                virt_dom.managedSave(0)

        snapshot_backend = self.image_backend.snapshot(instance,
                                disk_path,
                                image_type=source_format)

        if live_snapshot:
            LOG.info(_LI("Beginning live snapshot process"),
                     instance=instance)
        else:
            LOG.info(_LI("Beginning cold snapshot process"),
                     instance=instance)

        update_task_state(task_state=task_states.IMAGE_PENDING_UPLOAD)
        snapshot_directory = CONF.libvirt.snapshots_directory
        fileutils.ensure_tree(snapshot_directory)
        with utils.tempdir(dir=snapshot_directory) as tmpdir:
            try:
                out_path = os.path.join(tmpdir, snapshot_name)
                if live_snapshot:
                    # openstack貌似不支持动态创建快照,所以不修改这里的代码
                    os.chmod(tmpdir, 0o701)
                    self._live_snapshot(virt_dom, disk_path, out_path, image_format)
                else:
                    # 修改的代码
                    # 获得虚拟机镜像的后端镜像的文件名
                    backing_file = libvirt_utils.get_disk_backing_file(disk_path, basename=False)
                    # 获得后端镜像的后缀
                    suffix = backing_file.split(".")[-1]
                    # 我们本身备份虚拟机时,会把原先虚拟机磁盘加一个“.base”后缀
                    # 现在如果suffix == “.base”,那么说明这个虚拟机已经备份过一次,那么之后我们需要合并overlay1和overlay2
                    # 如果虚拟机备份过一次,我们还需要将overlay1再加个后缀“.base”
                    if suffix == ".base":
                        utils.execute('mv', backing_file, backing_file + '.base')    
                        backing_file = backing_file + '.base'
                    out_path = disk_path
                    utils.execute('mv', disk_path, disk_path + '.base')
                    disk_path = disk_path + '.base'
                    # 然后在原磁盘文件上创建增量文件
                    libvirt_utils.create_overlay(disk_path, out_path, image_format)
                    # 强制重启虚拟机
                    virt_dom.reset()
                    # 判断虚拟机是否备份过一次
                    # 如果备份过一次,现在我们的快照链如下:
                    # base <- overlay1 <- overlay2 <- overlay3
                    if suffix == ".base":
                        # 因为overlay1改名了,所以先要更改overlay2的后端镜像名
                        utils.execute('qemu-img', 'rebase', '-b', backing_file, disk_path)
                        # 然后将overlay2融入overlay1中
                        utils.execute('qemu-img', 'commit', disk_path)
                        # 然后将overlay1作为overlay3的后端镜像
                        utils.execute('qemu-img', 'rebase', '-u', '-b', backing_file, out_path)
                        # 最后删除overlay2这个镜像
                        utils.execute('rm', disk_path)
                        # 这是为了之后上传后端镜像时,上传的是backing_file
                        disk_path = backing_file
            finally:
                new_dom = None
                if CONF.libvirt.type != 'lxc' and not live_snapshot:
                    if state == power_state.RUNNING:
                        new_dom = self._create_domain(domain=virt_dom)
                    elif state == power_state.PAUSED:
                        new_dom = self._create_domain(domain=virt_dom,
                                launch_flags=libvirt.VIR_DOMAIN_START_PAUSED)
                    if new_dom is not None:
                        self._attach_pci_devices(new_dom, pci_manager.get_instance_pci_devs(instance))
                        self._attach_sriov_ports(context, instance, new_dom)
                LOG.info(_LI("Snapshot extracted, beginning image upload"), instance=instance)
            
            update_task_state(task_state=task_states.IMAGE_UPLOADING, 
                              expected_state=task_states.IMAGE_PENDING_UPLOAD)
            
            # 将原先的磁盘文件上传到镜像服务器
            with libvirt_utils.file_open(disk_path) as image_file:
                self._image_api.update(context,
                                       image_id,
                                       metadata,
                                       image_file)
                LOG.info(_LI("Snapshot image upload complete"),
                         instance=instance) 
    \end{lstlisting}

\end{document}