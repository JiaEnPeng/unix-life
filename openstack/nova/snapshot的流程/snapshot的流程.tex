% !TeX spellcheck = en_US
%% 字体:方正静蕾简体
%%		 方正粗宋
\documentclass[a4paper,left=1.5cm,right=1.5cm,11pt]{article}

\usepackage[utf8]{inputenc}
\usepackage{fontspec}
\usepackage{cite}
\usepackage{xeCJK}
\usepackage{indentfirst}
\usepackage{titlesec}
\usepackage{etoolbox}%
\makeatletter
\patchcmd{\ttlh@hang}{\parindent\z@}{\parindent\z@\leavevmode}{}{}%
\patchcmd{\ttlh@hang}{\noindent}{}{}{}%
\makeatother

\usepackage{hyperref}
\usepackage{longtable}
\usepackage{empheq}
\usepackage{graphicx}
\usepackage{float}
\usepackage{rotating}
\usepackage{subfigure}
\usepackage{tabu}
\usepackage{amsmath}
\usepackage{setspace}
\usepackage{amsfonts}
\usepackage{appendix}
\usepackage{listings}
\usepackage{xcolor}
\usepackage{geometry}
\setcounter{secnumdepth}{4}
%\titleformat*{\section}{\LARGE}
%\renewcommand\refname{参考文献}
%\titleformat{\chapter}{\centering\bfseries\huge}{}{0.7em}{}{}
\titleformat{\section}{\LARGE\bf}{\thesection}{1em}{}{}
\titleformat{\subsection}{\Large\bfseries}{\thesubsection}{1em}{}{}
\titleformat{\subsubsection}{\large\bfseries}{\thesubsubsection}{1em}{}{}
\renewcommand{\contentsname}{{ \centerline{目{  } 录}}}
\setCJKfamilyfont{cjkhwxk}{STXINGKA.TTF}
%\setCJKfamilyfont{cjkhwxk}{华文行楷}
%\setCJKfamilyfont{cjkfzcs}{方正粗宋简体}
%\newcommand*{\cjkfzcs}{\CJKfamily{cjkfzcs}}
\newcommand*{\cjkhwxk}{\CJKfamily{cjkhwxk}}
%\newfontfamily\wryh{Microsoft YaHei}
%\newfontfamily\hwzs{华文中宋}
%\newfontfamily\hwst{华文宋体}
%\newfontfamily\hwfs{华文仿宋}
%\newfontfamily\jljt{方正静蕾简体}
%\newfontfamily\hwxk{华文行楷}
\newcommand{\verylarge}{\fontsize{60pt}{\baselineskip}\selectfont}  
\newcommand{\chuhao}{\fontsize{44.9pt}{\baselineskip}\selectfont}  
\newcommand{\xiaochu}{\fontsize{38.5pt}{\baselineskip}\selectfont}  
\newcommand{\yihao}{\fontsize{27.8pt}{\baselineskip}\selectfont}  
\newcommand{\xiaoyi}{\fontsize{25.7pt}{\baselineskip}\selectfont}  
\newcommand{\erhao}{\fontsize{23.5pt}{\baselineskip}\selectfont}  
\newcommand{\xiaoerhao}{\fontsize{19.3pt}{\baselineskip}\selectfont} 
\newcommand{\sihao}{\fontsize{14pt}{\baselineskip}\selectfont}      % 字号设置  
\newcommand{\xiaosihao}{\fontsize{12pt}{\baselineskip}\selectfont}  % 字号设置  
\newcommand{\wuhao}{\fontsize{10.5pt}{\baselineskip}\selectfont}    % 字号设置  
\newcommand{\xiaowuhao}{\fontsize{9pt}{\baselineskip}\selectfont}   % 字号设置  
\newcommand{\liuhao}{\fontsize{7.875pt}{\baselineskip}\selectfont}  % 字号设置  
\newcommand{\qihao}{\fontsize{5.25pt}{\baselineskip}\selectfont}    % 字号设置 

\usepackage{diagbox}
\usepackage{multirow}
\boldmath
\XeTeXlinebreaklocale "zh"
\XeTeXlinebreakskip = 0pt plus 1pt minus 0.1pt
\definecolor{cred}{rgb}{0.8,0.8,0.8}
\definecolor{cgreen}{rgb}{0,0.3,0}
\definecolor{cpurple}{rgb}{0.5,0,0.35}
\definecolor{cdocblue}{rgb}{0,0,0.3}
\definecolor{cdark}{rgb}{0.95,1.0,1.0}
\lstset{
	language=bash,
	numbers=left,
	numberstyle=\tiny\color{black},
	showspaces=false,
	showstringspaces=false,
	basicstyle=\scriptsize,
	keywordstyle=\color{purple},
	commentstyle=\itshape\color{cgreen},
	stringstyle=\color{blue},
	frame=lines,
	% escapeinside=``,
	extendedchars=true, 
	xleftmargin=1em,
	xrightmargin=1em, 
	backgroundcolor=\color{cred},
	aboveskip=1em,
	breaklines=true,
	tabsize=4
} 

%\newfontfamily{\consolas}{Consolas}
%\newfontfamily{\monaco}{Monaco}
%\setmonofont[Mapping={}]{Consolas}	%英文引号之类的正常显示,相当于设置英文字体
%\setsansfont{Consolas} %设置英文字体 Monaco, Consolas,  Fantasque Sans Mono
%\setmainfont{Times New Roman}
%\setCJKmainfont{STZHONGS.TTF}
%\setmonofont{Consolas}
% \newfontfamily{\consolas}{YaHeiConsolas.ttf}
\newfontfamily{\monaco}{MONACO.TTF}
\setCJKmainfont{STZHONGS.TTF}
%\setmainfont{MONACO.TTF}
%\setsansfont{MONACO.TTF}

\newcommand{\fic}[1]{\begin{figure}[H]
		\center
		\includegraphics[width=0.8\textwidth]{#1}
	\end{figure}}
	
\newcommand{\sizedfic}[2]{\begin{figure}[H]
		\center
		\includegraphics[width=#1\textwidth]{#2}
	\end{figure}}

\newcommand{\codefile}[1]{\lstinputlisting{#1}}

\newcommand{\interval}{\vspace{0.5em}}

\newcommand{\tablestart}{
	\interval
	\begin{longtable}{p{2cm}p{10cm}}
	\hline}
\newcommand{\tableend}{
	\hline
	\end{longtable}
	\interval}

% 改变段间隔
\setlength{\parskip}{0.2em}
\linespread{1.1}

\usepackage{lastpage}
\usepackage{fancyhdr}
\pagestyle{fancy}
\lhead{\space \qquad \space}
\chead{snapshot的流程 \qquad}
\rhead{\qquad\thepage/\pageref{LastPage}}

\begin{document}

\tableofcontents

\clearpage

\section{snapshot的流程}
\subsection{数据结构}
\subsubsection{req}
    \begin{lstlisting}
    # nova/api/openstack/wsgi.py Request
    \end{lstlisting}

\subsubsection{body}
    \begin{lstlisting}
    {u'createImage': {u'name': u'snap1', u'metadata': {}}}
    \end{lstlisting}

\subsubsection{Instance}
    该类是代表虚拟机数据库抽象的类。

    \begin{lstlisting}
    # nova/objects/instance.py Instance
    # nova.db.sqlalchemy.models.Instance
    \end{lstlisting}

\subsubsection{bdms}
    \begin{lstlisting}
    # nova/objects/block_device.py BlockDeviceMappingList
    \end{lstlisting}

\subsubsection{Image}
    \begin{lstlisting}
    # glance.db.sqlalchemy.models.Image
    \end{lstlisting}

\subsubsection{image\_meta}
    \begin{lstlisting}
    {'status': u'queued', 'name': u'snap1', 'deleted': False, 'container_format': u'bare', 'created_at': datetime.datetime(2017, 2, 28, 5, 38, 57, tzinfo=<iso8601.iso8601.Utc object at 0x7fc8fc5a3d10>), 'disk_format': u'qcow2', 'updated_at': datetime.datetime(2017, 2, 28, 5, 38, 57, tzinfo=<iso8601.iso8601.Utc object at 0x7fc8fc5a3d10>), 'id': u'9ce400ab-7785-445c-9e89-9ea35d6de063', 'owner': u'd6fda80e5d464008825d806edf4ecc20', 'min_ram': 0, 'checksum': None, 'min_disk': 40, 'is_public': False, 'deleted_at': None, 'properties': {u'instance_uuid': u'ed38cd61-3b9c-47b8-b3e0-b9c511053b23', u'instance_type_memory_mb': u'4096', u'user_id': u'ea66cd61e4564f88b2a877868fe1b8a4', u'image_type': u'snapshot', u'instance_type_id': u'1', u'instance_type_name': u'm1.medium', u'instance_type_ephemeral_gb': u'0', u'instance_type_rxtx_factor': u'1.0', u'instance_type_root_gb': u'40', u'network_allocated': u'True', u'instance_type_flavorid': u'3', u'instance_type_vcpus': u'2', u'instance_type_swap': u'0', u'base_image_ref': u'1ed1b0e2-f9ae-4a9a-a0aa-166f6a75d5f2'}, 'size': 0}
    \end{lstlisting}

\subsection{nova-api部分}
    \begin{lstlisting}
    # nova/api/openstack/compute/servers.py Controller._action_create_image()
    def _action_create_image(self, req, id, body):
        # id是instance id
        context = req.environ['nova.context']
        entity = body.get("createImage", {})

        image_name = entity.get("name")

        ...

        props = {}
        metadata = entity.get('metadata', {})
        
        try:
            props.update(metadata)
        ...

        instance = self._get_server(context, req, id)

        bdms = objects.BlockDeviceMappingList.get_by_instance_uuid(
                    context, instance.uuid)

        try:
            # 判断root分区是否是volume
            if self.compute_api.is_volume_backed_instance(context, instance,
                                                          bdms):
                img = instance['image_ref']
                if not img:
                    properties = bdms.root_metadata(
                            context, self.compute_api.image_api,
                            self.compute_api.volume_api)
                    image_meta = {'properties': properties}
                else:
                    image_meta = self.compute_api.image_api.get(context, img)

                image = self.compute_api.snapshot_volume_backed(
                                                       context,
                                                       instance,
                                                       image_meta,
                                                       image_name,
                                                       extra_properties=props)
            else:
                image = self.compute_api.snapshot(context,
                                                  instance,
                                                  image_name,
                                                  extra_properties=props)
        ...

        # 以下代码的功能:build location of newly-created image entity
        image_id = str(image['id'])
        url_prefix = self._view_builder._update_glance_link_prefix(
                req.application_url)
        image_ref = os.path.join(url_prefix,
                                 context.project_id,
                                 'images',
                                 image_id)

        resp = webob.Response(status_int=202)
        resp.headers['Location'] = image_ref
        return resp
    \end{lstlisting}

    \begin{lstlisting}
    # nova/compute/api.py API.snapshot()
    def snapshot(self, context, instance, name, extra_properties=None):
        # instance: nova.db.sqlalchemy.models.Instance
        # name: name of the snapshot
        # extra_properties: dict of extra image properties to include
        #                   when creating the image.
        # returns: A dict containing image metadata

        # 函数功能:Create new image entry in the image service.  This new image will be reserved for the compute manager to upload a snapshot or backup.
        image_meta = self._create_image(context, instance, name,
                                        'snapshot',
                                        extra_properties=extra_properties)

        # 更改instance的task_state
        instance.task_state = task_states.IMAGE_SNAPSHOT_PENDING
        # 调用Instance.save()方法更改数据库
        instance.save(expected_task_state=[None])

        self.compute_rpcapi.snapshot_instance(context, instance,
                                              image_meta['id'])

        return image_meta
    \end{lstlisting}

    \begin{lstlisting}
    # nova/compute/rpcapi.py ComputeAPI.snapshot_instance()
    def snapshot_instance(self, ctxt, instance, image_id):
        # server: the destination host for a message.
        # server == instance['host']
        cctxt = self.client.prepare(server=_compute_host(None, instance),
                version=version)
        cctxt.cast(ctxt, 'snapshot_instance',
                   instance=instance,
                   image_id=image_id)
    \end{lstlisting}

\subsection{nova-compute部分}
    \begin{lstlisting}
    # nova/compute/manager.py ComputeManager.snapshot_instance()
    def snapshot_instance(self, context, image_id, instance):
        # context: security context
        # instance: a nova.objects.instance.Instance object
        # image_id: glance.db.sqlalchemy.models.Image.Id
        
        try:
            # 修改instance的task_state
            instance.task_state = task_states.IMAGE_SNAPSHOT
            # 调用Instance.save()方法更改数据库
            instance.save(
                        expected_task_state=task_states.IMAGE_SNAPSHOT_PENDING)
        ...

        self._snapshot_instance(context, image_id, instance,
                                task_states.IMAGE_SNAPSHOT)
    \end{lstlisting}

    \begin{lstlisting}
    # nova/compute/manager.py ComputeManager._snapshot_instance()
    def _snapshot_instance(self, context, image_id, instance,
                           expected_task_state):
        # self.driver.get_info(instance)["state"]
        current_power_state = self._get_power_state(context, instance)
        try:
            # 修改instance的power_state
            instance.power_state = current_power_state
            instance.save()

            if instance.power_state != power_state.RUNNING:
                state = instance.power_state
                running = power_state.RUNNING

            def update_task_state(task_state,
                                  expected_state=expected_task_state):
                instance.task_state = task_state
                instance.save(expected_task_state=expected_state)

            # 调用LibvirtDriver.snapshot()函数实现快照功能
            self.driver.snapshot(context, instance, image_id,
                                 update_task_state)
            
            # 记录instance状态
            instance.task_state = None
            instance.save(expected_task_state=task_states.IMAGE_UPLOADING)
        ...
    \end{lstlisting}

    \begin{lstlisting}
    def snapshot(self, context, instance, image_id, update_task_state):
        try:
            # 调用virConnect.lookupByName()返回virDomain对象
            virt_dom = self._lookup_by_name(instance['name'])
        ...

        base_image_ref = instance['image_ref']

        # 得到instance的image的相关数据
        base = compute_utils.get_image_metadata(
            context, self._image_api, base_image_ref, instance)
        
        # Retrieves the information record for a single disk image by image_id
        snapshot = self._image_api.get(context, image_id)

        # 得到instance的磁盘路径
        disk_path = libvirt_utils.find_disk(virt_dom)
        source_format = libvirt_utils.get_disk_type(disk_path)

        image_format = CONF.libvirt.snapshot_image_format or source_format

        # NOTE(bfilippov): save lvm and rbd as raw
        if image_format == 'lvm' or image_format == 'rbd':
            image_format = 'raw'

        metadata = self._create_snapshot_metadata(base,
                                                  instance,
                                                  image_format,
                                                  snapshot['name'])

        snapshot_name = uuid.uuid4().hex

        state = LIBVIRT_POWER_STATE[virt_dom.info()[0]]

        # NOTE(rmk): Live snapshots require QEMU 1.3 and Libvirt 1.0.0.
        #            These restrictions can be relaxed as other configurations
        #            can be validated.
        # NOTE(dgenin): Instances with LVM encrypted ephemeral storage require
        #               cold snapshots. Currently, checking for encryption is
        #               redundant because LVM supports only cold snapshots.
        #               It is necessary in case this situation changes in the
        #               future.
        if (self._has_min_version(MIN_LIBVIRT_LIVESNAPSHOT_VERSION,
                                  MIN_QEMU_LIVESNAPSHOT_VERSION,
                                  REQ_HYPERVISOR_LIVESNAPSHOT)
             and source_format not in ('lvm', 'rbd')
             and not CONF.ephemeral_storage_encryption.enabled):
            live_snapshot = True
            # Abort is an idempotent operation, so make sure any block
            # jobs which may have failed are ended. This operation also
            # confirms the running instance, as opposed to the system as a
            # whole, has a new enough version of the hypervisor (bug 1193146).
            try:
                virt_dom.blockJobAbort(disk_path, 0)
            except libvirt.libvirtError as ex:
                error_code = ex.get_error_code()
                if error_code == libvirt.VIR_ERR_CONFIG_UNSUPPORTED:
                    live_snapshot = False
                else:
                    pass
        else:
            live_snapshot = False

        # NOTE(rmk): We cannot perform live snapshots when a managedSave
        #            file is present, so we will use the cold/legacy method
        #            for instances which are shutdown.
        if state == power_state.SHUTDOWN:
            live_snapshot = False

        # NOTE(dkang): managedSave does not work for LXC
        if CONF.libvirt.virt_type != 'lxc' and not live_snapshot:
            if state == power_state.RUNNING or state == power_state.PAUSED:
                self._detach_pci_devices(virt_dom,
                    pci_manager.get_instance_pci_devs(instance))
                self._detach_sriov_ports(context, instance, virt_dom)
                virt_dom.managedSave(0)

        snapshot_backend = self.image_backend.snapshot(instance,
                disk_path,
                image_type=source_format)

        if live_snapshot:
            LOG.info(_LI("Beginning live snapshot process"),
                     instance=instance)
        else:
            LOG.info(_LI("Beginning cold snapshot process"),
                     instance=instance)

        update_task_state(task_state=task_states.IMAGE_PENDING_UPLOAD)
        snapshot_directory = CONF.libvirt.snapshots_directory
        fileutils.ensure_tree(snapshot_directory)
        with utils.tempdir(dir=snapshot_directory) as tmpdir:
            try:
                out_path = os.path.join(tmpdir, snapshot_name)
                if live_snapshot:
                    # NOTE(xqueralt): libvirt needs o+x in the temp directory
                    os.chmod(tmpdir, 0o701)
                    self._live_snapshot(virt_dom, disk_path, out_path,
                                        image_format)
                else:
                    snapshot_backend.snapshot_extract(out_path, image_format)
            finally:
                new_dom = None
                # NOTE(dkang): because previous managedSave is not called
                #              for LXC, _create_domain must not be called.
                if CONF.libvirt.virt_type != 'lxc' and not live_snapshot:
                    if state == power_state.RUNNING:
                        new_dom = self._create_domain(domain=virt_dom)
                    elif state == power_state.PAUSED:
                        new_dom = self._create_domain(domain=virt_dom,
                                launch_flags=libvirt.VIR_DOMAIN_START_PAUSED)
                    if new_dom is not None:
                        self._attach_pci_devices(new_dom,
                            pci_manager.get_instance_pci_devs(instance))
                        self._attach_sriov_ports(context, instance, new_dom)
                LOG.info(_LI("Snapshot extracted, beginning image upload"),
                         instance=instance)

            # Upload that image to the image service

            update_task_state(task_state=task_states.IMAGE_UPLOADING,
                     expected_state=task_states.IMAGE_PENDING_UPLOAD)
            with libvirt_utils.file_open(out_path) as image_file:
                self._image_api.update(context,
                                       image_id,
                                       metadata,
                                       image_file)
                LOG.info(_LI("Snapshot image upload complete"),
                         instance=instance)
    \end{lstlisting}

\end{document}