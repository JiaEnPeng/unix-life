% !TeX spellcheck = en_US
%% 字体:方正静蕾简体
%%		 方正粗宋
\documentclass[a4paper,left=1.5cm,right=1.5cm,11pt]{article}

\usepackage[utf8]{inputenc}
\usepackage{fontspec}
\usepackage{cite}
\usepackage{xeCJK}
\usepackage{indentfirst}
\usepackage{titlesec}
\usepackage{etoolbox}%
\makeatletter
\patchcmd{\ttlh@hang}{\parindent\z@}{\parindent\z@\leavevmode}{}{}%
\patchcmd{\ttlh@hang}{\noindent}{}{}{}%
\makeatother

\usepackage{hyperref}
\usepackage{longtable}
\usepackage{empheq}
\usepackage{graphicx}
\usepackage{float}
\usepackage{rotating}
\usepackage{subfigure}
\usepackage{tabu}
\usepackage{amsmath}
\usepackage{setspace}
\usepackage{amsfonts}
\usepackage{appendix}
\usepackage{listings}
\usepackage{xcolor}
\usepackage{geometry}
\setcounter{secnumdepth}{4}
%\titleformat*{\section}{\LARGE}
%\renewcommand\refname{参考文献}
%\titleformat{\chapter}{\centering\bfseries\huge}{}{0.7em}{}{}
\titleformat{\section}{\LARGE\bf}{\thesection}{1em}{}{}
\titleformat{\subsection}{\Large\bfseries}{\thesubsection}{1em}{}{}
\titleformat{\subsubsection}{\large\bfseries}{\thesubsubsection}{1em}{}{}
\renewcommand{\contentsname}{{ \centerline{目{  } 录}}}
\setCJKfamilyfont{cjkhwxk}{STXINGKA.TTF}
%\setCJKfamilyfont{cjkhwxk}{华文行楷}
%\setCJKfamilyfont{cjkfzcs}{方正粗宋简体}
%\newcommand*{\cjkfzcs}{\CJKfamily{cjkfzcs}}
\newcommand*{\cjkhwxk}{\CJKfamily{cjkhwxk}}
%\newfontfamily\wryh{Microsoft YaHei}
%\newfontfamily\hwzs{华文中宋}
%\newfontfamily\hwst{华文宋体}
%\newfontfamily\hwfs{华文仿宋}
%\newfontfamily\jljt{方正静蕾简体}
%\newfontfamily\hwxk{华文行楷}
\newcommand{\verylarge}{\fontsize{60pt}{\baselineskip}\selectfont}  
\newcommand{\chuhao}{\fontsize{44.9pt}{\baselineskip}\selectfont}  
\newcommand{\xiaochu}{\fontsize{38.5pt}{\baselineskip}\selectfont}  
\newcommand{\yihao}{\fontsize{27.8pt}{\baselineskip}\selectfont}  
\newcommand{\xiaoyi}{\fontsize{25.7pt}{\baselineskip}\selectfont}  
\newcommand{\erhao}{\fontsize{23.5pt}{\baselineskip}\selectfont}  
\newcommand{\xiaoerhao}{\fontsize{19.3pt}{\baselineskip}\selectfont} 
\newcommand{\sihao}{\fontsize{14pt}{\baselineskip}\selectfont}      % 字号设置  
\newcommand{\xiaosihao}{\fontsize{12pt}{\baselineskip}\selectfont}  % 字号设置  
\newcommand{\wuhao}{\fontsize{10.5pt}{\baselineskip}\selectfont}    % 字号设置  
\newcommand{\xiaowuhao}{\fontsize{9pt}{\baselineskip}\selectfont}   % 字号设置  
\newcommand{\liuhao}{\fontsize{7.875pt}{\baselineskip}\selectfont}  % 字号设置  
\newcommand{\qihao}{\fontsize{5.25pt}{\baselineskip}\selectfont}    % 字号设置 

\usepackage{diagbox}
\usepackage{multirow}
\boldmath
\XeTeXlinebreaklocale "zh"
\XeTeXlinebreakskip = 0pt plus 1pt minus 0.1pt
\definecolor{cred}{rgb}{0.8,0.8,0.8}
\definecolor{cgreen}{rgb}{0,0.3,0}
\definecolor{cpurple}{rgb}{0.5,0,0.35}
\definecolor{cdocblue}{rgb}{0,0,0.3}
\definecolor{cdark}{rgb}{0.95,1.0,1.0}
\lstset{
	language=python,
	numbers=left,
	numberstyle=\tiny\color{black},
	showspaces=false,
	showstringspaces=false,
	basicstyle=\scriptsize,
	keywordstyle=\color{purple},
	commentstyle=\itshape\color{cgreen},
	stringstyle=\color{blue},
	frame=lines,
	% escapeinside=``,
	extendedchars=true, 
	xleftmargin=1em,
	xrightmargin=1em, 
	backgroundcolor=\color{cred},
	aboveskip=1em,
	breaklines=true,
	tabsize=4
} 

%\newfontfamily{\consolas}{Consolas}
%\newfontfamily{\monaco}{Monaco}
%\setmonofont[Mapping={}]{Consolas}	%英文引号之类的正常显示,相当于设置英文字体
%\setsansfont{Consolas} %设置英文字体 Monaco, Consolas,  Fantasque Sans Mono
%\setmainfont{Times New Roman}
%\setCJKmainfont{STZHONGS.TTF}
%\setmonofont{Consolas}
% \newfontfamily{\consolas}{YaHeiConsolas.ttf}
\newfontfamily{\monaco}{MONACO.TTF}
\setCJKmainfont{STZHONGS.TTF}
%\setmainfont{MONACO.TTF}
%\setsansfont{MONACO.TTF}

\newcommand{\fic}[1]{\begin{figure}[H]
		\center
		\includegraphics[width=0.8\textwidth]{#1}
	\end{figure}}
	
\newcommand{\sizedfic}[2]{\begin{figure}[H]
		\center
		\includegraphics[width=#1\textwidth]{#2}
	\end{figure}}

\newcommand{\codefile}[1]{\lstinputlisting{#1}}

\newcommand{\interval}{\vspace{0.5em}}

\newcommand{\tablestart}{
	\interval
	\begin{longtable}{p{2cm}p{10cm}}
	\hline}
\newcommand{\tableend}{
	\hline
	\end{longtable}
	\interval}

% 改变段间隔
\setlength{\parskip}{0.2em}
\linespread{1.1}

\usepackage{lastpage}
\usepackage{fancyhdr}
\pagestyle{fancy}
\lhead{\space \qquad \space}
\chead{openstack中的重要概念 \qquad}
\rhead{\qquad\thepage/\pageref{LastPage}}

\begin{document}

\section{BlockDeviceMapping}
    \url{https://docs.openstack.org/developer/nova/block_device_mapping.html}

    \begin{lstlisting}
class BlockDeviceMapping(base.NovaPersistentObject, base.NovaObject):
    fields = {
        'id': fields.IntegerField(),
        'instance_uuid': fields.UUIDField(),
        'instance': fields.ObjectField('Instance', nullable=True),
        'source_type': fields.StringField(nullable=True),
        'destination_type': fields.StringField(nullable=True),
        'guest_format': fields.StringField(nullable=True),
        'device_type': fields.StringField(nullable=True),
        'disk_bus': fields.StringField(nullable=True),
        'boot_index': fields.IntegerField(nullable=True),
        'device_name': fields.StringField(nullable=True),
        'delete_on_termination': fields.BooleanField(default=False),
        'snapshot_id': fields.StringField(nullable=True),
        'volume_id': fields.StringField(nullable=True),
        'volume_size': fields.IntegerField(nullable=True),
        'image_id': fields.StringField(nullable=True),
        'no_device': fields.BooleanField(default=False),
        'connection_info': fields.StringField(nullable=True),
    }
    \end{lstlisting}
    
\section{flavor}
    \url{https://docs.openstack.org/admin-guide/compute-flavors.html}

    \url{https://docs.openstack.org/admin-guide/cli-manage-flavors.html}

    在OpenStack中,flavors定义了nova实例的CPU、内存、存储容量等数值。
    简单地说,一个flavor相当于一个实例的硬件配置。\par

    一个flavor包含了如下参数:
    \begin{itemize}
        \item[1.] Flavor ID,这是一个flavor的uuid。uuid一般是自动产生的。
        \item[2.] Name,一个flavor的名称。
        \item[3.] VCPUS,虚拟CPU的数量。
        \item[4.] Memory MB,RAM的大小,单位为MB。
        \item[5.] Root Disk GB,root分区的磁盘空间的大小,单位为GB。
        \item[6.] Ephemeral Disk GB,临时分区的磁盘空间的大小,单位为GB,
                  默认值为0。需要知道的是,当虚拟机关闭时,临时分区所有数据将丢失。
                  而且制作快照时,不会考虑临时分区中的数据。
        \item[7.] Swap,交换空间的大小,单位为MB,默认值为0。
        \item[8.] RXTX Factor,这是一个可选属性,用于创建不同带宽的server,默认值为1.0。
                  RXTX Factor仅适用于基于Xen或NSX的系统。
        \item[9.] Is Public,用于决定是否任何用户都可以使用这个flavor,默认值为True。
        \item[10.] Extra Specs,键和值的pair,用于定义flavor可以在哪些compute node上运行。
    \end{itemize}

    对于Newton而言,openstack没有默认的flavor,而Mitaka和更早的版本有如下的默认flavor:
    \fic{1.png}

\subsection{管理flavor}
    在openstack中,可以使用openstack flavor命令行工具来管理flavor。\par

    常用的openstack flavor命令如下所示:
    \begin{itemize}
        \item[1.] 列出flavors,并显示出flavor的属性,命令如下:
        \begin{lstlisting}
    openstack flavor list
        \end{lstlisting}

        \item[2.] 创建flavor,命令如下:
        \begin{lstlisting}
    openstack flavor create FLAVOR_NAME --id FLAVOR_ID --ram RAM_IN_MB --disk ROOT_DISK_IN_GB --vcpus NUMBER_OF_VCPUS
        \end{lstlisting}

        可以通过如下命令查看create更多的选项:
        \begin{lstlisting}
    openstack help flavor create
        \end{lstlisting}

        \item[3.] 将flavor分配给一个project,命令如下:
        \begin{lstlisting}
    # FLAVOR是flavor的名称或ID
    # TENANT_ID是project的ID
    nova flavor-access-add FLAVOR TENANT_ID
        \end{lstlisting}

        \item[4.] 删除flavor,命令如下:
        \begin{lstlisting}
    openstack flavor delete FLAVOR_ID
        \end{lstlisting}

        \item[5.] 查看flavor命令的帮助手册,命令如下:
        \begin{lstlisting}
    openstack flavor --help
        \end{lstlisting}
    \end{itemize}

\subsection{Extra Specs}
    这个网页可以查看Extra Specs上的值:
    \begin{lstlisting}
    https://docs.openstack.org/admin-guide/compute-flavors.html#extra-specs
    \end{lstlisting}

\section{Instance类}
    \url{https://developer.openstack.org/api-guide/compute/server_concepts.html}

    \begin{lstlisting}[language = python]
    # nova/objects/instance.py Instance
    Instance.save()  # 用于修改数据库
    \end{lstlisting}

    \begin{lstlisting}
class Instance(BASE, NovaBase):
    """Represents a guest VM."""
    __tablename__ = 'instances'
    __table_args__ = (
        Index('uuid', 'uuid', unique=True),
        Index('project_id', 'project_id'),
        Index('instances_host_deleted_idx',
              'host', 'deleted'),
        Index('instances_reservation_id_idx',
              'reservation_id'),
        Index('instances_terminated_at_launched_at_idx',
              'terminated_at', 'launched_at'),
        Index('instances_uuid_deleted_idx',
              'uuid', 'deleted'),
        Index('instances_task_state_updated_at_idx',
              'task_state', 'updated_at'),
        Index('instances_host_node_deleted_idx',
              'host', 'node', 'deleted'),
        Index('instances_host_deleted_cleaned_idx',
              'host', 'deleted', 'cleaned'),
    )
    injected_files = []

    id = Column(Integer, primary_key=True, autoincrement=True)

    @property
    def name(self):
        try:
            base_name = CONF.instance_name_template % self.id
        except TypeError:
            # Support templates like "uuid-%(uuid)s", etc.
            info = {}
            # NOTE(russellb): Don't use self.iteritems() here, as it will
            # result in infinite recursion on the name property.
            for column in iter(orm.object_mapper(self).columns):
                key = column.name
                # prevent recursion if someone specifies %(name)s
                # %(name)s will not be valid.
                if key == 'name':
                    continue
                info[key] = self[key]
            try:
                base_name = CONF.instance_name_template % info
            except KeyError:
                base_name = self.uuid
        return base_name

    @property
    def _extra_keys(self):
        return ['name']

    user_id = Column(String(255))
    project_id = Column(String(255))

    image_ref = Column(String(255)) # instance的后端镜像id
    kernel_id = Column(String(255))
    ramdisk_id = Column(String(255))
    hostname = Column(String(255))

    launch_index = Column(Integer)
    key_name = Column(String(255))
    key_data = Column(MediumText())

    power_state = Column(Integer) # instance的power_state
    vm_state = Column(String(255))
    task_state = Column(String(255)) # instance的task_state

    memory_mb = Column(Integer)
    vcpus = Column(Integer)
    root_gb = Column(Integer)
    ephemeral_gb = Column(Integer)
    ephemeral_key_uuid = Column(String(36))

    # This is not related to hostname, above.  It refers
    #  to the nova node.
    # instance所在的宿主机
    host = Column(String(255))  # , ForeignKey('hosts.id'))
    # To identify the "ComputeNode" which the instance resides in.
    # This equals to ComputeNode.hypervisor_hostname.
    node = Column(String(255))

    # *not* flavorid, this is the internal primary_key
    instance_type_id = Column(Integer)

    user_data = Column(MediumText())

    reservation_id = Column(String(255))

    scheduled_at = Column(DateTime)
    launched_at = Column(DateTime)
    terminated_at = Column(DateTime)

    availability_zone = Column(String(255))

    # User editable field for display in user-facing UIs
    display_name = Column(String(255))
    display_description = Column(String(255))

    # To remember on which host an instance booted.
    # An instance may have moved to another host by live migration.
    launched_on = Column(MediumText())

    # NOTE(jdillaman): locked deprecated in favor of locked_by,
    # to be removed in Icehouse
    locked = Column(Boolean)
    locked_by = Column(Enum('owner', 'admin'))

    os_type = Column(String(255))
    architecture = Column(String(255))
    vm_mode = Column(String(255))
    uuid = Column(String(36))

    root_device_name = Column(String(255))
    default_ephemeral_device = Column(String(255))
    default_swap_device = Column(String(255))
    config_drive = Column(String(255))

    # User editable field meant to represent what ip should be used
    # to connect to the instance
    access_ip_v4 = Column(types.IPAddress())
    access_ip_v6 = Column(types.IPAddress())

    auto_disk_config = Column(Boolean())
    progress = Column(Integer)

    # EC2 instance_initiated_shutdown_terminate
    # True: -> 'terminate'
    # False: -> 'stop'
    # Note(maoy): currently Nova will always stop instead of terminate
    # no matter what the flag says. So we set the default to False.
    shutdown_terminate = Column(Boolean(), default=False)

    # EC2 disable_api_termination
    disable_terminate = Column(Boolean(), default=False)

    # OpenStack compute cell name.  This will only be set at the top of
    # the cells tree and it'll be a full cell name such as 'api!hop1!hop2'
    cell_name = Column(String(255))
    internal_id = Column(Integer)

    # Records whether an instance has been deleted from disk
    cleaned = Column(Integer, default=0)
    \end{lstlisting}

\section{image metadata}
    image metadata的概念:
    \begin{lstlisting}
    Another common term for “image properties” is “image metadata” because what we’re talking about here are properties that describe the image data that can be consumed by various OpenStack services 
    \end{lstlisting}

    image metadata的示例:
    \begin{lstlisting}
    {'status': u'queued', 
     'name': u'snap1', 
     'deleted': False, 
     'container_format': u'bare', 
     'created_at': datetime.datetime(2017, 2, 28, 5, 38, 57, tzinfo=<iso8601.iso8601.Utc object at 0x7fc8fc5a3d10>), 
     'disk_format': u'qcow2', 
     'updated_at': datetime.datetime(2017, 2, 28, 5, 38, 57, tzinfo=<iso8601.iso8601.Utc object at 0x7fc8fc5a3d10>), 
     'id': u'9ce400ab-7785-445c-9e89-9ea35d6de063', 
     'owner': u'd6fda80e5d464008825d806edf4ecc20', 
     'min_ram': 0, 
     'checksum': None, 
     'min_disk': 40, 
     'is_public': False, 
     'deleted_at': None, 
     'properties': {u'instance_uuid': u'ed38cd61-3b9c-47b8-b3e0-b9c511053b23', 
                    u'instance_type_memory_mb': u'4096', 
                    u'user_id': u'ea66cd61e4564f88b2a877868fe1b8a4', 
                    u'image_type': u'snapshot', 
                    u'instance_type_id': u'1', 
                    u'instance_type_name': u'm1.medium', 
                    u'instance_type_ephemeral_gb': u'0', 
                    u'instance_type_rxtx_factor': u'1.0', 
                    u'instance_type_root_gb': u'40', 
                    u'network_allocated': u'True', 
                    u'instance_type_flavorid': u'3', 
                    u'instance_type_vcpus': u'2', 
                    u'instance_type_swap': u'0', 
                    u'base_image_ref': u'1ed1b0e2-f9ae-4a9a-a0aa-166f6a75d5f2'}, 
     'size': 0}
    \end{lstlisting}

\section{Image}
    \begin{lstlisting}
class Image(BASE, GlanceBase):
    """Represents an image in the datastore."""
    __tablename__ = 'images'
    __table_args__ = (Index('checksum_image_idx', 'checksum'),
                      Index('visibility_image_idx', 'visibility'),
                      Index('ix_images_deleted', 'deleted'),
                      Index('owner_image_idx', 'owner'),
                      Index('created_at_image_idx', 'created_at'),
                      Index('updated_at_image_idx', 'updated_at'))

    id = Column(String(36), primary_key=True,
                default=lambda: str(uuid.uuid4()))
    name = Column(String(255))
    disk_format = Column(String(20))
    container_format = Column(String(20))
    size = Column(BigInteger().with_variant(Integer, "sqlite"))
    virtual_size = Column(BigInteger().with_variant(Integer, "sqlite"))
    status = Column(String(30), nullable=False)
    visibility = Column(Enum('private', 'public', 'shared', 'community',
                        name='image_visibility'), nullable=False,
                        server_default='shared')
    checksum = Column(String(32))
    min_disk = Column(Integer, nullable=False, default=0)
    min_ram = Column(Integer, nullable=False, default=0)
    owner = Column(String(255))
    protected = Column(Boolean, nullable=False, default=False,
                       server_default=sql.expression.false())
    \end{lstlisting}

\section{用户请求Request}
    \begin{lstlisting}
class Request(webob.Request):
    """Add some OpenStack API-specific logic to the base webob.Request."""
    \end{lstlisting}

\end{document}