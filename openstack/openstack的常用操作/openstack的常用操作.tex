% !TeX spellcheck = en_US
%% 字体:方正静蕾简体
%%		 方正粗宋
\documentclass[a4paper,left=1.5cm,right=1.5cm,11pt]{article}

\usepackage[utf8]{inputenc}
\usepackage{fontspec}
\usepackage{cite}
\usepackage{xeCJK}
\usepackage{indentfirst}
\usepackage{titlesec}
\usepackage{etoolbox}%
\makeatletter
\patchcmd{\ttlh@hang}{\parindent\z@}{\parindent\z@\leavevmode}{}{}%
\patchcmd{\ttlh@hang}{\noindent}{}{}{}%
\makeatother
\usepackage{hyperref}
\usepackage{longtable}
\usepackage{empheq}
\usepackage{graphicx}
\usepackage{float}
\usepackage{rotating}
\usepackage{subfigure}
\usepackage{tabu}
\usepackage{amsmath}
\usepackage{setspace}
\usepackage{amsfonts}
\usepackage{appendix}
\usepackage{listings}
\usepackage{xcolor}
\usepackage{geometry}
\setcounter{secnumdepth}{4}
%\titleformat*{\section}{\LARGE}
%\renewcommand\refname{参考文献}
%\titleformat{\chapter}{\centering\bfseries\huge}{}{0.7em}{}{}
\titleformat{\section}{\LARGE\bf}{\thesection}{1em}{}{}
\titleformat{\subsection}{\Large\bfseries}{\thesubsection}{1em}{}{}
\titleformat{\subsubsection}{\large\bfseries}{\thesubsubsection}{1em}{}{}
\renewcommand{\contentsname}{{ \centerline{目{  } 录}}}
\setCJKfamilyfont{cjkhwxk}{STXINGKA.TTF}
%\setCJKfamilyfont{cjkhwxk}{华文行楷}
%\setCJKfamilyfont{cjkfzcs}{方正粗宋简体}
%\newcommand*{\cjkfzcs}{\CJKfamily{cjkfzcs}}
\newcommand*{\cjkhwxk}{\CJKfamily{cjkhwxk}}
%\newfontfamily\wryh{Microsoft YaHei}
%\newfontfamily\hwzs{华文中宋}
%\newfontfamily\hwst{华文宋体}
%\newfontfamily\hwfs{华文仿宋}
%\newfontfamily\jljt{方正静蕾简体}
%\newfontfamily\hwxk{华文行楷}
\newcommand{\verylarge}{\fontsize{60pt}{\baselineskip}\selectfont}  
\newcommand{\chuhao}{\fontsize{44.9pt}{\baselineskip}\selectfont}  
\newcommand{\xiaochu}{\fontsize{38.5pt}{\baselineskip}\selectfont}  
\newcommand{\yihao}{\fontsize{27.8pt}{\baselineskip}\selectfont}  
\newcommand{\xiaoyi}{\fontsize{25.7pt}{\baselineskip}\selectfont}  
\newcommand{\erhao}{\fontsize{23.5pt}{\baselineskip}\selectfont}  
\newcommand{\xiaoerhao}{\fontsize{19.3pt}{\baselineskip}\selectfont} 
\newcommand{\sihao}{\fontsize{14pt}{\baselineskip}\selectfont}      % 字号设置  
\newcommand{\xiaosihao}{\fontsize{12pt}{\baselineskip}\selectfont}  % 字号设置  
\newcommand{\wuhao}{\fontsize{10.5pt}{\baselineskip}\selectfont}    % 字号设置  
\newcommand{\xiaowuhao}{\fontsize{9pt}{\baselineskip}\selectfont}   % 字号设置  
\newcommand{\liuhao}{\fontsize{7.875pt}{\baselineskip}\selectfont}  % 字号设置  
\newcommand{\qihao}{\fontsize{5.25pt}{\baselineskip}\selectfont}    % 字号设置 

\usepackage{diagbox}
\usepackage{multirow}
\boldmath
\XeTeXlinebreaklocale "zh"
\XeTeXlinebreakskip = 0pt plus 1pt minus 0.1pt
\definecolor{cred}{rgb}{0.8,0.8,0.8}
\definecolor{cgreen}{rgb}{0,0.3,0}
\definecolor{cpurple}{rgb}{0.5,0,0.35}
\definecolor{cdocblue}{rgb}{0,0,0.3}
\definecolor{cdark}{rgb}{0.95,1.0,1.0}
\lstset{
	language=python,
	numbers=left,
	numberstyle=\tiny\color{black},
	showspaces=false,
	showstringspaces=false,
	basicstyle=\scriptsize,
	keywordstyle=\color{purple},
	commentstyle=\color{cgreen},
	stringstyle=\color{blue},
	frame=lines,
	% escapeinside=``,
	extendedchars=true, 
	xleftmargin=1em,
	xrightmargin=1em, 
	backgroundcolor=\color{cred},
	aboveskip=1em,
	breaklines=true,
	tabsize=4
} 

%\newfontfamily{\consolas}{Consolas}
%\newfontfamily{\monaco}{Monaco}
%\setmonofont[Mapping={}]{Consolas}	%英文引号之类的正常显示,相当于设置英文字体
%\setsansfont{Consolas} %设置英文字体 Monaco, Consolas,  Fantasque Sans Mono
%\setmainfont{Times New Roman}
%\setCJKmainfont{STZHONGS.TTF}
%\setmonofont{Consolas}
% \newfontfamily{\consolas}{YaHeiConsolas.ttf}
\newfontfamily{\monaco}{MONACO.TTF}
\setCJKmainfont{STZHONGS.TTF}
%\setmainfont{MONACO.TTF}
%\setsansfont{MONACO.TTF}

\newcommand{\fic}[1]{\begin{figure}[H]
		\center
		\includegraphics[width=0.8\textwidth]{#1}
	\end{figure}}
	
\newcommand{\sizedfic}[2]{\begin{figure}[H]
		\center
		\includegraphics[width=#1\textwidth]{#2}
	\end{figure}}

\newcommand{\codefile}[1]{\lstinputlisting{#1}}

\newcommand{\interval}{\vspace{0.5em}}

\newcommand{\tablestart}{
	\interval
	\begin{longtable}{p{2cm}p{10cm}}
	\hline}
\newcommand{\tableend}{
	\hline
	\end{longtable}
	\interval}

% 改变段间隔
\setlength{\parskip}{0.2em}
\linespread{1.1}

\usepackage{lastpage}
\usepackage{fancyhdr}
\pagestyle{fancy}
\lhead{\space \qquad \space}
\chead{openstack的常用操作 \qquad}
\rhead{\qquad\thepage/\pageref{LastPage}}

\begin{document}

\tableofcontents

\clearpage

\section{openstack baremetal}
\subsection{查看所有chassis}
	命令如下:
	\begin{lstlisting}
	openstack baremetal chassis list
	\end{lstlisting}

\section{openstack endpoint}
\subsection{查看所有的endpoint}
	命令如下:
	\begin{lstlisting}
	openstack endpoint list
	\end{lstlisting}

\subsection{创建一个服务的endpoint}
	命令如下:
	\begin{lstlisting}
	openstack endpoint create baremetal public http://10.250.1.3:6385 --region RegionOne
	openstack endpoint create baremetal admin http://10.250.1.3:6385 --region RegionOne
	openstack endpoint create baremetal internal http://10.250.1.3:6385 --region RegionOne
	\end{lstlisting}

\section{openstack hypervisor}
\subsection{列出所有的hypervisor}
	命令如下:
	\begin{lstlisting}
	openstack hypervisor list
	\end{lstlisting}

\subsection{查看某个hypervisor的信息}
	命令如下:
	\begin{lstlisting}
	openstack hypervisor show <hypervisor-uuid>
	\end{lstlisting}

\subsection{查看hypervisor的整体状态}
	命令如下:
	\begin{lstlisting}
	openstack hypervisor stats show
	\end{lstlisting}

\section{openstack image}
\subsection{列出所有镜像文件}
	命令如下:
	\begin{lstlisting}
	openstack image list
	\end{lstlisting}

\section{openstack network}

\section{openstack project}
\subsection{查看openstack中的project}
	命令如下:
	\begin{lstlisting}
	openstack project list
	\end{lstlisting}

\subsection{创建一个project}
	命令如下:
	\begin{lstlisting}
	openstack project create service --domain=Default
	openstack project create invisible_to_admin --domain=default
	openstack project create demo --domain=default
	openstack project create alt_demo --domain=default
	openstack project create swiftprojecttest1 --domain=default
	openstack project create swiftprojecttest2 --domain=default
	openstack project create swiftprojecttest4 --domain=7101c820856a4114bd3eb89a3b96cdae
	openstack project create service --domain=default
	\end{lstlisting}

\section{openstack role}
\subsection{查看openstack有哪些role}
	命令如下:
	\begin{lstlisting}
	openstack role list
	\end{lstlisting}

\subsection{创建一个role}
	命令如下:
	\begin{lstlisting}
	openstack role create service
	openstack role create ResellerAdmin
	openstack role create Member
	openstack role create member
	openstack role create anotherrole
	openstack role create anotherrole
	openstack role create baremetal_admin
	openstack role create baremetal_observer
	\end{lstlisting}

\subsection{赋予用户操作权限}
	命令如下:
	\begin{lstlisting}
	openstack role add admin --user 10ec861de94c473ab6051644891c48da --project 554c828e8fd347e6be93c2e5e855fa1f
	openstack role add admin --user 34ee1ff9a328464987c2229dfa1ac3e5 --project 3b7952fbf71b43dabfc70008a5867ee8
	openstack role add service --user placement --project service --user-domain Default --project-domain Default
	openstack role add admin --user placement --project service --user-domain Default --project-domain Default
	openstack role add service --user ironic --project service --user-domain Default --project-domain Default
	openstack role add admin --user ironic --project service --user-domain Default --project-domain Default
	openstack role add baremetal_admin --user nova --project service
	openstack role add baremetal_observer --user demo --project demo
	\end{lstlisting}

\section{openstack router}
\subsection{查看openstack中的router}
	命令如下:
	\begin{lstlisting}
	openstack router list
	\end{lstlisting}

\subsection{创建一个router}
	命令如下:
	\begin{lstlisting}
	openstack --os-cloud devstack-admin --os-region RegionOne router create --project 6eaff7ecc1694fcd94532bad5f09e17f router1
	\end{lstlisting}

\subsection{将一个subnet加入一个router}
	命令如下:
	\begin{lstlisting}
	openstack --os-cloud devstack-admin --os-region RegionOne router add subnet ff71a343-a4fb-4a6a-b7fc-f013566b31cc 26e240db-8cc9-4938-831a-5710d689426a
	\end{lstlisting}

\section{openstack service}
\subsection{查看openstack有哪些服务}
	命令如下:
	\begin{lstlisting}
	openstack service list
	\end{lstlisting}

\subsection{创建服务实体}
	命令如下:
	\begin{lstlisting}
	openstack service create network --name neutron '--description=Neutron Service'
	\end{lstlisting}
	
\subsection{查看openstack中某个服务的信息}
	命令如下:
	\begin{lstlisting}
	# 显示network的信息
	openstack service show network
	# 以某种形式显示这个信息
	openstack service show network -f json
	# 单独显示信息中的某个值
	openstack service show network -c id
	\end{lstlisting}

\section{openstack user}
\subsection{查看所有的user}
	命令如下:
	\begin{lstlisting}
	openstack user list
	\end{lstlisting}

\subsection{创建相应的user}
	命令如下:
	\begin{lstlisting}
	openstack user create demo --password p1111111 --domain=default --email=demo@example.com
	openstack user create alt_demo --password p1111111 --domain=default --email=alt_demo@example.com
	openstack user create nova --password p1111111 --domain=Default
	openstack user create glance --password p1111111 --domain=Default
	openstack user create glance-swift --password p1111111 --domain=Default
	openstack user create neutron --password p1111111 --domain=Default
	openstack user create swift --password p1111111 --domain=Default
	openstack user create swiftusertest1 --password testing --domain=default --email=test@example.com
	openstack user create swiftusertest3 --password testing3 --domain=default --email=test3@example.com
	openstack user create swiftusertest2 --password testing2 --domain=default --email=test2@example.com
	openstack user create swiftusertest4 --password testing4 --domain=7101c820856a4114bd3eb89a3b96cdae --email=test4@example.com --or-show -f value -c id
	openstack user create placement --password p1111111 --domain=Default
	openstack user create ironic --password p1111111 --domain=Default
	\end{lstlisting}

\section{openstack subnet}

\end{document}