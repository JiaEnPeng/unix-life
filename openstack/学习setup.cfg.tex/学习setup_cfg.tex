% !TeX spellcheck = en_US
%% 字体:方正静蕾简体
%%		 方正粗宋
\documentclass[a4paper,left=1.5cm,right=1.5cm,11pt]{article}

\usepackage[utf8]{inputenc}
\usepackage{fontspec}
\usepackage{cite}
\usepackage{xeCJK}
\usepackage{indentfirst}
\usepackage{titlesec}
\usepackage{etoolbox}%
\makeatletter
\patchcmd{\ttlh@hang}{\parindent\z@}{\parindent\z@\leavevmode}{}{}%
\patchcmd{\ttlh@hang}{\noindent}{}{}{}%
\makeatother

\usepackage{longtable}
\usepackage{empheq}
\usepackage{graphicx}
\usepackage{float}
\usepackage{rotating}
\usepackage{subfigure}
\usepackage{tabu}
\usepackage{amsmath}
\usepackage{setspace}
\usepackage{amsfonts}
\usepackage{appendix}
\usepackage{listings}
\usepackage{xcolor}
\usepackage{geometry}
\setcounter{secnumdepth}{4}
%\titleformat*{\section}{\LARGE}
%\renewcommand\refname{参考文献}
%\titleformat{\chapter}{\centering\bfseries\huge}{}{0.7em}{}{}
\titleformat{\section}{\LARGE\bf}{\thesection}{1em}{}{}
\titleformat{\subsection}{\Large\bfseries}{\thesubsection}{1em}{}{}
\titleformat{\subsubsection}{\large\bfseries}{\thesubsubsection}{1em}{}{}
\renewcommand{\contentsname}{{ \centerline{目{  } 录}}}
\setCJKfamilyfont{cjkhwxk}{STXINGKA.TTF}
%\setCJKfamilyfont{cjkhwxk}{华文行楷}
%\setCJKfamilyfont{cjkfzcs}{方正粗宋简体}
%\newcommand*{\cjkfzcs}{\CJKfamily{cjkfzcs}}
\newcommand*{\cjkhwxk}{\CJKfamily{cjkhwxk}}
%\newfontfamily\wryh{Microsoft YaHei}
%\newfontfamily\hwzs{华文中宋}
%\newfontfamily\hwst{华文宋体}
%\newfontfamily\hwfs{华文仿宋}
%\newfontfamily\jljt{方正静蕾简体}
%\newfontfamily\hwxk{华文行楷}
\newcommand{\verylarge}{\fontsize{60pt}{\baselineskip}\selectfont}  
\newcommand{\chuhao}{\fontsize{44.9pt}{\baselineskip}\selectfont}  
\newcommand{\xiaochu}{\fontsize{38.5pt}{\baselineskip}\selectfont}  
\newcommand{\yihao}{\fontsize{27.8pt}{\baselineskip}\selectfont}  
\newcommand{\xiaoyi}{\fontsize{25.7pt}{\baselineskip}\selectfont}  
\newcommand{\erhao}{\fontsize{23.5pt}{\baselineskip}\selectfont}  
\newcommand{\xiaoerhao}{\fontsize{19.3pt}{\baselineskip}\selectfont} 
\newcommand{\sihao}{\fontsize{14pt}{\baselineskip}\selectfont}      % 字号设置  
\newcommand{\xiaosihao}{\fontsize{12pt}{\baselineskip}\selectfont}  % 字号设置  
\newcommand{\wuhao}{\fontsize{10.5pt}{\baselineskip}\selectfont}    % 字号设置  
\newcommand{\xiaowuhao}{\fontsize{9pt}{\baselineskip}\selectfont}   % 字号设置  
\newcommand{\liuhao}{\fontsize{7.875pt}{\baselineskip}\selectfont}  % 字号设置  
\newcommand{\qihao}{\fontsize{5.25pt}{\baselineskip}\selectfont}    % 字号设置 

\usepackage{diagbox}
\usepackage{multirow}
\boldmath
\XeTeXlinebreaklocale "zh"
\XeTeXlinebreakskip = 0pt plus 1pt minus 0.1pt
\definecolor{cred}{rgb}{0.8,0.8,0.8}
\definecolor{cgreen}{rgb}{0,0.3,0}
\definecolor{cpurple}{rgb}{0.5,0,0.35}
\definecolor{cdocblue}{rgb}{0,0,0.3}
\definecolor{cdark}{rgb}{0.95,1.0,1.0}
\lstset{
	language=bash,
	numbers=left,
	numberstyle=\tiny\color{black},
	showspaces=false,
	showstringspaces=false,
	basicstyle=\scriptsize,
	keywordstyle=\color{purple},
	commentstyle=\itshape\color{cgreen},
	stringstyle=\color{blue},
	frame=lines,
	% escapeinside=``,
	extendedchars=true, 
	xleftmargin=1em,
	xrightmargin=1em, 
	backgroundcolor=\color{cred},
	aboveskip=1em,
	breaklines=true,
	tabsize=4
} 

%\newfontfamily{\consolas}{Consolas}
%\newfontfamily{\monaco}{Monaco}
%\setmonofont[Mapping={}]{Consolas}	%英文引号之类的正常显示,相当于设置英文字体
%\setsansfont{Consolas} %设置英文字体 Monaco, Consolas,  Fantasque Sans Mono
%\setmainfont{Times New Roman}
%\setCJKmainfont{STZHONGS.TTF}
%\setmonofont{Consolas}
% \newfontfamily{\consolas}{YaHeiConsolas.ttf}
\newfontfamily{\monaco}{MONACO.TTF}
\setCJKmainfont{STZHONGS.TTF}
%\setmainfont{MONACO.TTF}
%\setsansfont{MONACO.TTF}

\newcommand{\fic}[1]{\begin{figure}[H]
		\center
		\includegraphics[width=0.8\textwidth]{#1}
	\end{figure}}
	
\newcommand{\sizedfic}[2]{\begin{figure}[H]
		\center
		\includegraphics[width=#1\textwidth]{#2}
	\end{figure}}

\newcommand{\codefile}[1]{\lstinputlisting{#1}}

\newcommand{\interval}{\vspace{0.5em}}

\newcommand{\tablestart}{
	\interval
	\begin{longtable}{p{2cm}p{10cm}}
	\hline}
\newcommand{\tableend}{
	\hline
	\end{longtable}
	\interval}

% 改变段间隔
\setlength{\parskip}{0.2em}
\linespread{1.1}

\usepackage{lastpage}
\usepackage{fancyhdr}
\pagestyle{fancy}
\lhead{\space \qquad \space}
\chead{学习setup.cfg \qquad}
\rhead{\qquad\thepage/\pageref{LastPage}}

\begin{document}

\tableofcontents

\clearpage

\section{学习setup.cfg}
\subsection{基本术语}
	模块:python中可复用的基本代码单元,可由其他代码import。\par
	纯python模块:由python编写的模块,包含在单独的py文件中。\par
	扩展模块:由实现python的底层语言编写的模块,通常包含在单独的动态加载文件中。\par
	包:包是含其他模块的模块,经常由包含\_\_init\_\_.py文件的目录发布。\par
	模块发布:一些python模块的集合,这些模块将被一起安装。\par
	纯模块发布:一个只包含纯python模块和包的模块发布。\par
	非纯模块发布:至少包含一个扩展模块的模块发布。

\subsection{Distutils简介}
	Distutils可以用来在python环境中构建和安装额外的模块。\par

	为了使用Distutils,除了编写源码,还需要:
	\begin{itemize}
		\item[1.] 编写setup.py。
		\item[2.] 编写setup.cfg。
		\item[3.] 创建一个源码分布。
		\item[4.] 创建一个或多个二进制发布。
	\end{itemize}

\subsection{setup.cfg格式}
	setup.cfg是Distutils的配置文件,他的文件格式类似于:
	\begin{lstlisting}
	[command]
	option=value
	...
	\end{lstlisting}

	其中,command是Distutils的命令参数,option是相应的参数选项。\par

	setup.py文件是和setup.cfg配套存在的,可以通过setup.py查看command都有哪些参数选项:
	\begin{lstlisting}
	python setup.py command --help
	\end{lstlisting}

	举个例子:
	\begin{lstlisting}
	# setup.cfg文件的内容
	[build]
	build-base=blib
	force=1

	# 以上文件配置相当于下面的命令
	python setup.py build --build-base=blib --force	
	\end{lstlisting}

	setup.cfg文件的内容由很多个section组成,比如global、metadata、file等,提供了软件包的名称、作者等信息。
	如果我们想去理解代码结构,只需要关注[entry\_points]这一节。

\subsubsection{entry\_points}
	entry\_points提供了一个基于文件系统对象名的注册和import机制。\par

	也就是说,entry\_points会把python对象和某个name关联起来,
	之后其他的代码只要使用这个name就可以找到对应的对象,而不用关心这个对象具体所在的位置。\par

	举一个例子:
	\begin{lstlisting}[language = python]
	# 创建一个函数对象
	def the_function():
		"function whose name is 'the_function'"
		print "hello from the_function"

	# 这个函数对象所在的模块是myns.mypkg.mymodule.py
	# entry_points就可以这么写
	[entry_points]
	# my_ep_group_id相当于一个entry_points组,可以包含多个entry_point
	my_ep_group_id = 
		my_ep_func = myns.mypkg.mymodule: the_function
	\end{lstlisting}

	然后通过“python setup.py install”安装以后,就可以通过“pkg\_resources”去调用这些entry\_points:
	\begin{lstlisting}
	import pkg_resources

	named_objects = []
	for ep in pkg_resources.iter_entry_points(group='my_ep_group_id'):
		named_objects.append(ep.load())

	# 根据entry_points中mu_ep_group_id组的定义,可以知道named_objects[0]是the_function
	named_objects[0]() # 输出“hello from the_function”
	\end{lstlisting}

\subsubsection{openstack中使用entry\_points的例子}
	以Ceilometer为例,它setup.cfg中一部分内容如下:
	\begin{lstlisting}
	# ceilometer.compute.virt是一个entry_points组,包含了3个entry_point
	ceilometer.compute.virt = 
		libvirt = ceilometer.compute.virt.libvirt.inspector: LibvirtInspector
		hyperv = ceilometer.compute.virt.hyperv.inspector: HypervInspector
		vsphere = ceilometer.compute.virt.vmware.inspector: VsphereInspector
	\end{lstlisting}

	安装Ceilometer以后,其他程序可以利用下面几种方式调用这些entry\_point:
	\begin{itemize}
		\item[1.] 使用pkg\_resources,通过iter\_entry\_points遍历获得这些entry\_point:
		\begin{lstlisting}
	import pkg_resources
	def run_entry_point(data):
		group = 'ceilometer.compute.virt'
		for entrypoint in pkg_resources.iter_entry_points(group=group):
			plugin = entrypoint.load()
			plugin(data) # plugin指向ceilometer.compute.virt中entry_point中注册的对象
		\end{lstlisting}

		\item[2.] 使用pkg\_resources,通过load\_entry\_point函数和entry\_point的名称来获得注册的对象:
		\begin{lstlisting}
	from pkg_resources import load_entry_point
	fun = load_entry_point('ceilometer', 'ceilometer.compute.virt', 'libvirt')
	# fun指向LibvirtInspector这个函数对象
		\end{lstlisting}

		\item[3.] 使用stevedore,通过其中的driver类来获得注册的对象:
		\begin{lstlisting}
	from stevedore import driver

	def get_hypervisor_inspector():
		try:
			namespace = 'ceilometer.compute.virt'
			# cfg.CONF.hypervisor_inspector是oslo.config的一个配置选项
			mgr = driver.DriverManager(namespace, cfg.CONF.hypervisor_inspector, invoke_on_load=True)
			return mgr.driver()
		except ImportError as e:
			LOG.error(_("Unable ro load the hypervisor inspector: %s") % (e))
			return Inspector()
		\end{lstlisting}
	\end{itemize}

\end{document}