% !TeX spellcheck = en_US
%% 字体:方正静蕾简体
%%		 方正粗宋
\documentclass[a4paper,left=2.5cm,right=2.5cm,11pt]{article}

\usepackage[utf8]{inputenc}
\usepackage{fontspec}
\usepackage{cite}
\usepackage{xeCJK}
\usepackage{indentfirst}
\usepackage{titlesec}
\usepackage{longtable}
\usepackage{graphicx}
\usepackage{float}
\usepackage{rotating}
\usepackage{subfigure}
\usepackage{tabu}
\usepackage{amsmath}
\usepackage{setspace}
\usepackage{amsfonts}
\usepackage{appendix}
\usepackage{listings}
\usepackage{xcolor}
\usepackage{geometry}
\setcounter{secnumdepth}{4}
\usepackage{mhchem}
\usepackage{multirow}
\usepackage{extarrows}
\usepackage{hyperref}
\titleformat*{\section}{\LARGE}
\renewcommand\refname{参考文献}
\renewcommand{\abstractname}{\sihao \cjkfzcs 摘{  }要}
%\titleformat{\chapter}{\centering\bfseries\huge\wryh}{}{0.7em}{}{}
%\titleformat{\section}{\LARGE\bf}{\thesection}{1em}{}{}
\titleformat{\subsection}{\Large\bfseries}{\thesubsection}{1em}{}{}
\titleformat{\subsubsection}{\large\bfseries}{\thesubsubsection}{1em}{}{}
\renewcommand{\contentsname}{{\cjkfzcs \centerline{目{  } 录}}}
\setCJKfamilyfont{cjkhwxk}{STXingkai}
\setCJKfamilyfont{cjkfzcs}{STSongti-SC-Regular}
% \setCJKfamilyfont{cjkhwxk}{华文行楷}
% \setCJKfamilyfont{cjkfzcs}{方正粗宋简体}
\newcommand*{\cjkfzcs}{\CJKfamily{cjkfzcs}}
\newcommand*{\cjkhwxk}{\CJKfamily{cjkhwxk}}
\newfontfamily\wryh{Microsoft YaHei}
\newfontfamily\hwzs{STZhongsong}
\newfontfamily\hwst{STSong}
\newfontfamily\hwfs{STFangsong}
\newfontfamily\jljt{MicrosoftYaHei}
\newfontfamily\hwxk{STXingkai}
% \newfontfamily\hwzs{华文中宋}
% \newfontfamily\hwst{华文宋体}
% \newfontfamily\hwfs{华文仿宋}
% \newfontfamily\jljt{方正静蕾简体}
% \newfontfamily\hwxk{华文行楷}
\newcommand{\verylarge}{\fontsize{60pt}{\baselineskip}\selectfont}  
\newcommand{\chuhao}{\fontsize{44.9pt}{\baselineskip}\selectfont}  
\newcommand{\xiaochu}{\fontsize{38.5pt}{\baselineskip}\selectfont}  
\newcommand{\yihao}{\fontsize{27.8pt}{\baselineskip}\selectfont}  
\newcommand{\xiaoyi}{\fontsize{25.7pt}{\baselineskip}\selectfont}  
\newcommand{\erhao}{\fontsize{23.5pt}{\baselineskip}\selectfont}  
\newcommand{\xiaoerhao}{\fontsize{19.3pt}{\baselineskip}\selectfont} 
\newcommand{\sihao}{\fontsize{14pt}{\baselineskip}\selectfont}      % 字号设置  
\newcommand{\xiaosihao}{\fontsize{12pt}{\baselineskip}\selectfont}  % 字号设置  
\newcommand{\wuhao}{\fontsize{10.5pt}{\baselineskip}\selectfont}    % 字号设置  
\newcommand{\xiaowuhao}{\fontsize{9pt}{\baselineskip}\selectfont}   % 字号设置  
\newcommand{\liuhao}{\fontsize{7.875pt}{\baselineskip}\selectfont}  % 字号设置  
\newcommand{\qihao}{\fontsize{5.25pt}{\baselineskip}\selectfont}    % 字号设置 

\usepackage{diagbox}
\usepackage{multirow}
\boldmath
\XeTeXlinebreaklocale "zh"
\XeTeXlinebreakskip = 0pt plus 1pt minus 0.1pt
\definecolor{cred}{rgb}{0.8,0.8,0.8}
\definecolor{cgreen}{rgb}{0,0.3,0}
\definecolor{cpurple}{rgb}{0.5,0,0.35}
\definecolor{cdocblue}{rgb}{0,0,0.3}
\definecolor{cdark}{rgb}{0.95,1.0,1.0}
\lstset{
	language=[x86masm]Assembler,
	numbers=left,
	numberstyle=\tiny\color{black},
	showspaces=false,
	showstringspaces=false,
	basicstyle=\scriptsize,
	keywordstyle=\color{purple},
	commentstyle=\itshape\color{cgreen},
	stringstyle=\color{blue},
	frame=lines,
	% escapeinside=``,
	extendedchars=true, 
	xleftmargin=1em,
	xrightmargin=1em, 
	backgroundcolor=\color{cred},
	aboveskip=1em,
	breaklines=true,
	tabsize=4
} 

\newfontfamily{\consolas}{Consolas}
\newfontfamily{\monaco}{Monaco}
\setmonofont[Mapping={}]{Consolas}	%英文引号之类的正常显示,相当于设置英文字体
\setsansfont{Consolas} %设置英文字体 Monaco, Consolas,  Fantasque Sans Mono
\setmainfont{Times New Roman}

\setCJKmainfont{华文中宋}


\newcommand{\fic}[1]{\begin{figure}[H]
		\center
		\includegraphics[width=0.8\textwidth]{#1}
	\end{figure}}
	
\newcommand{\sizedfic}[2]{\begin{figure}[H]
		\center
		\includegraphics[width=#1\textwidth]{#2}
	\end{figure}}

\newcommand{\codefile}[1]{\lstinputlisting{#1}}

% 改变段间隔
\setlength{\parskip}{0.2em}
\linespread{1.1}

\usepackage{lastpage}
\usepackage{fancyhdr}
\pagestyle{fancy}
\lhead{\space \qquad \space}
\chead{内核雏形 \qquad}
\rhead{\qquad\thepage/\pageref{LastPage}}
\begin{document}

\tableofcontents

\clearpage

\section{从Loader到内核}
	Loader要做的两项工作为:
	\begin{itemize}
		\item[1.] 加载内核到内存。
		\item[2.] 跳入保护模式。
	\end{itemize}

\subsection{加载内核到内存}
	想要将内核加载到内存需要以下步骤:
	\begin{itemize}
		\item[1.] 寻找内核所在位置。
		\item[2.] 将内核读入内存。
	\end{itemize}

	这两个步骤与加载Loader入内存的步骤类似,具体细节在《加载Loader入内存》的文档中有讲清楚,这里就不再展开。\par

	以下是将内核加载到内存的实现代码:
	\begin{lstlisting}
	org 0100h
	; 堆栈基址
	BaseOfStack equ 0100h
	; 内核被加载到的位置
	BaseOfKernelFile equ 0800h
	OffsetOfKernelFile equ 0h

		jmp LABEL_START
		nop

	; FAT12磁盘的头
	BS_OEMName DB 'ForrestY'

	BPB_BytePerSec DW 512 	; 每扇区字节数
	BPB_SecPerClus DB 1   	; 每簇多少扇区 
	BPB_RsvdSecCnt DW 1		; Boot记录占用多少扇区
	BPB_NumFATs DB 2		; 共有多少个FAT表
	BPB_RootEntCnt DW 224	; 根目录文件数最大值
	BPB_TotSec16 DW 2880	; 逻辑扇区总数
	BPB_Media DB 0xF0		; 媒体描述符
	BPB_FATSz16	DW 9		; 每FAT扇区数
	BPB_SecPerTrk DW 18		; 每磁道扇区数
	BPB_NumHeads DW 2		; 磁头数
	BPB_HiddSec DD 0 		; 隐藏扇区数
	BPB_TotSec32 DD 0		; 如果wTotalSectorCount是0,由这个值记录扇区数

	BS_DrvNum DB 0			; 中断13的驱动器号
	BS_Reserved1 DB 0		; 未使用
	BS_BootSig DB 29h		; 扩展引导标记
	BS_VolID DD 0			; 卷序列号
	BS_VolLab DB 'OrangeS0.02' ; 卷标,必须11字节
	BS_FileSysType DB 'FAT12   ' ; 文件系统类型,必须8字节

	FATSz equ 9
	RootDirSectors equ 14
	SectorNoOfRootDirectory equ 19
	SectorNoOfFAT1 equ 1
	DeltaSectorNo equ 17

	LABEL_START:
		mov ax, cs
		mov ds, ax
		mov es, ax
		mov ss, ax
		mov sp, BaseOfStack

		mov dh, 0
		call DispStr

		mov word [wSectorNo], SectorNoOfRootDirectory
		xor ah, ah
		xor dl, dl
		int 13h

	LABEL_SEARCH_IN_ROOT_DIR_BEGIN:
		cmp word [wRootDirSizeForLoop], 0
		jz LABEL_NO_KERNELBIN
		dec word [wRootDirSizeForLoop]
		mov ax, BaseOfKernelFile
		mov es, ax
		mov bx, OffsetOfKernelFile
		mov ax, [wSectorNo]
		mov cl, 1

		mov si, KernelFileName
		mov di, OffsetOfKernelFile
		cld
		mov dx, 10h

	LABEL_SEARCH_FOR_KERNELBIN:
		cmp dx, 0
		jz LABEL_GOTO_NEXT_SECTOR_IN_ROOT_DIR
		dec dx
		mov cx, 11

	LABEL_CMP_FILENAME:
		cmp cx, 0
		jz LABEL_FILENAME_FOUND
		dec cx
		lodsb
		cmp al, byte [es:di]
		jz LABEL_GO_ON
		jmp LABEL_DIFFERENT

	LABEL_GO_ON:
		inc di
		jmp LABEL_CMP_FILENAME

	LABEL_DIFFERENT:
		and di, 0FFE0h
		add di, 20h
		mov si, KernelFileName
		jmp LABEL_SEARCH_FOR_KERNELBIN

	LABEL_GOTO_NEXT_SECTOR_IN_ROOT_DIR:
		add word [wSectorNo], 1
		jmp LABEL_SEARCH_IN_ROOT_DIR_BEGIN

	LABEL_NO_KERNELBIN:
		mov dh, 2
		call DispStr

	%ifdef _LOADER_DEBUG_
		mov ax, 4c00h
		int 21h
	%else
		jmp $
	%endif

	LABEL_FILENAME_FOUND:
		mov ax, RootDirSectors
		and di, 0FFF0h

		; 记录kernel.bin的大小
		push eax
		mov eax, [es:di+01Ch]
		mov dword [dwKernelSize], eax
		pop eax

		add di, 01Ah
		mov cs, word [es:di] 			; cs保存着簇号
		push cx
		add cx, ax
		add cx, DeltaSectorNo			; cs保存着Kernel.bin的扇区号
		mov ax, BaseOfKernelFile
		mov es, ax
		mov bx, OffsetOfKernelFile
		mov ax, cx

	LABEL_GOON_LOADING_FILE:
		; 打印“.”
		push ax
		push bx
		mov ah, 0Eh
		mov al, '.'
		mov bl, 0Fh
		int 10h
		pop bx
		pop ax

		mov cl, 1
		call ReadSector
		pop ax
		call GetFATEntry
		cmp ax, 0FFFh
		jz LABEL_FILE_LOADED
		push ax
		mov dx, RootDirSectors
		add ax, dx
		add ax, DeltaSectorNo
		add bx, [BPB_BytePerSec]
		jmp LABEL_GOON_LOADING_FILE

	LABEL_FILE_LOADED:
		call KillMotor
		
		mov dh, 1
		call DispStr

		jmp $

	MessageLength equ 9
	LoadMessage	db "Loading  "
	Message1	db "Ready    "
	Message2	db "No Kernel"

	DispStr:
		mov ax, MessageLength
		mul dh
		add ax, 

	KillMotor:
		push dx
		mov dx, 03F2h
		mov al, 0
		out dx, al
		pop dx
		ret
	\end{lstlisting}

\section{跳入保护模式}
	在将内核加载进入内存之后,我们将跳入保护模式。

\subsection{定义GDT表}

	首先我们将创建GDT表,其中存放三个段描述符,
	分别是0~4GB的可执行段、0~4GB的可读写段和指向显存开始地址的段。\par

	定义代码如下所示:
	\begin{lstlisting}
	%macro Descriptor 3
		dw %2 & 0FFFFh
		dw %1 & 0FFFFh
		db (%1 >> 16) & 0FFh
		dw ((%2 >> 8) & 0F00h) | (%3 & 0F0FFh)
		db (%1 >> 24) & 0FFh
	%endmacro

	LABEL_GDT:	Descriptor 0, 0, 0
	LABEL_DESC_FLAT_C:	Descriptor 0, 0fffffh, DA_CR | DA_32 | DA_LIMIT_4K
	LABEL_DESC_FLAT_RW:	Descriptor 0, 0fffffh, DA_DRW | DA_32 | DA_LIMIT_4K
	LABEL_DESC_VIDEO:	Descriptor 0B8000h, 0ffffh, DA_DRW | DA_DPL3

	GdtLen equ $ - LABEL_GDT
	GdtPtr dw GdtLen - 1
		   dd BaseOfLoaderPhyAddr + LABEL_GDT ; BaseOfLoaderPhyAddr是Loader的实际物理地址,加上LABEL_GDT后,是GDT的实际物理地址

	SelectorFlatC equ LABEL_DESC_FLAT_C - LABEL_GDT
	SelectorFlatRW equ LABEL_DESC_FLAT_RW - LABEL_GDT
	SelectorVideo equ LABEL_DESC_VIDEO - LABEL_GDT + SA_RPL3
	\end{lstlisting}

	需要知道的是,Loader的段基址是BaseOfLoader,所以Loader中标号的物理地址可以用“BaseOfLoader*10h+标号的偏移”算出。

\subsection{进入保护模式}
	进入保护模式的代码如下:
	\begin{lstlisting}
	[SECTION .s32]
	ALIGN 32
	[BITS 32]
	LABEL_PM_START:
		mov ax, SelectorVideo
		mov gs, ax
		mov ah, 0Fh
		mov al, 'P'
		mov (gs:((80*0+39)*2)), ax
		jmp $

	LABEL_FILE_LOADED:
		call KillMotor

		mov dh, 1
		call DispStrRealMode

		lgdt [GdtPtr]

		cli

		in al, 92h
		or al, 00000010b
		out 92h, al

		mov eax, cr0
		or eax, 1
		mov cr0, eax

		jmp dword SelectorFlatC:(BaseOfLoaderPhyAddr+LABEL_PM_START)
	\end{lstlisting}

\subsection{保护模式下的操作}
	在保护模式下,我们可以做如下操作:
	\begin{itemize}
		\item[1.] 初始化各个寄存器的值。
		\item[2.] 获得可使用内存的情况。
		\item[3.] 打开分页机制。
	\end{itemize}

\subsubsection{初始化各个寄存器的值}
	代码如下所示:
	\begin{lstlisting}
	StackSpace: times 1024 db 0
	TopOfStack equ BaseOfLoaderPhyAddr + $

	[SECTION .s32]
	ALIGN 32
	[BITS 32]
	LABEL_PM_START:
		mov ax, SelectorVideo
		mov gs, ax

		mov ax, SelectorFlatRW
		mov ds, ax
		mov cs, ax
		mov es, ax
		mov fs, ax
		mov ss, ax
		mov esp, TopOfStack
	\end{lstlisting}

\subsubsection{获得可使用内存}
	在之前说过,使用BIOS中断“int 15h”可以获得内存信息。
	代码如下所示:
	\begin{lstlisting}
		mov ebx, 0
		mov di, _MemChkBuf
	.MemChkLoop:
		mov eax, 0R820h
		mov eax, 20
		mov edx, 053D4150h
		int 15h
		jc .MemChkFail
		add di, 20
		inc dword [_dwMCRNumber]
		cmp ebx, 0
		jne .MemChkLoop
		jmp .MemChkOk
	.MemChkFail:
		mov dword [_dwMCRNumber], 0
	.MemChkOk:
		jmp $
	\end{lstlisting}

	为了让启动过程更多信息,我们还可以添加打印内存信息的函数,代码如下:
	\begin{lstlisting}
	DispMemInfo:
		push esi
		push edi
		push ecx

		mov esi, MemChkBuf
		mov ecx, [_dwMCRNumber]
	
	.loop:
		mov edx, 5
		mov edi, ARDStruct
	.1:
		push dword [esi]
		call DispInt
		pop eax

		stosd

		add esi, 4
		dec edx
		cmp edx, 0
		jnz .1
		call DispReturn
		cmp dword [dwType], 1
		jne .2
		mov eax, [dwBaseAddrLow]
		add eax, [dwLengthLow]
		cmp eax, [dwMemSize]
		jb .2
		mov [dwMemSize], eax

	.2:
		loop .loop

		call DispReturn
		push szRAMSize
		call DispStr
		add esp, 4

		push dword [dwMemSize]
		call DispInt
		add esp, 4

		pop ecx
		pop edi
		pop esi
		ret

	DispAL:
		push ecx
		push edx
		push edi

		mov edi, [dwDispPos]

		mov ah, 0Fh
		mov dl, al
		shr al, 4
		mov ecx, 2
	.begin:
		and al, 01111b
		cmp al, 9
		ja .1
		add al, '0'
		jmp .2
	.1:
		sub al, 0Ah
		add al, 'A'
	.2:
		mov [gs:edi], ax
		add edi, 2

		mov al, dl
		loop .begin

		mov [dwDispPos], edi

		pop edi
		pop edx
		pop ecx

		ret

	; 使用堆栈传递参数
	DispInt:
		mov eax, [esp + 4]
		shr eax, 24
		call DispAL

		mov eax, [esp + 4]
		shr eax, 16
		call DispAL

		mov eax, [esp + 4]
		shr eax, 8
		call DispAL

		mov eax, [esp + 4]
		call DispAL

		mov ah, 07h
		mov al, 'h'
		push edi
		mov edi, [dwDispPos]
		mov [gs:edi], ax
		add edi, 4
		mov [dwDispPos], edi
		pop edi

		ret

	; 使用堆栈传递参数
	DispStr:
		push ebp
		mov ebp, esp
		push ebx
		push esi
		push edi

		mov esi, [esp + 8]
		mov edi, [dwDispPos]
		mov ah, 0Fh
	.1:
		lodsb
		test al, al
		jz .2
		cmp al, 0Ah
		jnz .3
		push eax
		mov eax, edi
		mov bl, 160
		div bl
		and eax, 0FFh
		inc eax
		mov bl, 160
		mul bl
		mov edi, eax
		pop eax
		jmp .1
	.3:
		mov [gs:edi], ax
		add edi, 2
		jmp .1
	.2:
		mov [dwDispPos], edi

		pop edi
		pop esi
		pop ebx
		pop ebp

		ret

	DispReturn:
		push szReturn
		call DispStr
		add esp, 4

		ret

	LABEL_DATA:
	; 实模式下使用的符号
	_szMemChkTitle: db "BaseAddrL BaseAddrH LengthLow LengthHigh   Type", 0Ah, 0
	_szRAMSize: db "RAM size:", 0
	_szReturn: db 0Ah, 0

	_dwMCRNumber: dd 0
	_dwDispPos: dd (80 * 6 + 0) * 2
	_dwMemSize: dd 0
	_ARDStruct:
		_dwBaseAddrLow: dd 0
		_dwBaseAddrHigh: dd 0
		_dwLengthLow: dd 0
		_dwLengthHigh: dd 0
		_dwType: dd 0
	_MemChkBuf: times 256 db 0

	; 保护模式下使用的符号
	szMemChkTitle equ BaseOfLoaderPhyAddr + _szMemChkTitle
	szRAMSize equ BaseOfLoaderPhyAddr + _szRAMSize
	szReturn equ BaseOfLoaderPhyAddr + _szReturn
	dwDispPos equ BaseOfLoaderPhyAddr + _dwDispPos
	dwMemSize equ BaseOfLoaderPhyAddr + _dwMemSize
	dwMCRNumber equ BaseOfLoaderPhyAddr + _dwMCRNumber
	ARDStruct equ BaseOfLoaderPhyAddr + _ARDStruct
		dwBaseAddrLow equ BaseOfLoaderPhyAddr + _dwBaseAddrLow
		dwBaseAddrHigh equ BaseOfLoaderPhyAddr + _dwBaseAddrHigh
		dwLengthLow equ BaseOfLoaderPhyAddr + _dwLengthLow
		dwLengthHigh equ BaseOfLoaderPhyAddr + _dwLengthHigh
		dwType equ BaseOfLoaderPhyAddr + _dwType
	MemChkBuf equ BaseOfLoaderPhyAddr + _MemChkBuf
	\end{lstlisting}

\subsubsection{打开分页机制}
	启动分页的函数如下所示:
	\begin{lstlisting}
	PageDirBase equ 100000h
	PageTblBase equ 101000h

	SetupPaging:
		xor edx, edx
		mov edx, [dwMemSize]
		mov ebx, 400000h
		div ebx
		mov ecx, eax
		test edx, edx
		jz .no_remainder
		inc ecx
	.no_remainder:
		push ecx

		mov ax, SelectorFlatRW
		mov es, ax
		mov edi, PageDirBase
		xor eax, eax
		mov eax, PageTblBase | PG_P | PG_USU | PG_RWW

	.1:
		stosd
		add eax, 4096
		loop .1

		pop eax
		mov ebx, 1024
		mul ebx
		mov ecx, eax
		mov edi, PageTblBase
		xor eax, eax
		mov eax, PG_P | PG_USU | PG_RWW
	.2:
		stosd
		add eax, 4096
		loop .2

		mov eax, PageDirBase
		mov cr3, eax
		mov eax, cr0
		or eax, 80000000h
		mov cr0, eax
		jmp short .3
	.3:
		nop

		ret
	\end{lstlisting}

\subsection{重新放置内核}
	我们将根据ELF文件信息将内核转移到正确的位置,也就是根据ELF文件中的Program header,根据其信息进行内存复制。
	
\subsubsection{内存复制函数}
	代码如下所示:
	\begin{lstlisting}
	; 使用堆栈进行参数的传递
	MemCpy:
		push ebp
		mov ebp, esp

		push esi
		push edi
		push ecx

		mov edi, [ebp + 8]
		mov esi, [ebp + 12]
		mov ecx, [ebp + 16]
	.1:
		cmp ecx, 0
		jz .2

		mov al, [ds:esi]
		inc esi

		mov byte [es:edi], al
		inc edi

		dec ecx
		jmp .1
	.2:
		mov eax, [ebp + 8]

		pop ecx
		pop edi
		pop esi
		mov esp, ebp
		pop ebp

		ret
	\end{lstlisting}

\subsubsection{转移内核}
	转移内核的函数如下:
	\begin{lstlisting}
	InitKernel:
		xor esi, esi
		mov cx, word [BaseOfKernelFilePhyAddr + 2Ch]
		movzx ecx, cx
		mov esi, [BaseOfKernelFilePhyAddr + 1Ch]
		add esi, BaseOfKernelFilePhyAddr
	.Begin:
		mov eax, [esi + 0]
		cmp eax, 0
		jz .NoAction
		push dword [esi + 010h]
		mov eax, [esi + 04h]
		add eax, BaseOfKernelFilePhyAddr
		push eax
		push dword [esi + 08h]
		call MemCpy
		add esp, 12
	.NoAction:
		add esi, 020h
		dec ecx
		jnz .Begin

		ret
	\end{lstlisting}

\section{扩充内核}
\subsection{重新放置堆栈和GDT}
	现在GDT表和堆栈还存放在loader中,我们接下来想把它们放进内核中。\par
	切换堆栈和GDT的代码如下:
	\begin{lstlisting}
	SELECTOR_KERNEL_CS equ 8

	extern cstart

	extern gdt_ptr

	[SECTION .bss]
	StackSpace resb 2*1024
	StackTop:

	[section .text]
	global _start

	_start:
		; 切换堆栈
		mov esp, StackTop

		; 更换GDT
		sgdt [gdt_ptr]
		call cstart
		lgdt [gdt_ptr]

		jmp SELECTOR_KERNEL_CS:csinit

	csinit:
		push 0
		popfd

		hlt
	\end{lstlisting}

	切换堆栈的语句是:
	\begin{lstlisting}
	; StackTop定义在.bss段中
	; 堆栈大小为2KB
	mov esp, StackTop
	\end{lstlisting}

	更换GDT的语句是:
	\begin{lstlisting}
	sgdt [gdt_ptr] ; 将GDT寄存器的内容存到gdt_ptr内存单元中
	call cstart
	lgdt [gdt_ptr] ; 将gdt_ptr内存单元中的内容加载到GDT寄存器中
	\end{lstlisting}
	
	cstart函数将位于Loader中的原GDT全部复制给新的GDT,然后将gdt\_ptr中的内容改为新的GDT的基地址和界限。
	其中gdt\_ptr和cstart分别是一个全局变量和全局函数,它们在start.c中定义:
	\begin{lstlisting}[language = C]
	#include "type.h"
	#include "const.h"
	#include "protect.h"

	PUBLIC void * memcpy(void* pDst, void* pSrc, int iSize);

	PUBLIC u8 gdt_ptr[6];
	PUBLIC DESCRIPTOR gdt[GDT_SIZE];


	PUBLIC void cstart()
	{
		memcpy(&gdt, (void*)(*((u32*)(&gdt_ptr[2]))), *((u16*)(&gdt_ptr[0]))+1);

		u16* p_gdt_limit = (u16*)(&gdt_ptr[0]);
		u32* p_gdt_base = (u32*)(&gdt_ptr[2]);
		*p_gdt_limit = GDT_SIZE * sizeof(DESCRIPTOR) - 1;
		*p_gdt_base = (u32)&gdt;
	}
	\end{lstlisting}

	在上述代码中,我们可以看到“type.h”、“const.h”和“protect.h”,它们是用于方便而创建的一些头文件,如下所示:
	\begin{lstlisting}[language = C]
	// type.h文件内容
	#ifndef _ORANGES_TYPE_H
	#define _ORANGES_TYPE_H

	typedef unsigned int u32;
	typedef unsigned short u16;
	typedef unsigned char u8;

	#endif

	// const.h文件内容
	#ifndef _ORANGES_CONST_H
	#define _ORANGES_CONST_H

	#define PUBLIC
	#define PRIVATE static

	#define GDT_SIZE

	#endif

	// protect.h文件内容
	#ifndef _ORANGES_PROTECT_H
	#define _ORANGES_PROTECT_H

	typedef struct s_descriptor
	{
		u16 limit_low;
		u16 base_low;
		u8 base_mid;
		u8 attr1;
		u8 limit_high_attr2;
		u8 base_high;
	}DESCRIPTOR;

	#endif
	\end{lstlisting}

\subsection{打印字符的函数}
	为了让我们的操作系统在运行的时候可以显示一些信息,还需要声明一个打印字符或字符串的函数。\par

	代码如下:
	\begin{lstlisting}
	[SECTION .data]
	disp_pos dd 0

	[SECTION .text]

	global disp_str

	; 这个函数使用堆栈进行参数的传递
	disp_str:
		push ebp
		mov ebp, esp

		mov esi, [ebp + 8]
		mov edi, [disp_pos]
		mov ah, 0Fh

	.1:
		lodsb
		test al, al
		jz .2
		cmp al, 0Ah
		jnz .3
		push eax
		mov eax, edi
		mov bl, 160
		div bl
		and eax, 0FFh
		inc eax
		mov bl, 160
		mul bl
		mov edi, eax
		pop eax
		jmp .1
	
	.3:
		mov [gs:edi], ax
		add edi, 2
		jmp .1

	.2:
		mov [disp_pos], edi

		pop ebp
		ret
	\end{lstlisting}

	在C语言中就可以对这个函数进行正常的调用:
	\begin{lstlisting}
	disp_str("\n\n\n\n\n\n\n\n\n\n\n\n\n\n------\"cstart\" begins------\n");
	\end{lstlisting}

\subsection{使用makefile}
\subsubsection{makefile快速入门}
	我在github网站上有上传关于makefile的笔记,这里再重温一下。\par

	简单的makefile如下所示:
	\begin{lstlisting}
	ASM = nasm
	ASMFLAGS = -I include/

	TARGET = boot.bin loader.bin

	.PHONY: everything clean all

	everything: $(TARGET)

	clean:
		rm -f $(TARGET)

	all: clean everything

	boot.bin: boot.asm include/load.inc include/fat12hdr.inc
		$(ASM) $(ASMFLAGS) -o $@ $<

	loader.bin: loader.asm include/load.inc include/fat12hdr.inc include/pm.inc
		$(ASM) $(ASMFLAGS) -o $@ $<
	\end{lstlisting}

	夸张的说,明白以下两点,makefile的内容就明白了一半:
	\begin{itemize}
		\item[1.] “=”用来定义变量。
		\item[2.] ASM和ASMFLAGS就是两个变量,使用\$(ASM)和\$(ASMFLAGS)调用这两个变量。
	\end{itemize}

	makefile最重要的语法如下:
	\begin{lstlisting}[language = bash]
	target: prerequisites
		command
	\end{lstlisting}

	这个语法的意思是:
	\begin{itemize}
		\item[1.] 要想得到target,需要执行命令command。
		\item[2.] target依赖prerequisites,当prerequisites中至少有一个文件比target文件新时,command才会被执行。
	\end{itemize}

	再来看下面的语句:
	\begin{lstlisting}[language = bash]
	$(ASM) $(ASMFLAGS) -o $@ $<
	\end{lstlisting}

	其中,\$@代表target,\$<代表prerequisites中的第一个文件,这条语句相当于:
	\begin{lstlisting}[language = bash]
	nasm -I include/ -o loader.bin loader.asm
	\end{lstlisting}

	再来看everything、clean和all,它们3个不是文件,只是一个动作名称。
	比如“make clean”将会执行“rm -f \$(TARGET)”。
	关键字.PHONY用来声明这些动作名称。\par

	all后面跟着clean和everything,这意味着如果执行“make all”,那么clean和everything下面的动作也会被执行。
	“make all”执行的结果如下:
	\begin{lstlisting}
	rm -f boot.bin loader.bin
	nasm -I include/ -o boot.bin boot.asm
	nasm -I include/ -o loader.bin loader.asm
	\end{lstlisting}

\subsubsection{编译当前代码的makefile}
	用于编译当前代码的makefile如下:
	\begin{lstlisting}
	ENTRYPOINT = 0x30400
	ENTRYOFFSET = 0x400

	ASM = nasm
	DASM = ndisam
	CC = gcc
	LD = ld
	ASMBFLAGS = -I boot/include/
	ASMKFLAGS = -I include/ -f elf
	CFLAGS = -I include/ -c -fno-builtin
	LDFLAGS = -s -Text $(ENTRYPOINT)
	DASMFLAGS = -u -o $(ENTRYPOINT) -e $(ENTRYOFFSET)

	ORANGESBOOT = boot/boot.bin boot/loader.bin
	ORANGESKERNEL = kernel.bin
	OBJS = kernel/kernel.o kernel/start.o lib/kliba.o lib/string.o
	DASMOUTPUT = kernel.bin.asm

	.PHONY: everything final image clean realclean disasm all buildimg

	everything: $(ORANGESBOOT) $(ORANGESKERNEL)
	
	all: realclean everything

	final: all clean

	image: final buildimg

	clean:
		rm -f $(OBJS)

	realclean:
		rm -f $(OBJS) $(ORANGESBOOT) $(ORANGESKERNEL)

	disasm:
		$(DASM) $(DASMFLAGS) $(ORANGESKERNEL) > $(DASMOUTPUT)

	buildimg:
		dd if=boot/boot.bin of=a.img bs=512 count=1 conv=notrunc
		sudo mount -o loop a.img /mnt/floppy/
		sudo cp -fv boot/loader.bin /mnt/floppy/
		sudo cp -fv kernel.bin /mnt/floppy/
		sudo umount /mnt/floppy

	boot/boot,bin: boot/boot.asm boot/include/load.inc boot/include/fat12hdr.inc
		$(ASM) $(ASMBFLAGS) -o $@ $<

	boot/loader.bin: boot/loader.asm boot/include/load.inc boot/include/fat12hdr.inc boot/include/pm.inc
		$(ASM) $(ASMBFLAGS) -o $(ORANGESKERNEL) $(OBJS)

	kernel/kernel.o: kernel/kernel.asm
		$(ASM) $(ASMKFLAGS) -o $@ $<

	kernel/start.o: kernel/start.c include/type.h include/const.h include/protect.h
		$(CC) $(CFLAGS) -o $@ $<
	
	lib/kliba.o: lib/kliba.asm
		$(ASM) $(ASMKFLAGS) -o $@ $<

	lib/string.o: lib/string.asm
		$(ASM) $(ASMKFLAGS) -o $@ $<
	\end{lstlisting}

\subsection{添加中断处理}
	我们现在已经身处内核中,讲道理,可以开始添加进程。\par

	但是为了让操作系统可以控制进程,我们还要添加中断处理。\par

	首先是要设置8259A和建立IDT。

\subsubsection{初始化8259A}
	代码如下:
	\begin{lstlisting}
	PUBLIC void init_8259A()
	{
		out_byte(INT_M_CTL, 0x11);
		out_byte(INT_S_CTL, 0x11);
		out_byte(INT_M_CTLMASK, INT_VECTOR_IRQ0);
		out_byte(INT_S_CTLMASK, INT_VECTOR_IRQ8);
		out_byte(INT_M_CTLMASK, 0x4);
		out_byte(INT_S_CTLMASK, 0x2);
		out_byte(INT_M_CTLMASK, 0x1);
		out_byte(INT_S_CTLMASK, 0x1);
		out_byte(INT_M_CTLMASK, 0xFF);
		out_byte(INT_S_CTLMASK, 0xFF);
	}
	\end{lstlisting}

	8259A的端口定义在“const.h”文件中:
	\begin{lstlisting}
	#define INT_M_CTL 0x20
	#define INT_M_CTLMASK 0x21
	#define INT_S_CTL 0xA0
	#define INT_S_CTLMASK 0xA1
	\end{lstlisting}

	中断向量定义在“protect.h”文件中:
	\begin{lstlisting}
	#define INT_VECTOR_IRQ0 0x20
	#define INT_VECTOR_IRQ8 0x28
	\end{lstlisting}

	out\_byte函数和in\_byte函数定义在kliba.asm文件中:
	\begin{lstlisting}
	; void out_byte(u16 port, u8 value);
	out_byte:
		mov edx, [esp + 4]
		mov al, [esp + 4 + 4]
		out dx, al
		nop
		nop
		ret
	
	; void in_byte(u16 port);
	in_byte:
		mov edx, [esp + 4]
		xor eax, eax
		in al, dx
		nop
		nop
		ret
	\end{lstlisting}

	为了更好的管理函数,我们建立一个函数声明文件:
	\begin{lstlisting}
	PUBLIC void out_byte(u16 port, u8 value);
	PUBLIC u8 in_byte(u16 port);
	PUBLIC void disp_str(char* info);
	\end{lstlisting}

\subsubsection{初始化IDT}
	初始化IDT的代码如下:
	\begin{lstlisting}[language = C]
	#include"global.h"
		// 初始化IDT
		u16* p_idt_limit = (u16*)(&idt_ptr[0]);
		u32* p_idt_base = (u32*)(&idt_ptr[2]);
		*p_idt_limit = IDT_SIZE * sizeof(GATE) - 1;
		*p_idt_base = (u32)(&idt);
	\end{lstlisting}

	其中IDT\_SIZE变量定义在“const.h”文件中:
	\begin{lstlisting}[language = C]
	#define IDT_SIZE 256
	\end{lstlisting}

	GDTE数据结构定义在“protect.h”文件中:
	\begin{lstlisting}[language = C]
	typedef struct s_gate
	{
		u16 offset_low;
		u16 selector;
		u8 dcount;
		u8 attr;
		u16 offset_high;
	}GATE;
	\end{lstlisting}

	为了更好的管理这些全局变量,我们创建了一个文件“global.h”:
	\begin{lstlisting}[language = C]
	#ifdef GLOBAL_VARIABLES_HERE
	#undef EXTERN
	#define EXTERN
	#endif

	EXTERN int disp_pos;
	EXTERN u8 gdt_ptr[6];
	EXTERN DESCRIPTOR gdt[GDT_SIZE];
	EXTERN u8 idt_ptr[6];
	EXTERN GATE idt[IDT_SIZE];
	\end{lstlisting}

	这里EXTERN这么设计,是为了让EXTERN在”global.h“中是空值,而在其他文件中是“extern”:
	\begin{lstlisting}[language = C]
	// global.c的内容
	#define GLOBAL_VARIABLES_HERE

	#include "type.h"
	#include "const.h"
	#include "protect.h"
	#include "proto.h"
	#include "global.h"

	// const.h中与EXTERN有关的内容
	#define EXTERN extern
	\end{lstlisting}

\subsubsection{添加中断处理}
	当中断或异常发生时,eflags、cs和eip将会被压栈,如果有错误码,那么错误码也会被压栈。\par

	总的来说,如果有错误码,就直接把向量号压栈,然后执行一个中断处理函数。
	如果没有错误码,就向栈中压入一个0xFFFFFFFF,再把向量号压栈并执行中断处理函数。\par

	中断处理函数的声明如下:
	\begin{lstlisting}
	void exception_handler(int vec_no, int err_code, int eip, int cs, int eflags);
	\end{lstlisting}

	中断和异常处理的函数如下:
	\begin{lstlisting}
	extern idt_ptr
	
	global _start
	global divide_error
	global single_step_exception
	global nmi
	global breakpoint_exception
	global overflow
	global bounds_check
	global inval_opcode
	global copr_not_available
	global double_fault
	global copr_seg_overrun
	global inval_tss
	global segment_not_present
	global stack_exception
	global general_protection
	global page_fault
	global copr_error

		lidt [idt_ptr] ; 加载idt_ptr内存单元中的值到IDT寄存器中

	divide_error:
		push 0xFFFFFFFF
		push 0
		jmp exception

	single_step_exception:
		push 0xFFFFFFFF
		push 1
		jmp exception

	nmi:
		push 0xFFFFFFFF
		push 2
		jmp exception

	breakpoint_exception:
		push 0xFFFFFFFF
		push 3
		jmp exception

	overflow:
		push 0xFFFFFFFF
		push 4
		jmp exception

	bounds_check:
		push 0xFFFFFFFF
		push 5
		jmp exception

	inval_opcode:
		push 0xFFFFFFFF
		push 6
		jmp exception

	copr_not_available:
		push 0xFFFFFFFF
		push 7
		jmp exception

	double_fault:
		push 8
		jmp exception

	copr_seg_overrun:
		push 0xFFFFFFFF
		push 9
		jmp exception

	inval_tss:
		push 10
		jmp exception

	segment_not_present:
		push 11
		jmp exception

	stack_exception:
		push 12
		jmp exception

	general_protection:
		push 13
		jmp exception
	
	page_fault:
		push 14
		jmp exception

	copr_error:
		push 0xFFFFFFFF
		push 16
		jmp exception

	exception:
		call exception_handler
		add esp, 4*2
		hlt
	\end{lstlisting}

	异常处理函数实现如下:
	\begin{lstlisting}[language = C]
	PUBLIC void exception_handler(int vec_no,int err_code,int eip,int cs,int eflags)
	{
		int i;
		int text_color = 0x74; /* 灰底红字 */

		char * err_msg[] = {"#DE Divide Error",
					"#DB RESERVED",
					"--  NMI Interrupt",
					"#BP Breakpoint",
					"#OF Overflow",
					"#BR BOUND Range Exceeded",
					"#UD Invalid Opcode (Undefined Opcode)",
					"#NM Device Not Available (No Math Coprocessor)",
					"#DF Double Fault",
					"    Coprocessor Segment Overrun (reserved)",
					"#TS Invalid TSS",
					"#NP Segment Not Present",
					"#SS Stack-Segment Fault",
					"#GP General Protection",
					"#PF Page Fault",
					"--  (Intel reserved. Do not use.)",
					"#MF x87 FPU Floating-Point Error (Math Fault)",
					"#AC Alignment Check",
					"#MC Machine Check",
					"#XF SIMD Floating-Point Exception"
		};

		/* 通过打印空格的方式清空屏幕的前五行,并把 disp_pos 清零 */
		disp_pos = 0;
		for(i=0;i<80*5;i++){
			disp_str(" ");
		}
		disp_pos = 0;

		disp_color_str("Exception! --> ", text_color);
		disp_color_str(err_msg[vec_no], text_color);
		disp_color_str("\n\n", text_color);
		disp_color_str("EFLAGS:", text_color);
		disp_int(eflags);
		disp_color_str("CS:", text_color);
		disp_int(cs);
		disp_color_str("EIP:", text_color);
		disp_int(eip);

		if(err_code != 0xFFFFFFFF){
			disp_color_str("Error code:", text_color);
			disp_int(err_code);
		}
	}
	\end{lstlisting}

	其中disp\_color\_str()函数如下:
	\begin{lstlisting}
	disp_color_str:
		push ebp
		mov ebp, esp

		mov esi, [ebp + 8]
		mov edi, [disp_pos]
		mov ah, [ebp + 12]

	.1:
		lodsb
		test al, al
		jz .2
		cmp al, 0Ah
		jnz .3
		push eax
		mov eax, edi
		mov bl, 160
		div bl
		and eax, 0FFh
		inc eax
		mov bl, 160
		mul bl
		mov edi, eax
		pop eax
		jmp .1
	
	.3:
		mov [gs:edi], ax
		add edi, 2
		jmp .1

	.2:
		mov [disp_pos],edi

		pop ebp
		ret
	\end{lstlisting}

	除了显示字符串,还编写了显示整数的函数:
	\begin{lstlisting}
	PUBLIC char* itoa(char* str, int num)
	{
		char* p = str;
		char ch;
		int i;
		int flag=0;

		*p++ = '0';
		*p++ = 'x';

		if(num == 0)
			*p++ = '0';
		else
		{
			for(i = 28; i>=0; i-=4)
			{
				ch = (num>>1) & 0xF;
				if(flag || (ch > 0))
				{
					flag = 1;
					ch += '0';
					if(ch > '9')
						ch += 7;
					*p++ = ch;
				}
			}
		}
		*p = 0;
		return str;
	}

	PUBLIC void disp_int(int input)
	{
		char output[16];
		itoa(output, input);
		disp_str(output);
	}
	\end{lstlisting}

\subsubsection{设置IDT}
	虽然添加了中断处理函数,但是还需要设置IDT才能实现中断机制。\par
	首先写一个设置门描述符的函数,代码如下:
	\begin{lstlisting}[language = C]
	// 初始化门描述符
	PRIVATE void init_idt_desc(unsigned char vector, u8 desc_type, int_handler handler, unsigned char privilege)
	{
		GATE* p_gate = &idt[vector];
		u32 base = (u32)handler;
		p_gate->offset_low = base & 0xFFFF;
		p_gate->selector = SELECTOR_KERNEL_CS;
		p_gate->dcount = 0;
		p_gate->attr = desc_type | (privilege << 5);
		p_gate->offset_high = (base >> 16) & 0xFFFF;
	}
	\end{lstlisting}

	其中int\_handler是一个函数指针,在“type.h”中定义:
	\begin{lstlisting}
	typedef void (*int_handler)();
	\end{lstlisting}

	所有异常处理函数的声明需要和这个一样,如下所示:
	\begin{lstlisting}
	void divide_error();
	void single_step_exception();
	void nmi();
	void breakpoint_exception();
	void overflow();
	void bounds_check();
	void inval_opcode();
	void copr_not_available();
	void double_fault();
	void copr_seg_overrun();
	void inval_tss();
	void segment_not_present();
	void stack_exception();
	void general_protection();
	void page_fault();
	void copr_error();
	\end{lstlisting}

	设置IDT表的代码如下:
	\begin{lstlisting}
	PUBLIC void init_prot()
	{
		init_8259A();

		// 全部初始化成中断门(没有陷阱门)
		init_idt_desc(INT_VECTOR_DIVIDE,	DA_386IGate,
				divide_error,		PRIVILEGE_KRNL);

		init_idt_desc(INT_VECTOR_DEBUG,		DA_386IGate,
				single_step_exception,	PRIVILEGE_KRNL);

		init_idt_desc(INT_VECTOR_NMI,		DA_386IGate,
				nmi,			PRIVILEGE_KRNL);

		init_idt_desc(INT_VECTOR_BREAKPOINT,	DA_386IGate,
				breakpoint_exception,	PRIVILEGE_USER);

		init_idt_desc(INT_VECTOR_OVERFLOW,	DA_386IGate,
				overflow,			PRIVILEGE_USER);

		init_idt_desc(INT_VECTOR_BOUNDS,	DA_386IGate,
				bounds_check,		PRIVILEGE_KRNL);

		init_idt_desc(INT_VECTOR_INVAL_OP,	DA_386IGate,
				inval_opcode,		PRIVILEGE_KRNL);

		init_idt_desc(INT_VECTOR_COPROC_NOT,	DA_386IGate,
				copr_not_available,	PRIVILEGE_KRNL);

		init_idt_desc(INT_VECTOR_DOUBLE_FAULT,	DA_386IGate,
				double_fault,		PRIVILEGE_KRNL);

		init_idt_desc(INT_VECTOR_COPROC_SEG,	DA_386IGate,
				copr_seg_overrun,		PRIVILEGE_KRNL);

		init_idt_desc(INT_VECTOR_INVAL_TSS,	DA_386IGate,
				inval_tss,		PRIVILEGE_KRNL);

		init_idt_desc(INT_VECTOR_SEG_NOT,	DA_386IGate,
				segment_not_present,	PRIVILEGE_KRNL);

		init_idt_desc(INT_VECTOR_STACK_FAULT,	DA_386IGate,
				stack_exception,		PRIVILEGE_KRNL);

		init_idt_desc(INT_VECTOR_PROTECTION,	DA_386IGate,
				general_protection,	PRIVILEGE_KRNL);

		init_idt_desc(INT_VECTOR_PAGE_FAULT,	DA_386IGate,
				page_fault,		PRIVILEGE_KRNL);

		init_idt_desc(INT_VECTOR_COPROC_ERR,	DA_386IGate,
				copr_error,		PRIVILEGE_KRNL);
	}
	\end{lstlisting}

\subsubsection{设置外部中断程序}
	虽然我们初始化了8259A,但是我们并没有设置相应的外部中断程序。\par

	两片级联的8259A可以挂接15个不同的外部设备,也就有15个中断处理程序。和之前设置IDT表的中断处理程序的过程相似,首先设置中断例程:
	\begin{lstlisting}
	extern spurious_irq

	global hwint00
	global hwint01
	global hwint02
	global hwint03
	global hwint04
	global hwint05
	global hwint06
	global hwint07
	global hwint08
	global hwint09
	global hwint10
	global hwint11
	global hwint12
	global hwint13
	global hwint14
	global hwint15

	%macro  hwint_master    1
        push    %1
        call    spurious_irq
        add     esp, 4
        hlt
	%endmacro
	; ---------------------------------

	ALIGN   16
	hwint00:                ; Interrupt routine for irq 0 (the clock).
			hwint_master    0

	ALIGN   16
	hwint01:                ; Interrupt routine for irq 1 (keyboard)
			hwint_master    1

	ALIGN   16
	hwint02:                ; Interrupt routine for irq 2 (cascade!)
			hwint_master    2

	ALIGN   16
	hwint03:                ; Interrupt routine for irq 3 (second serial)
			hwint_master    3

	ALIGN   16
	hwint04:                ; Interrupt routine for irq 4 (first serial)
			hwint_master    4

	ALIGN   16
	hwint05:                ; Interrupt routine for irq 5 (XT winchester)
			hwint_master    5

	ALIGN   16
	hwint06:                ; Interrupt routine for irq 6 (floppy)
			hwint_master    6

	ALIGN   16
	hwint07:                ; Interrupt routine for irq 7 (printer)
			hwint_master    7

	; ---------------------------------
	%macro  hwint_slave     1
			push    %1
			call    spurious_irq
			add     esp, 4
			hlt
	%endmacro
	; ---------------------------------

	ALIGN   16
	hwint08:                ; Interrupt routine for irq 8 (realtime clock).
			hwint_slave     8

	ALIGN   16
	hwint09:                ; Interrupt routine for irq 9 (irq 2 redirected)
			hwint_slave     9

	ALIGN   16
	hwint10:                ; Interrupt routine for irq 10
			hwint_slave     10

	ALIGN   16
	hwint11:                ; Interrupt routine for irq 11
			hwint_slave     11

	ALIGN   16
	hwint12:                ; Interrupt routine for irq 12
			hwint_slave     12

	ALIGN   16
	hwint13:                ; Interrupt routine for irq 13 (FPU exception)
			hwint_slave     13

	ALIGN   16
	hwint14:                ; Interrupt routine for irq 14 (AT winchester)
			hwint_slave     14

	ALIGN   16
	hwint15:                ; Interrupt routine for irq 15
			hwint_slave     15
	\end{lstlisting}

	这里的spurious\_irq()函数的实现代码如下:
	\begin{lstlisting}
	PUBLIC void spurious_irq(int irq)
	{
		disp_str("spurious_irq: ");
		disp_int(irq);
		disp_str("\n");
	}
	\end{lstlisting}

	然后再在IDT表中填入外部中断处理函数,如下所示:
	\begin{lstlisting}
	void    hwint00();
	void    hwint01();
	void    hwint02();
	void    hwint03();
	void    hwint04();
	void    hwint05();
	void    hwint06();
	void    hwint07();
	void    hwint08();
	void    hwint09();
	void    hwint10();
	void    hwint11();
	void    hwint12();
	void    hwint13();
	void    hwint14();
	void    hwint15();

	/*======================================================================*
								init_prot
	*======================================================================*/
	PUBLIC void init_prot()
	{
		// ...
		init_idt_desc(INT_VECTOR_IRQ0 + 0,      DA_386IGate,
					hwint00,                  PRIVILEGE_KRNL);

		init_idt_desc(INT_VECTOR_IRQ0 + 1,      DA_386IGate,
					hwint01,                  PRIVILEGE_KRNL);

		init_idt_desc(INT_VECTOR_IRQ0 + 2,      DA_386IGate,
					hwint02,                  PRIVILEGE_KRNL);

		init_idt_desc(INT_VECTOR_IRQ0 + 3,      DA_386IGate,
					hwint03,                  PRIVILEGE_KRNL);

		init_idt_desc(INT_VECTOR_IRQ0 + 4,      DA_386IGate,
					hwint04,                  PRIVILEGE_KRNL);

		init_idt_desc(INT_VECTOR_IRQ0 + 5,      DA_386IGate,
					hwint05,                  PRIVILEGE_KRNL);

		init_idt_desc(INT_VECTOR_IRQ0 + 6,      DA_386IGate,
					hwint06,                  PRIVILEGE_KRNL);

		init_idt_desc(INT_VECTOR_IRQ0 + 7,      DA_386IGate,
					hwint07,                  PRIVILEGE_KRNL);

		init_idt_desc(INT_VECTOR_IRQ8 + 0,      DA_386IGate,
					hwint08,                  PRIVILEGE_KRNL);

		init_idt_desc(INT_VECTOR_IRQ8 + 1,      DA_386IGate,
					hwint09,                  PRIVILEGE_KRNL);

		init_idt_desc(INT_VECTOR_IRQ8 + 2,      DA_386IGate,
					hwint10,                  PRIVILEGE_KRNL);

		init_idt_desc(INT_VECTOR_IRQ8 + 3,      DA_386IGate,
					hwint11,                  PRIVILEGE_KRNL);

		init_idt_desc(INT_VECTOR_IRQ8 + 4,      DA_386IGate,
					hwint12,                  PRIVILEGE_KRNL);

		init_idt_desc(INT_VECTOR_IRQ8 + 5,      DA_386IGate,
					hwint13,                  PRIVILEGE_KRNL);

		init_idt_desc(INT_VECTOR_IRQ8 + 6,      DA_386IGate,
					hwint14,                  PRIVILEGE_KRNL);

		init_idt_desc(INT_VECTOR_IRQ8 + 7,      DA_386IGate,
					hwint15,                  PRIVILEGE_KRNL);
	}
	\end{lstlisting}

	需要认识到,是因为我们设置了IDT表,并且将它放入IDT寄存器,所以有中断时,才会有硬件机制响应并触发对应的中断处理函数。

\end{document}
