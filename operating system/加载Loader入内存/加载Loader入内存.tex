% !TeX spellcheck = en_US
%% 字体:方正静蕾简体
%%		 方正粗宋
\documentclass[a4paper,left=2.5cm,right=2.5cm,11pt]{article}

\usepackage[utf8]{inputenc}
\usepackage{fontspec}
\usepackage{cite}
\usepackage{xeCJK}
\usepackage{indentfirst}
\usepackage{titlesec}
\usepackage{longtable}
\usepackage{graphicx}
\usepackage{float}
\usepackage{rotating}
\usepackage{subfigure}
\usepackage{tabu}
\usepackage{amsmath}
\usepackage{setspace}
\usepackage{amsfonts}
\usepackage{appendix}
\usepackage{listings}
\usepackage{xcolor}
\usepackage{geometry}
\setcounter{secnumdepth}{4}
\usepackage{mhchem}
\usepackage{multirow}
\usepackage{extarrows}
\usepackage{hyperref}
\titleformat*{\section}{\LARGE}
\renewcommand\refname{参考文献}
\renewcommand{\abstractname}{\sihao \cjkfzcs 摘{  }要}
%\titleformat{\chapter}{\centering\bfseries\huge\wryh}{}{0.7em}{}{}
%\titleformat{\section}{\LARGE\bf}{\thesection}{1em}{}{}
\titleformat{\subsection}{\Large\bfseries}{\thesubsection}{1em}{}{}
\titleformat{\subsubsection}{\large\bfseries}{\thesubsubsection}{1em}{}{}
\renewcommand{\contentsname}{{\cjkfzcs \centerline{目{  } 录}}}
\setCJKfamilyfont{cjkhwxk}{STXingkai}
\setCJKfamilyfont{cjkfzcs}{STSongti-SC-Regular}
% \setCJKfamilyfont{cjkhwxk}{华文行楷}
% \setCJKfamilyfont{cjkfzcs}{方正粗宋简体}
\newcommand*{\cjkfzcs}{\CJKfamily{cjkfzcs}}
\newcommand*{\cjkhwxk}{\CJKfamily{cjkhwxk}}
\newfontfamily\wryh{Microsoft YaHei}
\newfontfamily\hwzs{STZhongsong}
\newfontfamily\hwst{STSong}
\newfontfamily\hwfs{STFangsong}
\newfontfamily\jljt{MicrosoftYaHei}
\newfontfamily\hwxk{STXingkai}
% \newfontfamily\hwzs{华文中宋}
% \newfontfamily\hwst{华文宋体}
% \newfontfamily\hwfs{华文仿宋}
% \newfontfamily\jljt{方正静蕾简体}
% \newfontfamily\hwxk{华文行楷}
\newcommand{\verylarge}{\fontsize{60pt}{\baselineskip}\selectfont}  
\newcommand{\chuhao}{\fontsize{44.9pt}{\baselineskip}\selectfont}  
\newcommand{\xiaochu}{\fontsize{38.5pt}{\baselineskip}\selectfont}  
\newcommand{\yihao}{\fontsize{27.8pt}{\baselineskip}\selectfont}  
\newcommand{\xiaoyi}{\fontsize{25.7pt}{\baselineskip}\selectfont}  
\newcommand{\erhao}{\fontsize{23.5pt}{\baselineskip}\selectfont}  
\newcommand{\xiaoerhao}{\fontsize{19.3pt}{\baselineskip}\selectfont} 
\newcommand{\sihao}{\fontsize{14pt}{\baselineskip}\selectfont}      % 字号设置  
\newcommand{\xiaosihao}{\fontsize{12pt}{\baselineskip}\selectfont}  % 字号设置  
\newcommand{\wuhao}{\fontsize{10.5pt}{\baselineskip}\selectfont}    % 字号设置  
\newcommand{\xiaowuhao}{\fontsize{9pt}{\baselineskip}\selectfont}   % 字号设置  
\newcommand{\liuhao}{\fontsize{7.875pt}{\baselineskip}\selectfont}  % 字号设置  
\newcommand{\qihao}{\fontsize{5.25pt}{\baselineskip}\selectfont}    % 字号设置 

\usepackage{diagbox}
\usepackage{multirow}
\boldmath
\XeTeXlinebreaklocale "zh"
\XeTeXlinebreakskip = 0pt plus 1pt minus 0.1pt
\definecolor{cred}{rgb}{0.8,0.8,0.8}
\definecolor{cgreen}{rgb}{0,0.3,0}
\definecolor{cpurple}{rgb}{0.5,0,0.35}
\definecolor{cdocblue}{rgb}{0,0,0.3}
\definecolor{cdark}{rgb}{0.95,1.0,1.0}
\lstset{
	language=[x86masm]Assembler,
	numbers=left,
	numberstyle=\tiny\color{black},
	showspaces=false,
	showstringspaces=false,
	basicstyle=\scriptsize,
	keywordstyle=\color{purple},
	commentstyle=\itshape\color{cgreen},
	stringstyle=\color{blue},
	frame=lines,
	% escapeinside=``,
	extendedchars=true, 
	xleftmargin=1em,
	xrightmargin=1em, 
	backgroundcolor=\color{cred},
	aboveskip=1em,
	breaklines=true,
	tabsize=4
} 

\newfontfamily{\consolas}{Consolas}
\newfontfamily{\monaco}{Monaco}
\setmonofont[Mapping={}]{Consolas}	%英文引号之类的正常显示,相当于设置英文字体
\setsansfont{Consolas} %设置英文字体 Monaco, Consolas,  Fantasque Sans Mono
\setmainfont{Times New Roman}

\setCJKmainfont{华文中宋}


\newcommand{\fic}[1]{\begin{figure}[H]
		\center
		\includegraphics[width=0.8\textwidth]{#1}
	\end{figure}}
	
\newcommand{\sizedfic}[2]{\begin{figure}[H]
		\center
		\includegraphics[width=#1\textwidth]{#2}
	\end{figure}}

\newcommand{\codefile}[1]{\lstinputlisting{#1}}

% 改变段间隔
\setlength{\parskip}{0.2em}
\linespread{1.1}

\usepackage{lastpage}
\usepackage{fancyhdr}
\pagestyle{fancy}
\lhead{\space \qquad \space}
\chead{加载Loader入内存 \qquad}
\rhead{\qquad\thepage/\pageref{LastPage}}
\begin{document}

\tableofcontents

\clearpage

\section{加载Loader入内存}
	如果要加载一个Loader入内存,就需要读软盘,所以首先了解一下软盘的结构。

\subsection{FAT12格式}
	1.44MB软盘上使用的是FAT12文件系统,软盘由引导扇区、FAT1、FAT2、根目录区和数据区组成。
	如下图所示:
	\fic{3.png}

	我们想将Loader读入内存,首先要知道Loader在软盘中的位置。\par

	Loader一般存放在数据区中,而数据区信息存放在FAT2后面的根目录区中。\par

\subsubsection{根目录区条目}
	根目录区开始的扇区号为19,由若干个目录条目组成。每一个条目占用32字节,结构如下图所示:
	\fic{4.png}

	由上图可以看出,每个条目定义了文件的名称、属性大小、日期以及此条目对应的数据区中的开始簇号。
	因为一个簇号只包含一个扇区,所以条目记录着对应的扇区号。
	需要注意的是,数据区的第一个簇的簇号是2。如果DIR\_FstClus为2,说明该文件的数据位于数据区的第一个簇。\par

	随后还要算出根目录区所占的扇区数。一个扇区为512字节,一个条目为32字节,
	只要先算出根目录区总字节数,再除以512字节,就能得到根目录区所占的扇区数。如果不能整除,需要将扇区数加1。\par

	直到数据区开始扇区号,又知道文件在数据区中的开始簇号,就能得到总的扇区号。

\subsubsection{FAT表}
	文件有时候会大于512字节,所以会占用超过512字节,所以我们需要FAT表来找到所有的簇。\par

	FAT表有两个,FAT2是FAT1表的备份。FAT1的开始扇区号是1,由FAT项组成。
	每个FAT项占12位,包含一个字节和另一个字节的一半。\par

	当根目录区中条目记录的是数据区中开始的第3个簇,那么就对应第3个FAT项。
	如果这个FAT项值不为0xFFF而是0x008,那么说明文件还未结束,文件的下一部分内容在数据区的第8个簇中。
	同时我们需要再找到第8个FAT项,依次查找下去,直到FAT项的值为0xFFF,则代表着文件结束。\par

	需要注意的是,在实际编写代码的过程中,要考虑一个FAT项跨越两个扇区的情况。

\subsection{读软盘函数}
	如果要加载一个文件入内存,就需要读软盘,此时就需要用到BIOS中断int 13h,它的用法如下所示:
	\fic{1.png}

	从上图可以看到,int 13h需要用al指明要读的扇区数,ch指定磁道号,cl指定起始扇区号,
	dh指定磁头号。然后用es:bx指定数据缓冲区。\par

	对于1.44MB软盘而言,一共有两个磁头号(磁头号0和1),每一面有80个磁道(0~79),每个磁道有18个扇区(1~18)。
	需要注意的是,软盘两面的磁道号是交错排序的,磁头号为0的磁道号为偶数,磁头号为1的磁道号为奇数。\par

	根据上面的已知条件,当我们拥有Loader所在的扇区号时,就可以算出磁头号、磁道号和起始扇区号。
	因为一个磁道有18个扇区,所以扇区号除以18就得到它所在的磁道号,同时余数就是起始扇区号。
	需要注意的是,因为起始扇区号是从1开始的,所以还需要加1。
	而得到的磁道号也需要处理,因为这个磁道号是软盘的磁道号,不是软盘某一面上的磁道号。
	通过判断磁道号的奇偶就能得到磁头号,然后将总体的磁道号除以2,就能得到软盘某一面上的磁道号。
	计算方法如下图所示:
	\fic{2.png}

	根据上述的描述,我们就可以写一个读取软盘的函数了。
	函数使用堆栈暂时保存数据,所以还会建立一个堆栈。
	函数使用ax传递要读取的扇区号,使用cl传递要读取的扇区数。
	代码如下所示:
	\begin{lstlisting}
	BaseOfStack equ 07c00h ; 相当于0~07c00h都可以当作堆栈

	BPB_SecPerTrk DB 18

	; ...
		; 初始化堆栈
		mov ax, cs
		mov ds, ax
		mov es, ax
		mov ss, ax
		mov sp, BaseOfStack

	ReadSector:
		push bp
		mov bp, sp
		; 这里想用堆栈暂时保存要读取的扇区数
		; 但是因为push和pop默认处理的是字单元
		; 所以需要通过mov指令来另外处理
		sub esp, 2
		mov byte [bp-2], cl ; cl存放着要读取的扇区数
		push bx				; 保存bx

		mov bl, [BPB_SecPerTrk]
		div bl				; ax存放着要读取的扇区号,bl为18,指令执行后ah存放余数,al存放商
		inc ah				; 得到起始扇区号
		mov cl, ah			; cl指定起始扇区号
		mov dh, al			
		and dh, 1			; 得到磁头号
		shr al, 1			; 得到扇区号
		mov ch, al

		pop					; 恢复bx
		mov dl, [BS_DrvNum]	; 指定驱动器号
	.GoOnReading:
		mov ah, 2
		mov al, byte [bp-2]
		int 13h
		jc .GoOnReading

		add esp, 2
		pop bp

		ret
	\end{lstlisting}

\subsection{寻找Loader}
	寻找Loader只需要遍历根目录区中的所有扇区,然后将每一个扇区加载入内存。
	然后从中寻找文件名为Loader.bin的条目,直到找到为止。
	代码如下:
	\begin{lstlisting}
	BaseOfLoader equ 09000h
	OffsetOfLoader equ 0100h
	RootDirSectors equ 14				;根目录区占用的扇区数
	SectorNoOfRootDirectory equ 19		; 根目录区的开始扇区号

		; 首先复位软驱,通过int 13h中断实现
		xor ah, ah
		xor dl, dl
		int 13h

		mov word [wSectorNo], SectorNoOfRootDirectory
	
	LABEL_SEARCH_IN_ROOT_DIR_BEGIN:
		cmp word [wRootDirSizeForLoop], 0 ; 判断根目录区是否已经读完
		jz LABEL_NO_LOADERBIN
		dec word [wRootDirSizeForLoop]
		mov ax, BaseOfLoader
		mov es, ax
		mov bx, OffsetOfLoader 	; es:bx指向数据缓冲区
		mov ax, [wSectorNo] 	; wSectorNo存放着扇区号
		mov cl, 1				; cl存放着要读取的扇区数
		call ReadSector

		mov si, LoaderFileName
		mov di, OffsetOfLoader
		cld
		mov dx, 10h
	
	LABEL_SEARCH_FOR_LOADERBIN:
		cmp dx, 0
		jz LABEL_GOTO_NEXT_SECTOR_IN_ROOT_DIR
		dec dx
		mov cx, 11
	
	LABEL_CMP_FILENAME:
		cmp cx, 0
		jz LABEL_FILENAME_FOUND
		dev cx
		lodsb		; si加1
		cmp al, byte [es:di]
		jz LABEL_GO_ON
		jmp LABEL_DIFFERENT
	
	LABEL_GO_ON:
		inc di		; di加1
		jmp LABEL_CMP_FILENAME
	
	LABEL_DIFFERENT:
		; 跳到下一个目录条目
		and di, 0FFE0h
		and di, 20h
		mov si, LoaderFileName
		jmp LABEL_SEARCH_FOR_LOADERBIN
	
	LABEL_GOTO_NEXT_SECTOR_IN_ROOT_DIR:
		add wor [wSectorNo], 1
		jmp LABEL_SEARCH_IN_ROOT_DIR_BEGIN

	LABEL_NO_LOADERBIN:
		mov dh, 2
		call DispStr
	
	%ifdef _BOOT_DEBUG
		mov ax, 4c00h
		int 21h
	%else
		jmp $
	%endif

	LABEL_FILENAME_FOUND:
		jmp $

	wRootDirSizeForLoop dw RootDirSectors
	wSectorNo dw 0
	LoaderFileName db "LOADER  BIN",0	; 最后加个0用于判断字符串是否结束
	\end{lstlisting}

	需要注意的是,我们上述代码还调用了DispStr函数。
	这个函数是自己定义的,通过dh传递函数参数,用于选择显示哪条信息。
	具体实现代码如下所示:
	\begin{lstlisting}
	MessageLength equ 9
	BootMessage db "Booting  "
	Message1	db "Ready.   "
	Message2	db "NO LOADER"

	DispStr:
		mov ax, MessageLength
		mul dh
		add ax, BootMessage
		mov bp, ax
		mov ax, ds
		mov es, ax
		mov cx, MessageLength
		mov ax, 01301h
		mob bx, 0007h
		mov dl, 0
		int 10h
		ret
	\end{lstlisting}

\subsection{计算FAT项的值}
	在上一节中,我们找到了Loader在根目录区中的条目,然后就可以根据这个条目得到Loader在数据区中对应的开始簇号。
	同时我们还可以根据这个簇号得到FAT表中的FAT项,从而找到Loader占用的其余所有扇区。\par

	在将Loader装入内存的BaseOfLoader:OffsetOfLoader之前,我们先写一个函数,用于求FAT项的值。
	函数使用ax传递扇区号,同时使用ax存放返回的FAT项的值。
	实现代码如下:
	\begin{lstlisting}
	SectorNoOfFAT1 equ 1

	GetFATEntry:
		push es
		push bx
		push ax
		mov ax, BaseOfLoader	; BaseOfLoader值为0900h
		sub ax, 0100h			
		mov es, ax				; 改变es后,数据缓冲区为0800h:bx,FAT项将存入此处
		pop ax

		mov byte [bOdd], 0
		mov bx, 3
		mul bx
		mob bx, 2
		div bx			; ax存放着FAT项的偏移字节,dx存放着余数
		cmp dx, 0
		jz LABEL_EVEN
		mov byte [bOdd], 1

	LABEL_EVEN:
		xor dx, dx
		mov bx, [BPB_SecPerSec]
		div bx		; ax存放着FAT项在FAT1中的开始扇区号,dx存放着FAT项在扇区内的偏移
		push dx
		mov bx, 0	; bx为0,FAT项将被读入0800h:0处
		add ax, SectorNoOfFAT1
		mov cl, 2	; 一次读两个扇区,防止FAT跨扇区
		call ReadSector

		pop dx
		add bx, dx	; bx存放着FAT项在扇区的偏移字节
		mov ax, [es:bx]
		cmp byte [bOdd], 1
		jnz LABEL_EVEN_2
		shr ax, 4
	LABEL_EVEN_2:
		and ax, 0FFFh

	LABEL_GET_FAT_ENTRY_OK:
		pop bx
		pop es
		ret

	bOdd DB 0
	\end{lstlisting}

\subsection{加载Loader入内存}
	有了前面的准备,我们现在就可以开始直接加载Loader进入内存了,实现代码如下:
	\begin{lstlisting}
	LABEL_FILENAME_FOUND:
		mov ax, RootDirSectors
		and di, 0FFE0h
		add di, 01Ah
		mov cx, word [es:di]	; cx存放着FAT项的序号
		push cx
		add cx, ax				; ax存放着RootDirSectors,也就是根目录区占用的扇区数
		add cx, DeltaSectorNo	; cx存放着Loader的扇区号,DeltaSectorNo存放着19-2=17,19是根目录区开始扇区号,2是因为数据区开始簇号为2
		mov ax, BaseOfLoader
		mov es, ax
		mov bx, OffsetOfLoader
		mov ax, cx

	LABEL_GOON_LOADING_FILE:
		; 打印一个al存放的“.”
		push ax
		push bx
		mov ah, 0Eh
		mov al, '.'
		mov bl, 0Fh
		int 10h
		pop bx
		pop ax

		mov cl, 1
		call ReadSector
		pop ax				; ax存放着FAT项的序号
		call GetFATEntry
		cmp ax, 0FFFh
		jz LABEL_FILE_LOADED
		push ax
		mov dx, RootDirSectors
		add ax, dx
		add ax, DeltaSectorNo
		add bx, [BPB_BytsPerSec]	; 数据缓冲区指向下一个512字节
		jmp LABEL_GOON_LOADING_FILE

	LABEL_FILE_LOADED:
		jmp $
	\end{lstlisting}

\end{document}
