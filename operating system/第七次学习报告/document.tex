% !TeX spellcheck = en_US
%% 字体:方正静蕾简体
%%		 方正粗宋
\documentclass[a4paper,left=2.5cm,right=2.5cm,11pt]{article}

\usepackage[utf8]{inputenc}
\usepackage{fontspec}
\usepackage{cite}
\usepackage{xeCJK}
\usepackage{indentfirst}
\usepackage{titlesec}
\usepackage{longtable}
\usepackage{graphicx}
\usepackage{float}
\usepackage{rotating}
\usepackage{subfigure}
\usepackage{tabu}
\usepackage{amsmath}
\usepackage{setspace}
\usepackage{amsfonts}
\usepackage{appendix}
\usepackage{listings}
\usepackage{xcolor}
\usepackage{geometry}
\setcounter{secnumdepth}{4}
\usepackage{mhchem}
\usepackage{multirow}
\usepackage{extarrows}
\usepackage{hyperref}
\titleformat*{\section}{\LARGE}
\renewcommand\refname{参考文献}
\renewcommand{\abstractname}{\sihao \cjkfzcs 摘{  }要}
%\titleformat{\chapter}{\centering\bfseries\huge\wryh}{}{0.7em}{}{}
%\titleformat{\section}{\LARGE\bf}{\thesection}{1em}{}{}
\titleformat{\subsection}{\Large\bfseries}{\thesubsection}{1em}{}{}
\titleformat{\subsubsection}{\large\bfseries}{\thesubsubsection}{1em}{}{}
\renewcommand{\contentsname}{{\cjkfzcs \centerline{目{  } 录}}}
\setCJKfamilyfont{cjkhwxk}{STXingkai}
\setCJKfamilyfont{cjkfzcs}{STSongti-SC-Regular}
% \setCJKfamilyfont{cjkhwxk}{华文行楷}
% \setCJKfamilyfont{cjkfzcs}{方正粗宋简体}
\newcommand*{\cjkfzcs}{\CJKfamily{cjkfzcs}}
\newcommand*{\cjkhwxk}{\CJKfamily{cjkhwxk}}
\newfontfamily\wryh{Microsoft YaHei}
\newfontfamily\hwzs{STZhongsong}
\newfontfamily\hwst{STSong}
\newfontfamily\hwfs{STFangsong}
\newfontfamily\jljt{MicrosoftYaHei}
\newfontfamily\hwxk{STXingkai}
% \newfontfamily\hwzs{华文中宋}
% \newfontfamily\hwst{华文宋体}
% \newfontfamily\hwfs{华文仿宋}
% \newfontfamily\jljt{方正静蕾简体}
% \newfontfamily\hwxk{华文行楷}
\newcommand{\verylarge}{\fontsize{60pt}{\baselineskip}\selectfont}  
\newcommand{\chuhao}{\fontsize{44.9pt}{\baselineskip}\selectfont}  
\newcommand{\xiaochu}{\fontsize{38.5pt}{\baselineskip}\selectfont}  
\newcommand{\yihao}{\fontsize{27.8pt}{\baselineskip}\selectfont}  
\newcommand{\xiaoyi}{\fontsize{25.7pt}{\baselineskip}\selectfont}  
\newcommand{\erhao}{\fontsize{23.5pt}{\baselineskip}\selectfont}  
\newcommand{\xiaoerhao}{\fontsize{19.3pt}{\baselineskip}\selectfont} 
\newcommand{\sihao}{\fontsize{14pt}{\baselineskip}\selectfont}      % 字号设置  
\newcommand{\xiaosihao}{\fontsize{12pt}{\baselineskip}\selectfont}  % 字号设置  
\newcommand{\wuhao}{\fontsize{10.5pt}{\baselineskip}\selectfont}    % 字号设置  
\newcommand{\xiaowuhao}{\fontsize{9pt}{\baselineskip}\selectfont}   % 字号设置  
\newcommand{\liuhao}{\fontsize{7.875pt}{\baselineskip}\selectfont}  % 字号设置  
\newcommand{\qihao}{\fontsize{5.25pt}{\baselineskip}\selectfont}    % 字号设置 

\usepackage{diagbox}
\usepackage{multirow}
\boldmath
\XeTeXlinebreaklocale "zh"
\XeTeXlinebreakskip = 0pt plus 1pt minus 0.1pt
\definecolor{cred}{rgb}{0.8,0.8,0.8}
\definecolor{cgreen}{rgb}{0,0.3,0}
\definecolor{cpurple}{rgb}{0.5,0,0.35}
\definecolor{cdocblue}{rgb}{0,0,0.3}
\definecolor{cdark}{rgb}{0.95,1.0,1.0}
\lstset{
	language=[x86masm]Assembler,
	numbers=left,
	numberstyle=\tiny\color{black},
	showspaces=false,
	showstringspaces=false,
	basicstyle=\scriptsize,
	keywordstyle=\color{purple},
	commentstyle=\itshape\color{cgreen},
	stringstyle=\color{blue},
	frame=lines,
	% escapeinside=``,
	extendedchars=true, 
	xleftmargin=1em,
	xrightmargin=1em, 
	backgroundcolor=\color{cred},
	aboveskip=1em,
	breaklines=true,
	tabsize=4
} 

\newfontfamily{\consolas}{Consolas}
\newfontfamily{\monaco}{Monaco}
\setmonofont[Mapping={}]{Consolas}	%英文引号之类的正常显示,相当于设置英文字体
\setsansfont{Consolas} %设置英文字体 Monaco, Consolas,  Fantasque Sans Mono
\setmainfont{Times New Roman}

\setCJKmainfont{华文中宋}


\newcommand{\fic}[1]{\begin{figure}[H]
		\center
		\includegraphics[width=0.8\textwidth]{#1}
	\end{figure}}
	
\newcommand{\sizedfic}[2]{\begin{figure}[H]
		\center
		\includegraphics[width=#1\textwidth]{#2}
	\end{figure}}

\newcommand{\codefile}[1]{\lstinputlisting{#1}}

% 改变段间隔
\setlength{\parskip}{0.2em}
\linespread{1.1}

\usepackage{lastpage}
\usepackage{fancyhdr}
\pagestyle{fancy}
\lhead{\space \qquad \space}
\chead{第七次学习报告 \qquad}
\rhead{\qquad\thepage/\pageref{LastPage}}
\begin{document}

\tableofcontents

\clearpage

\section{实现引导扇区}
	一个操作系统从开机到开始运行,需要经历“引导,加载内核进入内存,跳入保护模式,开始执行内核”。
	操作系统使用Loader模块来加载内核进入内存,并跳入保护模式。
	引导扇区负责将Loader加载入内存。
\subsection{引导扇区格式}
	引导扇区是软盘的第0个扇区。
	我们把Loader模块复制到软盘上,然后引导扇区将找到并加载它。\par
	引导扇区开头有一个很重要的数据结构,叫做BPB。这个BPB数据结构使得软盘被操作系统识别。
	加上BPB数据结构之后,引导扇区的格式如下面的代码所示:
	\begin{lstlisting}
	jmp short LABEL_START
	nop

	BS_OEMName DB 'ForrestY' ; 生厂商名字
	BPB_BytsPerSec DW 512 ; 每扇区字节数
	BPB_SecPerClus DB 1 ; 每簇多少扇区
	BPB_RsvdSecCnt DB 1 ; Boot记录占用多少扇区
	BPB_NumFATS DB 2 ; 共有多少FAT表
	BPB_RootEntCnt DW 224 ; 根目录文件数最大值
	BPB_TotSec16 DW 2880 ; 逻辑扇区总数
	BPB_Media DB 0xF0 ; 媒体描述符
	BPB_FATSz16 DW 9 ; 每FAT扇区数
	BPB_SecPerTrk DW 18 ; 每磁道扇区数
	BPB_NumHeads DW 2 ; 磁头数
	BPB_HiddSec DD 0 ;  隐藏扇区数
	BPB_TotSec32 DD 0 ; 记录扇区数
	BS_DrvNum DB 0 ; 中断13的驱动器号
	BS_Reservedl DB 0 ; 未使用
	BS_BootSig DB 29h ; 扩展引导标记
	BS_VolID DD 0 ; 卷序列号
	BS_VolLab DB 'pengsida001' ; 卷标,必须11个字节
	BS_FileSysType DB 'FAT12' ; 文件系统类型

LABEL_START:
	; ...
	\end{lstlisting}

\subsection{加载Loader进入内存}
	为了加载Loader文件,我们首先需要知道Loader模块所在的位置。
	我们假设Loader模块存放在根目录中,而根目录信息存放在根目录区中。根目录区从第19个扇区开始,由BPB\_RootEntCnt个目录条目组成。\par
	目录条目占用32字节,它的格式如下:
	\sizedfic{0.5}{1.png}

	所以,我们只要Loader模块的目录条目,就可以根据DIR\_FstClus的值找到Loader模块。
\subsubsection{寻找Loader的目录条目}
	我们通过遍历根目录区来寻找Loader模块目录条目,代码如下:
	\begin{lstlisting}
	; 根目录的第一个扇区号是19
	SectorNoOfRootDirectory equ 19
	; 数据缓冲区的基地址
	BaseOfLoader equ 09000h
	; 数据缓冲区的偏移地址
	OffsetOfLoader equ 0100h
	; Loader模块的名字
	LoaderFileName db "LOADER BIN", 0

	; wSectorNo地址单元存放着要读取的扇区号
	mov word [wSectorNo], SectorNoOfRootDirectory
LABEL_SEARCH_IN_ROOT_DIR_BEGIN:
	; wRootDirSizeForLoop地址单元存放着扇区数
	cmp word [wRootDirSizeForLoop], 0
	jz LABEL_NO_LOADERBIN
	; 每遍历一次,扇区数减一
	dec word [wRootDirSizeForLoop]
	; 设置es:bx,指定数据缓冲区的地址
	mov ax, BaseOfLoader 
	mov es, ax
	mov bx, OffsetOfLoader
	; 设置要读的扇区号
	mov ax, [wSectorNo]
	; 设置要读的扇区数
	mov cl, 1
	; 读取一个扇区的内容
	call ReadSector

	mov si, LoaderFileName
	mov di, OffsetOfLoader
	cld
	; dx代表着接下来的循环次数
	; 一个扇区512字节,一个根目录条目32字节,所以需要循环16次
	mov dx, 10h
LABEL_SEARCH_FOR_LOADERBIN:
	cmp dx, 0
	jz LABEL_GOTO_NEXT_SECTOR_IN_ROOT_DIR
	dec dx
	; 比较文件名称,文件名称为11个字节,所以比较11次
	mov cx, 11
LABEL_CMP_FILENAME:
	cmp cx, 0
	jz LABEL_FILENAME_FOUND
	dec cx
	; 将LoaderFileName处的字节读入al
	lodsb
	; 比较数据缓冲区中的字节
	cmp al, byte [es:di]
	jz LABEL_GO_ON
	jmp LABEL_DIFFERENT
LABEL_GO_ON:
	inc di
	jmp LABEL_CMP_FILENAME
LABEL_DIFFERENT:
	; 让di指向下一个条目
	; 每个目录条目为32字节
	and di, 0FFE0h
	add di, 20h
	mov si, LoaderFileName
	jmp LABEL_SEARCH_FOR_LOADERBIN
LABEL_GOTO_NEXT_SECTOR_IN_ROOT_DIR:
	; 读取下一个扇区号
	add word [wSectorNo], 1
	jmp LABEL_SEARCH_IN_ROOT_DIR_BEGIN
LABEL_NO_LOADERBIN:
	mov dh, 2
	call DispStr
	jmp $
LABEL_FILENAME_FOUND:
	jmp $
	\end{lstlisting}

	上述代码中,用到了读取一个扇区内容的函数ReadSector。
	需要读软盘的时候,要用到BIOS中断int 13h。\par
	当ah=00h时,int 13h用于复位软驱,此时使用dl指定驱动器号。\par
	当ah=02h时,int 13h用于从磁盘将数据读入es:bx指向的缓冲区中。当读取错误时,CF会被置一。
	此时需要设置如下的寄存器值:
	\begin{lstlisting}
	al = 要读扇区数
	ch = 磁道号
	cl = 起始扇区号
	dh = 磁头号
	dl = 驱动器号
	es:bx指定数据缓冲区
	\end{lstlisting}

	在代码中添加一个ReadSector函数,用于读取软盘。\par
	ReadSector函数将al和cl作为参数,al是相对扇区号,cl是要读取的扇区数。
	在函数中,程序根据相对扇区号获得磁道号、起始扇区号和磁头号。
	软盘中一个磁道有18个扇区,于是将相对扇区号除以18,得到的商和余数分别是总磁道号和起始扇区号。
	软盘中,因为有两面,所以分别要磁头号0和磁头号1标记。
	现在还需要确定的是在哪个磁头号的第几个磁道号。
	软盘结构中,不是先排完0磁头号再排1磁头号的,而是交错排列。
	总磁头号为偶数的位于磁头号0,总磁头号为奇数的位于磁头号1。
	所以只要判断总磁头号的奇偶就能得到磁头号。将总磁头号与1相与,为0的话磁头号就是0,为1的话磁头号就是1。
	然后将总磁头号除以2,就能得到相对于磁头的起始磁道号。
	读取软盘扇区的代码如下;
	\begin{lstlisting}
ReadSector:
	push bp
	mov bp, sp
	sub esp, 2
	; 处理int 13h所需要的参数
	; cl存放着要读取的扇区数
	mov byte [bp-2], cl
	push bx
	; bl存放着每个磁道上的扇区数
	mov bl, [BPB_SecPerTrk]
	; ax/bl,商放在al中,余数放在ah中
	div bl
	; 得到当前磁道中的起始扇区号
	inc ah
	; 设置cl的值为起始扇区号
	mov cl, ah
	mov dh, al
	shr al, 1
	; 设置ch的值为磁道号
	mov ch, al
	; 设置dh的值为磁头号
	and dh, 1
	pop bx
	; 设置dl的值为驱动器号
	mov dl, [BS_DrvNum]
.GoOnReading:
	mov ah, 2
	mov al, byte [bp-2]
	int 13h
	jc .GoOnReading

	add esp, 2
	pop bp
	ret
	\end{lstlisting}

\subsubsection{寻找FAT项}
	现在我们得到Loader模块的目录条目了,也就能得到Loader模块对应的开始簇号,也就能够得到Loader.bin的起始扇区号。
	通过这个扇区号,我们可以做到两件事:
	\begin{itemize}
		\item 把起始扇区装入内存。
		\item 通过这个扇区号找到FAT中的项,从而找到Loader占用的其他所有扇区。\par
		在此先介绍一下软盘上的文件系统FAT12,它由引导扇区、两个FAT表、根目录区和数据区组成,格式如下图:
		\sizedfic{0.6}{2.png}

		我们之前获得的根目录条目就存放在根目录区中。现在我们拥有Loader模块的簇号,需要根据FAT表来找到Loader的所有簇。
		FAT表由FAT项组成,每个FAT项长为12字节,FAT项的值代表文件下一个簇号。如果FAT项的值大等于0xFF8,那么代表当前簇是本文件的最后一个簇。
		如果FAT项的值为0xFF7,表示它是一个坏簇。
		所以我们只要拥有起始簇号n,就对应着FAT表中第n个FAT项,从而就能找到找到文件的所有簇号,也就找到了文件所占用的所有扇区。\par
		FAT项的格式如下:
		\sizedfic{0.6}{3.png}
	\end{itemize}

	实现代码如下:
	\begin{lstlisting}
	bOdd db 0
	SectorNoOfFAT1 equ 1
	GetFATEntry:
		push es
		push bx
		; ax存放着起始扇区号,对应着第[ax]个FAT项
		push ax
		; Loader模块在内存中的起始地址
		mov ax, BaseOfLoader
		; 在Loader模块后面留出4k空间用于存放FAT项,作为数据缓冲区
		sub ax, 0100h
		mov es, ax
		; 恢复ax的值
		pop ax
		; bOdd用于判断FAT项从第0位开始还是从第4位开始
		mov byte [bOdd], 0
		; FAT项占用1.5个字节,所以ax先乘以3,再除以2
		mov bx, 3
		mul bx
		mov bx, 2
		; 商放在ax中,余数放在dx中
		div bx
		; 判断FAT项从第0位开始还是从第4位开始
		; ax为奇数时,FAT项从第4位开始。ax为偶数时,FAT项从第0位开始
		cmp dx, 0
		jz LABEL_EVEN
		mov byte [bOdd], 1
	LABEL_EVEN:
		xor dx, dx
		; BPB_BytsPerSec是每个扇区占用的字节数
		mov bx, [BPB_BytsPerSec]
		; ax存放着FAT项相对于FAT的扇区号,bx存放着FAT项在扇区中的偏移
		div bx
		push dx
		mov bx, 0
		; ax加上FAT的扇区号,当前值为FAT项所在的扇区号
		add ax, SectorNoOfFAT1
		; cl存放着要读取的扇区数
		mov cl, 2
		; 读取FAT项所在的扇区,为了防止FAT项跨越两个扇区,所以一次读取两个扇区
		call ReadSector
		pop dx
		add bx, dx
		; es是数据缓冲区基地址,bx是FAT项在扇区中的偏移
		; 现在ax中存放着FAT项的值
		mov ax, [es:bx]
		cmp byte [bOdd], 1
		jnz LABEL_EVEN_2
		; 如果FAT从第4位开始,就将ax右移4位
		shr ax, 4
	LABEL_EVEN_2:
		; 如果FAT从第0位开始,就只保留ax的低12位
		and ax, 0FFFh
	LABEL_GET_FAT_ENTRY_OK:
		pop bx
		pop es
		ret
	\end{lstlisting}

\subsubsection{加载Loader}
	现在我们可以通过遍历根目录区来找到Loader模块对应的根目录条目。
	从根目录条目中找到相应的簇号,然后根据FAT表中的FAT项找到文件的下一个簇号。
	这里的簇号是相对于数据区的簇号。为了获得整个软盘中的簇号,需要加上根目录区的起始簇号,在加上根目录区簇的数量。
	因为数据区的簇号是从2开始的,所以还要减去2。最后根据实际的簇号得到Loader模块所在的扇区号,然后将扇区号作为ReadSector函数的参数,读取相应扇区的数据。
	实现代码如下:
	\begin{lstlisting}
	LABEL_FILENAME_FOUND:
		mov ax, RootDirSectors
		and di, 0FFE0h
		add di, 01Ah
		mov cx, word [es:di]
		push cx
		add cx, ax
		add cx, DeltaSectorNo
		mov ax, BaseOfLoader
		mov es, ax
		mov bx, OffsetOfLoader
		mov ax, cx
	LABEL_GOON_LOADING_FILE:
		mov cl, 1
		call ReadSector
		pop ax
		call GetFATEntry
		; 检查FAT项的值是否是0FFFh
		cmp ax, 0FFFh
		jz LABEL_FILE_LOADED
		push ax
		; RootDirSectors是根目录扇区数
		mov dx, RootDirSectors
		add ax, dx
		; 根目录开始扇区号为19,数据区第一个簇的簇号是2
		; 为了正确求得FAT项对应的簇号,定义了DeltaSectorNo equ 17
		; FAT项对应的簇号 = RootDirSectors + DeltaSectorNo + 起始簇号
		add ax, DeltaSectorNo
		; es:bx指向数据缓冲区
		; 读取一个扇区结束后,bx的值加512字节
		add bx, [BPB_BytsPerSec]
		jmp LABEL_GOON_LOADING_FILE
	\end{lstlisting}

\subsubsection{执行Loader模块的代码}
	前几个小结将Loader模块加载进了内存,放在BaseOfLoader:OffsetOfLoader处。现在只要将程序跳转到数据缓冲区的地址,就可以开始执行Loader模块的代码。
	实现代码如下:
	\begin{lstlisting}
	jmp BaseOfLoader:OffsetOfLoader
	\end{lstlisting}

	回想本节一开始说的,一个操作系统从开机到开始运行,需要经历“引导,加载内核进入内存,跳入保护模式,开始执行内核”。
	现在我们只做到了引导,也就是将Loader模块加载进内存。Loader要做的事情还有两件:
	\begin{itemize}
		\item 加载内核入内存。
		\item 跳入保护模式。
	\end{itemize}

\subsection{引导扇区完整的实现代码}
	前面小节虽然有贴代码,但是都是一些细节上的代码。现在感受一下完整的实现代码,以此对引导扇区整个的实现框架有清楚的认识。
	在阅读代码之前,先了解一下软盘上FAT12文件系统引导扇区的格式,从而对代码框架有更全面的认识。FAT12引导扇区格式如下所示:
	\fic{4.png}
	
	引导扇区实现代码如下:
	\codefile{boot.asm}

\clearpage

\section{实现内核雏形}

\subsection{汇编与C混合编程}
	因为编写内核的过程中,需要用到汇编与C的混合编程,所以在此通过一个例子去熟悉它。\par
	程序入口\_start在foo.asm中,一开始程序会调用bar.c中的函数choose(),在choose()函数中又将调用foo.asm中的函数myprint()来打印字符串。
	整个过程如下图:
	\fic{5.png}

	例子如下,先看foo.asm的代码:
	\begin{lstlisting}
	; foo.asm
	extern choose

	[section .data]

	num1st dd 3
	num2nd dd 4

	[section .text]

	global _start
	global myprint

	_start:
		push dword [num2nd]
		push dword [num1st]
		call choose
		add esp, 8
		mov ebx, 0
		mov eax, 1
		int 0x80
	
	myprint:
		mov edx, [esp + 8]
		mov ecx, [esp + 4]
		mov ebx, 1
		mov eax, 4
		int 0x80
		ret
	\end{lstlisting}

	在看bar.c的代码:
	\begin{lstlisting}[language = C]
	void myprint(char* msg, int len);
	
	int choose(int a, int b)
	{
		if(a >= b)
			myprint("the first one\n", 15);
		else
			myprint("the second one\n", 16);
		return 0;
	}
	\end{lstlisting}

	以上的例子有几点需要说明的:
	\begin{itemize}
		\item foo.asm中一定要定义"\_start",而且还要用过global关键字将它导出,这样链接程序才能找到它,作为程序的入口点。
		\item 因为bar.c要用到foo.asm中的myprint()函数,所以需要用global关键字将其导出。
		\item 因为foo.asm要用到bar.c中的函数choose(),所以要用extern关键字声明。
	\end{itemize}

\subsection{ELF文件格式}
	内核在linux编译后是ELF格式。为了让Loader将内核装载进内存,我们还需要研究ELF格式。\par
	ELF文件由4部分组成,分别是ELF头、程序头表、节和节头表。格式如下:
	\fic{6.png}

	ELF文件为了能够支持8位到32位不同架构的处理器,定义了下列数据结构:
	\begin{lstlisting}[language = C]
	// Elf32_Addr 无符号程序地址,4字节大小
	// Elf32_Half 无符号中等大小整数,2字节大小
	// Elf32_Off 无符号文件偏移,4字节大小
	// Elf32_SWord 有符号大整数,4字节大小
	// Elf32_Word 无符号大整数,4字节大小
	\end{lstlisting}

	接下来将分别介绍ELF文件的这4个部分。

\subsubsection{ELF头}
	ELF头的定义代码如下:
	\begin{lstlisting}[language = C]
	#define EI_NIDENT 16
	typedef struct
	{
		// e_ident为16字节的字符数组,头4个字节为“.ELF”,用于表明该文件是一个ELF文件。接下来的12个字节是一些与机器无关的信息。
		unsigned char e_ident[EI_NIDENT];
		// e_type用于表示该文件的类型
		Elf32_Half e_type;
		// e_machine用于表明该程序需要的体系结构
		Elf32_Half e_machine;
		// e_version用于表明文件的版本
		Elf32_Word e_version;
		// e_entry为程序的入口地址
		Elf32_Addr e_entry;
		// e_phoff代表了程序头表在文件中的偏移量
		Elf32_Off e_phoff;
		// e_shoff代表了节头表在文件中的偏移量
		Elf32_Off e_shoff;
		// e_flags对于IA32而言,此项为0
		Elf32_Word e_flags;
		// e_ehsize代表ELF头的大小
		Elf32_Half e_ehsize;
		// e_phentsize代表程序头表中每个条目的大小
		Elf32_Half e_phentsize;
		// e_phnum代表程序头表中条目的数目
		Elf32_Half e_phnum;
		// e_shentsize代表节头表中每个条目的大小
		Elf32_Half e_shentsize;
		// e_shnum代表节头表中条目的数目
		Elf32_Half e_shnum;
		// e_shstrndx表示了包含节名称的字符串表所在节的序号
		Elf32_Half e_shstrndx;
	}Elf32_Ehdr;
	\end{lstlisting}

\subsubsection{程序头}
	程序头描述的是一个段在文件中的位置、大小以及它被放入内存后所在的位置和大小。
	如果我们想把一个文件加载进内存,就需要用到这些信息。以下是程序头的定义代码:
	\begin{lstlisting}[language = C]
	typedef struct
	{
		Elf32_Word p_type;
		Elf32_Off p_offset;
		Elf32_Addr p_vaddr;
		Elf32_Addr p_paddr;
		Elf32_Word p_filesz;
		Elf32_Word p_memsz;
		Elf32_Word p_flags;
		Elf32_Word p_align;
	}Elf32_Phdr;
	\end{lstlisting}

\end{document}
