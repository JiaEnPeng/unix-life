% !TeX spellcheck = en_US
%% 字体:方正静蕾简体
%%		 方正粗宋
\documentclass[a4paper,left=2.5cm,right=2.5cm,11pt]{article}

\usepackage[utf8]{inputenc}
\usepackage{fontspec}
\usepackage{cite}
\usepackage{xeCJK}
\usepackage{indentfirst}
\usepackage{titlesec}
\usepackage{longtable}
\usepackage{graphicx}
\usepackage{float}
\usepackage{rotating}
\usepackage{subfigure}
\usepackage{tabu}
\usepackage{amsmath}
\usepackage{setspace}
\usepackage{amsfonts}
\usepackage{appendix}
\usepackage{listings}
\usepackage{xcolor}
\usepackage{geometry}
\setcounter{secnumdepth}{4}
\usepackage{mhchem}
\usepackage{multirow}
\usepackage{extarrows}
\usepackage{hyperref}
\titleformat*{\section}{\LARGE}
\renewcommand\refname{参考文献}
\renewcommand{\abstractname}{\sihao \cjkfzcs 摘{  }要}
%\titleformat{\chapter}{\centering\bfseries\huge\wryh}{}{0.7em}{}{}
%\titleformat{\section}{\LARGE\bf}{\thesection}{1em}{}{}
\titleformat{\subsection}{\Large\bfseries}{\thesubsection}{1em}{}{}
\titleformat{\subsubsection}{\large\bfseries}{\thesubsubsection}{1em}{}{}
\renewcommand{\contentsname}{{\cjkfzcs \centerline{目{  } 录}}}
\setCJKfamilyfont{cjkhwxk}{STXingkai}
\setCJKfamilyfont{cjkfzcs}{STSongti-SC-Regular}
% \setCJKfamilyfont{cjkhwxk}{华文行楷}
% \setCJKfamilyfont{cjkfzcs}{方正粗宋简体}
\newcommand*{\cjkfzcs}{\CJKfamily{cjkfzcs}}
\newcommand*{\cjkhwxk}{\CJKfamily{cjkhwxk}}
\newfontfamily\wryh{Microsoft YaHei}
\newfontfamily\hwzs{STZhongsong}
\newfontfamily\hwst{STSong}
\newfontfamily\hwfs{STFangsong}
\newfontfamily\jljt{MicrosoftYaHei}
\newfontfamily\hwxk{STXingkai}
% \newfontfamily\hwzs{华文中宋}
% \newfontfamily\hwst{华文宋体}
% \newfontfamily\hwfs{华文仿宋}
% \newfontfamily\jljt{方正静蕾简体}
% \newfontfamily\hwxk{华文行楷}
\newcommand{\verylarge}{\fontsize{60pt}{\baselineskip}\selectfont}  
\newcommand{\chuhao}{\fontsize{44.9pt}{\baselineskip}\selectfont}  
\newcommand{\xiaochu}{\fontsize{38.5pt}{\baselineskip}\selectfont}  
\newcommand{\yihao}{\fontsize{27.8pt}{\baselineskip}\selectfont}  
\newcommand{\xiaoyi}{\fontsize{25.7pt}{\baselineskip}\selectfont}  
\newcommand{\erhao}{\fontsize{23.5pt}{\baselineskip}\selectfont}  
\newcommand{\xiaoerhao}{\fontsize{19.3pt}{\baselineskip}\selectfont} 
\newcommand{\sihao}{\fontsize{14pt}{\baselineskip}\selectfont}      % 字号设置  
\newcommand{\xiaosihao}{\fontsize{12pt}{\baselineskip}\selectfont}  % 字号设置  
\newcommand{\wuhao}{\fontsize{10.5pt}{\baselineskip}\selectfont}    % 字号设置  
\newcommand{\xiaowuhao}{\fontsize{9pt}{\baselineskip}\selectfont}   % 字号设置  
\newcommand{\liuhao}{\fontsize{7.875pt}{\baselineskip}\selectfont}  % 字号设置  
\newcommand{\qihao}{\fontsize{5.25pt}{\baselineskip}\selectfont}    % 字号设置 

\usepackage{diagbox}
\usepackage{multirow}
\boldmath
\XeTeXlinebreaklocale "zh"
\XeTeXlinebreakskip = 0pt plus 1pt minus 0.1pt
\definecolor{cred}{rgb}{0.8,0.8,0.8}
\definecolor{cgreen}{rgb}{0,0.3,0}
\definecolor{cpurple}{rgb}{0.5,0,0.35}
\definecolor{cdocblue}{rgb}{0,0,0.3}
\definecolor{cdark}{rgb}{0.95,1.0,1.0}
\lstset{
	language=[x86masm]Assembler,
	numbers=left,
	numberstyle=\tiny\color{black},
	showspaces=false,
	showstringspaces=false,
	basicstyle=\scriptsize,
	keywordstyle=\color{purple},
	commentstyle=\itshape\color{cgreen},
	stringstyle=\color{blue},
	frame=lines,
	% escapeinside=``,
	extendedchars=true, 
	xleftmargin=1em,
	xrightmargin=1em, 
	backgroundcolor=\color{cred},
	aboveskip=1em,
	breaklines=true,
	tabsize=4
} 

\newfontfamily{\consolas}{Consolas}
\newfontfamily{\monaco}{Monaco}
\setmonofont[Mapping={}]{Consolas}	%英文引号之类的正常显示,相当于设置英文字体
\setsansfont{Consolas} %设置英文字体 Monaco, Consolas,  Fantasque Sans Mono
\setmainfont{Times New Roman}

\setCJKmainfont{华文中宋}


\newcommand{\fic}[1]{\begin{figure}[H]
		\center
		\includegraphics[width=0.8\textwidth]{#1}
	\end{figure}}
	
\newcommand{\sizedfic}[2]{\begin{figure}[H]
		\center
		\includegraphics[width=#1\textwidth]{#2}
	\end{figure}}

\newcommand{\codefile}[1]{\lstinputlisting{#1}}

\newcommand{\interval}{\vspace{0.5em}}

% 改变段间隔
\setlength{\parskip}{0.2em}
\linespread{1.1}

\usepackage{lastpage}
\usepackage{fancyhdr}
\pagestyle{fancy}
\lhead{\space \qquad \space}
\chead{第七次学习报告 \qquad}
\rhead{\qquad\thepage/\pageref{LastPage}}
\begin{document}

\tableofcontents

\clearpage

\section{实现引导扇区}
	一个操作系统从开机到开始运行,需要经历“引导,加载内核进入内存,跳入保护模式,开始执行内核”。
	操作系统使用Loader模块来加载内核进入内存,并跳入保护模式。
	引导扇区负责将Loader加载入内存。
\subsection{引导扇区格式}
	引导扇区是软盘的第0个扇区。
	我们把Loader模块复制到软盘上,然后引导扇区将找到并加载它。\par
	引导扇区开头有一个很重要的数据结构,叫做BPB。这个BPB数据结构使得软盘被操作系统识别。
	加上BPB数据结构之后,引导扇区的格式如下面的代码所示:
	\begin{lstlisting}
	jmp short LABEL_START
	nop

	BS_OEMName DB 'ForrestY' ; 生厂商名字
	BPB_BytsPerSec DW 512 ; 每扇区字节数
	BPB_SecPerClus DB 1 ; 每簇多少扇区
	BPB_RsvdSecCnt DB 1 ; Boot记录占用多少扇区
	BPB_NumFATS DB 2 ; 共有多少FAT表
	BPB_RootEntCnt DW 224 ; 根目录文件数最大值
	BPB_TotSec16 DW 2880 ; 逻辑扇区总数
	BPB_Media DB 0xF0 ; 媒体描述符
	BPB_FATSz16 DW 9 ; 每FAT扇区数
	BPB_SecPerTrk DW 18 ; 每磁道扇区数
	BPB_NumHeads DW 2 ; 磁头数
	BPB_HiddSec DD 0 ;  隐藏扇区数
	BPB_TotSec32 DD 0 ; 记录扇区数
	BS_DrvNum DB 0 ; 中断13的驱动器号
	BS_Reservedl DB 0 ; 未使用
	BS_BootSig DB 29h ; 扩展引导标记
	BS_VolID DD 0 ; 卷序列号
	BS_VolLab DB 'pengsida001' ; 卷标,必须11个字节
	BS_FileSysType DB 'FAT12' ; 文件系统类型

LABEL_START:
	; ...
	\end{lstlisting}

\subsection{加载Loader进入内存}
	为了加载Loader文件,我们首先需要知道Loader模块所在的位置。
	我们假设Loader模块存放在根目录中,而根目录信息存放在根目录区中。根目录区从第19个扇区开始,由BPB\_RootEntCnt个目录条目组成。\par
	目录条目占用32字节,它的格式如下:
	\sizedfic{0.5}{1.png}

	所以,我们只要Loader模块的目录条目,就可以根据DIR\_FstClus的值找到Loader模块。
\subsubsection{寻找Loader的目录条目}
	我们通过遍历根目录区来寻找Loader模块目录条目,代码如下:
	\begin{lstlisting}
	; 根目录的第一个扇区号是19
	SectorNoOfRootDirectory equ 19
	; 数据缓冲区的基地址
	BaseOfLoader equ 09000h
	; 数据缓冲区的偏移地址
	OffsetOfLoader equ 0100h
	; Loader模块的名字
	LoaderFileName db "LOADER BIN", 0

	; wSectorNo地址单元存放着要读取的扇区号
	mov word [wSectorNo], SectorNoOfRootDirectory
LABEL_SEARCH_IN_ROOT_DIR_BEGIN:
	; wRootDirSizeForLoop地址单元存放着扇区数
	cmp word [wRootDirSizeForLoop], 0
	jz LABEL_NO_LOADERBIN
	; 每遍历一次,扇区数减一
	dec word [wRootDirSizeForLoop]
	; 设置es:bx,指定数据缓冲区的地址
	mov ax, BaseOfLoader 
	mov es, ax
	mov bx, OffsetOfLoader
	; 设置要读的扇区号
	mov ax, [wSectorNo]
	; 设置要读的扇区数
	mov cl, 1
	; 读取一个扇区的内容
	call ReadSector

	mov si, LoaderFileName
	mov di, OffsetOfLoader
	cld
	; dx代表着接下来的循环次数
	; 一个扇区512字节,一个根目录条目32字节,所以需要循环16次
	mov dx, 10h
LABEL_SEARCH_FOR_LOADERBIN:
	cmp dx, 0
	jz LABEL_GOTO_NEXT_SECTOR_IN_ROOT_DIR
	dec dx
	; 比较文件名称,文件名称为11个字节,所以比较11次
	mov cx, 11
LABEL_CMP_FILENAME:
	cmp cx, 0
	jz LABEL_FILENAME_FOUND
	dec cx
	; 将LoaderFileName处的字节读入al
	lodsb
	; 比较数据缓冲区中的字节
	cmp al, byte [es:di]
	jz LABEL_GO_ON
	jmp LABEL_DIFFERENT
LABEL_GO_ON:
	inc di
	jmp LABEL_CMP_FILENAME
LABEL_DIFFERENT:
	; 让di指向下一个条目
	; 每个目录条目为32字节
	and di, 0FFE0h
	add di, 20h
	mov si, LoaderFileName
	jmp LABEL_SEARCH_FOR_LOADERBIN
LABEL_GOTO_NEXT_SECTOR_IN_ROOT_DIR:
	; 读取下一个扇区号
	add word [wSectorNo], 1
	jmp LABEL_SEARCH_IN_ROOT_DIR_BEGIN
LABEL_NO_LOADERBIN:
	mov dh, 2
	call DispStr
	jmp $
LABEL_FILENAME_FOUND:
	jmp $
	\end{lstlisting}

	上述代码中,用到了读取一个扇区内容的函数ReadSector。
	需要读软盘的时候,要用到BIOS中断int 13h。\par
	当ah=00h时,int 13h用于复位软驱,此时使用dl指定驱动器号。\par
	当ah=02h时,int 13h用于从磁盘将数据读入es:bx指向的缓冲区中。当读取错误时,CF会被置一。
	此时需要设置如下的寄存器值:
	\begin{lstlisting}
	al = 要读扇区数
	ch = 磁道号
	cl = 起始扇区号
	dh = 磁头号
	dl = 驱动器号
	es:bx指定数据缓冲区
	\end{lstlisting}

	在代码中添加一个ReadSector函数,用于读取软盘。\par
	ReadSector函数将al和cl作为参数,al是相对扇区号,cl是要读取的扇区数。
	在函数中,程序根据相对扇区号获得磁道号、起始扇区号和磁头号。
	软盘中一个磁道有18个扇区,于是将相对扇区号除以18,得到的商和余数分别是总磁道号和起始扇区号。
	软盘中,因为有两面,所以分别要磁头号0和磁头号1标记。
	现在还需要确定的是在哪个磁头号的第几个磁道号。
	软盘结构中,不是先排完0磁头号再排1磁头号的,而是交错排列。
	总磁头号为偶数的位于磁头号0,总磁头号为奇数的位于磁头号1。
	所以只要判断总磁头号的奇偶就能得到磁头号。将总磁头号与1相与,为0的话磁头号就是0,为1的话磁头号就是1。
	然后将总磁头号除以2,就能得到相对于磁头的起始磁道号。
	读取软盘扇区的代码如下;
	\begin{lstlisting}
ReadSector:
	push bp
	mov bp, sp
	sub esp, 2
	; 处理int 13h所需要的参数
	; cl存放着要读取的扇区数
	mov byte [bp-2], cl
	push bx
	; bl存放着每个磁道上的扇区数
	mov bl, [BPB_SecPerTrk]
	; ax/bl,商放在al中,余数放在ah中
	div bl
	; 得到当前磁道中的起始扇区号
	inc ah
	; 设置cl的值为起始扇区号
	mov cl, ah
	mov dh, al
	shr al, 1
	; 设置ch的值为磁道号
	mov ch, al
	; 设置dh的值为磁头号
	and dh, 1
	pop bx
	; 设置dl的值为驱动器号
	mov dl, [BS_DrvNum]
.GoOnReading:
	mov ah, 2
	mov al, byte [bp-2]
	int 13h
	jc .GoOnReading

	add esp, 2
	pop bp
	ret
	\end{lstlisting}

\subsubsection{寻找FAT项}
	现在我们得到Loader模块的目录条目了,也就能得到Loader模块对应的开始簇号,也就能够得到Loader.bin的起始扇区号。
	通过这个扇区号,我们可以做到两件事:
	\begin{itemize}
		\item 把起始扇区装入内存。
		\item 通过这个扇区号找到FAT中的项,从而找到Loader占用的其他所有扇区。\par
		在此先介绍一下软盘上的文件系统FAT12,它由引导扇区、两个FAT表、根目录区和数据区组成,格式如下图:
		\sizedfic{0.6}{2.png}

		我们之前获得的根目录条目就存放在根目录区中。现在我们拥有Loader模块的簇号,需要根据FAT表来找到Loader的所有簇。
		FAT表由FAT项组成,每个FAT项长为12字节,FAT项的值代表文件下一个簇号。如果FAT项的值大等于0xFF8,那么代表当前簇是本文件的最后一个簇。
		如果FAT项的值为0xFF7,表示它是一个坏簇。
		所以我们只要拥有起始簇号n,就对应着FAT表中第n个FAT项,从而就能找到找到文件的所有簇号,也就找到了文件所占用的所有扇区。\par
		FAT项的格式如下:
		\sizedfic{0.6}{3.png}
	\end{itemize}

	实现代码如下:
	\begin{lstlisting}
	bOdd db 0
	SectorNoOfFAT1 equ 1
	GetFATEntry:
		push es
		push bx
		; ax存放着起始扇区号,对应着第[ax]个FAT项
		push ax
		; Loader模块在内存中的起始地址
		mov ax, BaseOfLoader
		; 在Loader模块后面留出4k空间用于存放FAT项,作为数据缓冲区
		sub ax, 0100h
		mov es, ax
		; 恢复ax的值
		pop ax
		; bOdd用于判断FAT项从第0位开始还是从第4位开始
		mov byte [bOdd], 0
		; FAT项占用1.5个字节,所以ax先乘以3,再除以2
		mov bx, 3
		mul bx
		mov bx, 2
		; 商放在ax中,余数放在dx中
		div bx
		; 判断FAT项从第0位开始还是从第4位开始
		; ax为奇数时,FAT项从第4位开始。ax为偶数时,FAT项从第0位开始
		cmp dx, 0
		jz LABEL_EVEN
		mov byte [bOdd], 1
	LABEL_EVEN:
		xor dx, dx
		; BPB_BytsPerSec是每个扇区占用的字节数
		mov bx, [BPB_BytsPerSec]
		; ax存放着FAT项相对于FAT的扇区号,bx存放着FAT项在扇区中的偏移
		div bx
		push dx
		mov bx, 0
		; ax加上FAT的扇区号,当前值为FAT项所在的扇区号
		add ax, SectorNoOfFAT1
		; cl存放着要读取的扇区数
		mov cl, 2
		; 读取FAT项所在的扇区,为了防止FAT项跨越两个扇区,所以一次读取两个扇区
		call ReadSector
		pop dx
		add bx, dx
		; es是数据缓冲区基地址,bx是FAT项在扇区中的偏移
		; 现在ax中存放着FAT项的值
		mov ax, [es:bx]
		cmp byte [bOdd], 1
		jnz LABEL_EVEN_2
		; 如果FAT从第4位开始,就将ax右移4位
		shr ax, 4
	LABEL_EVEN_2:
		; 如果FAT从第0位开始,就只保留ax的低12位
		and ax, 0FFFh
	LABEL_GET_FAT_ENTRY_OK:
		pop bx
		pop es
		ret
	\end{lstlisting}

\subsubsection{加载Loader}
	现在我们可以通过遍历根目录区来找到Loader模块对应的根目录条目。
	从根目录条目中找到相应的簇号,然后根据FAT表中的FAT项找到文件的下一个簇号。
	这里的簇号是相对于数据区的簇号。为了获得整个软盘中的簇号,需要加上根目录区的起始簇号,在加上根目录区簇的数量。
	因为数据区的簇号是从2开始的,所以还要减去2。最后根据实际的簇号得到Loader模块所在的扇区号,然后将扇区号作为ReadSector函数的参数,读取相应扇区的数据。
	实现代码如下:
	\begin{lstlisting}
	LABEL_FILENAME_FOUND:
		mov ax, RootDirSectors
		and di, 0FFE0h
		add di, 01Ah
		mov cx, word [es:di]
		push cx
		add cx, ax
		add cx, DeltaSectorNo
		mov ax, BaseOfLoader
		mov es, ax
		mov bx, OffsetOfLoader
		mov ax, cx
	LABEL_GOON_LOADING_FILE:
		mov cl, 1
		call ReadSector
		pop ax
		call GetFATEntry
		; 检查FAT项的值是否是0FFFh
		cmp ax, 0FFFh
		jz LABEL_FILE_LOADED
		push ax
		; RootDirSectors是根目录扇区数
		mov dx, RootDirSectors
		add ax, dx
		; 根目录开始扇区号为19,数据区第一个簇的簇号是2
		; 为了正确求得FAT项对应的簇号,定义了DeltaSectorNo equ 17
		; FAT项对应的簇号 = RootDirSectors + DeltaSectorNo + 起始簇号
		add ax, DeltaSectorNo
		; es:bx指向数据缓冲区
		; 读取一个扇区结束后,bx的值加512字节
		add bx, [BPB_BytsPerSec]
		jmp LABEL_GOON_LOADING_FILE
	\end{lstlisting}

\subsubsection{执行Loader模块的代码}
	前几个小结将Loader模块加载进了内存,放在BaseOfLoader:OffsetOfLoader处。现在只要将程序跳转到数据缓冲区的地址,就可以开始执行Loader模块的代码。
	实现代码如下:
	\begin{lstlisting}
	jmp BaseOfLoader:OffsetOfLoader
	\end{lstlisting}

	回想本节一开始说的,一个操作系统从开机到开始运行,需要经历“引导,加载内核进入内存,跳入保护模式,开始执行内核”。
	现在我们只做到了引导,也就是将Loader模块加载进内存。Loader要做的事情还有两件:
	\begin{itemize}
		\item 加载内核入内存。
		\item 跳入保护模式。
	\end{itemize}

\subsection{引导扇区完整的实现代码}
	前面小节虽然有贴代码,但是都是一些细节上的代码。现在感受一下完整的实现代码,以此对引导扇区整个的实现框架有清楚的认识。
	在阅读代码之前,先了解一下软盘上FAT12文件系统引导扇区的格式,从而对代码框架有更全面的认识。FAT12引导扇区格式如下所示:
	\fic{4.png}
	
	引导扇区实现代码如下:
	\codefile{boot.asm}

\clearpage

\section{汇编与C混合编程}
	因为编写内核的过程中,需要用到汇编与C的混合编程,所以在此通过一个例子去熟悉它。\par
	程序入口\_start在foo.asm中,一开始程序会调用bar.c中的函数choose(),在choose()函数中又将调用foo.asm中的函数myprint()来打印字符串。
	整个过程如下图:
	\fic{5.png}

	例子如下,先看foo.asm的代码:
	\begin{lstlisting}
	; foo.asm
	extern choose

	[section .data]

	num1st dd 3
	num2nd dd 4

	[section .text]

	global _start
	global myprint

	_start:
		push dword [num2nd]
		push dword [num1st]
		call choose
		add esp, 8
		mov ebx, 0
		mov eax, 1
		int 0x80
	
	myprint:
		mov edx, [esp + 8]
		mov ecx, [esp + 4]
		mov ebx, 1
		mov eax, 4
		int 0x80
		ret
	\end{lstlisting}

	在看bar.c的代码:
	\begin{lstlisting}[language = C]
	void myprint(char* msg, int len);
	
	int choose(int a, int b)
	{
		if(a >= b)
			myprint("the first one\n", 15);
		else
			myprint("the second one\n", 16);
		return 0;
	}
	\end{lstlisting}

	以上的例子有几点需要说明的:
	\begin{itemize}
		\item foo.asm中一定要定义"\_start",而且还要用过global关键字将它导出,这样链接程序才能找到它,作为程序的入口点。
		\item 因为bar.c要用到foo.asm中的myprint()函数,所以需要用global关键字将其导出。
		\item 因为foo.asm要用到bar.c中的函数choose(),所以要用extern关键字声明。
	\end{itemize}

\clearpage

\section{ELF文件}
	内核在linux编译后是ELF格式。为了让Loader将内核装载进内存,我们还需要研究ELF格式。\par
	iABI object文件格式是UNIX系统实验室作为应用程序二进制借口(ABI)而开发和发布的,又被称为ELF文件格式。
	在ELF目标文件中有三种主要的类型,如下:
	\begin{itemize}
		\item 可重定位文件,保存着代码和数据,用来和其他的object文件一起来创建一个可执行文件或共享文件。
		\item 可执行文件,保存着用来执行的程序,该文件指出了exec如来创建程序进程映像。
		\item 共享文件,保存着代码和数据,用来被链接编辑器和动态链接器连接。链接编辑器可以将当前共享文件与其他的可重定位和共享文件链接,从而创建其他的目标文件。
		动态链接器可以将当前共享文件与一个可执行文件和其他的共享文件链接,从而创建一个进程映像。
	\end{itemize}

	ELF目标文件由汇编器和链接编辑器创建,是程序的二进制表现形式,用于在处理器上直接运行。

\subsection{文件格式}
	object文件参与了程序的链接和程序的运行。object文件为这两个过程提供了不同的视角,因此链接过程中和执行过程中的object文件格式是不一样的。\par
	以下是链接过程中object文件的格式:
	\sizedfic{0.4}{7.png}

	以下是执行过程中object文件的格式:
	\sizedfic{0.3}{8.png}

	可以看出,object文件由4部分组成,如下:
	\begin{itemize}
		\item ELF header,它位于文件的开头,描述了该文件的结构。
		\item Program header table,用于告诉系统如何来创建一个进程的内存映像。执行过程中必须有这个程序头表。
		\item Sections,包含了指令、数据、符号表和重定位信息等。
		\item Section header table,包含了section的信息。每个section在这个表中有一个entry,每个entry给出了相应section的名字、大小等信息。链接过程中必须有这个节头表。
	\end{itemize}

\subsubsection{数据表示}
	ELF文件为了能够支持8位到32位不同架构的处理器,定义了一些与机器无关的数据类型:
	\begin{lstlisting}[language = C]
	// Elf32_Addr 无符号程序地址,4字节大小
	// Elf32_Half 无符号中等大小整数,2字节大小
	// Elf32_Off 无符号文件偏移,4字节大小
	// Elf32_SWord 有符号大整数,4字节大小
	// Elf32_Word 无符号大整数,4字节大小
	// unsigned char 无符号小整数,1字节大小
	\end{lstlisting}


\subsection{ELF header}
	ELF头的定义代码如下:
	\begin{lstlisting}[language = C]
	#define EI_NIDENT 16
	typedef struct
	{
		unsigned char e_ident[EI_NIDENT];
		Elf32_Half e_type;
		Elf32_Half e_machine;
		Elf32_Word e_version;
		Elf32_Addr e_entry;
		Elf32_Off e_phoff;
		Elf32_Off e_shoff;
		Elf32_Word e_flags;
		Elf32_Half e_ehsize;
		Elf32_Half e_phentsize;
		Elf32_Half e_phnum;
		Elf32_Half e_shentsize;
		Elf32_Half e_shnum;
		Elf32_Half e_shstrndx;
	}Elf32_Ehdr;
	\end{lstlisting}

	ELF header各个成员解释如下:
	\begin{itemize}
		\item e\_ident,为16字节的字符数组,头4个字节为“.ELF”,用于表明该文件是一个ELF文件。接下来的12个字节是一些与机器无关的信息。
		\item e\_type,用于确定该object文件的类型。下图是e\_type的值与对应的文件类型。
		\sizedfic{0.5}{9.png}

		\item e\_machine用于表明该程序需要的体系结构。下图是e\_machine的值与对应的文件类型。
		\sizedfic{0.5}{10.png}

		\item e\_version用于表明该object文件的版本。下图是e\_version的值与对应的文件类型。
		\sizedfic{0.5}{11.png}

		\item e\_entry为程序的入口地址。如果程序没有这个入口点,该成员的值将保持为0。
		\item e\_phoff代表了程序头表在文件中的偏移量。如果object文件中没有程序头表,该成员的值将保持为0。
		\item e\_shoff代表了节头表在文件中的偏移量。如果object文件中没有节头表,该成员的值将保持为0。
		\item e\_flags存放着与object文件有关的处理器标志。
		\item e\_ehsize代表ELF头的大小。
		\item e\_phentsize代表程序头表中entry的大小。程序头表中每个entry的大小相同。
		\item e\_phnum代表程序头表中entry的数目。
		\item e\_shentsize代表节头表中每个条目的大小。
		\item e\_shnum代表节头表中条目的数目。
		\item e\_shstrndx代表着节头表中一个条目的序号,这个条目与section名字字符表有关。
	\end{itemize}

\subsubsection{ELF鉴别}
	在前面的小节有提到,ELF文件可以支持多种机器类型。为了支持这个特性,object文件的头16个字节被用来说明如何解释该文件。这16个字节与处理器无关,和object文件剩下的内容也无关。
	ELF header的e\_ident成员是一个数组,用于存放这16个字节。接下来解释该数组中的各个序号以及相应的值:
	\begin{itemize}
		\item EI\_MAG0~EI\_MAG3是4个宏,值分别为0~3。数组中该序号相应的值如下所示:
		\sizedfic{0.5}{12.png}

		这4个字节用于表示该文件格式为ELF。

		\item 宏EI\_CLASS,值为4,e\_ident[EI\_CLASS]用于确定object文件的位数。值和相应的意义如下所示:
		\sizedfic{0.5}{13.png}

		\item 宏EI\_DATA,值为5,e\_ident[EI\_DATA]指定了object中特定处理器数据的编码方式。\par
			当e\_ident[EI\_DATA]值为0时,表明了无效的编码方式。\par
			当e\_ident[EI\_DATA]值为1时,表明了ELFDATA2LSB编码方式,值最小的字节占着最低的地址,如下所示:
			\sizedfic{0.7}{14.png}

			当e\_ident[EI\_DATA]值为2时,表明了ELFDATA2MSB编码方式,值最大的字节占着最低的地址,如下所示:
			\sizedfic{0.7}{15.png}

		\item 宏EI\_VERSION,值为6,e\_ident[EI\_VERSION]指定了ELF header的版本。
		\item 宏EI\_PAD,值为7,用于标识e\_ident未使用字节的开头。接下来e\_ident[7]~e\_ident[15]都不会被用到,这些值都被置为0。
	\end{itemize}

% \subsubsection{程序头}
% 	程序头描述的是一个段在文件中的位置、大小以及它被放入内存后所在的位置和大小。
% 	如果我们想把一个文件加载进内存,就需要用到这些信息。以下是程序头的定义代码:
% 	\begin{lstlisting}[language = C]
% 	typedef struct
% 	{
% 		Elf32_Word p_type;
% 		Elf32_Off p_offset;
% 		Elf32_Addr p_vaddr;
% 		Elf32_Addr p_paddr;
% 		Elf32_Word p_filesz;
% 		Elf32_Word p_memsz;
% 		Elf32_Word p_flags;
% 		Elf32_Word p_align;
% 	}Elf32_Phdr;
% 	\end{lstlisting}

\subsection{Sections}
	object文件中的section header table用于定位所有的section。一个section table header的索引就是这个数组的下标。
	节头表的一些索引是比较特殊的,如下所示:
	\begin{itemize}
		\item SHN\_UNDEF,值为0,指向了没有定义的或无意义的section。
		\item SHN\_LORESERVE和SHN\_LOPROC,值为0xff00,是被保留的索引的最小值。
		\item SHN\_HIPROC,值为0xff1f,被保留的索引的最大值。从SHN\_LOPROC到SHN\_HIPROC的索引用于处理器特定的意义。
		\item SHN\_ABS,值为0xfff1,指向的section的符号是绝对地址。
		\item SHN\_COMMON,值为0xfff2,指向的section的符号是一个公共的符号。
		\item SHN\_HIRESERVE,值为0xffff,是被保留的索引的最大值。从SHN\_LORESERVE到SHN\_HIRESERVE的索引都是被保留的。
	\end{itemize}

	Seciton的格式需要满足以下几点条件:
	\begin{itemize}
		\item object文件中的每个section都有一个相应的section header,用于描述section。
		\item 每个section占据的空间都是连续的。
		\item sections之间不可以相互重叠。
	\end{itemize}

\subsubsection{Section header}
	Section header的定义代码如下:
	\begin{lstlisting}[language = C]
	typedef struct
	{
		Elf32_Word sh_name;
		Elf32_Word sh_type;
		Elf32_Word sh_flags;
		Elf32_Addr sh_addr;
		Elf32_Off sh_offset;
		Elf32_Word sh_size;
		Elf32_Word sh_link;
		Elf32_Word sh_info;
		Elf32_Word sh_addralign;
		Elf32_Word sh_entsize;
	}Elf32_Shdr;
	\end{lstlisting}

	Section header的成员解释如下:
	\begin{itemize}
		\item sh\_name,是section名字字符表的索引,用于查找当前section的名字。
		\item sh\_type,将section按内容和意义分类。
		\item sh\_flags,用于描述section的属性。
		\item sh\_addr,用于指定section在内存的地址。如果section不会出现在进程的内存映像空间中,那么这个值为0。
		\item sh\_offset,用于表明该section相对于object开头的偏移。
		\item sh\_size,表示了该section的大小。
		\item sh\_link,该成员包含了该section在section header table中的索引链接。该索引链接的解释依靠于该section的类型。
		\item sh\_info,它包含了多余的信息。这些信息的解释依赖于节的类型。
		\item sh\_addralign,一些sections有地址对齐的约束。
		\item sh\_entsize,一些sections保存着一张存放着entry的表,该成员值代表着entry的大小。如果section没有这张表,那么成员值为0。
	\end{itemize}

\subsubsection{sh\_type}
	sh\_type定义了下列宏:
	\sizedfic{0.4}{16.png}

	以下是对sh\_type各个宏的解释:
	\begin{itemize}
		\item SHT\_NULL,表明该节头无效。
		\item SHT\_PROGBITS,代表着该section保存了一些由程序定义的信息。
		\item SHT\_SYMTAB,为链接器提供符号。
		\item SHT\_DYNSYM,代表着该section保存着一个动态链接时所需最小的符号集合来节省空间。
		\item SHT\_STRTAB,代表着该section保存着一个字符串表。
		\item SHT\_RELA,代表着该section保存着具有明确加数的重定位入口。
		\item SHT\_HASH,代表着该section保存着一个符号哈希表。
		\item SHT\_DYNAMIC,代表该section保存着动态链接的信息。
		\item SHT\_NOTE,代表该section保存着其他一些标志文件的信息。
		\item SHT\_NOBITS,代表着该section在文件中不占空间。
		\item SHT\_REL,代表该section保存着一个重定位入口,但是这个重定位入口没有明确加数。
		\item SHT\_SHLIB,代表该section类型是保留的。如果程序包含该类型的section,那么这个程序不符合ABI。
		\item SHT\_LOPROC到SHT\_HIPROC,这个范围内的值被保留,用于处理器特定的语义。
		\item SHT\_LOUSER,该变量是为应用程序保留的索引的最小值。
		\item SHT\_HIUSER,该变量是为应用程序保留的索引的最大值。在SHT\_LOUSER和SHT\_HIUSER之间的section类型可能会被应用程序使用。
	\end{itemize}

	在之前说过,section header结构体中sh\_link成员和sh\_info成员包含的信息与section的类型有关。
	既然已经列出了section的各个类型,那么就来看看这两个成员与section类型的关系,如下表所示。
	
	\vspace{1em}

	\begin{longtable}{p{3cm}|p{5cm}|p{5cm}}
	\hline
	sh\_type & sh\_link & sh\_info \\
	\hline
	SHT\_DYNAMIC & 该section header在字符表中的索引值被该section中的entry所使用 & 0\\
	\hline
	SHT\_HASH & 该section header在符号表中的索引值应用于符号哈希表中 & 0\\
	\hline
	SHT\_REL SHT\_RELA & 该索引值是相关的符号表的索引值 & 该section header的索引值应用于重定位\\
	\hline
	SHT\_SYMTAB SHT\_DYNSYM & 该索引值是相关的字符表的索引值 & 比字符表中最大的索引更大的一个值 \\
	\hline
	other & SHU\_UNDEF & 0 \\
	\hline
	\end{longtable}

	\vspace{1em}

\subsubsection{sh\_flags}

	sh\_flags定义了如下的宏:
	\sizedfic{0.4}{17.png}

	以下是对sh\_flags各个宏的解释:
	\begin{itemize}
		\item SHF\_WRITE,代表该section包含了一些数据,这些数据在程序执行时可以被改写。
		\item SHF\_ALLOC,代表该section在程序执行时占用着内存。
		\item SHF\_EXECINSTR,代表该section包含可执行的程序指令。
		\item SHF\_MASKPROC,这个宏中的各个位被保留,用于处理器特定的语义。
	\end{itemize}

	在section header table的索引值0处的section header的值如下所示:
	\begin{lstlisting}[language = C]
	Elf32_Shdr header = table[0];
	header.sh_name = 0; // 代表该section没有名字
	header.sh_type = SHT_NULL;
	header.sh_flags = 0;
	header.sh_addr = 0;
	header.sh_offset = 0;
	header.sh_size = 0;
	header.sh_link = SHN_UNDEF;
	header.sh_info = 0;
	header.sh_addralign = 0;
	header.entsize = 0;
	\end{lstlisting}

\subsubsection{特殊的节}
	下面是一些特定的节,它们被系统使用,有着特定的类型和属性。
	\vspace{0.5em}

	\begin{longtable}{p{2cm}|p{3cm}|p{7cm}}
	\hline
	Name & Type & Attributes \\
	\hline
	.bss & SHT\_NOBITS & SHF\_ALLOC + SHF\_WRITE \\
	.comment & SHT\_PROGBITS & none \\
	.data & SHT\_PROGBITS & SHF\_ALLOC + SHF\_WRITE\\
	.data1 & SHT\_PROGBITS & SHF\_ALLOC + SHF\_WRITE\\
	.debug & SHT\_PROGBITS & none\\
	.dynamic & SHT\_DYNAMIC & see below\\
	.dynstr & SNT\_STRTAB & SHF\_ALLOC\\
	.dynsym & SHT\_DYNSYM & SHF\_ALLOC\\
	.fini & SHT\_PROGBITS & SHF\_ALLOC + SHF\_WRITE\\
	.got & SHT\_PROGBITS & see below\\
	.hash & SHT\_HASH & SHF\_ALLOC \\
	.init & SHT\_PROGBITS & SHF\_ALLOC + SHF\_WRITE\\
	.interp & SHT\_PROGBITS & see below\\
	.line & SHT\_PROGBITS & none\\
	.note & SNT\_NOTE & none\\
	.plt & SHT\_PROGBITS & see below\\
	.relname & SHT\_REL & see below\\
	.relaname & SHT\_RELA & see below\\
	.rodate & SHT\_PROGBITS & SHF\_ALLOC\\
	.rodata1 & SHT\_PROGBITS & SHF\_ALLOC\\
	.shstrtab & SHT\_STRTAB & none\\
	.strtab & SHT\_STRTAB & see below\\
	.symtab & SHT\_SYMTAB & see below \\
	.text & SHT\_PROGBITS & SHF\_ALLOC + SHF\_EXECINSTR \\
	\hline
	\end{longtable}

	\vspace{0.5em}

	下面是对这些节的介绍:
	\vspace{0.5em}
	\begin{longtable}{p{1.5cm}p{10cm}}
	.bss & 这个section保存着未初始化的数据,这些数据存放在程序内存映像中。当程序开始运行时,这些数据被初始化为0。该section不占文件空间。 \\
	.comment & 该section保存着版本控制的信息。\\
	.data .data1 & 这两个sections保存着初始化了的数据,这些数据存放在程序内存映像中。\\
	.debug & 该section保存着为符号调试的信息,这些信息内容是未指明的。\\
	.dynamic & 该section保存着动态链接的信息。\\
	.dynstr & 该section保存着动态链接是需要的字符串。一般情况下,名字字符串与符号表的entry相关联。\\
	.dynsym & 该section保存着动态符号表。\\
	.fini & 该section保存着可执行指令,这些指令构成了进程的终止代码。当一个程序正常退出时,系统将执行这个section中的指令。\\
	.got & 该section保存着全局的偏移量表。\\
	.hash & 该section保存着一个符号的哈希表。\\
	.init & 该section保存着可执行指令,这些指令构成了进程的初始化代码。当一个程序的main函数被调用之前,系统将执行这个section中的指令。\\
	.interp & 该section保存着程序的interpreter的路径。\\
	.line & 该section包含了用于符号debug的行数信息,它描述了源程序与机器代码之间的对应关系。该section的内容是未指明的。\\
	.plt & 该section保存着过程链接表。\\
	.rel .rela & 该section保存着重定位的信息。\\
	.rodata .rodata1 & 该section保存着只读数据,这些数据将在进程映像中构造不可写的段。\\
	.shstrtab &  该section保存着section的名称。\\
	.strtab & 该section保存着字符串。一般情况下,名字字符串与符号表的entry相关联。\\
	.symtab & 该section保存着一个符号表。\\
	.text & 该section保存着程序的可执行指令。\\
	\end{longtable}
	\vspace{0.5em}

\subsection{字符串表}
	字符串表保存着以NULL终止的一系列字符。object文件使用这些字符串来描述符号和section名。通过索引字符串表我们就能获得相应的字符串。
	字符串表的索引0是一个NULL字符,随后每个字符串都是以NULL终止的。例子如下:
	\fic{18.png}

	section header的sh\_name成员保存着字符串表的索引,用于指向section的名字。

\subsection{符号表}
	object文件的符号表存放着一个个条目,这些条目保存了一个程序在定位和重定位时需要的定义和引用的信息。
	符号表相当于一个数组,符号表索引是相应的下标。
	这些entry的定义代码如下:
	\begin{lstlisting}[language = C]
	typedef struct
	{
		Elf32_Word st_name;
		Elf32_Addr st_value;
		Elf32_Word st_size;
		unsigned char st_info;
		unsigned char st_other;
		Elf32_Half st_shndx;
	}Elf32_Sym;
	\end{lstlisting}

	下面介绍该结构体的各个成员:
	\vspace{0.5em}
	\begin{longtable}{p{2cm}p{8cm}}
	\hline
	st\_name & 这个成员的值是符号字符串表的索引。符号字符串表存放着符号的名字。如果该成员值为0,说明该符号无名。\\
	\hline
	st\_value & 该成员的值是相应的符号值。 \\
	\hline
	st\_size & 代表着符号的大小。\\
	\hline
	st\_info & 成员指出了符号的类型和相应的binding属性。\\
	\hline
	st\_other & 该成员值为0,没有相应的含义。\\
	\hline
	st\_shndx & 符号表条目有相对应的section。该成员的值是section header table的索引,用于定位section的section header。\\
	\hline
	\end{longtable}

\subsubsection{st\_info}
	st\_info包含了符号表的类型和binding属性。一个符号的binding决定了链接的可视性和行为。binding属性值定义了下列宏:
	\sizedfic{0.5}{19.png}

	下面介绍binding属性各个宏的意义:
	\vspace{0.5em}
	\begin{longtable}{p{3cm}p{8cm}}
	\hline
	STB\_LOCAL & 局部符号表,表示该符号只在当前object文件可见。\\
	\hline
	STB\_GLOBAL & 全局符号表,表示该符号对所有的object文件可见。\\
	\hline
	STB\_WEAK & 弱符号表,相当于全局符号,只是优先级较低。\\
	\hline
	STB\_LOPROC-STB\_HIPROC & 这个范围的值被保留,用于处理器特定的语义。\\
	\hline
	\end{longtable}
	\vspace{0.5em}

	全局符号与弱符号的区别如下:
	\begin{itemize}
		\item 当链接器链接几个可重定位的object文件时,它不允许STB\_GLOBAL符号的同名多重定义。如果全局符号与弱符号重名,将不会引起错误,而是选择忽略弱符号的定义。
			如果普通符号与弱符号重名,也不会引起错误,链接器同样是忽略弱符号的定义。
		\item 当链接器搜索档案库时,它将选出包含了未定义的全局符号的存档成员。这个成员的定义有可能是全局符号,也可能是弱符号。
		链接器不会为了解决一个未定义的弱符号而选出存档成员。未定义的弱符号值为0。
	\end{itemize}

	在每个符号表中,局部符号的优先级都高于弱符号和全局符号。\par

	st\_info还包含了符号的类型,它为相关的entry提供了分类。与类型相关的宏定义如下所示:
	\fic{20.png}

	下面介绍各个宏的具体含义:
	\interval
	\begin{longtable}{p{3cm}p{8cm}}
	\hline
	STT\_NOTYPE & 代表了该符号的类型没有指定。\\
	\hline
	STT\_OBJECT & 说明该符号和一个数据对象相关。\\
	\hline
	STT\_FUNC & 说明该符号和一个函数或其他可执行代码相关。这个类型的符号又称为函数符号。\\
	\hline
	STT\_SECTION & 说明该符号和一个section相关。这种类型的entry主要是为了重定位,一般情况下具有STB\_LOCAL约束。\\
	\hline
	STT\_FILE & 代表该符号给出了和目标文件相关的源文件名称。一个具有STB\_LOCAL约束的STT\_FILE类型的符号,它的section索引为SHN\_ABS。\\
	\hline
	STT\_LOPROC-STT\_HIPROC & 这个范围的值被保留,用于处理器特定的语义。\\
	\hline
	\end{longtable}
	\interval

	共享文件中函数符号具有特殊的意义。当其他的object文件从一个共享文件中引用一个函数时,链接器将自动地为引用符号创建一个链接表。
	除了函数符号,共享的目标符号将不会自动地通过链接表引用。

\subsubsection{st\_shndx}
	该成员的值是section header table的索引,用于定位section的section header。当重定位时,section将不断移动,该成员的值也会相应的变化。
	某些特殊的section有着相应特殊的st\_shndx,于是有相应的宏。这些宏的定义在介绍Sections的小节中有提到,如下所示:
	\interval
	\begin{longtable}{p{3cm}p{7cm}}
	\hline
	SHN\_ABS & 代表该符号的值不会随着重定位而变化。\\
	\hline
	SHN\_COMMON & 代表该符号标识了一个没有被分配的普通块,该符号的大小指出了需要的字节数。\\
	\hline
	SHN\_UNDEF & 代表该符号是未定义的。当链接器将该object文件和另一个定义了该符号的文件相链接时,该object文件中此符号将会被定义另一个文件中同名符号的定义。\\
	\hline
	\end{longtable}
	\interval

\subsubsection{st\_value}
	不同类型的object文件中st\_value成员的解释不同,相应的解释如下:
	\begin{itemize}
		\item 在可重定位文件中,st\_value的值代表一个符号相对于当前section的偏移量。如果当前section的索引是SHN\_COMMON,那么st\_value保存着符号的强制对齐值。
		\item 在可执行文件和可共享文件中,st\_value保存着一个虚拟地址。为了使得符号对于动态链接器更为有效,文件中的section偏移量将让步于内存层面上的虚拟地址。
	\end{itemize}

\subsubsection{符号表的0索引}
	符号表的0索引是保留的,它的各个值如下所示:
	\begin{lstlisting}[language = C]
	Elf32_Sym zero_index = table[0];
	zero_index.st_name = 0;
	zero_index.st_valu = 0;
	zero_index.st_size = 0;
	zero_index.st_info = 0;
	zero_index.st_other = 0;
	zero_index.st_shndx = SHN_UNDEF;
	\end{lstlisting}

\subsection{重定位}
	重定位是将符号引用与符号定义链接起来的过程。比如,当一个程序调用一个函数的时候,相关的调用必须在执行时把控制权转移到正确的目标地址。
	为了实现这个过程,重定位文件应当包含如何修改section内容的信息,从而保证可执行文件和可共享文件中存放着用于程序映像的信息。
	重定位条目定义如下:
	\begin{lstlisting}[language = C]
	typedef struct
	{
		Elf32_Addr r_offset;
		Elf32_Word r_info;
	}Elf32_Rel;

	typedef struct
	{
		Elf32_Addr r_offset;
		Elf32_Word r_info;
		Elf32_SWord r_addend;
	}Elf32_Rela;
	\end{lstlisting}

	对于结构体各个成员的解释如下:
	\interval
	\begin{longtable}{p{2cm}p{9cm}}
	\hline
	r\_offset & 该成员给出了重定位行为执行的位置。对于一个重定位文件而言,成员值是受重定位影响的存储单元相对于当前section的偏移量。
			    对于一个可执行文件或共享文件而言,成员值是受到重定位影响的存储单元的虚拟地址。\\
	\hline
	r\_info & 该成员给出了受重定位影响的符号表的索引值,同时还给出了所执行的重定位的类型。比如,call指令的重定位entry的r\_info存放着被调用函数的符号表索引值。
			  如果符号表索引值是STN\_UNDEF,那么成员值为0。重定位的类型与处理器有关。\\
	\hline
	r\_addend & 该成员给出了一个常量加数,用于计算重定位域的值。\\
	\hline
	\end{longtable}
	\interval

	从这两个结构体的定义可以看出,只有Elf32\_Rela的条目包含了显式的加数,Elf32\_Rel的条目存放着隐式的加数。
	对于不同的处理器架构而言,这两种entry中的一个可能是必要的或更为方便的。因此,对于特定的机器,使用哪一种entr将取决于上下文。\par

	一个重定位的section与一个符号表和一个可修改的section有关。section header中的sh\_info和sh\_link成员指明了这种关系。

\subsubsection{重定位类型}
	在介绍重定位类型的宏之前,为了简化描述,做了如下的标记。

	\interval
	\begin{longtable}{p{2cm}p{9cm}}
	\hline
	A & 表示用于计算重定位域的加数 \\
	\hline
	B & 表示在执行过程中一个共享目标被加载到内存时的基地址。一般情况下,一个共享文件使用的虚拟基地址为0。\\
	\hline
	G & 表示了相对于全局偏移表的偏移。重定位entry的符号在执行时将会驻留在这个表中。\\
	\hline
	GOT & 表示了全局偏移表的地址。\\
	\hline
	L & 表示了一个符号的过程链接表的位置。一个过程链接表entry将重定位一个函数调用到正确的目的单元。链接器将创建初始的过程链接表,在执行过程中动态链接器将会修改这些entry。\\
	\hline
	P & 表示了存储单元被重定位以后的位置。\\
	\hline
	S & 表示了某些特定符号的值,这些符号的索引驻留在重定位entry中。\\
	\hline
	\end{longtable}
	\interval

	重定位类型如下图所示:
	


\subsection{Program header}
\subsection{程序装载}
\subsection{动态链接}

\clearpage

\section{实现内核雏形}

\end{document}
