% !TeX spellcheck = en_US
%% 字体:方正静蕾简体
%%		 方正粗宋
\documentclass[a4paper,left=2.5cm,right=2.5cm,11pt]{article}

\usepackage[utf8]{inputenc}
\usepackage{fontspec}
\usepackage{cite}
\usepackage{xeCJK}
\usepackage{indentfirst}
\usepackage{titlesec}
\usepackage{longtable}
\usepackage{graphicx}
\usepackage{float}
\usepackage{rotating}
\usepackage{subfigure}
\usepackage{tabu}
\usepackage{amsmath}
\usepackage{setspace}
\usepackage{amsfonts}
\usepackage{appendix}
\usepackage{listings}
\usepackage{xcolor}
\usepackage{geometry}
\setcounter{secnumdepth}{4}
\usepackage{mhchem}
\usepackage{multirow}
\usepackage{extarrows}
\usepackage{hyperref}
\titleformat*{\section}{\LARGE}
\renewcommand\refname{参考文献}
\renewcommand{\abstractname}{\sihao \cjkfzcs 摘{  }要}
%\titleformat{\chapter}{\centering\bfseries\huge\wryh}{}{0.7em}{}{}
%\titleformat{\section}{\LARGE\bf}{\thesection}{1em}{}{}
\titleformat{\subsection}{\Large\bfseries}{\thesubsection}{1em}{}{}
\titleformat{\subsubsection}{\large\bfseries}{\thesubsubsection}{1em}{}{}
\renewcommand{\contentsname}{{\cjkfzcs \centerline{目{  } 录}}}
\setCJKfamilyfont{cjkhwxk}{STXingkai}
\setCJKfamilyfont{cjkfzcs}{STSongti-SC-Regular}
% \setCJKfamilyfont{cjkhwxk}{华文行楷}
% \setCJKfamilyfont{cjkfzcs}{方正粗宋简体}
\newcommand*{\cjkfzcs}{\CJKfamily{cjkfzcs}}
\newcommand*{\cjkhwxk}{\CJKfamily{cjkhwxk}}
\newfontfamily\wryh{Microsoft YaHei}
\newfontfamily\hwzs{STZhongsong}
\newfontfamily\hwst{STSong}
\newfontfamily\hwfs{STFangsong}
\newfontfamily\jljt{MicrosoftYaHei}
\newfontfamily\hwxk{STXingkai}
% \newfontfamily\hwzs{华文中宋}
% \newfontfamily\hwst{华文宋体}
% \newfontfamily\hwfs{华文仿宋}
% \newfontfamily\jljt{方正静蕾简体}
% \newfontfamily\hwxk{华文行楷}
\newcommand{\verylarge}{\fontsize{60pt}{\baselineskip}\selectfont}  
\newcommand{\chuhao}{\fontsize{44.9pt}{\baselineskip}\selectfont}  
\newcommand{\xiaochu}{\fontsize{38.5pt}{\baselineskip}\selectfont}  
\newcommand{\yihao}{\fontsize{27.8pt}{\baselineskip}\selectfont}  
\newcommand{\xiaoyi}{\fontsize{25.7pt}{\baselineskip}\selectfont}  
\newcommand{\erhao}{\fontsize{23.5pt}{\baselineskip}\selectfont}  
\newcommand{\xiaoerhao}{\fontsize{19.3pt}{\baselineskip}\selectfont} 
\newcommand{\sihao}{\fontsize{14pt}{\baselineskip}\selectfont}      % 字号设置  
\newcommand{\xiaosihao}{\fontsize{12pt}{\baselineskip}\selectfont}  % 字号设置  
\newcommand{\wuhao}{\fontsize{10.5pt}{\baselineskip}\selectfont}    % 字号设置  
\newcommand{\xiaowuhao}{\fontsize{9pt}{\baselineskip}\selectfont}   % 字号设置  
\newcommand{\liuhao}{\fontsize{7.875pt}{\baselineskip}\selectfont}  % 字号设置  
\newcommand{\qihao}{\fontsize{5.25pt}{\baselineskip}\selectfont}    % 字号设置 

\usepackage{diagbox}
\usepackage{multirow}
\boldmath
\XeTeXlinebreaklocale "zh"
\XeTeXlinebreakskip = 0pt plus 1pt minus 0.1pt
\definecolor{cred}{rgb}{0.8,0.8,0.8}
\definecolor{cgreen}{rgb}{0,0.3,0}
\definecolor{cpurple}{rgb}{0.5,0,0.35}
\definecolor{cdocblue}{rgb}{0,0,0.3}
\definecolor{cdark}{rgb}{0.95,1.0,1.0}
\lstset{
	language=c,
	numbers=left,
	numberstyle=\tiny\color{black},
	showspaces=false,
	showstringspaces=false,
	basicstyle=\scriptsize,
	keywordstyle=\color{purple},
	commentstyle=\itshape\color{cgreen},
	stringstyle=\color{blue},
	frame=lines,
	% escapeinside=``,
	extendedchars=true, 
	xleftmargin=1em,
	xrightmargin=1em, 
	backgroundcolor=\color{cred},
	aboveskip=1em,
	breaklines=true,
	tabsize=4
} 

\newfontfamily{\consolas}{Consolas}
\newfontfamily{\monaco}{Monaco}
\setmonofont[Mapping={}]{Consolas}	%英文引号之类的正常显示,相当于设置英文字体
\setsansfont{Consolas} %设置英文字体 Monaco, Consolas,  Fantasque Sans Mono
\setmainfont{Times New Roman}

\setCJKmainfont{华文中宋}


\newcommand{\fic}[1]{\begin{figure}[H]
		\center
		\includegraphics[width=0.8\textwidth]{#1}
	\end{figure}}
	
\newcommand{\sizedfic}[2]{\begin{figure}[H]
		\center
		\includegraphics[width=#1\textwidth]{#2}
	\end{figure}}

\newcommand{\codefile}[1]{\lstinputlisting{#1}}

% 改变段间隔
\setlength{\parskip}{0.2em}
\linespread{1.1}

\usepackage{lastpage}
\usepackage{fancyhdr}
\pagestyle{fancy}
\lhead{\space \qquad \space}
\chead{输入输出系统 \qquad}
\rhead{\qquad\thepage/\pageref{LastPage}}
\begin{document}

\tableofcontents

\clearpage

\section{键盘}
	对于处理键盘输入,我们本能是想用中断去处理。在此之前,我们先学习计算机响应键盘的过程。\par

\subsection{键盘敲击的过程}
	在键盘中存在一枚叫做键盘编码器的芯片Intel 8048,
	在计算机中有一个键盘控制器,是intel 8042芯片。\par

	过程如下:
	\begin{itemize}
		\item[1.] 8048芯片检测到一个键的动作,将相应的扫描码发送给8042。
		\item[2.] 8042讲扫描码转换成相应的Scan code set1扫描码,然后放置在输入缓冲区中。
		\item[3.] 此后8042告诉8259A产生中断,如果此时键盘又有新的键被按下,8042将不再接收,一直到缓冲区被清空,8042才能收到更多的扫描码。
	\end{itemize}

\subsection{读取键盘输入}
	8042的输出缓冲区的端口是0x60,我们可以直接去通过in\_byte()函数去读取。\par

	但为了真正处理键盘输入,我们还需要了解键盘编码:
	\begin{lstlisting}
	敲击键盘产生的编码被称为扫描码:
		当一个键被按下时,将产生Make Code,
		当一个键弹起时,产生Break Code。
	除了Pause键之外,每一个按键都对应一个Make Code和一个Break Code。
	\end{lstlisting}

	在代码中,为了处理Break Code和Make Code,可以创建一个数组:
	\begin{lstlisting}
u32 keymap[NR_SCAN_CODES * MAP_COLS] = {

/* scan-code			!Shift		Shift		E0 XX	*/
/* ==================================================================== */
/* 0x00 - none		*/	0,		0,		0,
/* 0x01 - ESC		*/	ESC,		ESC,		0,
/* 0x02 - '1'		*/	'1',		'!',		0,
/* 0x03 - '2'		*/	'2',		'@',		0,
/* 0x04 - '3'		*/	'3',		'#',		0,
/* 0x05 - '4'		*/	'4',		'$',		0,
/* 0x06 - '5'		*/	'5',		'%',		0,
/* 0x07 - '6'		*/	'6',		'^',		0,
/* 0x08 - '7'		*/	'7',		'&',		0,
/* 0x09 - '8'		*/	'8',		'*',		0,
/* 0x0A - '9'		*/	'9',		'(',		0,
/* 0x0B - '0'		*/	'0',		')',		0,
/* 0x0C - '-'		*/	'-',		'_',		0,
/* 0x0D - '='		*/	'=',		'+',		0,
/* 0x0E - BS		*/	BACKSPACE,	BACKSPACE,	0,
/* 0x0F - TAB		*/	TAB,		TAB,		0,
/* 0x10 - 'q'		*/	'q',		'Q',		0,
/* 0x11 - 'w'		*/	'w',		'W',		0,
/* 0x12 - 'e'		*/	'e',		'E',		0,
/* 0x13 - 'r'		*/	'r',		'R',		0,
/* 0x14 - 't'		*/	't',		'T',		0,
/* 0x15 - 'y'		*/	'y',		'Y',		0,
/* 0x16 - 'u'		*/	'u',		'U',		0,
/* 0x17 - 'i'		*/	'i',		'I',		0,
/* 0x18 - 'o'		*/	'o',		'O',		0,
/* 0x19 - 'p'		*/	'p',		'P',		0,
/* 0x1A - '['		*/	'[',		'{',		0,
/* 0x1B - ']'		*/	']',		'}',		0,
/* 0x1C - CR/LF		*/	ENTER,		ENTER,		PAD_ENTER,
/* 0x1D - l. Ctrl	*/	CTRL_L,		CTRL_L,		CTRL_R,
/* 0x1E - 'a'		*/	'a',		'A',		0,
/* 0x1F - 's'		*/	's',		'S',		0,
/* 0x20 - 'd'		*/	'd',		'D',		0,
/* 0x21 - 'f'		*/	'f',		'F',		0,
/* 0x22 - 'g'		*/	'g',		'G',		0,
/* 0x23 - 'h'		*/	'h',		'H',		0,
/* 0x24 - 'j'		*/	'j',		'J',		0,
/* 0x25 - 'k'		*/	'k',		'K',		0,
/* 0x26 - 'l'		*/	'l',		'L',		0,
/* 0x27 - ';'		*/	';',		':',		0,
/* 0x28 - '\''		*/	'\'',		'"',		0,
/* 0x29 - '`'		*/	'`',		'~',		0,
/* 0x2A - l. SHIFT	*/	SHIFT_L,	SHIFT_L,	0,
/* 0x2B - '\'		*/	'\\',		'|',		0,
/* 0x2C - 'z'		*/	'z',		'Z',		0,
/* 0x2D - 'x'		*/	'x',		'X',		0,
/* 0x2E - 'c'		*/	'c',		'C',		0,
/* 0x2F - 'v'		*/	'v',		'V',		0,
/* 0x30 - 'b'		*/	'b',		'B',		0,
/* 0x31 - 'n'		*/	'n',		'N',		0,
/* 0x32 - 'm'		*/	'm',		'M',		0,
/* 0x33 - ','		*/	',',		'<',		0,
/* 0x34 - '.'		*/	'.',		'>',		0,
/* 0x35 - '/'		*/	'/',		'?',		PAD_SLASH,
/* 0x36 - r. SHIFT	*/	SHIFT_R,	SHIFT_R,	0,
/* 0x37 - '*'		*/	'*',		'*',    	0,
/* 0x38 - ALT		*/	ALT_L,		ALT_L,  	ALT_R,
/* 0x39 - ' '		*/	' ',		' ',		0,
/* 0x3A - CapsLock	*/	CAPS_LOCK,	CAPS_LOCK,	0,
/* 0x3B - F1		*/	F1,		F1,		0,
/* 0x3C - F2		*/	F2,		F2,		0,
/* 0x3D - F3		*/	F3,		F3,		0,
/* 0x3E - F4		*/	F4,		F4,		0,
/* 0x3F - F5		*/	F5,		F5,		0,
/* 0x40 - F6		*/	F6,		F6,		0,
/* 0x41 - F7		*/	F7,		F7,		0,
/* 0x42 - F8		*/	F8,		F8,		0,
/* 0x43 - F9		*/	F9,		F9,		0,
/* 0x44 - F10		*/	F10,		F10,		0,
/* 0x45 - NumLock	*/	NUM_LOCK,	NUM_LOCK,	0,
/* 0x46 - ScrLock	*/	SCROLL_LOCK,	SCROLL_LOCK,	0,
/* 0x47 - Home		*/	PAD_HOME,	'7',		HOME,
/* 0x48 - CurUp		*/	PAD_UP,		'8',		UP,
/* 0x49 - PgUp		*/	PAD_PAGEUP,	'9',		PAGEUP,
/* 0x4A - '-'		*/	PAD_MINUS,	'-',		0,
/* 0x4B - Left		*/	PAD_LEFT,	'4',		LEFT,
/* 0x4C - MID		*/	PAD_MID,	'5',		0,
/* 0x4D - Right		*/	PAD_RIGHT,	'6',		RIGHT,
/* 0x4E - '+'		*/	PAD_PLUS,	'+',		0,
/* 0x4F - End		*/	PAD_END,	'1',		END,
/* 0x50 - Down		*/	PAD_DOWN,	'2',		DOWN,
/* 0x51 - PgDown	*/	PAD_PAGEDOWN,	'3',		PAGEDOWN,
/* 0x52 - Insert	*/	PAD_INS,	'0',		INSERT,
/* 0x53 - Delete	*/	PAD_DOT,	'.',		DELETE,
/* 0x54 - Enter		*/	0,		0,		0,
/* 0x55 - ???		*/	0,		0,		0,
/* 0x56 - ???		*/	0,		0,		0,
/* 0x57 - F11		*/	F11,		F11,		0,	
/* 0x58 - F12		*/	F12,		F12,		0,	
/* 0x59 - ???		*/	0,		0,		0,	
/* 0x5A - ???		*/	0,		0,		0,	
/* 0x5B - ???		*/	0,		0,		GUI_L,	
/* 0x5C - ???		*/	0,		0,		GUI_R,	
/* 0x5D - ???		*/	0,		0,		APPS,	
/* 0x5E - ???		*/	0,		0,		0,	
/* 0x5F - ???		*/	0,		0,		0,
/* 0x60 - ???		*/	0,		0,		0,
/* 0x61 - ???		*/	0,		0,		0,	
/* 0x62 - ???		*/	0,		0,		0,	
/* 0x63 - ???		*/	0,		0,		0,	
/* 0x64 - ???		*/	0,		0,		0,	
/* 0x65 - ???		*/	0,		0,		0,	
/* 0x66 - ???		*/	0,		0,		0,	
/* 0x67 - ???		*/	0,		0,		0,	
/* 0x68 - ???		*/	0,		0,		0,	
/* 0x69 - ???		*/	0,		0,		0,	
/* 0x6A - ???		*/	0,		0,		0,	
/* 0x6B - ???		*/	0,		0,		0,	
/* 0x6C - ???		*/	0,		0,		0,	
/* 0x6D - ???		*/	0,		0,		0,	
/* 0x6E - ???		*/	0,		0,		0,	
/* 0x6F - ???		*/	0,		0,		0,	
/* 0x70 - ???		*/	0,		0,		0,	
/* 0x71 - ???		*/	0,		0,		0,	
/* 0x72 - ???		*/	0,		0,		0,	
/* 0x73 - ???		*/	0,		0,		0,	
/* 0x74 - ???		*/	0,		0,		0,	
/* 0x75 - ???		*/	0,		0,		0,	
/* 0x76 - ???		*/	0,		0,		0,	
/* 0x77 - ???		*/	0,		0,		0,	
/* 0x78 - ???		*/	0,		0,		0,	
/* 0x78 - ???		*/	0,		0,		0,	
/* 0x7A - ???		*/	0,		0,		0,	
/* 0x7B - ???		*/	0,		0,		0,	
/* 0x7C - ???		*/	0,		0,		0,	
/* 0x7D - ???		*/	0,		0,		0,	
/* 0x7E - ???		*/	0,		0,		0,
/* 0x7F - ???		*/	0,		0,		0
};
	\end{lstlisting}

	建立这个数组以后,我们就可以通过扫描码直接对应到真正的字母了。\par

	这个数组还考虑到了按住shift键时的字母对应关系。

\subsection{建立键盘输入缓冲区}
	因为我们输入键盘时可以会同时按住几个键,所以我们要在中断处理程序中处理这种情况。\par

	考虑到因为扫描码的值和长度不一样,如果直接处理扫描码,可能会搞得代码很冗长。
	在这里建立一个键盘输入的缓冲区,让中断处理程序将每次收到的扫描码放入这个缓冲区,然后建立一个新的任务专门用来解析这些扫描码并做相应的处理。\par

	键盘缓冲区如下:
	\begin{lstlisting}
	typedef struct s_kb
	{
		char* p_head;
		char* p_tail;
		int count;
		char buf[KB_IN_BYTES];
	}KB_INPUT;	
	\end{lstlisting}

	创建对应的键盘中断处理程序:
	\begin{lstlisting}
	PRIVATE KB_INPUT kb_in;

	PUBLIC void keyboard_handler(int irq)
	{
		u8 scan_code = in_byte(0x60);
		if(kb_in.count < KB_IN_BYTES)
		{
			*(kv_in.p_head) = scan_code;
			kb_in.p_head++;
			if(kv_in.p_head == kb_in.buf + KB_IN_BYTES)
				kb_in.p_head = kb_in.buf;
			kb_in.count++;
		}
	}

	// 初始化键盘中断处理程序
	PUBLIC void init_keyboard()
	{
		kb_in.count = 0;
		kb_in.p_head = kb_in.p_tail = kb_in.buf;

		put_irq_handler(KEYBOARD_IRQ, keyboard_handler);
		enable_irq(KEYBOARD_IRQ);
	}

	// 打开键盘中断
	PUBLIC int kernel_main()
	{
		...
		init_keyboard();
		...
	}
	\end{lstlisting}

\subsection{添加新的进程处理缓冲区}
\subsubsection{读取缓冲区}
	函数体如下:
	\begin{lstlisting}
	PUBLIC void task_tty()
	{
		while(1)
		{
			keyboard_read();
		}
	}

	PUBLIC void keyboard_read()
	{
		u8 scan_code;
		if(kb_in.count > 0)
		{
			disable_int();
			scan_code = *(kb_in.p_tail);
			kb_in.p_tail++;
			if(kb_in.p_tail == kb_in.buf + KB_IN_BYTES)
				kb_in.p_tail = kb_in.buf;
			kb_in.count--;
			enable_int();
			disp_int(scan_code);
		}
	}
	\end{lstlisting}

\subsection{解析扫描码}
\subsubsection{显示普通的字符}
	查看扫描码,会知道以0xE0和0xE1开头的扫描码对应的按键一般比较特殊,我们先跳过处理。\par

	下面先进行最基本的字符显示:
	\begin{lstlisting}
	PUBLIC void keyboard_read()
	{
		char output[2];
		int make;

		memset(output, 0, 2);

		if(kb_in.count > 0)
		{
			...
			// 扫描码已经读入scan_code
			if(scan_code == 0xE1)
			{

			}
			else if(scan_code == 0xE0)
			{

			}
			else
			{
				// 如果make是FALSE,代表这个码是break code,否则就是make code
				make = (scan_code & FLAG_BREAK ? FALSE : TRUE);
				output[0] = keymap[(scan_code&0x7F)*MAP_COLS];

				if(make)
				{
					disp_str(output);
				}
			}
		}
	}
	\end{lstlisting}

\subsubsection{处理shift、alt和ctrl}
	处理这些组合按键的算法如下:
	\begin{itemize}
		\item[1.] 首先判断是否是shift、alt或者ctrl。这里对这些按键都声明一个标识符,如果按下就为1,没按下就为0,可以直接用make这个标识符来对它们赋值。此外,还另外标记一个key标识符,如果是这些特殊字符或者break code,key就为0,否则为按键实际值。
		\item[2.] 还记得我们定义keymap数组时,定义了三列数据。第二列数据就是shift按下时的字符显示情况。如果在处理中,遇到上述shift标识符为1的情况,就对列进行相应的操作。如果是其他标识符,就进行其他的操作。
		\item[3.] 检查key标识符,如果是1就显示字符。
	\end{itemize}

	代码如下:
	\begin{lstlisting}
	PRIVATE int shift_l;
	PRIVATE int shift_r;
	PRIVATE int alt_l;
	PRIVATE int alt_r;
	PRIVATE int ctrl_l;
	PRIVATE int ctrl_r;
	PRIVATE int caps_lock;
	PRIVATE int num_lock;
	PRIVATE int scroll_lock;
	PRIVATE int column;

	PUBLIC void keyboard_read()
	{
		u32 key = 0;
		u32* keyrow;

		if(kb_in.count > 0)
		{
			...
			if(scan_code == 0xE1)
			{

			}
			else if(scan_code == 0xE0)
			{

			}
			else
			{
				// 得到扫描码对应的那一行数据
				keyrow = &keymap[(scan_code & 0x7F) * MAP_COLS];
				column = 0;

				if(shift_l || shift_r)
					column = 1;
				
				key = keyrow[column];

				switch(key)
				{
					case SHIFT_L:
						shift_l = make;
						key = 0;
						break;
					case SHIFT_R:
						shift_r = make;
						key = 0;
						break;
					case CTRL_L:
						ctrl_l = make;
						key = 0;
						break;
					case CTRL_R:
						ctrl_r = make;
						key = 0;
						break;
					case ALT_L:
						alt_l = make;
						key = 0;
						break;
					case ALT_R:
						alt_r = make;
						key = 0;
						break;
					default:
						if(!make)
							key = 0;
						break;
				}

				if(key)
				{
					output[0] = key;
					disp_str(output);
				}
			}
		}
	}
	\end{lstlisting}

\subsubsection{处理不可打印的字符}
	这里我们将keyboard\_read()函数再做一个处理,将它变成单纯读取键盘缓冲区的函数,然后增加一个in\_process()函数,用于专门处理读出的字符。\par

	注意,我们还处理了不可打印的字符,算法如下:
	\begin{itemize}
		\item[1.] 为每一个不可打印的按键定义一个宏,键值key在传入in\_process()函数之前先根据实际情况选择是否与这些宏相或。
		\item[2.] 在in\_process()函数中,再与一个宏相与,从而决定是否打印该字符。
	\end{itemize}

	代码如下:
	\begin{lstlisting}
	#define SHIFT_L (0x08 + FLAG_EXT)
	#define SHIFT_R (0x09 + FLAG_EXT)
	#define CTRL_L (0x0A + FLAG_EXT)
	#define CTRL_R (0x0B + FLAG_EXT)
	#define ALT_L (0x0C + FLAG_EXT)
	#define ALT_R (0x0D + FLAG_EXT)

	PUBLIC void keyboard_read()
	{
		...
		// 键值key在传入in_process()函数之前先根据实际情况选择是否与这些宏相或
		if(make)
		{
			key |= (shift_l ? FLAG_SHIFT_L : 0);
			key |= (shift_r ? FLAG_SHIFT_R : 0);
			key |= (ctrl_l ? FLAG_CTRL_L : 0);
			key |= (ctrl_r ? FLAG_CTRL_R : 0);
			key |= (alt_l ? FLAG_ALT_L : 0);
			key |= (alt_r ? FLAG_ALT_R : 0);

			in_process(key);
		}
	}

	PUBLIC void in_process(u32 key)
	{
		char output[2];
		memset(output, 0, 2);

		// 在in_process()函数中,再与一个宏相与,从而决定是否打印该字符
		if(!(key & FLAG_EXT))
		{
			output[0] = key & 0xFF;
			disp_str(output);
		}
	}
	\end{lstlisting}

\section{TTY任务}
	我们接下来要实现TTY,首先需要知道一些基本框架:
	\begin{itemize}
		\item[1.] 每一个TTY都有自己读和写的动作,所以在keyboard\_read()函数中需要知道当前是在哪个TTY下,所以我们需要为这个函数传入一个参数。
		\item[2.] TTY任务应该自己负责自己的显示工作,所以in\_process()函数不再负责显示任务,而是交给TTY来完成。
		\item[3.] 每个TTY回显字符时操作的console是不同的,所以需要有一个成员来记载其对应的console信息。
	\end{itemize}

\subsection{TTY的一些数据结构}
	为了让keyboard\_read()函数知道是哪个TTY调用它,需要定义一个TTY结构。
	为了将TTY结构与特定的console关联起来,需要定义一个console结构,并作为TTY结构的一个成员:
	\begin{lstlisting}
	struct s_console;

	typedef struct s_tty
	{
		u32 in_buf[TTY_IN_BYTES];
		u32* p_inbuf_head;
		u32* p_inbuf_tail;
		int inbuf_count;

		struct s_console* p_console;
	}TTY;

	typedef struct s_console
	{
		unsigned int current_start_addr;
		unsigned int original_addr;
		unsigned int v_mem_limit;
		unsigned int cursor;
	}CONSOLE;
	\end{lstlisting}

\subsection{构建TTY框架}
	首先文字描述一下这个框架:
	\begin{itemize}
		\item[1.] 定义三个TTY结构,在tty任务中对这三个TTY进行循环,一次进行读操作和写操作。
		\item[2.] 如果控制台是当前控制台,TTY就可以读取键盘。
		\item[3.] 在keyboard\_read()函数中,首先处理键盘扫描码,然后将键值key传入in\_process()函数。
		\item[4.] 在in\_process()函数中,将键值key写入TTY缓冲区。
		\item[5.] tty任务执行tty\_do\_write()函数,然后直接向显存写入要显示的字符,还通过操作VGA的寄存器来设置光标。
	\end{itemize}

	首先改写tty任务的结构体,加入对TTY结构体的操作:
	\begin{lstlisting}
	#define NR_CONSOLES 3

	PUBLIC TTY tty_table[NR_CONSOLES];
	PUBLIC CONSOLE console_table[NR_CONSOLES];

	#define TTY_FIRST (tty_table)
	#define TTY_END (tty_table + NR_CONSOLES)

	PUBLIC void init_tty(TTY* p_tty);

	PUBLIC void task_tty()
	{
		TTY* p_tty;
		init_keyboard();
		for(p_tty = TTY_FIRST; p_tty < TTY_END; ++p_tty)
		{
			init_tty(p_tty);
		}

		nr_current_console = 0;
		while(1)
		{
			for(p_tty=TTY_FIRST; p_tty<TTY_END; ++p_tty)
			{
				tty_do_read(p_tty);
				tty_do_write(p_tty);
			}
		}
	}

	PRIVATE void init_tty(TTY* p_tty)
	{
		p_tty->inbuf_count = 0;
		p_tty->p_inbuf_head = p_tty->p_inbuf_tail = p_tty->in_buf;

		int nr_tty = p_tty - tty_table;
		p_tty->p_console = console_table + nr_tty;
	}
	\end{lstlisting}

	在调用tty\_do\_read()函数时,如果是当前console,就可以读取键盘缓冲区,然后用in\_process()函数将读取的键值key写入tty缓冲区:
	\begin{lstlisting}
	PUBLIC int is_current_console(CONSOLE* p_console);

	PRIVATE void tty_do_read(TTY* p_tty)
	{
		if(is_current_console(p_tty->p_console))
			keyboard_read(p_tty);
	}

	PUBLIC int is_current_console(CONSOLE* p_console)
	{
		return (p_console == &console_table[nr_current_console]);
	}

	PUBLIC void in_process(TTY* p_tty, u32 key)
	{
		char output[2];
		memset(output, 0, 2);

		if(!(key & FLAG_EXT))
		{
			if(p_tty->inbuf_count < TTY_IN_BYTES)
			{
				*(p_tty->p_inbuf_head) = key;
				p_tty->p_inbuf_head++;
				if(p_tty->p_inbuf_head == p_tty->in_buf + TTY_IN_BYTES)
					p_tty->p_inbuf_head = p_tty->in_buf;
				p_tty->inbuf_count++;
			}
		}
	}
	\end{lstlisting}

	tty任务执行tty\_do\_write()函数,然后直接向显存写入要显示的字符:
	\begin{lstlisting}
	PUBLIC void out_char(CONSOLE* p_console, char ch);

	PRIVATE void tty_do_write(TTY* p_tty)
	{
		if(p_tty->inbuf_count > 0)
		{
			char ch = *(p_tty->p_inbuf_tail);
			
			p_tty->p_inbuf_tail++;
			if(p_tty->p_inbuf_tail == p_tty->in_buf + TTY_IN_BYTES)
				p_tty->p_inbuf_tail = p_tty->in_buf;
			
			out_char(p_tty->p_console, ch);
		}
	}

	PUBLIC void out_char(CONSOLE* p_console, char ch)
	{
		u8* p_vmem = (u8*)(V_MEM_BASE + disp_pos);

		*p_vmem++ = ch;
		*p_vmem++ = DEFAULT_CHAR_COLOR;
		disp_pos += 2;

		set_cursor(disp_pos/2);
	}
	\end{lstlisting}

	这里的set\_cursor()函数是通过向寄存器写入值来进行光标的设置:
	\begin{lstlisting}
	PRIVATE void set_cursor(unsigned int position)
	{
		disable_int();
		out_byte(CRTC_ADDR_REG, CURSOR_H);
		out_byte(CRTC_DATA_REG, (position >> 8) & 0xFF);
		out_byte(CRTC_ADDR_REG, CURSOR_L);
		out_byte(CRTC_DATA_REG, position & 0xFF);
	}
	\end{lstlisting}

\subsection{多控制台}
	为了显示多个控制台,我们应该构建如下框架:
	\begin{itemize}
		\item[1.] 初始化每个tty结构中的console成员,设置它们初始对应的显存。
		\item[2.] 将outchar()与console结构体挂钩,根据结构体中的cursor成员进行字符的显示。
		\item[3.] 在in\_process()函数中,添加对Alt+Fn的处理。当按下Alt+Fn时,进行控制台的切换。
	\end{itemize}

	初始化每个tty结构中的console成员,设置它们初始对应的显存:
	\begin{lstlisting}
	PUBLIC void init_screen(TTY* p_tty)
	{
		int nr_tty = p_tty - tty_table;
		p_tty->p_console = console_table + nr_tty;
		int v_mem_size = V_MEM_BASE >> 1;
		int con_v_mem_size = v_mem_size / NR_CONSOLES;
		p_tty->p_console->original_addr = nr_tty * con_v_mem_size;
		p_tty->p_console->v_mem_limit = con_v_mem_size;
		p_tty->p_console->current_start_addr = p_tty->console->original_addr;
		p_tty->p_console->cursor = p_tty->p_console->original_addr;

		if(nr_tty == 0)
		{
			p_tty->p_console->cursor = disp_pos / 2;
			disp_pos = 0;
		}
		else
		{
			out_char(p_tty->p_console, nr_tty + '0');
			out_char(p_tty->p_console, '#');
		}

		set_cursor(p_tty->p_console->cursor);
	}
	\end{lstlisting}

	将outchar()与console结构体挂钩,根据结构体中的cursor成员进行字符的显示:
	\begin{lstlisting}
	PUBLIC void out_char(CONSOLE* p_console, char ch)
	{
		u8* p_vmem = (u8*)(V_MEM_BASE + p_console->cursor * 2);
		*p_vmem++ = ch;
		*p_vmem++ = DEFAULT_CHAR_COLOR;
		p_console->cursor++;
		set_cursor(p_console->cursor);
	}
	\end{lstlisting}

	在in\_process()函数中,添加对Alt+Fn的处理。当按下Alt+Fn时,进行控制台的切换:
	\begin{lstlisting}
	PUBLIC void select_console(int nr_console)
	{
		if((nr_console < 0) || (nr_console >= NR_CONSOLES))
			return;

		nr_current_console = nr_console;
		set_cursor(console_table[nr_console].cursor);
		set_vedio_start_addr(console_table[nr_console].current_start_addr);
	}

	PUBLIC void in_process(TTY* p_tty, u32 key)
	{
		...
		else
		{
			case F1:
			case F2:
			case F3:
			case F4:
			case F5:
			case F6:
			case F7:
			case F8:
			case F9:
			case F10:
			case F11:
			case F12:
				if((key & FLAG_ALT_L) || (key & FLAG_ALT_R))
					select_console(raw_code - F1);
				break;
		}
	}
	\end{lstlisting}	

\section{实现printf函数}
	这里的printf函数就是我们熟悉的printf函数,当某个进程调用printf()函数时,操作系统就会向控制台输出字符。\par

	实现printf函数的步骤如下:
	\begin{itemize}
		\item[1.] 向进程表添加成员,为每个进程指定特定的TTY。
		\item[2.] 实现系统调用write,这里的write()函数借助我们之前实现的TTY结构和outchar()进行实现。
		\item[3.] 借助系统调用write()实现printf函数。
	\end{itemize}

	向进程表添加成员,为每个进程指定特定的TTY:
	\begin{lstlisting}
	typedef struct s_proc
	{
		STACK_FRAME regs;
		u16 ldt_sel;

		int ticks;
		int priority;

		u32 pid;
		char p_name[16];

		int nr_tty;
	}
	\end{lstlisting}

	实现系统调用write():
	\begin{lstlisting}
	// 添加关于write()函数的声明
	PUBLIC void sys_write(char* buf, int len, PROCESS* p_proc);
	PUBLIC void write(char* buf, int len);

	// 编写write()的函数体
	write:
		mov eax, _NR_write
		mov ebx, [esp + 4]
		mov ecx, [esp + 8]
		int INT_VECTOR_SYS_CALL
		ret

	// 编写sys_write()的函数体
	PUBLIC void tty_write(TTY* p_tty, char* buf, int len)
	{
		char* p = buf;
		int i = len;

		while(i)
		{
			out_char(p_tty->p_console, *p++);
			i--;
		}
	}

	PUBLIC int sys_write(char* buf, int len, PROCESS* p_proc)
	{
		tty_write(&tty_table[p_proc->nr_tty], buf, len);
		return 0;
	}
	\end{lstlisting}

	借助系统调用write()实现printf函数:
	\begin{lstlisting}
	int printf(const char* fmt, ...)
	{
		int i;
		char buf[256];

		va_list arg = (va_list)((char*)(&fmt) + 4);
		i = vsprintf(buf, fmt, arg);
		write(buf, i);
		return i;
	}

	int vsprintf(char* buf, const char* fmt, va_list args)
	{
		char* p;
		char tmp[256];
		va_list p_next_arg = args;

		for(p=buf; *fmt; fmt++)
		{
			if(*fmt != '%')
			{
				*p++ = *fmt;
				continue;
			}

			fmt++;

			switch(*fmt)
			{
				case 'x':
					itoa(tmp, *((int*)p_next_arg));
					strcpy(p, tmp);
					p_next_arg += 4;
					p += strlen(tmp);
					break;
				case 's':
					while(((char*)p_next_arg) != '\0')
					{
						*p = (char*)p_next_arg;
						p++;
						p_next_arg++;
					}
					break;
				default:
					break;
			}
		}
	}
	\end{lstlisting}

\end{document}
