% !TeX spellcheck = en_US
%% 字体:方正静蕾简体
%%		 方正粗宋
\documentclass[a4paper,left=2.5cm,right=2.5cm,11pt]{article}

\usepackage[utf8]{inputenc}
\usepackage{fontspec}
\usepackage{cite}
\usepackage{xeCJK}
\usepackage{indentfirst}
\usepackage{titlesec}
\usepackage{longtable}
\usepackage{graphicx}
\usepackage{float}
\usepackage{rotating}
\usepackage{subfigure}
\usepackage{tabu}
\usepackage{amsmath}
\usepackage{setspace}
\usepackage{amsfonts}
\usepackage{appendix}
\usepackage{listings}
\usepackage{xcolor}
\usepackage{geometry}
\setcounter{secnumdepth}{4}
\usepackage{mhchem}
\usepackage{multirow}
\usepackage{extarrows}
\usepackage{hyperref}
\titleformat*{\section}{\LARGE}
\renewcommand\refname{参考文献}
\renewcommand{\abstractname}{\sihao \cjkfzcs 摘{  }要}
%\titleformat{\chapter}{\centering\bfseries\huge\wryh}{}{0.7em}{}{}
%\titleformat{\section}{\LARGE\bf}{\thesection}{1em}{}{}
\titleformat{\subsection}{\Large\bfseries}{\thesubsection}{1em}{}{}
\titleformat{\subsubsection}{\large\bfseries}{\thesubsubsection}{1em}{}{}
\renewcommand{\contentsname}{{\cjkfzcs \centerline{目{  } 录}}}
\setCJKfamilyfont{cjkhwxk}{STXingkai}
\setCJKfamilyfont{cjkfzcs}{STSongti-SC-Regular}
% \setCJKfamilyfont{cjkhwxk}{华文行楷}
% \setCJKfamilyfont{cjkfzcs}{方正粗宋简体}
\newcommand*{\cjkfzcs}{\CJKfamily{cjkfzcs}}
\newcommand*{\cjkhwxk}{\CJKfamily{cjkhwxk}}
\newfontfamily\wryh{Microsoft YaHei}
\newfontfamily\hwzs{STZhongsong}
\newfontfamily\hwst{STSong}
\newfontfamily\hwfs{STFangsong}
\newfontfamily\jljt{MicrosoftYaHei}
\newfontfamily\hwxk{STXingkai}
% \newfontfamily\hwzs{华文中宋}
% \newfontfamily\hwst{华文宋体}
% \newfontfamily\hwfs{华文仿宋}
% \newfontfamily\jljt{方正静蕾简体}
% \newfontfamily\hwxk{华文行楷}
\newcommand{\verylarge}{\fontsize{60pt}{\baselineskip}\selectfont}  
\newcommand{\chuhao}{\fontsize{44.9pt}{\baselineskip}\selectfont}  
\newcommand{\xiaochu}{\fontsize{38.5pt}{\baselineskip}\selectfont}  
\newcommand{\yihao}{\fontsize{27.8pt}{\baselineskip}\selectfont}  
\newcommand{\xiaoyi}{\fontsize{25.7pt}{\baselineskip}\selectfont}  
\newcommand{\erhao}{\fontsize{23.5pt}{\baselineskip}\selectfont}  
\newcommand{\xiaoerhao}{\fontsize{19.3pt}{\baselineskip}\selectfont} 
\newcommand{\sihao}{\fontsize{14pt}{\baselineskip}\selectfont}      % 字号设置  
\newcommand{\xiaosihao}{\fontsize{12pt}{\baselineskip}\selectfont}  % 字号设置  
\newcommand{\wuhao}{\fontsize{10.5pt}{\baselineskip}\selectfont}    % 字号设置  
\newcommand{\xiaowuhao}{\fontsize{9pt}{\baselineskip}\selectfont}   % 字号设置  
\newcommand{\liuhao}{\fontsize{7.875pt}{\baselineskip}\selectfont}  % 字号设置  
\newcommand{\qihao}{\fontsize{5.25pt}{\baselineskip}\selectfont}    % 字号设置 

\usepackage{diagbox}
\usepackage{multirow}
\boldmath
\XeTeXlinebreaklocale "zh"
\XeTeXlinebreakskip = 0pt plus 1pt minus 0.1pt
\definecolor{cred}{rgb}{0.8,0.8,0.8}
\definecolor{cgreen}{rgb}{0,0.3,0}
\definecolor{cpurple}{rgb}{0.5,0,0.35}
\definecolor{cdocblue}{rgb}{0,0,0.3}
\definecolor{cdark}{rgb}{0.95,1.0,1.0}
\lstset{
	language=[x86masm]Assembler,
	numbers=left,
	numberstyle=\tiny\color{black},
	showspaces=false,
	showstringspaces=false,
	basicstyle=\scriptsize,
	keywordstyle=\color{purple},
	commentstyle=\itshape\color{cgreen},
	stringstyle=\color{blue},
	frame=lines,
	% escapeinside=``,
	extendedchars=true, 
	xleftmargin=1em,
	xrightmargin=1em, 
	backgroundcolor=\color{cred},
	aboveskip=1em,
	breaklines=true,
	tabsize=4
} 

\newfontfamily{\consolas}{Consolas}
\newfontfamily{\monaco}{Monaco}
\setmonofont[Mapping={}]{Consolas}	%英文引号之类的正常显示,相当于设置英文字体
\setsansfont{Consolas} %设置英文字体 Monaco, Consolas,  Fantasque Sans Mono
\setmainfont{Times New Roman}

\setCJKmainfont{华文中宋}


\newcommand{\fic}[1]{\begin{figure}[H]
		\center
		\includegraphics[width=0.8\textwidth]{#1}
	\end{figure}}
	
\newcommand{\sizedfic}[2]{\begin{figure}[H]
		\center
		\includegraphics[width=#1\textwidth]{#2}
	\end{figure}}

\newcommand{\codefile}[1]{\lstinputlisting{#1}}

% 改变段间隔
\setlength{\parskip}{0.2em}
\linespread{1.1}

\usepackage{lastpage}
\usepackage{fancyhdr}
\pagestyle{fancy}
\lhead{\space \qquad \space}
\chead{进程间通信 \qquad}
\rhead{\qquad\thepage/\pageref{LastPage}}
\begin{document}

\tableofcontents

\clearpage

\section{实现IPC}
	实现框架如下:
	\begin{itemize}
		\item[1.] 增加sendrec系统调用,这里的系统调用供进程调用,最终根据类型决定调用msg\_receive()还是msg\_send()函数。
		\item[2.] 新定义MESSAGE结构体和扩展proc结构体,并且声明一个phys\_copy()函数,用于进程之间复制消息体。
		\item[3.] 实现msg\_send()函数,用于将sender进程的消息体内容传递给目标进程。
		\item[4.] 实现msg\_receive()函数,用于接收来自其它进程的消息。
	\end{itemize}

\subsection{增加sendrec系统调用}
	两个函数体分别如下:
	\begin{lstlisting}
	sendrec:
		mov eax, _NR_sendrec
		mov ebx, [esp + 4]
		mov ecx, [esp + 8]
		mov edx, [esp + 12]
		int INT_VECTOR_SYS_CALL
		ret

	// ==============================

	PUBLIC int sys_sendrec(int function, int src_dest, MESSAGE* m, struct proc* p)
	{
		assert(k_reenter == 0);
		assert((src_dest >= 0 && src_dest <= NR_TASKS + NR_PROCS) || src_dest == ANY || src_dest == INTERRUPT);

		int ret = 0;
		int caller = proc2pid(p);
		MESSAGE* mla = (MESSAGE*)va2la(caller, m);
		mla->source = caller;

		assert(mla->source != src_dest);

		if(function == SEND)
		{
			ret = msg_send(p, src_dest, m);
			if(ret != 0)
				return ret;
		}
		else if(function == RECEIVE)
		{
			ret = msg_receive(p, src_dest, m);
			if(ret != 0)
				return ret;
		}
		else
		{
			panic("{sys_sendrec} invalid function: %d (SEND:%d RECEIVE:%d)", function, SEND, RECEIVE);
		}
	}
	\end{lstlisting}

	我们这里还用了send\_recv()函数去封装sendrec()系统调用:
	\begin{lstlisting}
	PUBLIC int send_recv(int function, int src_dest, MESSAGE* msg)
	{
		int ret = 0;
		if(function == RECEIVE)
			memset(msg, 0, sizeof(MESSAGE));

		switch(function)
		{
			case BOTH:
				ret = sendrec(SEND, src_dest, msg);
				if(ret == 0)
					ret = sendrec(RECEIVE, src_dest, msg);
				case SEND:
				case RECEIVE:
					ret = sendrec(function, src_dest, msg);
					break;
				default:
					assert((function == BOTH) || (function == SEND) || (function == RECEIVE));
					break;
		}
		return ret;
	}
	\end{lstlisting}

\subsection{为消息的接收与发送做准备}
\subsubsection{扩展进程结构体}
	定义MESSAGE结构体:
	\begin{lstlisting}
	struct mess1 
	{
		int m1i1;
		int m1i2;
		int m1i3;
		int m1i4;
	};

	struct mess2 
	{
		void* m2p1;
		void* m2p2;
		void* m2p3;
		void* m2p4;
	};

	struct mess3 
	{
		int	m3i1;
		int	m3i2;
		int	m3i3;
		int	m3i4;
		u64	m3l1;
		u64	m3l2;
		void*	m3p1;
		void*	m3p2;
	};

	#define	RETVAL		u.m3.m3i1

	typedef struct 
	{
		int source;
		int type;
		union {
			struct mess1 m1;
			struct mess2 m2;
			struct mess3 m3;
		} u;
	} MESSAGE;
	\end{lstlisting}

	接收方和发送方斗维护着一个消息结构体,发送方的结构体携带了消息内容,而接收方是空的。\par

	为了使进程可以通信,我们要在进程体结构中增加几个成员:
	\begin{lstlisting}
	struct proc
	{
		struct stackframe regs;

		u16 ldt_sel;
		struct descriptor ldts[LDT_SIZE];

		int ticks;
		int priority;

		u32 pid;
		char name[16];

		int nr_tty;

		// 以下是扩展的成员
		int p_flags;
		/*
		   用于表明进程的状态
		   0,表示进程正在运行或准备运行
		   SENDING,进程处于发送消息的状态,消息还未送达,进程被阻塞
		   RECEIVING。进程处于接收消息的状态,消息还未收到,进程被阻塞
		*/

		MESSAGE * p_msg; // 指向消息体

		int p_recvfrom;
		/*
		   记录进程想要从谁那里接收消息
		*/

		int p_sendto;
		/*
		   记录进程想要发送消息给谁
		*/

		int has_int_msg;
		/*
			系统是否正在等待一个中断发生
		*/

		struct proc * q_sending;
		/*
			向进程发送消息的进程队列中,q_sending指向第一个试图发送消息的进程
		*/

		struct proc * next_sending;
		/*
			向进程发送消息的进程队列中,进程如果处在这个队列中,next_sending指向下一个进程
		*/
	}
	\end{lstlisting}

\subsubsection{实现phys\_copy()函数}
	这里的phys\_copy()函数可以直接借助memcpy()函数来实现:
	\begin{lstlisting}
	PUBLIC	void*	memcpy(void* p_dst, void* p_src, int size);
	
	#define	phys_copy	memcpy
	\end{lstlisting}

	为了使用phys\_copy()函数,我们需要先把消息的地址转为线性地址,算法如下:
	\begin{itemize}
		\item[1.] 根据上述进程结构体的定义,消息MESSAGE是proc结构体的一个成员,如果我们拥有一个MESSAGE,那么它一定是附属于进程结构体的。
		          那么这个消息的地址值一个是相对于进程结构体的偏离地址。
		\item[2.] 现在我们拥有相对于进程结构体的偏离地址,只需要再求出进程结构体的线性地址,两者相加就是消息的线性地址。
	\end{itemize}
	\begin{lstlisting}
	// 每个进程都有自己的LDT
	// 通过进程结构体中的LDT中的描述符可以得到相应段的基地址
	PUBLIC int ldt_seg_linear(struct proc* p, int idx)
	{
		struct descriptor* d = &p->ldts[idx];
		return d->base_high << 24 | d->base_mid << 16 | d->base_low;
	}

	PUBLIC void* va2la(int pid, void* va)
	{
		struct proc* p = &proc_table[pid];

		u32 seg_base = ldt_seg_linear(p, INDEX_LDT_RW);
		u32 la = seg_base + u32(va);

		return (void*)la;
	}
	\end{lstlisting}

\subsection{实现msg\_send()函数}
	这个函数的算法如下:
	\begin{itemize}
		\item[1.] 首先判断是否发生死锁。
		\item[2.] 判断目标进程dest是否正在等待sender进程的消息。
		\item[3.] 如果是,就把消息复制给目标进程,目标进程被解除阻塞,继续运行。
				  如果不是,sender进程被阻塞,并加入目标进程的发送队列中。
	\end{itemize}

	\begin{lstlisting}
	PRIVATE int msg_send(struct proc* current, int dest, MESSAGE* m)
	{
		struct proc* sender = current;
		struct proc* p_dest = proc_table + dest;

		assert(proc2pid(sender) != dest);

		// 检测是否发生死锁
		if(deadlock(proc2pid(), dest))
		{
			panic(">>DEADLOCK<< %s->%s", sender->name, p_dest->name);
		}

		// 判断目标进程p_dest是否在等待sender进程的消息
		if((p_dest->p_flags & RECEIVING) && (p_dest->p_recvfrom == proc2pid(sender)) || p_dest->p_recvfrom == ANY)
		{
			assert(p_dest->p_msg);
			assert(m);

			// 将消息复制给目标进程p_dest
			phys_copy(va2la(dest, p_dest->p_msg), va2la(proc2pid(sender), m), sizeof(MESSAGE));

			// 恢复p_dest的状态
			p_dest->p_msg = 0;
			p_dest->p_flags &= ~RECEIVING;
			p_dest->p_recvfrom = NO_TASK;

			// 将目标进程解除阻塞
			unblock(p_dest);

			assert(p_dest->p_flags == 0);
			assert(p_dest->p_msg == 0);
			assert(p_dest->p_recvfrom == NO_TASK);
			assert(p_dest->p_sendto == NO_TASK);
			assert(sender->p_flags == 0);
			assert(sender->p_msg == 0);
			assert(sender->p_recvfrom == NO_TASK);
			assert(sender->p_sendto == NO_TASK);
		}
		else // 目标进程没有在等待sender进程,sender进程被阻塞,加入到目标进程的发送队列中
		{
			sender->p_flags |= SENDING;
			assert(sender->p_flags == SENDING);
			sender->p_sendto = dest;
			sender->p_msg = m;

			struct proc* p;

			// 将sender进程加入目标进程的发送队列中
			if(p_dest->q_sending)
			{
				p = p_dest->q_sending;
				while(p->next_sending)
					p = p->next_sending;
				p->next_sending = sender;
			}
			else
			{
				p_dest->q_sending = sender;
			}
			sender->next_sending = 0;
			
			// 阻塞sender进程
			block(sender);

			assert(sender->p_flags == SENDING);
			assert(sender->p_msg != 0);
			assert(sender->p_recvfrom == NO_TASK);
			assert(sender->p_sendto == dest);
		}
		return 0;
	}
	\end{lstlisting}

\subsubsection{实现一些辅助函数}
	在实现msg\_send()函数的过程中,我们使用了block()和unblock()函数,用于阻塞和解锁进程。
	在这里可以借助进程调度来实现进程的阻塞与解锁:
	\begin{lstlisting}
	PRIVATE void block(struct proc* p)
	{
		assert(p->p_flags); // 首先判断进程的状态不为0,也就是不是runnable的状态
		schedule();
	}

	PRIVATE void unblock(struct proc* p)
	{
		assert(p->p_flags == 0); // 判断进程为runnable的状态
	}
	\end{lstlisting}

	这里unblock()和block()的函数很简单,主要是利用p\_flags这个状态,然后在进程调度函数中做一些手脚:
	\begin{lstlisting}
	PUBLIC void schedule()
	{
		struct proc* p;
		int greatest_ticks = 0;

		while(!greatest_ticks)
		{
			for(p = &FIRST_PROC; p <= &LAST_PROC; p++)
			{
				// 进程的p_flags只有为0,才可能分配到cpu
				if(p->p_flags == 0)
				{
					if(p->ticks > greatest_ticks)
					{
						greatest_ticks = p->ticks;
						p_proc_ready = p;
					}
				}
			}

			// 如果进程初始ticks都为0
			if(!greatest_ticks)
			{
				for(p = &FIRST_PROC; p <= &LAST_PROC; ++p)
				{
					// 只有进程的p_flags只有为0,才可能分配到ticks
					if(p->p_flags == 0)
						p->ticks = p->priority;
				}
			}
		}
	}
	\end{lstlisting}

	在实现消息传递机制中,我们还要谨防死锁。
	这里通过判断消息的发送是否构成一个环来判断。
	如果构成一个环,则意味着发生死锁,比如A试图发消息给B,同时B试图给C,C试图给A发消息,那么死锁就发生了。deadlock()的实现如下:
	\begin{lstlisting}
	PRIVATE int deadlock(int src, int dest)
	{
		struct proc* p = proc_table + dest;
		while(1)
		{
			// 用于检测是否构成一个环
			if(p->p_flags & SENDING)
			{
				if(p->p_sendto == src)
				{
					// 如果发现可以构成一个环时,将这个进程环打印出来
					p = proc_table + dest;
					printl("=_=%s",p->name);
					do
					{
						assert(p->msg);
						p = proc_table + p->p_sendto;
						printl("->%s", p->name);
					}while(p != proc_table + src);
					return 1;
				}
				p = proc_table + p->p_sendto;
			}
			else
				break;
		}
		return 0;
	}
	\end{lstlisting}

\subsection{实现msg\_receive()函数}
	这个函数的算法如下:
	\begin{itemize}
		\item[1.] 首先判断进程是否有一个来自硬件的消息,如果是,并且进程的消息源为ANY或INTERRUPT,就准备一个消息给进程,并返回。
		\item[2.] 如果进程的消息源为ANY,就从自己的q\_sending中选取一个消息源,将其该源进程的消息复制给进程。
		\item[3.] 如果进程的消息源为特定进程A,则先判断A是否在等待向自己发送消息。如果是,就把消息复制给进程。
		\item[4.] 如果此时没有任何进程发消息给本进程,则该进程将被阻塞。
	\end{itemize}

	\begin{lstlisting}
	PRIVATE int msg_receive(struct proc* current, int src, MESSAGE* m)
	{
		struct proc* p_who_wanna_recv = current;
		struct proc* p_from = 0;
		struct proc* prev = 0;
		int copyok = 0;

		assert(proc2pid(p_who_wanna_recv) != src);

		// 判断进程是否有一个来自硬件的消息,并且进程的消息源为ANY或INTERRUPT
		if((p_who_wanna_recv->has_int_msg) && ((src == ANY) || (src == INTERRUPT)))
		{
			MESSAGE msg;
			reset_msg(&msg);
			msg.source = HARD_INT;
			assert(m);
			phys_copy(va2la(proc2pid(p_who_wanna_recv), m), &msg, sizeof(MESSAGE));
			
			p_who_wanna_recv->has_int_msg = 0;

			assert(p_who_wanna_recv->p_flags == 0);
			assert(p_who_wanna_recv->p_msg == 0);
			assert(p_who_wanna_recv->p_sendto == NO_TASK);
			assert(p_who_wanna_recv->has_int_msg == 0);

			return 0;
		}

		// 如果进程的消息源为ANY
		if(src == ANY)
		{
			if(p_who_wanna_recv->q_sending)
			{
				// 从自己的q_sending中选取一个消息
				p_from = p_who_wanna_recv->q_sending;
				copyok = 1;

				assert(p_who_wanna_recv->p_flags == 0);
				assert(p_who_wanna_recv->p_msg == 0);
				assert(p_who_wanna_recv->p_recvfrom == NO_TASK);
				assert(p_who_wanna_recv->p_sendto == NO_TASK);
				assert(p_who_wanna_recv->q_sending != 0);
				
				assert(p_from->p_flags == SENDING);
				assert(p_from->p_msg != 0);
				assert(p_from->p_recvfrom == NO_TASK);
				assert(p_from->p_sendto == proc(p_who_wanna_recv));
			}
		}
		else // 如果进程的消息源为特定进程
		{
			p_from = &proc_table[src];

			// 判断进程是否在发消息,并且目标为本进程
			if((p_from->p_flags & SENDING) && (p_from->p_sendto == proc2pid(p_who_wanna_recv)))
			{
				copyok = 1;
				
				struct proc* p = p_who_wanna_recv->q_sending;
				assert(p);

				// 该循环用于找到进程队列中源进程的前一个进程prev,用于维护进程队列
				while(p)
				{
					assert(p_from->p_flags & SENDING);
					if(proc2pid(p) == src)
					{
						p_from = p;
						break;
					}
					prev = p;
					p = p->next_sending;
				}
			}

			assert(p_who_wanna_recv->p_flags == 0);
			assert(p_who_wanna_recv->p_msg == 0);
			assert(p_who_wanna_recv->p_recvfrom == NO_TASK);
			assert(p_who_wanna_recv->p_sendto == NO_TASK);
			assert(p_who_wanna_recv->q_sending != 0);

			assert(p_from->p_flags == SENDING);
			assert(p_from->p_msg != 0);
			assert(p_from->p_recvfrom == NO_TASK);
			assert(p_from->p_sendto == proc2pid(p_who_wanna_recv));
		}

		// 如果有相应的消息源
		if(copyok)
		{
			// 如果p_from就是进程队列的第一个进程
			if(p_from == p_who_wanna_recv->q_sending)
			{
				assert(prev == 0);
				// 更新进程队列
				p_who_wanna_recv->q_sending = p_from->next_sending;
				p_from->next_sending = 0;
			}
			else // 如果不是,同样是维护进程队列
			{
				assert(prev);
				prev->next_sending = p_from->next_sending;
				p_from->next_sending = 0;
			}

			assert(m);
			assert(p_from->p_msg);
			
			// 将消息体复制给目标进程
			phys_copy(va2la(proc2pid(p_who_wanna_recv), m), va2la(proc2pid(p_from), p_from->p_msg), sizeof(MESSAGE));
			
			p_from->p_msg = 0;
			p_from->p_sendto = NO_TASK;
			p_from->p_flags &= ~SENDING; // 将p_from的p_flags设为0
			
			// 解除对p_from的阻塞
			unblock(p_from);
		}
		else
		{
			// 将p_who_wanna_recv的p_flags设为RECEIVING
			p_who_wanna_recv->p_flags |= RECEIVING;
			p_who_wanna_recv->p_msg = m;

			if(src == ANY)
				p_who_wanna_recv->p_recvfrom = ANY;
			else
				p_who_wanna_recv->p_recvfrom = proc2pid(p_from);
			
			// 阻塞目标进程p_who_wanna_recv
			block(p_who_wanna_recv);

			assert(p_who_wanna_recv->p_flags == RECEIVING);
			assert(p_who_wanna_recv->p_msg != 0);
			assert(p_who_wanna_recv->p_recvfrom != NO_TASK);
			assert(p_who_wanna_recv->p_sendto == NO_TASK);
			assert(p_who_wanna_recv->has_int_msg == 0);
		}
	}
	\end{lstlisting}

\section{使用IPC机制实现get\_ticks()函数}
	这里的get\_ticks()函数首先需要向某个系统任务发出请求ticks的值,随后等待该系统任务的响应。
	这样的行为我们定义为BOTH,也就是发送一个消息,随后马上等待接收一个消息。实现如下:
	\begin{lstlisting}
	PUBLIC int get_ticks()
	{
		MESSAGE msg;
		memset(&msg, 0, sizeof(MESSAGE));
		msg.type = GET_TICKS;
		send_recv(BOTH, TASK_SYS, &msg);
		return msg.RETVAL;
	}

	PUBLIC void task_sys()
	{
		MESSAGE msg;
		while(1)
		{
			// 等待其他进程的消息
			send_recv(RECEIVE, ANY, &msg);
			int src = msg.source;

			switch(msg.type)
			{
				case GET_TICKS:
					msg.RETVAL = ticks;
					// 向特定进程发送消息
					send_recv(SEND, src, &msg);
					break;
				default:
					panic("unknown msg type");
					break;
			}
		}
	}
	\end{lstlisting}

\end{document}
