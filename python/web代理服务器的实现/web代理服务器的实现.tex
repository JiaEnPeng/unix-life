% !TeX spellcheck = en_US
%% 字体:方正静蕾简体
%%		 方正粗宋
\documentclass[a4paper,left=2.5cm,right=2.5cm,11pt]{article}

\usepackage[utf8]{inputenc}
\usepackage{fontspec}
\usepackage{cite}
\usepackage{xeCJK}
\usepackage{indentfirst}
\usepackage{titlesec}
\usepackage{longtable}
\usepackage{graphicx}
\usepackage{float}
\usepackage{rotating}
\usepackage{subfigure}
\usepackage{tabu}
\usepackage{amsmath}
\usepackage{setspace}
\usepackage{amsfonts}
\usepackage{appendix}
\usepackage{listings}
\usepackage{xcolor}
\usepackage{geometry}
\setcounter{secnumdepth}{4}
\usepackage{mhchem}
\usepackage{multirow}
\usepackage{extarrows}
\usepackage{hyperref}
\titleformat*{\section}{\LARGE}
\renewcommand\refname{参考文献}
\renewcommand{\abstractname}{\sihao \cjkfzcs 摘{  }要}
%\titleformat{\chapter}{\centering\bfseries\huge\wryh}{}{0.7em}{}{}
%\titleformat{\section}{\LARGE\bf}{\thesection}{1em}{}{}
\titleformat{\subsection}{\Large\bfseries}{\thesubsection}{1em}{}{}
\titleformat{\subsubsection}{\large\bfseries}{\thesubsubsection}{1em}{}{}
\renewcommand{\contentsname}{{\cjkfzcs \centerline{目{  } 录}}}
\setCJKfamilyfont{cjkhwxk}{STXingkai}
\setCJKfamilyfont{cjkfzcs}{STSongti-SC-Regular}
% \setCJKfamilyfont{cjkhwxk}{华文行楷}
% \setCJKfamilyfont{cjkfzcs}{方正粗宋简体}
\newcommand*{\cjkfzcs}{\CJKfamily{cjkfzcs}}
\newcommand*{\cjkhwxk}{\CJKfamily{cjkhwxk}}
\newfontfamily\wryh{Microsoft YaHei}
\newfontfamily\hwzs{STZhongsong}
\newfontfamily\hwst{STSong}
\newfontfamily\hwfs{STFangsong}
\newfontfamily\jljt{MicrosoftYaHei}
\newfontfamily\hwxk{STXingkai}
% \newfontfamily\hwzs{华文中宋}
% \newfontfamily\hwst{华文宋体}
% \newfontfamily\hwfs{华文仿宋}
% \newfontfamily\jljt{方正静蕾简体}
% \newfontfamily\hwxk{华文行楷}
\newcommand{\verylarge}{\fontsize{60pt}{\baselineskip}\selectfont}  
\newcommand{\chuhao}{\fontsize{44.9pt}{\baselineskip}\selectfont}  
\newcommand{\xiaochu}{\fontsize{38.5pt}{\baselineskip}\selectfont}  
\newcommand{\yihao}{\fontsize{27.8pt}{\baselineskip}\selectfont}  
\newcommand{\xiaoyi}{\fontsize{25.7pt}{\baselineskip}\selectfont}  
\newcommand{\erhao}{\fontsize{23.5pt}{\baselineskip}\selectfont}  
\newcommand{\xiaoerhao}{\fontsize{19.3pt}{\baselineskip}\selectfont} 
\newcommand{\sihao}{\fontsize{14pt}{\baselineskip}\selectfont}      % 字号设置  
\newcommand{\xiaosihao}{\fontsize{12pt}{\baselineskip}\selectfont}  % 字号设置  
\newcommand{\wuhao}{\fontsize{10.5pt}{\baselineskip}\selectfont}    % 字号设置  
\newcommand{\xiaowuhao}{\fontsize{9pt}{\baselineskip}\selectfont}   % 字号设置  
\newcommand{\liuhao}{\fontsize{7.875pt}{\baselineskip}\selectfont}  % 字号设置  
\newcommand{\qihao}{\fontsize{5.25pt}{\baselineskip}\selectfont}    % 字号设置 

\usepackage{diagbox}
\usepackage{multirow}
\boldmath
\XeTeXlinebreaklocale "zh"
\XeTeXlinebreakskip = 0pt plus 1pt minus 0.1pt
\definecolor{cred}{rgb}{0.8,0.8,0.8}
\definecolor{cgreen}{rgb}{0,0.3,0}
\definecolor{cpurple}{rgb}{0.5,0,0.35}
\definecolor{cdocblue}{rgb}{0,0,0.3}
\definecolor{cdark}{rgb}{0.95,1.0,1.0}
\lstset{
	language=bash,
	numbers=left,
	numberstyle=\tiny\color{black},
	showspaces=false,
	showstringspaces=false,
	basicstyle=\scriptsize,
	keywordstyle=\color{purple},
	commentstyle=\itshape\color{cgreen},
	stringstyle=\color{blue},
	frame=lines,
	% escapeinside=``,
	extendedchars=true, 
	xleftmargin=1em,
	xrightmargin=1em, 
	backgroundcolor=\color{cred},
	aboveskip=1em,
	breaklines=true,
	tabsize=4
} 

\newfontfamily{\consolas}{Consolas}
\newfontfamily{\monaco}{Monaco}
\setmonofont[Mapping={}]{Consolas}	%英文引号之类的正常显示,相当于设置英文字体
\setsansfont{Consolas} %设置英文字体 Monaco, Consolas,  Fantasque Sans Mono
\setmainfont{Times New Roman}

\setCJKmainfont{华文中宋}


\newcommand{\fic}[1]{\begin{figure}[H]
		\center
		\includegraphics[width=0.8\textwidth]{#1}
	\end{figure}}
	
\newcommand{\sizedfic}[2]{\begin{figure}[H]
		\center
		\includegraphics[width=#1\textwidth]{#2}
	\end{figure}}

\newcommand{\codefile}[1]{\lstinputlisting{#1}}

\newcommand{\interval}{\vspace{0.5em}}

% 改变段间隔
\setlength{\parskip}{0.2em}
\linespread{1.1}

\usepackage{lastpage}
\usepackage{fancyhdr}
\pagestyle{fancy}
\lhead{\space \qquad \space}
\chead{web代理服务器的实现 \qquad}
\rhead{\qquad\thepage/\pageref{LastPage}}
\begin{document}

\tableofcontents

\clearpage

\section{web代理服务器的实现}
	这份报告首先描述了web代理服务器实现了什么功能,然后讲述如何去实现这个web代理服务器。
	阅读者只要有python编程和socket编程的基础,就可以根据这篇文档写一个代理服务器。

\subsection{web代理服务器的功能}
\subsubsection{HTTP请求转发}
	这个web代理服务器实现了HTTP请求转发,也就是当我的代理服务器从一个浏览器接收到对某对象的HTTP请求时,
	它生成对相同对象的一个新HTTP请求,并向初始服务器发送。\par

	如下图所示:
	\fic{1.png}

\clearpage

\subsubsection{回送HTTP内容替代}
	这个web代理服务器还实现了回送HTTP内容替代,也就是代理服务器从初始服务器接收到具有该对象的HTTP响应时,
	它生成一个包括该对象的新HTTP响应,并发送给该客户。\par

	如下图所示:
	\fic{2.png}

\clearpage

\subsubsection{支持多用户访问功能}
	这个web代理服务器支持多线程,也就是支持多用户访问功能。这个功能比较难描述,我主要讲一下单线程代理服务器和多线程代理服务器的区别。\par

	如果是单线程代理服务器,我在请求一个网页的时候,如果这个网页还未加载完成,那么随即请求另一个网页就无法成功。
	而如果是多线程代理服务器,就可以同时请求多个网页。\par

	单线程代理服务器的效果如下图,也就是左边的网页还未加载结束,而同时请求右边的网页,右边的网页就不会有任何显示:
	\sizedfic{0.7}{4.png}

	多线程代理服务器的效果如下图,也就是左边的网页还未加载结束,而同时请求右边的网页,右边的网页也可以开始加载:
	\sizedfic{0.7}{3.jpeg}

\subsection{web代理服务器的实现思路}
	这部分只是讲解web代理服务器功能实现的思路,一些具体的细节仍需要参见完整的源码,源码在附件中。

\subsubsection{如何实现HTTP请求转发}
	实现思路如下:
	\begin{itemize}
		\item[1.] 建立一个代理socket,监听tcp连接请求。
		\item[2.] 建立tcp连接以后,从客户socket获得请求报文。
		\item[3.] 从请求报文中提取初始服务器,并建立与初始服务器的连接。
		\item[4.] 将HTTP请求转发给初始服务器。
	\end{itemize}

	建立一个代理socket,监听tcp连接请求:
	\begin{lstlisting}
	# 建立一个代理socket
	port = 12000
	proxy_socket = socket.socket(socket.AF_INET, socket.SOCK_STREAM)
	proxy_socket.bind(('', port))
	proxy_socket.listen(5)
	while 1:
		# 监听tcp连接请求
		client_socket, client_addr = proxy_socket.accept()
		# 开始处理来自客户的请求
		proxy_thread(client_socket, client_addr)
	\end{lstlisting}

	建立tcp连接以后,从客户socket获得请求报文:
	\begin{lstlisting}
	request = client_socket.recv(999999)
	\end{lstlisting}

	从请求报文中提取初始服务器,并建立与初始服务器的连接:
	\begin{lstlisting}
	# 提取初始服务器的主机地址和端口
	host, port = get_host_and_port(request)
	server_socket = socket.socket(socket.AF_INET, socket.SOCK_STREAM)
	# 建立与初始服务器的连接
	server_socket.connect((host, port))
	\end{lstlisting}

	这里get\_host\_and\_port()是我另外写的一个函数,用于提取请求报文中初始服务器的主机地址和端口。
	一个典型的HTTP请求报文如下所示:
	\begin{lstlisting}
GET http://cn.bing.com/ HTTP/1.1
Host: cn.bing.com
Accept: text/html,application/xhtml+xml,application/xml;q=0.9,*/*;q=0.8
Proxy-Connection: keep-alive
Upgrade-Insecure-Requests: 1
Accept-Language: zh-cn
Accept-Encoding: gzip, deflate
Connection: keep-alive
	\end{lstlisting}

	初始服务器的主机地址和端口存放在第一行,如果第一行中没有声明端口,说明初始服务器的web服务器运行在80端口。
	我们可以使用如下函数提取:
	\begin{lstlisting}
	def get_host_and_port(request):
		first_line = request.split('\n')[0]
		url = first_line.split(' ')[1]

		http_pos = url.find("://")
		if http_pos == -1:
			temp = url
		else:
			temp = url[(http_pos+3):]
		
		host_pos = temp.find("/")
		if host_pos == -1:
			host_pos = len(temp)

		port_pos = temp.find(":")

		if (port_pos == -1 or host_pos < port_pos):
			port = 80
			host = temp[:host_pos]
		else:
			port = int((temp[(port_pos+1):])[:host_pos-port_pos-1])
			host = temp[:port_pos]

		return host, port
	\end{lstlisting}
	
	最后将HTTP请求转发给初始服务器:
	\begin{lstlisting}
	server_socket.send(request)
	\end{lstlisting}

\subsubsection{如何实现回送HTTP内容替代}
	实现思路如下:
	\begin{itemize}
		\item[1.] 首先接收来自初始服务器的数据。
		\item[2.] 然后将数据发送给客户。
		\item[3.] 最后关闭客户器端socket和服务器端socket。
	\end{itemize}

	接收来自初始服务器的数据:
	\begin{lstlisting}
	data = server_socket.recv(999999)
	\end{lstlisting}

	将数据发送给客户:
	\begin{lstlisting}
	client_socket.send(data)
	\end{lstlisting}

	最后关闭客户器端socket和服务器端socket:
	\begin{lstlisting}
	server_socket.close()
	client_socket.close()
	\end{lstlisting}

\subsubsection{如何实现支持多用户访问功能}
	这个只要使用多线程编程就可以了,只要有客户请求,我们就开一个新线程处理这个请求,代码如下:
	\begin{lstlisting}
	while 1:
		client_socket, client_addr = proxy_socket.accept()
		thread.start_new_thread(proxy_thread, (client_socket, client_addr))
	\end{lstlisting}

\end{document}
