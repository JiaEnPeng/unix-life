% !TeX spellcheck = en_US
%% 字体:方正静蕾简体
%%		 方正粗宋
\documentclass[a4paper,left=2.5cm,right=2.5cm,11pt]{article}

\usepackage[utf8]{inputenc}
\usepackage{fontspec}
\usepackage{cite}
\usepackage{xeCJK}
\usepackage{indentfirst}
\usepackage{titlesec}
\usepackage{longtable}
\usepackage{graphicx}
\usepackage{float}
\usepackage{rotating}
\usepackage{subfigure}
\usepackage{tabu}
\usepackage{amsmath}
\usepackage{setspace}
\usepackage{amsfonts}
\usepackage{appendix}
\usepackage{listings}
\usepackage{xcolor}
\usepackage{geometry}
\setcounter{secnumdepth}{4}
\usepackage{mhchem}
\usepackage{multirow}
\usepackage{extarrows}
\usepackage{hyperref}
\titleformat*{\section}{\LARGE}
\renewcommand\refname{参考文献}
\renewcommand{\abstractname}{\sihao \cjkfzcs 摘{  }要}
%\titleformat{\chapter}{\centering\bfseries\huge\wryh}{}{0.7em}{}{}
%\titleformat{\section}{\LARGE\bf}{\thesection}{1em}{}{}
\titleformat{\subsection}{\Large\bfseries}{\thesubsection}{1em}{}{}
\titleformat{\subsubsection}{\large\bfseries}{\thesubsubsection}{1em}{}{}
\renewcommand{\contentsname}{{\cjkfzcs \centerline{目{  } 录}}}
\setCJKfamilyfont{cjkhwxk}{STXingkai}
\setCJKfamilyfont{cjkfzcs}{STSongti-SC-Regular}
% \setCJKfamilyfont{cjkhwxk}{华文行楷}
% \setCJKfamilyfont{cjkfzcs}{方正粗宋简体}
\newcommand*{\cjkfzcs}{\CJKfamily{cjkfzcs}}
\newcommand*{\cjkhwxk}{\CJKfamily{cjkhwxk}}
\newfontfamily\wryh{Microsoft YaHei}
\newfontfamily\hwzs{STZhongsong}
\newfontfamily\hwst{STSong}
\newfontfamily\hwfs{STFangsong}
\newfontfamily\jljt{MicrosoftYaHei}
\newfontfamily\hwxk{STXingkai}
% \newfontfamily\hwzs{华文中宋}
% \newfontfamily\hwst{华文宋体}
% \newfontfamily\hwfs{华文仿宋}
% \newfontfamily\jljt{方正静蕾简体}
% \newfontfamily\hwxk{华文行楷}
\newcommand{\verylarge}{\fontsize{60pt}{\baselineskip}\selectfont}  
\newcommand{\chuhao}{\fontsize{44.9pt}{\baselineskip}\selectfont}  
\newcommand{\xiaochu}{\fontsize{38.5pt}{\baselineskip}\selectfont}  
\newcommand{\yihao}{\fontsize{27.8pt}{\baselineskip}\selectfont}  
\newcommand{\xiaoyi}{\fontsize{25.7pt}{\baselineskip}\selectfont}  
\newcommand{\erhao}{\fontsize{23.5pt}{\baselineskip}\selectfont}  
\newcommand{\xiaoerhao}{\fontsize{19.3pt}{\baselineskip}\selectfont} 
\newcommand{\sihao}{\fontsize{14pt}{\baselineskip}\selectfont}      % 字号设置  
\newcommand{\xiaosihao}{\fontsize{12pt}{\baselineskip}\selectfont}  % 字号设置  
\newcommand{\wuhao}{\fontsize{10.5pt}{\baselineskip}\selectfont}    % 字号设置  
\newcommand{\xiaowuhao}{\fontsize{9pt}{\baselineskip}\selectfont}   % 字号设置  
\newcommand{\liuhao}{\fontsize{7.875pt}{\baselineskip}\selectfont}  % 字号设置  
\newcommand{\qihao}{\fontsize{5.25pt}{\baselineskip}\selectfont}    % 字号设置 

\usepackage{diagbox}
\usepackage{multirow}
\boldmath
\XeTeXlinebreaklocale "zh"
\XeTeXlinebreakskip = 0pt plus 1pt minus 0.1pt
\definecolor{cred}{rgb}{0.8,0.8,0.8}
\definecolor{cgreen}{rgb}{0,0.3,0}
\definecolor{cpurple}{rgb}{0.5,0,0.35}
\definecolor{cdocblue}{rgb}{0,0,0.3}
\definecolor{cdark}{rgb}{0.95,1.0,1.0}
\lstset{
	language=bash,
	numbers=left,
	numberstyle=\tiny\color{black},
	showspaces=false,
	showstringspaces=false,
	basicstyle=\scriptsize,
	keywordstyle=\color{purple},
	commentstyle=\itshape\color{cgreen},
	stringstyle=\color{blue},
	frame=lines,
	% escapeinside=``,
	extendedchars=true, 
	xleftmargin=1em,
	xrightmargin=1em, 
	backgroundcolor=\color{cred},
	aboveskip=1em,
	breaklines=true,
	tabsize=4
} 

\newfontfamily{\consolas}{Consolas}
\newfontfamily{\monaco}{Monaco}
\setmonofont[Mapping={}]{Consolas}	%英文引号之类的正常显示,相当于设置英文字体
\setsansfont{Consolas} %设置英文字体 Monaco, Consolas,  Fantasque Sans Mono
\setmainfont{Times New Roman}

\setCJKmainfont{华文中宋}


\newcommand{\fic}[1]{\begin{figure}[H]
		\center
		\includegraphics[width=0.8\textwidth]{#1}
	\end{figure}}
	
\newcommand{\sizedfic}[2]{\begin{figure}[H]
		\center
		\includegraphics[width=#1\textwidth]{#2}
	\end{figure}}

\newcommand{\codefile}[1]{\lstinputlisting{#1}}

\newcommand{\interval}{\vspace{0.5em}}

\newcommand{\tablestart}{
	\interval
	\begin{longtable}{p{2cm}p{10cm}}
	\hline}
\newcommand{\tableend}{
	\hline
	\end{longtable}
	\interval}

% 改变段间隔
\setlength{\parskip}{0.2em}
\linespread{1.1}

\usepackage{lastpage}
\usepackage{fancyhdr}
\pagestyle{fancy}
\lhead{\space \qquad \space}
\chead{expect脚本 \qquad}
\rhead{\qquad\thepage/\pageref{LastPage}}
\begin{document}

\tableofcontents

\clearpage

\section{expect脚本的参数}
	expect通过命令行参数传递参数,通过\$argc获取参数个数,\$argv是参数数组,使用[lindex \$argv n-1]获取第n个参数。\par

	使用例子如下:
	\begin{lstlisting}
	#!/usr/bin/expect

	set num $argc # 获得参数个数
	set username [lindex $argv 0] # 获得第1个参数
	set passwd [lindex $argv 1] # 获得第2个参数
	\end{lstlisting}

\section{expect脚本中的常用命令}
	\begin{itemize}
		\item[1.] spawn,用于启动一个进程,例子如下:
		\begin{lstlisting}
	#!/usr/bin/expect

	set username [lindex $argv 1]
	spawn ssh -l username 192.168.1.1
		\end{lstlisting}

		\item[2.] expect,等待进程的某些字符串。expect支持正规表达式并能同时等待多个字符串,并对每一个字符串执行不同的操作。格式如下:
		\begin{lstlisting}
	expect patlist1 action1 patlist2 action2
		\end{lstlisting}

		expect该命令一直等到当前进程的输出和以上的某一个模式相匹配,或者等到时间超过一个特定的时间长度,或者等到遇到了文件的结束为止。\par

		使用例子如下:
		\begin{lstlisting}
	expect "*welcome*" break
			"*busy*" {print busy; continue}
			"*failed*" abort
			timeout abort
		\end{lstlisting}

	\end{itemize}

\section{使用expect自动telnet会话}
	代码如下:
	\begin{lstlisting}
	#!/usr/bin/expect

	set ip [lindex $argv 0] # 获得第1个参数,作为IP
	set userid [lindex $argv 1] # 获得第2个参数,作为userid
	set mypassword [lindex $argv 2] # 获得第3个参数,作为密码
	set mycommand [lindex $argv 3] # 获得第4个参数,作为命令
	set timeout 10

	spawn telnet $ip
		expect "username:"
		send "$userid\r"

		expect "password:"
		send "$mypassword\r"

		expect "%"
		send "$mycommand\r"

		expect "%"
		set results $expect_out(buffer)

		send "exit\r"
		expect eof
	\end{lstlisting}

\section{使用expect自动建立FTP会话}
	\begin{lstlisting}
	#!/usr/bin/expect

	set ip [lindex $argv 0]
	set userid [lindex $argv 1]
	set mypassword [lindex $argv 2]
	set timeout 10

	spawn ftp $ip
		expect "username:"
		send "$userid\r"

		expect "password:"
		send "$mypassword\r"

		expect "ftp>"
		send "bin\r"

		expect "ftp>"
		send "prompt\r"

		expect "ftp>"
		send "mget *\r"

		expect "ftp>"
		send "bye\r"

		expect eof
	\end{lstlisting}

\section{使用expect自动ssh虚拟机}
	\begin{lstlisting}
	#!/usr/bin/expect

	set IP [lindex $argv 0]
	set USER [lindex $argv 1]
	set PASSWD [lindex $argv 2]
	set CMD [lindex $argv 3]

	spawn ssh $USER@$IP $CMD
	expect {
		"(yes/no)" {
			send "yes\r"
			expect "password:"
			send "$PASSWD\r"
		}
		"password:" {send "$PASSWD\r"}
		"* to host" {exit 1}
	}
	expect eof
	\end{lstlisting}

\end{document}
