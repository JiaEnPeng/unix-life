% !TeX spellcheck = en_US
%% 字体:方正静蕾简体
%%		 方正粗宋
\documentclass[a4paper,left=1.5cm,right=1.5cm,11pt]{article}

\usepackage[utf8]{inputenc}
\usepackage{fontspec}
\usepackage{cite}
\usepackage{xeCJK}
\usepackage{indentfirst}
\usepackage{titlesec}
\usepackage{etoolbox}%
\makeatletter
\patchcmd{\ttlh@hang}{\parindent\z@}{\parindent\z@\leavevmode}{}{}%
\patchcmd{\ttlh@hang}{\noindent}{}{}{}%
\makeatother

\usepackage{hyperref}
\usepackage{longtable}
\usepackage{empheq}
\usepackage{graphicx}
\usepackage{float}
\usepackage{rotating}
\usepackage{subfigure}
\usepackage{tabu}
\usepackage{amsmath}
\usepackage{setspace}
\usepackage{amsfonts}
\usepackage{appendix}
\usepackage{listings}
\usepackage{xcolor}
\usepackage{geometry}
\setcounter{secnumdepth}{4}
%\titleformat*{\section}{\LARGE}
%\renewcommand\refname{参考文献}
%\titleformat{\chapter}{\centering\bfseries\huge}{}{0.7em}{}{}
\titleformat{\section}{\LARGE\bf}{\thesection}{1em}{}{}
\titleformat{\subsection}{\Large\bfseries}{\thesubsection}{1em}{}{}
\titleformat{\subsubsection}{\large\bfseries}{\thesubsubsection}{1em}{}{}
\renewcommand{\contentsname}{{ \centerline{目{  } 录}}}
\setCJKfamilyfont{cjkhwxk}{STXINGKA.TTF}
%\setCJKfamilyfont{cjkhwxk}{华文行楷}
%\setCJKfamilyfont{cjkfzcs}{方正粗宋简体}
%\newcommand*{\cjkfzcs}{\CJKfamily{cjkfzcs}}
\newcommand*{\cjkhwxk}{\CJKfamily{cjkhwxk}}
%\newfontfamily\wryh{Microsoft YaHei}
%\newfontfamily\hwzs{华文中宋}
%\newfontfamily\hwst{华文宋体}
%\newfontfamily\hwfs{华文仿宋}
%\newfontfamily\jljt{方正静蕾简体}
%\newfontfamily\hwxk{华文行楷}
\newcommand{\verylarge}{\fontsize{60pt}{\baselineskip}\selectfont}  
\newcommand{\chuhao}{\fontsize{44.9pt}{\baselineskip}\selectfont}  
\newcommand{\xiaochu}{\fontsize{38.5pt}{\baselineskip}\selectfont}  
\newcommand{\yihao}{\fontsize{27.8pt}{\baselineskip}\selectfont}  
\newcommand{\xiaoyi}{\fontsize{25.7pt}{\baselineskip}\selectfont}  
\newcommand{\erhao}{\fontsize{23.5pt}{\baselineskip}\selectfont}  
\newcommand{\xiaoerhao}{\fontsize{19.3pt}{\baselineskip}\selectfont} 
\newcommand{\sihao}{\fontsize{14pt}{\baselineskip}\selectfont}      % 字号设置  
\newcommand{\xiaosihao}{\fontsize{12pt}{\baselineskip}\selectfont}  % 字号设置  
\newcommand{\wuhao}{\fontsize{10.5pt}{\baselineskip}\selectfont}    % 字号设置  
\newcommand{\xiaowuhao}{\fontsize{9pt}{\baselineskip}\selectfont}   % 字号设置  
\newcommand{\liuhao}{\fontsize{7.875pt}{\baselineskip}\selectfont}  % 字号设置  
\newcommand{\qihao}{\fontsize{5.25pt}{\baselineskip}\selectfont}    % 字号设置 

\usepackage{diagbox}
\usepackage{multirow}
\boldmath
\XeTeXlinebreaklocale "zh"
\XeTeXlinebreakskip = 0pt plus 1pt minus 0.1pt
\definecolor{cred}{rgb}{0.8,0.8,0.8}
\definecolor{cgreen}{rgb}{0,0.3,0}
\definecolor{cpurple}{rgb}{0.5,0,0.35}
\definecolor{cdocblue}{rgb}{0,0,0.3}
\definecolor{cdark}{rgb}{0.95,1.0,1.0}
\lstset{
	language=bash,
	numbers=left,
	numberstyle=\tiny\color{black},
	showspaces=false,
	showstringspaces=false,
	basicstyle=\scriptsize,
	keywordstyle=\color{purple},
	commentstyle=\itshape\color{cgreen},
	stringstyle=\color{blue},
	frame=lines,
	% escapeinside=``,
	extendedchars=true, 
	xleftmargin=1em,
	xrightmargin=1em, 
	backgroundcolor=\color{cred},
	aboveskip=1em,
	breaklines=true,
	tabsize=4
} 

%\newfontfamily{\consolas}{Consolas}
%\newfontfamily{\monaco}{Monaco}
%\setmonofont[Mapping={}]{Consolas}	%英文引号之类的正常显示,相当于设置英文字体
%\setsansfont{Consolas} %设置英文字体 Monaco, Consolas,  Fantasque Sans Mono
%\setmainfont{Times New Roman}
%\setCJKmainfont{STZHONGS.TTF}
%\setmonofont{Consolas}
% \newfontfamily{\consolas}{YaHeiConsolas.ttf}
\newfontfamily{\monaco}{MONACO.TTF}
\setCJKmainfont{STZHONGS.TTF}
%\setmainfont{MONACO.TTF}
%\setsansfont{MONACO.TTF}

\newcommand{\fic}[1]{\begin{figure}[H]
		\center
		\includegraphics[width=0.8\textwidth]{#1}
	\end{figure}}
	
\newcommand{\sizedfic}[2]{\begin{figure}[H]
		\center
		\includegraphics[width=#1\textwidth]{#2}
	\end{figure}}

\newcommand{\codefile}[1]{\lstinputlisting{#1}}

\newcommand{\interval}{\vspace{0.5em}}

\newcommand{\tablestart}{
	\interval
	\begin{longtable}{p{2cm}p{10cm}}
	\hline}
\newcommand{\tableend}{
	\hline
	\end{longtable}
	\interval}

% 改变段间隔
\setlength{\parskip}{0.2em}
\linespread{1.1}


\usepackage{lastpage}
\usepackage{fancyhdr}
\pagestyle{fancy}
\lhead{\space \qquad \space}
\chead{基本脚本编译 \qquad}
\rhead{\qquad\thepage/\pageref{LastPage}}
\begin{document}

\tableofcontents

\clearpage

\section{创建脚本文件}
	脚本文件中,文件的第一行必须指明所使用的shell,格式如下所示;
	\begin{lstlisting}
	#!/bin/bash
	\end{lstlisting}

\subsection{第一个脚本文件}
	下面是一个简单的脚本文件:
	\begin{lstlisting}
	#!/bin/shell
	data
	who
	\end{lstlisting}

	可以将两条命令放在同一行,不过这需要将它们用分号隔开,如下所示:
	\begin{lstlisting}
	#!/bin/shell
	data; who
	\end{lstlisting}

\subsection{脚本文件的权限}
	通过ls命令可以查看linux系统上的文件、目录和设备的文件权限,如下图所示:
	\fic{1.png}

	图中输出清单中的第一个字段就是描述文件和目录的权限的代码,共有10个字符。\par
	字段中的第一个字符定义了对象的类型,分别有以下的字符:
	\begin{longtable}{p{2cm}p{4cm}}
	\hline
	- & 表示文件 \\
	\hline
	d & 表示目录 \\
	\hline
	l & 表示链接 \\
	\hline
	c & 表示字符设备 \\
	\hline
	b & 表示块设备 \\
	\hline
	n & 表示网络设备 \\
	\hline
	\end{longtable}

	字段之后的9个字符分为3组,每组有3个字符,这三组字符从左到右分别为对象的所有者、拥有对象的用户组和系统上的其他任何人设置了权限。
	每组中有4个可能的字符:
	\begin{longtable}{p{2cm}p{4cm}}
	\hline
	r & 表示对象读权限 \\
	\hline
	w & 表示对象写权限 \\
	\hline
	x & 表示对象执行权限 \\
	\hline
	- & 表示对象没有该位置上的权限 \\
	\hline
	\end{longtable}

	linux文件权限代码如下图所示:
	\fic{2.png}

\subsubsection{修改脚本文件权限}
	可以通过umask和touch设置文件的权限。umask可以设置完整权限要减去的值,比如一个文件“temp.txt”的完整权限是666,输入命令“umask 022”,再输入命令“touch temp.txt”,
	那么这个文件的权限就变成了644。\par
	需要注意的是,如果之前已经输入命令“umask 022”,那么新创建的文件的权限都是644。还需要补充的是,文件的完整权限是666,而目录的完整权限是777。\par

	还可以使用chmod命令来修改文件和目录的权限,chmod命令的格式如下:
	\begin{lstlisting}
	chmod [options] <mode> <file>
	# options提供了一些额外的特性来扩展chmod命令的行为
	# mode是八进制文件权限
	\end{lstlisting}

	chmod只需要根据需要为文件制定标准的3位八进制代码,如下所示:
	\begin{lstlisting}
	chmod 760 temp.txt
	\end{lstlisting}

	除此之外,chmod还可以通过符号模式来制定权限的格式,命令格式如下:
	\begin{lstlisting}
	chmod [options] <ugoa><+-=><rwxXstugo> <file>
	\end{lstlisting}

	符号模式中的第一组字符定义了新权限使用的对象,可能的字符如下所示:
	\begin{longtable}{p{1.5cm}p{3cm}}
	\hline
	u & 表示用户 \\
	\hline
	g & 表示用户组 \\
	\hline
	o & 表示其他任何人 \\
	\hline
	a & 表示上述所有 \\
	\hline
	\end{longtable}

	符号模式中的第二组字符可能的字符如下所示:
	\begin{longtable}{p{1.5cm}p{6cm}}
	\hline
	+ & 表示在已有权限中添加权限 \\
	\hline
	- & 表示从已有权限中减去权限 \\
	\hline
	= & 表示为权限赋值 \\
	\hline
	\end{longtable}

	符号模式中的第三组字符可能的字符如下所示:
	\begin{longtable}{p{1.5cm}p{6cm}}
	\hline
	x & 用于指定执行权限,仅当对象为目录时有效 \\
	\hline
	s & 用于设置正在执行的UID或GID \\
	\hline
	t & 用于保存程序文本 \\
	\hline
	u & 用于将权限设置为所有者的权限 \\
	\hline
	g & 用于将权限设置为用户组的权限 \\
	\hline
	o & 用于将权限设置为其他人的权限 \\
	\hline
	\end{longtable}

\subsubsection{设置脚本文件为可执行文件}
	可以使用chmod来赋予脚本文件可执行的权限,如下所示:
	\begin{lstlisting}
	# test是脚本文件
	chmod u+x test
	\end{lstlisting}

	然后就可以运行这个脚本文件了,命令如下所示:
	\begin{lstlisting}
	# 在当前目录下
	./test
	\end{lstlisting}

\section{显示消息}
	可以使用echo来显示字符串,可以使用双引号或单引号来标记文本字符串,命令如下所示:
	\begin{lstlisting}
	echo "This is a test to see if you are paying attention"
	echo 'Rich says "scripting is easy"'
	\end{lstlisting}

	可以通过带“-n”参数项来使得输出的字符串最后没有带换行符,命令如下所示:
	\begin{lstlisting}
	echo -n "This is a test to see if you are paying attention"
	\end{lstlisting}

\section{使用变量}
	shell的变量和linux的环境变量一样。在shell脚本中可以使用系统变量,也可以自己定义用户变量。在脚本文件中使用变量的例子如下所示:
	\begin{lstlisting}
	#!/bin/bash
	# 需要注意,变量、等号和变量值之间不允许有空格
	days=10
	guest="Katie"
	echo "$guest checked in $days days ago"
	days=5
	guest="Jessica"
	echo "$guest checked in $days days ago"
	# 使用系统变量
	echo "HOME: $HOME"
	\end{lstlisting}

	如果想要显示"\$"符号,只需要在它前面加上反义符号就行了。

\subsection{反引号}
	反引号“`”可以将shell命令的输出赋值给变量,使用方式是将整个命令行命令用反引号包围起来,如下所示:
	\begin{lstlisting}
	# 这样就能将date命令的输出赋给testing变量
	testing=`date`
	# 显示testing变量的值
	echo "The data and time are: " $testing
	\end{lstlisting}

\section{重定向输入输出}
	重定向既可以用于输入也可以用于输出,可以重定向一个文件到命令输入,也可以重定向命令输出到另一个位置。

\subsection{输出重定向}
	输出重定向就是将一条命令的输出发送到一个文件中,使用方式如下所示:
	\begin{lstlisting}
	command > outputfile
	\end{lstlisting}

	需要注意的是,如果文件已经存在,那么它里面的内容会被重写。如果不想重写此文件的内容,而是想将命令的输出附加到现有文件中,需要使用两个大于号,使用方式如下所示:
	\begin{lstlisting}
	command >> outputfile
	\end{lstlisting}

\subsubsection{/dev/null}
	/dev/null代表空设备文件。“echo "123" > /dev/null”等于将输出重定向到空设备文件,也就是不输出任何信息到终端,相当于销毁了这个输出信息。\par

	接下来解释命令“echo log > /dev/null 2>\&1”:
	\begin{lstlisting}
	1:表示stdout标准输出,系统默认值是1,所以"echo "123" >/dev/null"等同于"echo "123" 1>/dev/null"
	2:表示stderr标准错误
	&:表示等同于的意思,2>&1,表示2的输出重定向等同于1
	"echo log > /dev/null 2>&1"代表着标准输出和标准错误输出都重定向到空设备文件中
	\end{lstlisting}

\subsection{输入重定向}
	输入重定向就是将一个文件的内容重定向到一个命令中,使用方式如下所示:
	\begin{lstlisting}
	command < inputfile
	\end{lstlisting}

	以下是使用输入重定向的例子:
	\begin{lstlisting}
	wc < temp.txt
	# wc命令可以对数据中的文本计数,然后输出文本的行数、文本的单词数和文本的字节数
	\end{lstlisting}

	还可以在命令行中为输入重定向指定数据,这种方法叫做内置输入重定向。通过使用两个小于号可以使用内置输入重定向,
	除此之外,还需要指定一个文本标记来说明输入数据的开始和结尾,如下所示:
	\begin{lstlisting}
	# 一般使用EOF作为marker
	command << marker
	data
	marker
	\end{lstlisting}

\section{管道}
	管道就是将一个命令的输出发送至另一个命令的输入,它的符号是“|”,格式如下所示:
	\begin{lstlisting}
	# 将command1的输出发送至command2的输入
	command1 | command2
	\end{lstlisting}

\section{数学计算}
\subsection{expr命令}
	expr的命令操作符如下图所示:
	\fic{3.png}

	expr的使用如下:
	\begin{lstlisting}
	expr ARG1 + ARG2
	\end{lstlisting}

	在脚本中使用expr命令如下所示:
	\begin{lstlisting}
	#!/bin/bash
	var1=10
	var2=10
	var3=`expr $var1 + $var2`
	echo "The result is $var3"
	\end{lstlisting}

\subsection{使用方括号}
	使用数学式时,还可以用美元符号和方括号将数学等式括起来,如下所示:
	\begin{lstlisting}
	$[operation]
	\end{lstlisting}

	在脚本中使用例子如下:
	\begin{lstlisting}
	#!/bin/bash
	var1=100
	var2=50
	var3=$[$var1 * $var2]
	echo "The final result is $var3"
	\end{lstlisting}

	需要注意的是,上面的数学运算都是整数运算。

\subsection{进行浮点运算}
	可以使用bash计算器进行浮点运算,bash计算器可以识别:
	\begin{itemize}
		\item 数字
		\item 变量
		\item 注释
		\item 表达式
		\item 编程语句
		\item 函数
	\end{itemize}
	
	可以使用bc命令从shell提示符访问bash计算器,如下所示:
	\begin{lstlisting}
	# 输入bc命令就直接跳进了计算器
	bc
	\end{lstlisting}

	还可以在脚本中使用bc,基本格式如下所示:
	\begin{lstlisting}
	variable=`echo "options; expression" | bc`
	\end{lstlisting}

	其中options用于设置变量,如果要设置多个变量,需要使用分号将它们分隔开,expression定义了使用bc计算的数学表达式,例子如下所示:
	\begin{lstlisting}
	# scale指定了变量为4为小数
	var1=`echo "scale=4; 3.44/5" | bc`
	\end{lstlisting}

	还可以使用多行数学表达式,格式如下所示:
	\begin{lstlisting}
	variable=`bc << EOF
	options
	statements
	expressions
	EOF
	`
	\end{lstlisting}

	在脚本中使用bash计算器的例子如下所示:
	\begin{lstlisting}
	#!/bin/bash
	var1=1046
	var2=43.67
	var3=33.2
	var4=71
	var5=`bc << EOF
	scale = 4
	a1 = ($var1 * $var2)
	b1 ($var3 * $var4)
	a1 + b1
	EOF
	`
	\end{lstlisting}

	需要注意的是,bash计算器中的变量只在bash计算器内有效,不能用在shell脚本中。

\section{退出脚本}
	可以通过查看\$?这个特殊变量来查看最后一条命令执行结束的退出状态,命令如下所示:
	\begin{lstlisting}
	echo $?
	\end{lstlisting}

	linux中退出状态代码如下图所示:
	\fic{5.png}

	可以通过exit命令来指定shell脚本的退出状态,如下所示:
	\begin{lstlisting}
	#!/bin/bash
	var1=10
	var2=20
	var3=$[$var1+var2]
	exit 5
	\end{lstlisting}

\end{document}
