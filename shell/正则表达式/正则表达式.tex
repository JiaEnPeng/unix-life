% !TeX spellcheck = en_US
%% 字体:方正静蕾简体
%%		 方正粗宋
\documentclass[a4paper,left=1.5cm,right=1.5cm,11pt]{article}

\usepackage[utf8]{inputenc}
\usepackage{fontspec}
\usepackage{cite}
\usepackage{xeCJK}
\usepackage{indentfirst}
\usepackage{titlesec}
\usepackage{etoolbox}%
\makeatletter
\patchcmd{\ttlh@hang}{\parindent\z@}{\parindent\z@\leavevmode}{}{}%
\patchcmd{\ttlh@hang}{\noindent}{}{}{}%
\makeatother

\usepackage{hyperref}
\usepackage{longtable}
\usepackage{empheq}
\usepackage{graphicx}
\usepackage{float}
\usepackage{rotating}
\usepackage{subfigure}
\usepackage{tabu}
\usepackage{amsmath}
\usepackage{setspace}
\usepackage{amsfonts}
\usepackage{appendix}
\usepackage{listings}
\usepackage{xcolor}
\usepackage{geometry}
\setcounter{secnumdepth}{4}
%\titleformat*{\section}{\LARGE}
%\renewcommand\refname{参考文献}
%\titleformat{\chapter}{\centering\bfseries\huge}{}{0.7em}{}{}
\titleformat{\section}{\LARGE\bf}{\thesection}{1em}{}{}
\titleformat{\subsection}{\Large\bfseries}{\thesubsection}{1em}{}{}
\titleformat{\subsubsection}{\large\bfseries}{\thesubsubsection}{1em}{}{}
\renewcommand{\contentsname}{{ \centerline{目{  } 录}}}
\setCJKfamilyfont{cjkhwxk}{STXINGKA.TTF}
%\setCJKfamilyfont{cjkhwxk}{华文行楷}
%\setCJKfamilyfont{cjkfzcs}{方正粗宋简体}
%\newcommand*{\cjkfzcs}{\CJKfamily{cjkfzcs}}
\newcommand*{\cjkhwxk}{\CJKfamily{cjkhwxk}}
%\newfontfamily\wryh{Microsoft YaHei}
%\newfontfamily\hwzs{华文中宋}
%\newfontfamily\hwst{华文宋体}
%\newfontfamily\hwfs{华文仿宋}
%\newfontfamily\jljt{方正静蕾简体}
%\newfontfamily\hwxk{华文行楷}
\newcommand{\verylarge}{\fontsize{60pt}{\baselineskip}\selectfont}  
\newcommand{\chuhao}{\fontsize{44.9pt}{\baselineskip}\selectfont}  
\newcommand{\xiaochu}{\fontsize{38.5pt}{\baselineskip}\selectfont}  
\newcommand{\yihao}{\fontsize{27.8pt}{\baselineskip}\selectfont}  
\newcommand{\xiaoyi}{\fontsize{25.7pt}{\baselineskip}\selectfont}  
\newcommand{\erhao}{\fontsize{23.5pt}{\baselineskip}\selectfont}  
\newcommand{\xiaoerhao}{\fontsize{19.3pt}{\baselineskip}\selectfont} 
\newcommand{\sihao}{\fontsize{14pt}{\baselineskip}\selectfont}      % 字号设置  
\newcommand{\xiaosihao}{\fontsize{12pt}{\baselineskip}\selectfont}  % 字号设置  
\newcommand{\wuhao}{\fontsize{10.5pt}{\baselineskip}\selectfont}    % 字号设置  
\newcommand{\xiaowuhao}{\fontsize{9pt}{\baselineskip}\selectfont}   % 字号设置  
\newcommand{\liuhao}{\fontsize{7.875pt}{\baselineskip}\selectfont}  % 字号设置  
\newcommand{\qihao}{\fontsize{5.25pt}{\baselineskip}\selectfont}    % 字号设置 

\usepackage{diagbox}
\usepackage{multirow}
\boldmath
\XeTeXlinebreaklocale "zh"
\XeTeXlinebreakskip = 0pt plus 1pt minus 0.1pt
\definecolor{cred}{rgb}{0.8,0.8,0.8}
\definecolor{cgreen}{rgb}{0,0.3,0}
\definecolor{cpurple}{rgb}{0.5,0,0.35}
\definecolor{cdocblue}{rgb}{0,0,0.3}
\definecolor{cdark}{rgb}{0.95,1.0,1.0}
\lstset{
	language=bash,
	numbers=left,
	numberstyle=\tiny\color{black},
	showspaces=false,
	showstringspaces=false,
	basicstyle=\scriptsize,
	keywordstyle=\color{purple},
	commentstyle=\itshape\color{cgreen},
	stringstyle=\color{blue},
	frame=lines,
	% escapeinside=``,
	extendedchars=true, 
	xleftmargin=1em,
	xrightmargin=1em, 
	backgroundcolor=\color{cred},
	aboveskip=1em,
	breaklines=true,
	tabsize=4
} 

%\newfontfamily{\consolas}{Consolas}
%\newfontfamily{\monaco}{Monaco}
%\setmonofont[Mapping={}]{Consolas}	%英文引号之类的正常显示,相当于设置英文字体
%\setsansfont{Consolas} %设置英文字体 Monaco, Consolas,  Fantasque Sans Mono
%\setmainfont{Times New Roman}
%\setCJKmainfont{STZHONGS.TTF}
%\setmonofont{Consolas}
% \newfontfamily{\consolas}{YaHeiConsolas.ttf}
\newfontfamily{\monaco}{MONACO.TTF}
\setCJKmainfont{STZHONGS.TTF}
%\setmainfont{MONACO.TTF}
%\setsansfont{MONACO.TTF}

\newcommand{\fic}[1]{\begin{figure}[H]
		\center
		\includegraphics[width=0.8\textwidth]{#1}
	\end{figure}}
	
\newcommand{\sizedfic}[2]{\begin{figure}[H]
		\center
		\includegraphics[width=#1\textwidth]{#2}
	\end{figure}}

\newcommand{\codefile}[1]{\lstinputlisting{#1}}

\newcommand{\interval}{\vspace{0.5em}}

\newcommand{\tablestart}{
	\interval
	\begin{longtable}{p{2cm}p{10cm}}
	\hline}
\newcommand{\tableend}{
	\hline
	\end{longtable}
	\interval}

% 改变段间隔
\setlength{\parskip}{0.2em}
\linespread{1.1}

\usepackage{lastpage}
\usepackage{fancyhdr}
\pagestyle{fancy}
\lhead{\space \qquad \space}
\chead{正则表达式 \qquad}
\rhead{\qquad\thepage/\pageref{LastPage}}
\begin{document}

\tableofcontents

\clearpage

\section{基本的正则表达式}
\subsection{纯文本匹配}
	sed编辑器就使用标准的文本字符串筛选数据,如下例所示:
	\begin{lstlisting}
	echo "This is a test" | sed -n '/test/p'
	echo "This is a test" | sed -n "/this/p"
	\end{lstlisting}

	在正则表达式中,文本匹配不局限于完整的单词,如果所定义的文本出现在数据流的任意位置,正则表达式都将匹配,如下例所示:
	\begin{lstlisting}
	echo "This books are expensive" | sed -n '/book/p'
	\end{lstlisting}

\subsection{特殊字符}
	正则表达式认可的特殊字符有:
	\begin{lstlisting}
	.*[]^${}\+?|()
	\end{lstlisting}

	如果要使用这些特殊字符作为文本字符,需要在特殊符号前加反斜杠。

\subsubsection{定位符}
	\^和\$用于将模式定位到数据流中行的开头和结尾。\par
	\^定义从数据流中文本行开头开始的模式,如果该模式位于文本行的其他任意位置,正则表达式匹配失败。
	如果要使用\^,需要将\^放在正则表达式指定的模式之前。
	如下例所示:
	\begin{lstlisting}
	echo "This book store" | sed -n '/^book/p'
	echo "books are greate" | sed -n '/^book/p'
	\end{lstlisting}

	需要注意的是,如果将脱字符\^放在模式的其他位置,它就充当普通字符而不再作为特殊字符,如下例所示:
	\begin{lstlisting}
	echo "This ^ is a test" | sed -n '/s ^/p'
	\end{lstlisting}

	\$定义从数据流中文本行结尾处的模式,如果在文本模式之后添加这个特殊符号,表示数据行必须以此文本模式结束,如下例所示:
	\begin{lstlisting}
	echo "This is a good book" | sed -n '/book$/p'
	echo "This book is good" | sed -n '/book$/p'
	\end{lstlisting}

	\^和\$在没有文本的模式中结合使用时可以筛选数据流中的空行,如下例所示:
	\begin{lstlisting}
	sed '/^$/d' data1
	\end{lstlisting}

\subsubsection{点字符}
	点字符可以匹配除换行符之外的任何单个字符,如下例所示:
	\begin{lstlisting}
	sed -n '/.at/p' data1
	\end{lstlisting}

\subsubsection{字符类}
	字符类用于限制要匹配的字符。如果要定义字符类,要使用方括号,如下例所示:
	\begin{lstlisting}
	# 仅匹配cat和hat
	sed -n '/[ch]at/p' data1
	# 仅匹配0、1、2或3
	sed -n '/[0123]/p' data1
	\end{lstlisting}

\subsubsection{否定字符类}
	否定字符类用于查找不在该字符类中的字符,当然这些字符仍然需要符合模式。
	通过在字符类范围的开头添加脱字符,可以使用否定字符类,如下例所示:
	\begin{lstlisting}
	# 这个用于匹配除cat和hat之外的所有.at
	sed -n '/[^ch]at/p' data1
	\end{lstlisting}

\subsubsection{使用范围}
	可以通过使用短划线符号在字符类中使用一系列字符范围,只需要指定范围内的第一个字符、短划线和范围内的最后一个字符。例子如下:
	\begin{lstlisting}
	# 用于匹配5位数字
	sed -n '/^[0-9][0-9][0-9][0-9][0-9]$/p' data1
	# 用于匹配cat、dat、eat、fat、gat或hat
	sed -n '/[c-h]at/p' data1
	\end{lstlisting}

	还可以在单个字符类中指定多个非连续的范围,如下例所示:
	\begin{lstlisting}
	# 用于匹配aat、bat、cat、hat、iat、jat、kat、lat和mat
	sed -n '/[a-ch-m]at/p' data1
	\end{lstlisting}

\subsubsection{星号}
	在某一个字符之后加一个星号,表示该字符在匹配模式中的文本中不出现或者出现大等于1次,如下例所示:
	\begin{lstlisting}
	echo "ik" | sed -n '/ie*k/p'
	echo "iek" | sed -n '/ie*k/p'
	echo "ieek" | sed -n '/ie*k/p'
	\end{lstlisting}

	还可以将星号应用于字符类,用于指定在文本中多次出现的一组或某一范围的字符,如下例所示:
	\begin{lstlisting}
	echo "bt" | sed -n '/b[ae]*t/p'
	echo "bat" | sed -n '/b[ae]*t/p'
	echo "bet" | sed -n '/b[ae]*t/p'
	echo "baaaeeet" | sed -n '/b[ae]*t/p'
	\end{lstlisting}

\subsection{特殊字符类}
	BRE定义了一些特殊的字符类,如下所示:
	\begin{longtable}{p{2cm}p{5cm}}
	\hline
	[[:alpha:]] & 匹配任意字母字符 \\
	\hline
	[[:alnum:]] & 匹配任意字母数字字符 \\
	\hline
	[[:blank:]] & 匹配空格或制表符字符 \\
	\hline
	[[:digit:]] & 匹配任意数字 \\
	\hline
	[[:lower:]] & 匹配任意小写字母字符 \\
	\hline
	[[:print:]] & 匹配任意可打印字符 \\
	\hline
	[[:punct:]] & 匹配任意标点符号 \\
	\hline
	[[:space:]] & 匹配任意空白字符 \\
	\hline
	[[:upper:]] & 匹配任意大写字母字符 \\
	\hline
	\end{longtable}

\section{扩展的正则表达式}
\subsection{问号}
	在某一个字符之后加一个问号,表示该字符在匹配模式中的文本中不出现或者只出现1次,如下例所示:
	\begin{lstlisting}
	echo "bt" | sed -n '/be?t/p'
	echo "bet" | sed -n '/be?t/p'
	echo "beet" | sed -n '/be?t/p'
	\end{lstlisting}

	问号还可以加在字符类后面,表示这个字符类在匹配模式中的文本中不出现或者只出现1次,如下例所示:
	\begin{lstlisting}
	echo "bt" | sed -n '/b[ae]?t/p'
	echo "bat" | sed -n '/b[ae]?t/p'
	echo "bet" | sed -n '/b[ae]?t/p'
	echo "baat" | sed -n '/b[ae]?t/p'
	echo "beet" | sed -n '/b[ae]?t/p'
	echo "baet" | sed -n '/b[ae]?t/p'
	\end{lstlisting}

\subsection{加号}
	在某一个字符之后加一个加号,表示该字符在匹配模式中的文本中出现大等于1次,如下例所示:
	\begin{lstlisting}
	echo "bt" | sed -n '/be+t/p'
	echo "bet" | sed -n '/be+t/p'
	echo "beet" | sed -n '/be+t/p'
	\end{lstlisting}

	类似的,加号也可以和字符类一起使用。

\subsection{使用大括号}
	可以使用大括号指定匹配模式的重复次数,这种方式称为间隔。有两种格式表示间隔:
	\begin{itemize}
		\item m:该正则表达式正好出现m次
		\item m,n:该正则表达式出现最少m次,最多n次
	\end{itemize}

	使用例子如下所示:
	\begin{lstlisting}
	echo "bt" | sed -n '/be{1,2}t'
	echo "bet" | sed -n '/be{1,2}t'
	echo "beet" | sed -n '/be{1,2}t'
	\end{lstlisting}

	类似的,间隔模式也可以和字符类一起使用。

\subsection{管道符号}
	管道符号允许正则表达式匹配数据流时使用两个或多个模式。如果任何一个模式与数据流文本匹配,该文本通过。如果没有一个模式匹配,数据流文本匹配失败。
	使用管道符号的格式如下:
	\begin{lstlisting}
	expr1|expr2
	\end{lstlisting}

	使用例子如下:
	\begin{lstlisting}
	echo "This cat is asleep" | sed '/cat|dog/p'
	echo "This dog is asleep" | sed '/cat|dog/p'
	\end{lstlisting}

\subsection{将表达式分组}
	使用圆括号可以将正则表达式模式分组,一个组合将被作为一个标准字符处理。
	可以像将特殊字符应用于正常字符一样,将特殊字符应用于组合。例子如下:
	\begin{lstlisting}
	echo "Sat" | sed -n '/Sat\(urday\)?/p'
	\end{lstlisting}

	常见的用法是将分组和管道符号结合使用,如下例所示:
	\begin{lstlisting}
	echo "cat" | sed -n '/\(c|b\)a\(b|t\)/p'
	\end{lstlisting}

\end{document}
