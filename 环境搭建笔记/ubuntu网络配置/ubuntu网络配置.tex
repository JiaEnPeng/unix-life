% !TeX spellcheck = en_US
%% 字体:方正静蕾简体
%%		 方正粗宋
\documentclass[a4paper,left=2.5cm,right=2.5cm,11pt]{article}

\usepackage[utf8]{inputenc}
\usepackage{fontspec}
\usepackage{cite}
\usepackage{xeCJK}
\usepackage{indentfirst}
\usepackage{titlesec}
\usepackage{longtable}
\usepackage{graphicx}
\usepackage{float}
\usepackage{rotating}
\usepackage{subfigure}
\usepackage{tabu}
\usepackage{amsmath}
\usepackage{setspace}
\usepackage{amsfonts}
\usepackage{appendix}
\usepackage{listings}
\usepackage{xcolor}
\usepackage{geometry}
\setcounter{secnumdepth}{4}
\usepackage{mhchem}
\usepackage{multirow}
\usepackage{extarrows}
\usepackage{hyperref}
\titleformat*{\section}{\LARGE}
\renewcommand\refname{参考文献}
\renewcommand{\abstractname}{\sihao \cjkfzcs 摘{  }要}
%\titleformat{\chapter}{\centering\bfseries\huge\wryh}{}{0.7em}{}{}
%\titleformat{\section}{\LARGE\bf}{\thesection}{1em}{}{}
\titleformat{\subsection}{\Large\bfseries}{\thesubsection}{1em}{}{}
\titleformat{\subsubsection}{\large\bfseries}{\thesubsubsection}{1em}{}{}
\renewcommand{\contentsname}{{\cjkfzcs \centerline{目{  } 录}}}
\setCJKfamilyfont{cjkhwxk}{STXingkai}
\setCJKfamilyfont{cjkfzcs}{STSongti-SC-Regular}
% \setCJKfamilyfont{cjkhwxk}{华文行楷}
% \setCJKfamilyfont{cjkfzcs}{方正粗宋简体}
\newcommand*{\cjkfzcs}{\CJKfamily{cjkfzcs}}
\newcommand*{\cjkhwxk}{\CJKfamily{cjkhwxk}}
\newfontfamily\wryh{Microsoft YaHei}
\newfontfamily\hwzs{STZhongsong}
\newfontfamily\hwst{STSong}
\newfontfamily\hwfs{STFangsong}
\newfontfamily\jljt{MicrosoftYaHei}
\newfontfamily\hwxk{STXingkai}
% \newfontfamily\hwzs{华文中宋}
% \newfontfamily\hwst{华文宋体}
% \newfontfamily\hwfs{华文仿宋}
% \newfontfamily\jljt{方正静蕾简体}
% \newfontfamily\hwxk{华文行楷}
\newcommand{\verylarge}{\fontsize{60pt}{\baselineskip}\selectfont}  
\newcommand{\chuhao}{\fontsize{44.9pt}{\baselineskip}\selectfont}  
\newcommand{\xiaochu}{\fontsize{38.5pt}{\baselineskip}\selectfont}  
\newcommand{\yihao}{\fontsize{27.8pt}{\baselineskip}\selectfont}  
\newcommand{\xiaoyi}{\fontsize{25.7pt}{\baselineskip}\selectfont}  
\newcommand{\erhao}{\fontsize{23.5pt}{\baselineskip}\selectfont}  
\newcommand{\xiaoerhao}{\fontsize{19.3pt}{\baselineskip}\selectfont} 
\newcommand{\sihao}{\fontsize{14pt}{\baselineskip}\selectfont}      % 字号设置  
\newcommand{\xiaosihao}{\fontsize{12pt}{\baselineskip}\selectfont}  % 字号设置  
\newcommand{\wuhao}{\fontsize{10.5pt}{\baselineskip}\selectfont}    % 字号设置  
\newcommand{\xiaowuhao}{\fontsize{9pt}{\baselineskip}\selectfont}   % 字号设置  
\newcommand{\liuhao}{\fontsize{7.875pt}{\baselineskip}\selectfont}  % 字号设置  
\newcommand{\qihao}{\fontsize{5.25pt}{\baselineskip}\selectfont}    % 字号设置 

\usepackage{diagbox}
\usepackage{multirow}
\boldmath
\XeTeXlinebreaklocale "zh"
\XeTeXlinebreakskip = 0pt plus 1pt minus 0.1pt
\definecolor{cred}{rgb}{0.8,0.8,0.8}
\definecolor{cgreen}{rgb}{0,0.3,0}
\definecolor{cpurple}{rgb}{0.5,0,0.35}
\definecolor{cdocblue}{rgb}{0,0,0.3}
\definecolor{cdark}{rgb}{0.95,1.0,1.0}
\lstset{
	language=bash,
	numbers=left,
	numberstyle=\tiny\color{white},
	showspaces=false,
	showstringspaces=false,
	basicstyle=\scriptsize,
	keywordstyle=\color{purple},
	commentstyle=\itshape\color{cgreen},
	stringstyle=\color{blue},
	frame=lines,
	% escapeinside=``,
	extendedchars=true, 
	xleftmargin=0em,
	xrightmargin=0em, 
	backgroundcolor=\color{cred},
	aboveskip=1em,
	breaklines=true,
	tabsize=4
} 

\newfontfamily{\consolas}{Consolas}
\newfontfamily{\monaco}{Monaco}
\setmonofont[Mapping={}]{Consolas}	%英文引号之类的正常显示,相当于设置英文字体
\setsansfont{Consolas} %设置英文字体 Monaco, Consolas,  Fantasque Sans Mono
\setmainfont{Times New Roman}

\setCJKmainfont{华文中宋}


\newcommand{\fic}[1]{\begin{figure}[H]
		\center
		\includegraphics[width=0.8\textwidth]{#1}
	\end{figure}}
	
\newcommand{\sizedfic}[2]{\begin{figure}[H]
		\center
		\includegraphics[width=#1\textwidth]{#2}
	\end{figure}}

\newcommand{\codefile}[1]{\lstinputlisting{#1}}

\newcommand{\interval}{\vspace{0.5em}}

\newcommand{\tablestart}{
	\interval
	\begin{longtable}{p{2cm}p{10cm}}
	\hline}
\newcommand{\tableend}{
	\hline
	\end{longtable}
	\interval}

% 改变段间隔
\setlength{\parskip}{0.2em}
\linespread{1.1}

\usepackage{lastpage}
\usepackage{fancyhdr}
\pagestyle{fancy}
\lhead{\space \qquad \space}
\chead{ubuntu网络配置 \qquad}
\rhead{\qquad\thepage/\pageref{LastPage}}
\begin{document}

\tableofcontents

\clearpage

\section{以太网接口}
	ubuntu系统使用ethX来认证以太网接口,其中X代表一个数值。
	第一个以太网接口为eth0,第二个以太网接口为eth1,依此类推。

\subsection{查询以太网接口}
	可以使用如下命令查询系统的以太网接口:
	\begin{lstlisting}
	ifconfig -a | grep eth
	\end{lstlisting}

	还可以使用如下命令查询系统中的所有以太网接口:
	\begin{lstlisting}
	lshw -class network
	\end{lstlisting}

\subsection{更改以太网接口名}
	以太网接口名在/etc/udev/rules.d/70-persistent-net.rules中配置,如果想要给以太网接口另取名字的话,
	可以先找到对应的mac地址的以太网接口,然后将如下值改为自己想要的名字:
	\begin{lstlisting}
	# ethX是原值,可以将它改为自己想要的名字
	NAME=ethX
	\end{lstlisting}

	修改之后,重启系统即可。

\subsection{设置以太网接口}
	以太网卡的设置包括:自动协商、端口速度、双工模式和Wake-on-LAN方式。
	可以使用ethtool工具设置以太网卡,安装命令如下:
	\begin{lstlisting}
	sudo apt install ethtool
	\end{lstlisting}

	通过以下命令可以查看以太网卡的设置:
	\begin{lstlisting}
	ethtool eth0
	\end{lstlisting}

	ethtool设置以太网卡的命令格式如下:
	\begin{lstlisting}
	# DEVNAME是网卡名
	ethtool [options] -s DEVNAME
	\end{lstlisting}

	ethtool的options可以通过“ethtool -h”查看。\par

	需要注意的是,ethtool对以太网卡的设置是暂时的,当系统重启后将恢复原样。
	如果想永久性地设置以太网卡,需要编辑/etc/network/interfaces文件。如下例所示:
	\begin{lstlisting}
	# /etc/network/interfaces
	auto eth0
	# 网卡可以不是静态
	iface eth0 inet static
	# 使用ethtool设置网卡,速度为1000Mb/s,全双工模式
	pre-up /sbin/ethtool -s eth0 speed 1000 duplex full
	\end{lstlisting}

\section{IP地址}
	这一节将介绍如何配置ubuntu的IP地址和默认网关,从而让系统可以在局域网和互联网在通信。

\subsection{暂时的IP地址分配}
	我们可以使用“ip”、“ifconfig”和“route”等命令来对网络进行暂时地配置。
	需要知道的是,这些配置不是永久性的,当系统重启后将恢复原样。\par

\subsubsection{使用ifconfig设置IP地址}

	如下例,用“ifconfig”命令暂时性地设置eth0的IP地址和子网掩码:
	\begin{lstlisting}
	sudo ifconfig eth0 10.0.0.100 netmask 255.255.255.0
	\end{lstlisting}

	使用“ifconfig”设置网络的命令如下:
	\begin{lstlisting}
	ifconfig <interfaces> [options]
	\end{lstlisting}

	上述命令的options可以通过“ifconfig -h”命令查看。\par

	还可以通过“ifconfig”来查看网卡的网络设置,如下例所示:
	\begin{lstlisting}
	ifconfig eth0
	\end{lstlisting}

\subsubsection{使用route设置默认网关}
	可以使用“route”命令为网卡接口添加默认网关,如下例所示:
	\begin{lstlisting}
	# 将eth0的默认网关设置为10.0.0.1
	sudo route add default gw 10.0.0.1 eth0
	\end{lstlisting}

	可以通过如下命令查看内核的IP路由表:
	\begin{lstlisting}
	route -n
	\end{lstlisting}

\subsubsection{暂时设置dns}
	可以通过编辑/etc/resolv.conf这个文件来暂时设置网络的域名服务器,如下例所示:
	\begin{lstlisting}
	nameserver 8.8.8.8
	nameserver 8.8.4.4
	\end{lstlisting}

	这个修改是暂时的,当系统重启后将恢复原状。

\subsubsection{恢复原状}
	如果想清除对一个网卡做出的配置,可以使用如下命令:
	\begin{lstlisting}
	# DEVNAME是网卡的名字
	ip addr flush DEVNAME
	\end{lstlisting}

	需要注意的是,上述命令并不会还原对域名服务器的改动。如果要还原DNS,需要重启系统或者修改/etc/resolv.conf这个文件。

\subsection{动态IP地址分配}
	通过编辑/etc/network/interfaces,可以将网卡ethX设置为动态IP地址分配:
	\begin{lstlisting}
	# /etc/network/interfaces
	auto ethX
	iface ethX inet dhcp
	\end{lstlisting}

	随后还需要通过“ifup”命令启动一个以太网接口,同时开始该网卡的DHCP进程,命令如下:
	\begin{lstlisting}
	sudo ifup ethX
	\end{lstlisting}

	如果想手动关闭一个以太网接口,可以使用“ifdown”命令,这条命令同时会启动DHCP释放进程关闭动态IP地址分配,命令格式如下:
	\begin{lstlisting}
	sudo ifdown ethX
	\end{lstlisting}

\subsection{静态IP地址分配}
	如果想让一个以太网接口使用静态IP地址分配,可以在/etc/network/interfaces添加如下内容,以eth0为例:
	\begin{lstlisting}
	# /etc/network/interfaces
	auto eth0
	iface eth0 inet static
	\end{lstlisting}

	同时还需要设置IP地址、子网掩码和网关,依然以eth0为例:
	\begin{lstlisting}
	# /etc/network/interfaces
	auto eth0
	iface eth0 inet static
	address 10.0.0.100
	netmask 255.255.255.0
	gateway 10.0.0.1
	\end{lstlisting}

	随后还需要手动启动eth0这个以太网接口,命令如下:
	\begin{lstlisting}
	sudo ifup eth0
	\end{lstlisting}

	如果想关闭eth0这个接口,可以使用如下命令:
	\begin{lstlisting}
	sudo ifdown eth0
	\end{lstlisting}

\subsection{回环接口lo}
	回环接口在ubuntu中称为lo,它的默认IP地址为127.0.0.1,可以通过“ifconfig”查看它的详细信息,命令如下所示:
	\begin{lstlisting}
	ifconfig lo
	\end{lstlisting}

	在/etc/network/interfaces文件中对lo有如下默认设置:
	\begin{lstlisting}
	# /etc/network/interfaces
	auto lo
	iface lo inet loopback
	\end{lstlisting}

	需要注意的是,如果没有明确目的,不要修改这个设置。

\section{名称解析}
	名称解析就是将IP地址映射为主机名的过程,从而使得资源在网络上更好辨别。
	接下来的部分将介绍使用DNS和静态主机名记录配置ubuntu系统,从而正确地进行名字解析。

\subsection{DNS客户端配置}
	一般来说,/etc/resolv.conf这个文件的内容是固定不变的。
	现在的电脑会频繁地切换网络,而resolvconf框架被用来跟踪这些变化并更新resolver的配置。
	resolvconf框架作为一个中介,用于关联提供nameserver信息的程序和需要nameserver信息的应用。
	resolvconf框架通过一组与网络接口配置有关的hook脚本来填充信息。
	对于用户最明显的区别是,任何对/etc/resolv.conf文件的手动修改都会被还原,因为只是resolvconf框架被触发,这个文件都会被覆写。
	而对于reolvconf框架而言,它使用DHCP客户端的hook和/etc/network/interfaces来修改/etc/resolv.conf。\par

	我们可以通过修改/etc/network/interfaces文件来配置resolver,添加的内容如下例所示:
	\begin{lstlisting}
	iface eth0 inet static
		address 192.168.3.3
		netmask 255.255.255.0
		gateway 192.168.3.1
		dns-search example.com
		dns-nameservers 192.168.3.45 192.168.8.10
	\end{lstlisting}

	其中,IP地址、子网掩码和网关需要正确配置,而以“dns-”为前缀的选项是可选的,用于配置resolv.conf的其他配置选项。\par

	需要知道的是,“dns-search”选项中的域名可以包含多个,如下例所示:
	\begin{lstlisting}
	iface eth0 inet static
		address 192.168.3.3
		netmask 255.255.255.0
		gateway 192.168.3.1
		dns-search example.com sales.example.com dev.example.com
		dns-nameservers 192.168.3.45 192.168.8.10
	\end{lstlisting}

\subsection{静态主机名}
	静态主机名就是主机名与IP地址的相互映射,这些映射关系在/etc/hosts文件中记录。
	在hosts文件中记录的条目优先于DNS,也就是说,如果需要解析的主机名在hosts文件中被记录,系统将不会再去查询DNS的映射关系。
	当主机名较少且不需要访问Internet时,优先选择配置/etc/hosts文件。\par

	如下是/etc/hosts文件的内容示例:
	\begin{lstlisting}
	127.0.0.1	localhost
	127.0.0.1	ubuntu-server
	10.0.0.11	server1 server1.example.com vpn
	10.0.0.12	server2 server2.example.com mail
	10.0.0.13	server3 server3.example.com www
	10.0.0.14	server4 server4.example.com file
	\end{lstlisting}

\subsection{配置name service switch}
	/etc/nsswitch.conf文件中配置着系统选择将主机名解析为IP地址的方法的顺序,默认配置如下所示:
	\begin{lstlisting}
	hosts:	files mdns4_minimal [NOTFOUND=return] dns mdns4
	\end{lstlisting}

	其中files代表/etc/hosts文件,mdns4\_minimal代表Multicast DNS,[NOTFOUND=return]代表解析到此结束,dns代表传统的单播DNS查询,mdns4代表Multicast DNS查询。
	如果想让dns作为第二优先级,可以将/etc/nsswitch.conf文件修改为如下所示:
	\begin{lstlisting}
	hosts:	files dns [NOTFOUND=return] mdns4_minimal mdns4
	\end{lstlisting}

\section{桥接}
	桥接多个接口比较复杂,但是在一些方案中很有用。
	一个方案是在多个网络接口之间建立网桥,然后使用一个防火墙来滤除两个网段之间的连接。
	另一个方案是利用网桥让虚拟机可以访问外部网络。\par
	接下来就介绍如何让一个虚拟机通过网桥访问外部网络:
	\begin{itemize}
		\item[1.] 首先在宿主机上安装bridge-utils包,命令如下:
	\begin{lstlisting}
	sudo apt install bridge-utils
	\end{lstlisting}

		\item[2.] 然后编辑宿主机上的/etc/network/interfaces文件,添加网桥的内容如下:
		\begin{lstlisting}
	# ...
	auto br0
	iface br0 inet static
		# address和netmask的值需要是一个合适的值
		address 192.168.0.10
		network 192.168.0.0
		netmask 255.255.255.0
		broadcast 192.168.0.255
		gateway 192.168.0.1
		bridge_ports eth0
		bridge_fd 9
		bridge_hello 2
		bridge_maxage 12
		bridge_stp off
		\end{lstlisting}

		\item[3.] 建立网桥,命令如下:
		\begin{lstlisting}
	sudo ifup br0
		\end{lstlisting}
	\end{itemize}

\end{document}
